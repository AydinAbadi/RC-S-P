% !TEX root =R-zkCSP.tex


\subsection{Recurring Contingent Service Payment (RC-S-P) Protocol}\label{sec::RC-S-P-protocol}


 
 
 In this section, we present  ``recurring contingent service payment'' (RC-S-P) protocol for a generic service. It utilises a novel combination of $\text{VSID}_{\scriptscriptstyle\text{light}}$, SAP, the private time bubble notion, and symmetric key encryption schemes along with the coin masking and padding techniques. At a high level the protocol works as follows.  The  client and server use SAP to provably agree on two private statements; first statement includes  payment details, while another one specifies a secret key, $k$, and a pad's length. They also  agree on  public parameters such as (a) the private time bubble's length, that is the total number of billing cycles: $z$, plus a waiting period, $\texttt{H}$, and (b) a smart contract which  specifies $z$ and  the total amount of masked coins each party should deposit. They deploy the contract.  Each party deposits its masked coins in the contract. If either party does not deposit enough coins on time,   later each party has a chance to withdraw its coins and terminate the contract. To start using/providing the service, they invoke $\text{VSID}_{\scriptscriptstyle\text{light}}$ protocol. In particular, they engage in   $\mathtt{VSID.keyGen}(.)$, $\mathtt{VSID.setup}(.)$, and $\mathtt{VSID.serve}(.)$ algorithms. If the server decides not to serve, e.g. it detects the client's misbehaviour,  it sends $0$  within a fixed time; in this case, the parties can withdraw their deposit and terminate the contract. Otherwise, the server sends $1$ to the contract. 
 
 
 
At the end of each billing cycle, the client generates an encrypted query, by calling $\mathtt{VSID.genQuery}(.)$ and encrypting its output using the key, $k$. It pads the encrypted query and  sends the result  to the contract. The encryption and pads ensure  nothing about the client's input (e.g. outsourced file) is revealed to the public within the private time bubble.  In the same cycle, the server retrieves the query, removes the pads and decrypts the result. Then, it locally checks its validity, by calling $\mathtt{VSID.checkQuery}(.)$. If the query is rejected, the server locally stores the index of the billing cycle and then  generates a dummy proof.  Otherwise, if the server accepts the query, it generates a proof of service by calling $\mathtt{VSID.prove}(.)$. In either case, the server encrypts the proof, pads it and sends the result to the contract. Note that sending (padded encrypted) dummy proof ensures that the public, during the private time bubble, does not learn if the client generates invalid queries. 

 After the server sends the messages to the contract, the client removes the pads, decrypts the proof and locally verifies it, by calling $\mathtt{VSID.verify}(.)$. If the verification is passed, then the client knows the server has delivered the service honestly. However, if the proof is rejected,  it waits until the private time bubble passes and dispute resolution time arrives. During the dispute resolution period, in the case the client or server  rejects any proofs, it sends a ``dispute'' message to the contract. The party also invokes the arbiter,  refers it to the invalid encrypted proofs in the contract, and  sends to it the decryption key and the pads' detail. The arbiter checks the validity of the key and pads, by using SAP. If they are accepted,  then the arbiter locally removes the pads from the encrypted proofs,   decrypts the related proofs, and runs $\mathtt{VSID.identify}(.)$ to check the validity of the party's claim. The arbiter sends to the contract a report of its findings that includes  the total number of times  the server and client provided invalid proofs. In the next phase, to distribute the coins, either client or server  sends: (a) ``pay'' message, (b) the agreed statement that specifies the payment details, and (c) the statement's proof to the contract which verifies the statement and if approved it distributes the coins according to the statement's detail, and the arbiter's report. 

Now we outline why RC-S-P addresses the issues.    In the setup, if the client provides ill-formed inputs (so later it can  accuse the server) then the server can detect and avoid serving it.  After the setup, if the client avoids sending any input,  then the server still gets paid for the service it provided. Also, in the case of a dispute between the parties, their claim is checked, and  the corrupt party is identified. The corrupt party has to pay the arbiter and if that is the client, then it has to pay the server as well.  These  features not only do guarantee the server's resource is not wasted, but also ensures fairness (i.e.  if a potentially malicious server is paid, then it must have provided the service and if a potentially malicious client does not pay, then it will learn nothing). Furthermore, as during the private time bubble  (a) no plaintext proof is given to the contract, and (b) no dispute resolution and coin transfer take place on contract, the public cannot figure out the outcome of each verification. This  preserves the server's privacy. Also, because the  deposited coins are masked and the agreed statement is kept private,   nothing about the detail of the service  is leaked to the public before the bubble bursts. This preserves the client's privacy. Also, as either party can prove to the contract the validity of the agreed statement, and ask the contract to distribute the coins, the coins will be not be locked forever. 

The RC-S-P protocol is presented below. It is assumed that (a) each party $\mathcal{R} \in\{\mathcal{C,S,A}r\}$ already has a blockchain public address, $adr_{\scriptscriptstyle\mathcal{R}}$, which is known to all parties, (b) it uses that (authorised) address to send transactions to the smart contract,  (c) the contract   before recording a transaction, ensures the transaction is  originated from an authorised address, and (d) there is a public price list $pl$ known to eveyone.

\begin{enumerate}
\item \textbf{Key Generation}. $\mathtt{RCSP}.\mathtt{keyGen}(1^{\scriptscriptstyle\lambda}, F)$
\begin{enumerate}


\item  $\mathcal C$ runs  $\mathtt{VSID.keyGen}(1^{\lambda},F)\rightarrow k:=(sk,pk)$. It picks a random secret key  $\bar{k}$ for a symmetric key encryption. Also,  it sets two parameters: $pad_{\scriptscriptstyle \pi}$ and $pad_{\scriptscriptstyle q}$, where  $pad_{\scriptscriptstyle \pi}$ and $pad_{\scriptscriptstyle q}$  refer to the number of  dummy values that will be used to pad encrypted proofs and encrypted queries respectively\footnote{\scriptsize The values of  $pad_{\scriptscriptstyle \pi}$ and $pad_{\scriptscriptstyle q}$ is determined as follows, $pad_{\scriptscriptstyle \pi}=\pi_{\scriptscriptstyle max}-\pi_{\scriptscriptstyle act}$ and $pad_{\scriptscriptstyle q}=q_{\scriptscriptstyle max}-q_{\scriptscriptstyle act}$, where $\pi_{\scriptscriptstyle max}$ and $\pi_{\scriptscriptstyle act}$ refer to the maximum and actual the service's proof size  while $q_{\scriptscriptstyle max}$ and $q_{\scriptscriptstyle act}$ refer to the maximum and actual the service's query size, respectively.},  determined by the security parameter and description of $F$.  Let $qp:=(pad_{\scriptscriptstyle \pi},pad_{\scriptscriptstyle q}, \bar{k})$. The keys' size is part of the security parameter.  Let  $\bm{k}=[k,k']$, where $k':=(sk',pk')$,   $sk'=qp$ and $pk':=(adr_{\scriptscriptstyle\mathcal{C}},adr_{\scriptscriptstyle\mathcal{S}})$. 
\end{enumerate}

\

\item \textbf{Client-side Initiation}. $\mathtt{RCSP}.\mathtt{cInit}(1^\lambda, u,\bm{k}, M,z,pl,enc)$
\begin{enumerate}

\item Calls  $\mathtt{VSID.setup}(1^{\lambda}, u,k,M)\rightarrow (u^{\scriptscriptstyle *}, {pp},e)$, to encode  service input, and generate metadata. {It appends $pp$ to $qp$}.



\item  Calls $\mathtt{SAP.init}(1^{\scriptscriptstyle\lambda}, adr_{\scriptscriptstyle\mathcal{C}}, adr_{\scriptscriptstyle\mathcal{S}},qp )\rightarrow(r_{\scriptscriptstyle qp}, g_{\scriptscriptstyle qp}, adr_{\scriptscriptstyle\text{SAP}_{\scriptscriptstyle 1}})$, to initiate an agreement (with $\mathcal{S}$) on $qp$.  Let $T_{\scriptscriptstyle qp}:=(\ddot{x}_{\scriptscriptstyle qp}, g_{\scriptscriptstyle qp})$ be proof/query encoding token,  where  $\ddot{x}_{\scriptscriptstyle qp}:=(qp,r_{\scriptscriptstyle qp})$ is the opening and $g_{\scriptscriptstyle qp}$ is the commitment stored on the contract as a result of running   SAP. 


\item   Sets coin parameters as   follows,   $o$: the amount of coins for each accepting proof, and $l$: the amount of coins  to cover the cost of each potential dispute resolution, given  price list $pl$.  %It appends $o$ and $l$ to $k'$.




\item  Sets  $cp:=(o,o_{\scriptscriptstyle max},l,l_{\scriptscriptstyle max},z)$, where $o_{\scriptscriptstyle max}$ is the maximum amount of coins  for an accepting service proof,  $l_{\scriptscriptstyle max}$  is the maximum amount of coins to resolve a potential dispute, and $z$ is the number of  service proofs/verifications. Then, $\mathcal C$ calls $\mathtt{SAP.init}(1^{\scriptscriptstyle\lambda}, adr_{\scriptscriptstyle\mathcal{C}}, adr_{\scriptscriptstyle\mathcal{S}},cp )\rightarrow(r_{\scriptscriptstyle cp},g_{\scriptscriptstyle cp},adr_{\scriptscriptstyle\text{SAP}_{\scriptscriptstyle 2}})$, to initiate an agreement (with $\mathcal{S}$) on $cp$. Let $T_{\scriptscriptstyle cp}:=(\ddot{x}_{\scriptscriptstyle cp},g_{\scriptscriptstyle cp})$ be coin encoding token,  where  $\ddot{x}_{\scriptscriptstyle cp}:=(cp,r_{\scriptscriptstyle cp})$ is the opening and $g_{\scriptscriptstyle cp}$ is the commitment stored on the contract as a result of executing   SAP.  





\item Set parameters  $coin^{\scriptscriptstyle *}_{\scriptscriptstyle\mathcal C}=z\cdot (o_{\scriptscriptstyle max}+l_{\scriptscriptstyle max})$ and $p_{\scriptscriptstyle\mathcal{S}}=z\cdot l_{\scriptscriptstyle max}$, where $coin^{\scriptscriptstyle *}_{\scriptscriptstyle\mathcal C}$ and $p_{\scriptscriptstyle\mathcal{S}}$ are   the total number of masked coins $\mathcal C$ and $\mathcal S$ should deposit respectively. It signs and deploys a smart contract, SC, that explicitly specifies  parameters $z$,  $coin^{\scriptscriptstyle *}_{\scriptscriptstyle\mathcal C}$, $p_{\scriptscriptstyle\mathcal{S}}$,  $adr_{\scriptscriptstyle\text{SAP}_{\scriptscriptstyle 1}}$, and $adr_{\scriptscriptstyle\text{SAP}_{\scriptscriptstyle 2}}$. Let $adr_{\scriptscriptstyle \text{SC}}$ be the address of the deployed SC, and $T:=\{T_{\scriptscriptstyle qp},T_{\scriptscriptstyle cp}\}$. It sets a set of time points/windows, $\texttt{Time}:\{ \texttt{T}_{\scriptscriptstyle 0},..., \texttt{T}_{\scriptscriptstyle 3},\texttt{G}_{\scriptscriptstyle 1,1},...,\texttt{G}_{\scriptscriptstyle z,2},\texttt{H}, \texttt{K}_{\scriptscriptstyle 1},...,\texttt{K}_{\scriptscriptstyle 3}, \texttt{L}\}$, that are explicitly specified in the contract who will accept a certain party's message only in a specified  time point/window. The time allocation will become clear in the next phases. 


\item Sets four counters $[y_{\scriptscriptstyle\mathcal C},y'_{\scriptscriptstyle\mathcal C},y_{\scriptscriptstyle\mathcal S}, y'_{\scriptscriptstyle\mathcal S}]$ in the contract, where their initial value is $0$. Let $\bm{y}: [y_{\scriptscriptstyle\mathcal C},y'_{\scriptscriptstyle\mathcal C},y_{\scriptscriptstyle\mathcal S}, y'_{\scriptscriptstyle\mathcal S}, \texttt{Time},adr_{\scriptscriptstyle \text{SC}}]$

\item Deposits $coin^{\scriptscriptstyle *}_{\scriptscriptstyle\mathcal C}$ coins in the contract.

\item Sends $u^{\scriptscriptstyle *}, e, \ddot{x}_{\scriptscriptstyle qp},\ddot{x}_{\scriptscriptstyle cp}$, and $p_{\scriptscriptstyle\mathcal{S}}$ (along with $adr_{\scriptscriptstyle \text{SC}}$) to $\mathcal S$. It sends  $(pk,pk')$ to SC. Let $\texttt{T}_{\scriptscriptstyle 0}$ be the time that the above process finishes. 



\end{enumerate}

\
\item \textbf{Server-side Initiation}\label{RCSP::Server-side-Initiation}. $\mathtt{RCSP}.\mathtt{sInit}(u^{\scriptscriptstyle *}, e, pk, z, T, p_{\scriptscriptstyle\mathcal{S}} ,\bm{y},enc)$

\begin{enumerate}

\item Checks the parameters  in $T$ (e.g. $qp$ and $cp$) and  in SC (e.g. $p_{\scriptscriptstyle\mathcal{S}}, \bm{y}$) and ensures sufficient amount of coins has been deposited by $\mathcal C$.  

\item Calls $\mathtt{SAP.agree}(qp,r_{\scriptscriptstyle qp},g_{\scriptscriptstyle qp},adr_{\scriptscriptstyle\mathcal{C}},adr_{\scriptscriptstyle\text{SAP}_{1}})\rightarrow (g'_{\scriptscriptstyle qp},b_{\scriptscriptstyle 1})$ and $\mathtt{SAP.agree}(cp,r_{\scriptscriptstyle cp},g_{\scriptscriptstyle cp},adr_{\scriptscriptstyle\mathcal{C}},adr_{\scriptscriptstyle\text{SAP}_{2}})\rightarrow (g'_{\scriptscriptstyle cp},b_{\scriptscriptstyle 2})$, to verify the correctness of tokens in $T$ and to agree on the tokens' parameters, where $qp,r_{\scriptscriptstyle qp}\in \ddot{x}_{\scriptscriptstyle qp}, \text{and }  cp,r_{\scriptscriptstyle cp} \in  \ddot{x}_{\scriptscriptstyle cp}$. Recall that if both $\mathcal{C}$ and $\mathcal{S}$ are honest, then $g_{\scriptscriptstyle qp}=g'_{\scriptscriptstyle qp}$ and $g_{\scriptscriptstyle cp}=g'_{\scriptscriptstyle cp}$. 




\item If any above check is rejected, then it sets $a=0$. Otherwise, it  calls $\mathtt{VSID.serve}(u^{\scriptscriptstyle *},e,pk, {pp})\rightarrow a$. 

\item Sends $a$ and $coin^{\scriptscriptstyle *}_{\scriptscriptstyle\mathcal S}=p_{\scriptscriptstyle\mathcal{S}}$ coins to SC at time $\texttt{T}_{\scriptscriptstyle 1}$, where  $coin^{\scriptscriptstyle *}_{\scriptscriptstyle\mathcal S}=\bot$ if $a=0$
\end{enumerate}
Note, $\mathcal S$  and  $\mathcal C$ can withdraw their coins at time $\texttt{T}_{\scriptscriptstyle 2}$, if  $\mathcal S$ sends $a=0$, fewer coins than $p_{\scriptscriptstyle\mathcal{S}}$, or nothing to the SC. To withdraw, $\mathcal S$  or  $\mathcal C$  simply sends a ``pay'' message to $\mathtt{RCSP}.\mathtt{pay}(.)$ algorithm (only) at time $\texttt{T}_{\scriptscriptstyle 2}$.

\

\textbf{\textit{Billing-cycles Onset}}. $\mathcal{C}$ and $\mathcal{S}$ engage in the following three phases, i.e. phase \ref{Billing-cycles-genQuery}-\ref{Client-sideProofVerification},  at the end of every $j$-th billing cycle, where $1\leq j\leq z$. Each $j$-th cycle includes two  time points, $\texttt{G}_{\scriptscriptstyle j,1}$ and $\texttt{G}_{\scriptscriptstyle j,2}$, where $\texttt{G}_{\scriptscriptstyle j,2}>\texttt{G}_{\scriptscriptstyle j,1}$, and $\texttt{G}_{\scriptscriptstyle 1,1}>\texttt{T}_{\scriptscriptstyle 2}$ 



\

\item\textbf{Client-side Query Generation}.\label{Billing-cycles-genQuery} $\mathtt{RCSP}.\mathtt{genQuery}(1^\lambda, \text{aux},k,Q, T_{\scriptscriptstyle qp}, enc)$

\begin{enumerate}
\item Calls $\mathtt{VSID.genQuery}(1^{\lambda},  \text{aux},k,Q, {pp})\rightarrow c_{\scriptscriptstyle j}:=(\bm{q}_{\scriptscriptstyle j},\bm{w}_{\scriptscriptstyle q_{_{\scaleto{j}{\size}}}})$, to generate a query-proof pair. 
\item Encrypts the pair, $\mathtt{Enc}(\bar{k},c_{\scriptscriptstyle j})=c'_{\scriptscriptstyle j}$, where $\bar{k}\in T_{\scriptscriptstyle qp}$. Then, it pads (each element of) the result with ${pad}_{\scriptscriptstyle q}\in T_{\scriptscriptstyle qp}$ random values that are picked uniformly at random from the encryption's output range, $U$. Let $c^{\scriptscriptstyle *}_{\scriptscriptstyle j}$ be the result. 
\item Sends the padded encrypted query-proof pair, $c^{\scriptscriptstyle *}_{\scriptscriptstyle j}$, to SC at time $\texttt{G}_{\scriptscriptstyle j,1}$
\end{enumerate}

\

\item\textbf{Server-side Proof Generation}. $\mathtt{RCSP}.\mathtt{prove}(u^{\scriptscriptstyle *},  \sigma,  c^{\scriptscriptstyle *}_{\scriptscriptstyle j}, pk, T_{\scriptscriptstyle qp}, enc)$


\begin{enumerate}

\item Constructs an empty  vector, $\bm{m}_{\scriptscriptstyle\mathcal S}=\bot$, if $j=1$.  
\item Removes the pads from $c^{\scriptscriptstyle *}_{\scriptscriptstyle j}$, using  parameters of $T_{\scriptscriptstyle qp}$. Let  $c'_{\scriptscriptstyle j}$ be the result. Next, it  decrypts  the result, $\mathtt{Dec(}\bar{k},c'_{\scriptscriptstyle j})=c_{\scriptscriptstyle j}$. Then,  it   runs $\mathtt{VSID.checkQuery}(c_{\scriptscriptstyle j}, pk, {pp})\rightarrow b_{\scriptscriptstyle j}$, to check the correctness of the queries.



\begin{itemize}
\item[$\bullet$] If $\mathcal S$ accepts the query, i.e. $b_{\scriptscriptstyle j}=1$, then  calls $\mathtt{VSID.prove}(u^{\scriptscriptstyle *},\sigma,c_{\scriptscriptstyle j},pk,{pp})\rightarrow \pi_{\scriptscriptstyle j}$, to generate the service proof. In this case, $\mathcal S$ encrypts it, $\mathtt{Enc}(\bar{k},\pi_{\scriptscriptstyle j})=\pi'_{\scriptscriptstyle j}$. Next, it   pads (every element of) the encrypted proof with ${pad}_{\scriptscriptstyle \pi}\in T_{\scriptscriptstyle qp}$ random values  picked uniformly at random from $U$. Let $\pi^{\scriptscriptstyle *}_{\scriptscriptstyle j}$ be the result. It sends the padded encrypted proof  to SC at time $\texttt{G}_{\scriptscriptstyle j,2}$ 

\item[$\bullet$] Otherwise (if $\mathcal S$ rejects the query), it appends  $j$  to $\bm{m}_{\scriptscriptstyle\mathcal{S}}$,  constructs a dummy proof $\pi'_{\scriptscriptstyle j}$, picked uniformly at random from $U$,  pads the result as above, and sends the padded dummy proof, $\pi^{\scriptscriptstyle *}_{\scriptscriptstyle j}$, to SC at time $\texttt{G}_{\scriptscriptstyle j,2}$ %, if the above proofs are rejected, i.e. $b_{\scriptscriptstyle j}=0$.

\end{itemize}
When $j=z$  and $\bm{m}_{\scriptscriptstyle\mathcal{S}}\neq\bot$, $\mathcal S$ sets    $m_{\scriptscriptstyle\mathcal S}:=(\bm{m}_{\scriptscriptstyle\mathcal{S}},adr_{\scriptscriptstyle \text{SC}})$.

\end{enumerate}

\

\item \textbf{Client-side Proof Verification}.\label{Client-sideProofVerification} $\mathtt{RCSP}.\mathtt{verify}(\pi^{\scriptscriptstyle *}_{\scriptscriptstyle j}, \bm{q}_{\scriptscriptstyle j},k,T_{\scriptscriptstyle qp},enc)$
\begin{enumerate}


\item Constructs an empty vector, $\bm{m}_{\scriptscriptstyle\mathcal C}=\bot$, if $j=1$. 

\item Removes the pads from $\pi^{\scriptscriptstyle *}_{\scriptscriptstyle j}$, utilising  parameters of $T_{\scriptscriptstyle qp}$. Let $\pi'_{\scriptscriptstyle j}$ be the result. It decrypts the service proof: $\mathtt{Dec}(\bar{k},\pi'_{\scriptscriptstyle j})=\pi''_{\scriptscriptstyle j}$ and then calls $\mathtt{VSID.verify}(\pi''_{\scriptscriptstyle j}, \bm{q}_{\scriptscriptstyle j},k, {pp})\rightarrow d_{\scriptscriptstyle j}$, to verify the proof. Note that if $\pi'_{\scriptscriptstyle j}=\mathtt{Enc}(\bar{k},\pi_{\scriptscriptstyle j})$, then $\pi''_{\scriptscriptstyle j}=\pi_{\scriptscriptstyle j}$. 

\begin{itemize}
\item[$\bullet$] If $\pi''_{\scriptscriptstyle j}$ passes the verification (i.e. $d_{\scriptscriptstyle j}=1$), then $\mathcal C$ concludes that the service  for this verification has been delivered successfully. 

\item[$\bullet$]  Otherwise (when $\pi''_{\scriptscriptstyle j}$ is rejected), $\mathcal C$ appends $j$  to $\bm{m}_{\scriptscriptstyle\mathcal{C}}$. 


\end{itemize}
\end{enumerate}
When $j=z$  and $\bm{m}_{\scriptscriptstyle\mathcal{C}}\neq\bot$, $\mathcal C$ sets    ${m}_{\scriptscriptstyle\mathcal C}:=(\bm{m}_{\scriptscriptstyle\mathcal{C}},adr_{\scriptscriptstyle \text{SC}},e')$, where  $e'$  contains the opening of $\mathtt{Com}_{\scriptscriptstyle sk}$ or  $\bot$, as stated in Remark \ref{remark::light-VSID}.




\

\item \textbf{Dispute Resolution}.\label{RCSP::Dispute-Resolution} $\mathtt{RCSP}.\mathtt{resolve}(m_{\scriptscriptstyle \mathcal{C}},m_{\scriptscriptstyle \mathcal{S}}, z, {\bm{\pi}}^{\scriptscriptstyle *}, {\bm{c}}^{\scriptscriptstyle *}, pk, T_{\scriptscriptstyle qp},enc)$



 The phase takes place only in  case of dispute, e.g. when $\mathcal C$ and/or $\mathcal S$ reject any proofs in the previous phases. %or  either party does not receive its counterparty's message on time. %A party who accepts all proofs does not need to get involved in this phase. 

\begin{enumerate}
%\item $\mathcal C$ and/or $\mathcal S$ at time $K_{\scriptscriptstyle 1}>G_{\scriptscriptstyle z,2}+H$ sends ``dispute'' message to the contract. 


\item The arbiter sets  counters: $y_{\scriptscriptstyle\mathcal C}, y'_{\scriptscriptstyle\mathcal C},y_{\scriptscriptstyle\mathcal S}$ and  $y'_{\scriptscriptstyle\mathcal S}$, that are initially set to $0$,  before time $\texttt{K}_{\scriptscriptstyle 1}$,  where $\texttt{K}_{\scriptscriptstyle 1}>\texttt{G}_{\scriptscriptstyle z,2}+\texttt{H}$. 
\item $\mathcal C$  sends $m_{\scriptscriptstyle\mathcal C}$ and $\ddot{x}_{\scriptscriptstyle qp}$ to the arbiter at time $\texttt{K}_{\scriptscriptstyle 1}$. Or,  $\mathcal S$ sends $m_{\scriptscriptstyle\mathcal S}$ and $\ddot{x}_{\scriptscriptstyle qp}$ to the arbiter at time $\texttt{K}_{\scriptscriptstyle 1}$ %,   where $|\vv{\bm{v}}_{\scriptscriptstyle\mathcal{C}}|\neq 0$ and $|\vv{\bm{v}}_{\scriptscriptstyle\mathcal{S}}|\neq 0$. 

 

\item\label{arbiter-checks-statement} At time $\texttt{K}_{\scriptscriptstyle 2}$, the arbiter checks the validity of  statement $\ddot{x}_{\scriptscriptstyle qp}$ sent by each party  $\forall \mathcal R\in \{\mathcal {C,S}\}$. To  do so, it sends  each $\ddot{x}_{\scriptscriptstyle qp}$ to SAP contract which  returns  either $1$ or $0$. The arbiter  constructs an empty vector: $\bm{v}$.  If  party $\mathcal R$'s statement is accepted, then 
 it appends every  element of $\bm{m}_{\scriptscriptstyle\mathcal{R}}$ to $\bm{v}$. It ensures $\bm{v}$ contains only distinct elements which are in the range $[1,z]$. Otherwise (if the party's statement is rejected) it discards the party's request, ${m}_{\scriptscriptstyle\mathcal{R}}$. It proceeds to the next step if $\bm{v}$ is not empty, otherwise it halts. 






\item\label{the-arbiter-loop} The arbiter  for every element $i\in\bm{v}$:

\begin{enumerate}

 
\item removes the pads from the related encrypted    query-proof pair  and from encrypted service proof. Let $c'_{\scriptscriptstyle i}$ and $\pi'_{\scriptscriptstyle i}$ be the result. %unpadded encrypted  query-proof pair  and encrypted service proof  respectively. 
\item decrypts the encrypted query-proof pair and encrypted service proof as follows, $ \mathtt{Dec}(\bar{k},c'_{\scriptscriptstyle i})=c_{\scriptscriptstyle i}$ and 
$ \mathtt{Dec}(\bar{k},\pi'_{\scriptscriptstyle i})=\pi''_{\scriptscriptstyle i}$ 
\item\label{arbiter-identify} calls   $\mathtt{VSID.identify}(\pi''_{\scriptscriptstyle i},c_{\scriptscriptstyle i}, k, e', {pp})\rightarrow I_{\scriptscriptstyle i}$
\begin{enumerate}
\item[$\bullet$] if $I_{\scriptscriptstyle i}=\mathcal C$, then it increments $y_{\scriptscriptstyle\mathcal C}$ by $1$ 
\item[$\bullet$]  if $I_{\scriptscriptstyle i}=\mathcal S$, then it increments $y_{\scriptscriptstyle\mathcal S}$ by $1$ 
\item[$\bullet$]  if $I_{\scriptscriptstyle i}=\bot$, then it increments  $y'_{\scriptscriptstyle\mathcal C}$ or $y'_{\scriptscriptstyle\mathcal S}$ by $1$, if $i$ is in  the complaint of $\mathcal {C}$ or $\mathcal {S}$ respectively. 
\end{enumerate}
\end{enumerate}

 
Let $\texttt{K}_{\scriptscriptstyle 3}$ be the time that the arbiter finishes the above checks.

\item The arbiter at time $\texttt{K}_{\scriptscriptstyle 3}$ sends $[y_{\scriptscriptstyle\mathcal C},y_{\scriptscriptstyle\mathcal S},  y'_{\scriptscriptstyle\mathcal C},y'_{\scriptscriptstyle\mathcal S}]$  to SC who accordingly overwrites the  elements it holds (i.e. elements of $\bm{y}$) by  the related vectors elements the arbiter sent.  
\end{enumerate}

\

\item \textbf{Coin Transfer}\label{RCSP::CoinTransfer}. $\mathtt{RCSP}.\mathtt{pay}(\bm{y},T_{\scriptscriptstyle cp},a,p_{\scriptscriptstyle\mathcal S}, coin^{\scriptscriptstyle *}_{\scriptscriptstyle\mathcal C},coin^{\scriptscriptstyle *}_{\scriptscriptstyle\mathcal S})$
\begin{enumerate}
 
 \item If SC  receives  ``pay'' message  at time $\texttt{T}_{2}$, where $a=0$ or $coins^{*}_{\scriptscriptstyle\mathcal{S}}<p_{\scriptscriptstyle\mathcal{S}}$, then it sends $coin^{\scriptscriptstyle *}_{\scriptscriptstyle\mathcal C}$ coins to $\mathcal C$ and $coin^{\scriptscriptstyle *}_{\scriptscriptstyle\mathcal S}$ coins to $\mathcal S$. In other words, the parties can withdraw their coins if they do not reach an agreement at the end of   phase \ref{RCSP::Server-side-Initiation}, i.e. server-side initiation.  Otherwise (i.e. they reach an agreement), they take the following step. 

\item Either $\mathcal C$ or $\mathcal S$ send ``pay'' message and the statement, $\ddot{x}_{\scriptscriptstyle cp}\in T_{\scriptscriptstyle cp}$,  to SC at time $\texttt{L}>\texttt{K}_{\scriptscriptstyle 3}$
\item SC checks the validity of the statement by sending $\ddot{x}_{\scriptscriptstyle cp}$ to SAP contract which  returns  either $1$ or $0$. SC only proceeds to the next step if the output is $1$. 
\item SC distributes the coins to the parties as follows:

\begin{itemize}
\item[$\bullet$]   $coin^{\scriptscriptstyle *}_{\scriptscriptstyle\mathcal C}-o(z-y_{\scriptscriptstyle\mathcal S})-l(y_{\scriptscriptstyle\mathcal C}+y'_{\scriptscriptstyle\mathcal C})$ coins  to $\mathcal C$
\item[$\bullet$] $coin^{\scriptscriptstyle *}_{\scriptscriptstyle\mathcal S}+o(z-y_{\scriptscriptstyle\mathcal S})-l(y_{\scriptscriptstyle\mathcal S}+y'_{\scriptscriptstyle\mathcal S})$ coins to $\mathcal S$
\item[$\bullet$] $l(y_{\scriptscriptstyle\mathcal S}+y_{\scriptscriptstyle\mathcal C}+y'_{\scriptscriptstyle\mathcal S}+y'_{\scriptscriptstyle\mathcal C})$ coins to the arbiter.  
\end{itemize}

\end{enumerate}
\end{enumerate}



\begin{remark} If all parties behave honestly,   then the server receives all its deposit back plus the amount of coins they initially agreed to pay the sever if it  delivers accepting proofs for all $z$ cycles, i.e. in total it receives $coin^{\scriptscriptstyle *}_{\scriptscriptstyle\mathcal S}+o\cdot z$ coins. Also, in this case an honest client receives all coins minus the amount of coins paid to the server for delivering accepting proofs for $z$ cycles, i.e. in total it receives $coin^{\scriptscriptstyle *}_{\scriptscriptstyle\mathcal C}-o\cdot z$ coins. However, the arbiter receives no coins, as it is never invoked. 
\end{remark}

\begin{remark} As stated in Section \ref{sec::RC-S-P-Definition},   algorithms $\mathtt{RCSP}.\mathtt{genQuery}(.),\mathtt{RCSP}.\mathtt{prove}(.),\mathtt{RCSP}.\mathtt{verify}(.)$ and $\mathtt{RCSP}.\mathtt{resolve}(.)$ implicitly  take $a, coin^{\scriptscriptstyle*}_{\scriptscriptstyle\mathcal{S}}, p_{\scriptscriptstyle\mathcal{S}}$ as another inputs and  execute only if $a=1$ and $coin^{\scriptscriptstyle*}_{\scriptscriptstyle\mathcal{S}}=p_{\scriptscriptstyle\mathcal{S}}$. For the sake of simplicity we  avoided explicitly stating it in the protocol.   
\end{remark}

\begin{remark}
Keeping track of  $(y'_{\scriptscriptstyle\mathcal C},y'_{\scriptscriptstyle\mathcal S})$ enables the arbiter to make malicious parties, who \emph{unnecessarily} invoke  it for  accepting proofs in step \ref{arbiter-identify}, pay  for the verifications it performs. 
\end{remark}


%\begin{remark}
%Keeping track of  $(y'_{\scriptscriptstyle\mathcal C},y'_{\scriptscriptstyle\mathcal S})$ enables the arbiter to make malicious parties, who \emph{unnecessarily} invoke  it for invalid statement in step \ref{arbiter-checks-statement} or  accepting proofs in step \ref{arbiter-identify}, pay  for the verifications it performs. 
%\end{remark}

%\begin{remark}
%If the client or server  receives no input from its counter-party, it still needs to invoke the arbiter, so that party can be paid (back) fairly. 
%\end{remark}





\begin{remark} The VSID scheme  does not (need to) preserve the privacy of the proofs. However, in RC-S-P  protocol  each proof's privacy must be preserved, for a certain time; otherwise, the proof itself can leak its status, e.g. when it can be publicly verified. This is the reason in RC-S-P protocol,  \emph{encrypted} proofs are sent to the contract.   
\end{remark}


\begin{remark}
For the sake of simplicity, in the above protocol, we assumed that each arbiter's invocation has a fixed  cost regardless of the number of steps it takes. To define a fine grained costing, one can simply allocate to each step the arbiter takes a certain rate and also separate counter  for the client and server.  
\end{remark}

\begin{remark}
In the case where $\mathtt{VSID.verify}(.)$ is privately verifiable and    the server invokes the arbiter, the client needs to provide inputs to the arbiter too. Otherwise (when it is publicly verifiable and  the server invokes the arbiter), the client's involvement is not required in the dispute resolution phase.  In contrast, if the client invokes the arbiter, the server's involvement is not required in that phase, regardless of the type of verifiability $\mathtt{VSID.verify}(.)$ supports. 
\end{remark}




 
 % !TEX root =R-zkCSP.tex

 \begin{lemma}
 If SAP is secure and  VSID scheme supports correctness, soundness, and detectable abort, then RC-S-P is secure against malicious server, w.r.t. Definition \ref{deff::RC-S-P-SecurityAgainstMaliciousServer}. 
 \end{lemma}
 

 \begin{proof}[sketch]
 We first consider event $F(u^{\scriptscriptstyle *},\bm{q}_{\scriptscriptstyle j})= h_{\scriptscriptstyle j}\ \wedge coin_{\scriptscriptstyle\mathcal{C},j}\neq  \frac{coin_{\scriptscriptstyle\mathcal C}^{\scriptscriptstyle*}}{z}-o$ that captures the case where the server provides an accepting service proof but makes an honest client withdraw an incorrect amount of coins, i.e. $coin_{\scriptscriptstyle\mathcal{C},j}\neq  \frac{coin_{\scriptscriptstyle\mathcal C}^{\scriptscriptstyle*}}{z}-o$. Recall, in the protocol, the total coins the client should receive after $z$ verifications is $coin^{\scriptscriptstyle *}_{\scriptscriptstyle\mathcal C}-o(z-y_{\scriptscriptstyle\mathcal S})-l\cdot y_{\scriptscriptstyle\mathcal C}$. Since we focus on  $j$-th verification, the amount of  coins that should be credited to the client for that verification is
  
  \begin{equation}\label{equ::what-client-recives-in-j}
   coin_{\scriptscriptstyle\mathcal{C},j}=\frac{coin^{\scriptscriptstyle *}_{\scriptscriptstyle\mathcal C}}{z}-o(1-y_{\scriptscriptstyle\mathcal S}[j])-l\cdot y_{\scriptscriptstyle\mathcal C}[j]
     \end{equation}

    
As the proof is valid, an honest client accepts it and does not raise any dispute. However, the server  would be able to make the client  withdraw incorrect amounts of coins, if it manages to either convince the arbiter that the client has misbehaved, by making the arbiter output  $y_{\scriptscriptstyle\mathcal {C}}[j]=1$ through dispute resolution phase, or submit to the contract, at the coin transfer phase,  an accepting  statement $\ddot{x}'_{\scriptscriptstyle cp}$ other than what was agreed at the initiation phase, i.e. $\ddot{x}'_{\scriptscriptstyle cp}\neq\ddot{x}_{\scriptscriptstyle cp}$, so it can change the payments parameters, e.g. $l$ or $o$. Nevertheless, it cannot falsely accuse the client of misbehaviour. Because,  due to the security of SAP, it cannot  convince the arbiter to accept different decryption key or pads other than what was agreed with the client in the initiation phase; specifically, it cannot persuade the arbiter to accept  $\ddot{x}'_{\scriptscriptstyle qp}$, where $\ddot{x}'_{\scriptscriptstyle qp}\neq \ddot{x}_{\scriptscriptstyle qp}$, except with a negligible probability, $\mu(\lambda)$. This ensures that the honest client's message is accessed by  the arbiter with a high probability, as the arbiter can extract the client's message using valid pad information and decryption key.  On the other hand, if the adversary provides a valid statement, i.e. $\ddot{x}_{\scriptscriptstyle qp}$, then due to the correctness of VSID, algorithm $\mathtt{VSID.identify}()$ outputs $I_{\scriptscriptstyle j}=\bot$. Therefore, due to the security of SAP and correctness of VSID,   $y_{\scriptscriptstyle\mathcal {C}}$ and $y_{\scriptscriptstyle\mathcal {S}}$ are not incremented by $1$  in $j$-th verification, i.e. $y_{\scriptscriptstyle\mathcal{C}}[j]=y_{\scriptscriptstyle\mathcal{S}}[j]=0$. So, according to Equation \ref{equ::what-client-recives-in-j}, the client is credited $\frac{coin_{\scriptscriptstyle\mathcal C}^{\scriptscriptstyle*}}{z}-o$ coins for that verification, with a high probability. Note,  due to the security of SAP, the server cannot change the payment parameters by persuading the contract to accept any statement $\ddot{x}'_{\scriptscriptstyle cp}$ other than what was agreed initially between the client and server, except with a negligible probability $\mu(\lambda)$ when it finds the hash function's collision. 
    
     We now move on to event $F(u^{\scriptscriptstyle *}, \bm{q}_{\scriptscriptstyle j})\neq h_{\scriptscriptstyle j} \ \wedge (d_{\scriptscriptstyle j}=1	\vee y_{\scriptscriptstyle \mathcal S}[j]=0 \ \vee (coin_{\scriptscriptstyle\mathcal{C},j}\neq \frac{coin_{\scriptscriptstyle\mathcal C}^{\scriptscriptstyle*}}{z} \ \vee  coin_{\scriptscriptstyle\mathcal{A}r,j}\neq l))$ which captures the case where the server provides an invalid service proof but either persuades the client to accept the proof, or  persuades the arbiter to accept the proof (e.g. when the client raises a dispute) or makes the client or arbiter withdraw an incorrect amount of coins, i.e.    $coin_{\scriptscriptstyle\mathcal{C},j}\neq \frac{coin_{\scriptscriptstyle\mathcal C}^{\scriptscriptstyle*}}{z}$ or $coin_{\scriptscriptstyle\mathcal{A}r,j}\neq l$ respectively.   Nevertheless, due to the soundness of VSID, the probability that a corrupt server can convince an honest client to accept invalid proof (i.e. outputs $d_{\scriptscriptstyle j}=1$), is negligible, $\mu(\lambda)$. So, the client detects it with a high probability and raises a dispute.  On the other hand, the server may try to convince the arbiter, and make it output $y_{\scriptscriptstyle \mathcal S}[j]=0$ or $y_{\scriptscriptstyle \mathcal C}[j]=1$, e.g. by sending a complaint. For $y_{\scriptscriptstyle \mathcal S}[j]=0$ to happen, it has to either provide a different accepting statement $\ddot{x}'_{\scriptscriptstyle qp}$, than what was initially agreed with the client (i.e. $\ddot{x}'_{\scriptscriptstyle qp}\neq \ddot{x}_{\scriptscriptstyle qp}$) and passes the verification, which   requires finding the hash function's collision, and its probability of success is   $\mu(\lambda)$. Or it makes the arbiter  accept an invalid proof, but due to the detectable abort property of VSID, its probability of success is also  $\mu(\lambda)$. As we discussed above, the probability that the adversary makes the arbiter output $y_{\scriptscriptstyle \mathcal C}[j]=1$ is  $\mu(\lambda)$ too. Therefore, the arbiter outputs $y_{\scriptscriptstyle \mathcal S}[j]=1$ and $y_{\scriptscriptstyle \mathcal C}[j]=0$  with a high probability, in both events when it is invoked by the client or  server. That means according to Equation \ref{equ::what-client-recives-in-j}, the client is credited $\frac{coin_{\scriptscriptstyle\mathcal C}^{\scriptscriptstyle*}}{z}$ coins for that verification, with a high probability. Recall, in the protocol, the total coins the arbiter should receive for $z$ verifications is $l(y_{\scriptscriptstyle\mathcal S}+y_{\scriptscriptstyle\mathcal C})$, so for the $j$-th the credited coins should be: 
     
     
     \begin{equation}\label{equ::arbiter}
    coin_{\scriptscriptstyle\mathcal{A}r,j}= l(y_{\scriptscriptstyle\mathcal S}[j]+y_{\scriptscriptstyle\mathcal C}[j])
     \end{equation}
     
   As we discussed above, $y_{\scriptscriptstyle \mathcal S}[j]=1$ and $y_{\scriptscriptstyle \mathcal C}[j]=0$. So, according to Equation \ref{equ::arbiter}, the arbiter is credited $l$ coins for that verification, with a high probability.  The adversary may  try to  make them withdraw an incorrect amount of coins, e.g. in the case where it does not succeed in convincing the client or arbiter. To this end,  at the coin transfer phase, it has to send a  different accepting statement  than what was initially agreed with the client. But, it would succeed only with a negligible probability, $\mu(\lambda)$, due to the security of SAP. 
     
  
Furthermore, in   both  events above, due to the security of SAP, the adversary cannot block an honest client's  messages, ``pay'' and $\ddot{x}_{\scriptscriptstyle cp}$, to the contract in the coin transfer phase.
  \hfill\(\Box\)\end{proof}

%We first consider event $F(u^{\scriptscriptstyle *},\bm{q}_{\scriptscriptstyle j})= h_{\scriptscriptstyle j}\ \wedge coin_{\scriptscriptstyle\mathcal{C},j}\neq  \frac{coin_{\scriptscriptstyle\mathcal C}^{\scriptscriptstyle*}}{z}-o$ that captures the case where the server provides an accepting service proof but makes an honest client withdraw an incorrect amount of coins, i.e. $coin_{\scriptscriptstyle\mathcal{C},j}\neq  \frac{coin_{\scriptscriptstyle\mathcal C}^{\scriptscriptstyle*}}{z}-o$. Since the proof is valid, an honest client accepts it and does not raise any dispute. 
%  However, the server  would be able to make the client  withdraw incorrect amounts of coins, if it manages to either convince the arbiter that the client has misbehaved (by making the arbiter increment $y_{\scriptscriptstyle\mathcal {C}}$  by $1$ through dispute resolution phase), or submit to the contract, at the coin transfer phase,  an accepting  statement $\ddot{x}'_{\scriptscriptstyle cp}$ other than what was agreed at the initiation phase, i.e. $\ddot{x}_{\scriptscriptstyle cp}$. Nevertheless, it cannot falsely accuse the client of misbehaviour. Because,  due to the security of SAP, it cannot  convince the arbiter to accept different decryption key or pads other than what was agreed with the client in the initiation phase; specifically, it cannot persuade the arbiter to accept  $\ddot{x}'_{\scriptscriptstyle qp}$, where $\ddot{x}'_{\scriptscriptstyle qp}\neq \ddot{x}_{\scriptscriptstyle qp}$, except with a negligible probability, $\mu(\lambda)$. This ensures that the honest client's message is accessed by  the arbiter with a high probability, as the arbiter can extract the client's message using valid pad information and decryption key.  On the other hand, if the adversary provides a valid statement, i.e. $\ddot{x}_{\scriptscriptstyle qp}$, then due to the correctness of VSID, algorithm $\mathtt{VSID.identify}()$ outputs $I_{\scriptscriptstyle j}=\bot$. Therefore, due to the security of SAP and correctness of VSID,   $y_{\scriptscriptstyle\mathcal {C}}$ and $y_{\scriptscriptstyle\mathcal {S}}$ are not incremented by $1$  in $j$-th verification, i.e. $y_{\scriptscriptstyle\mathcal{C}}[j]=y_{\scriptscriptstyle\mathcal{S}}[j]=0$.  Moreover,  due to the security of SAP, the server cannot persuade the contract to accept any statement other than what was agreed initially between the client and server, except with a negligible probability $\mu(\lambda)$ when it finds the hash function's collision. Recall, in the protocol, the total coins the client receives after $z$ verifications is $coin^{\scriptscriptstyle *}_{\scriptscriptstyle\mathcal C}-o(z-y_{\scriptscriptstyle\mathcal S})-l\cdot y_{\scriptscriptstyle\mathcal C}$. Since we focus on each $j$-th verification, the amount of  coins credited to the client for that verification is
%  
%  \begin{equation}\label{equ::what-client-recives-in-j}
%   \frac{coin^{\scriptscriptstyle *}_{\scriptscriptstyle\mathcal C}}{z}-o(1-y_{\scriptscriptstyle\mathcal S}[j])-l\cdot y_{\scriptscriptstyle\mathcal C}[j]
%     \end{equation}
%     
%     In the above event, as we discussed,  with a high probability $y_{\scriptscriptstyle\mathcal{C}}[j]=y_{\scriptscriptstyle\mathcal{S}}[j]=0$ , so the client is credited $\frac{coin_{\scriptscriptstyle\mathcal C}^{\scriptscriptstyle*}}{z}-o$ coins for that verification. 
%     
   




\begin{lemma}
 If SAP is secure and  VSID scheme supports correctness, inputs well-formedness, and detectable abort, then RC-S-P is secure against malicious client, w.r.t. Definition \ref{deff::RC-S-P-Security-Against Malicious-Client}. 
\end{lemma}


\begin{proof}[sketch]
  First, we consider event  $(M(u^{\scriptscriptstyle *},k)= \sigma  \wedge Q(\text{aux},k)= \bm{q}_{\scriptscriptstyle j}) \wedge
  (coin_{\scriptscriptstyle\mathcal{S},j}\neq  \frac{coin_{\scriptscriptstyle\mathcal S}^{\scriptscriptstyle*}}{z}+o)$ which captures the case where the client provides  accepting metadata and query  but makes the server withdraw an incorrect amount of coins, i.e. $coin_{\scriptscriptstyle\mathcal{S},j}\neq  \frac{coin_{\scriptscriptstyle\mathcal S}^{\scriptscriptstyle*}}{z}+o$. According to the RC-S-P protocol, the total coins the server should receive after $z$ verifications is $coin^{\scriptscriptstyle *}_{\scriptscriptstyle\mathcal S}+o(z-y_{\scriptscriptstyle\mathcal S})-l\cdot y_{\scriptscriptstyle\mathcal S}$. Since we focus on  $j$-th verification, the amount of  coins that should be credited to the server for $j$-th verification is
  
  \begin{equation}\label{equ::what-server-recives-in-j}
   coin_{\scriptscriptstyle\mathcal{S},j}=\frac{coin^{\scriptscriptstyle *}_{\scriptscriptstyle\mathcal S}}{z}+o(1-y_{\scriptscriptstyle\mathcal S}[j])-l\cdot y_{\scriptscriptstyle\mathcal S}[j]
     \end{equation}
  
  
  
  Since the metadata and query's proofs are valid, an honest server accepts them and does not raise any dispute, so we have $y_{\scriptscriptstyle\mathcal{C}}[j]=0$.  The client may want to  make the server  withdraw incorrect amounts of coins, if it manages to either convince the arbiter, in phase \ref{RCSP::Dispute-Resolution}, that the server has misbehaved, i.e. makes the arbiter output $y_{\scriptscriptstyle\mathcal{S}}[j]=1$, or submit to the contract  an accepting  statement $\ddot{x}'_{\scriptscriptstyle cp}$ other than what was agreed at the initiation phase, i.e. $\ddot{x}_{\scriptscriptstyle cp}$, in phase \ref{RCSP::CoinTransfer}. However, it cannot falsely accuse the server of misbehaviour. As,  due to the security of SAP, it cannot  convince the arbiter to accept different decryption key and pads, by providing a different accepting statement $\ddot{x}'_{\scriptscriptstyle qp}$  (where $\ddot{x}'_{\scriptscriptstyle qp}\neq \ddot{x}_{\scriptscriptstyle qp}$), than what was initially agreed with the server, except with  probability $\mu(\lambda)$. This ensures that  the arbiter is given the honest server's messages (with a high probability). On the other hand, if the adversary provides a valid statement, i.e. $\ddot{x}_{\scriptscriptstyle qp}$, then due to the correctness of VSID, algorithm $\mathtt{VSID.identify}()$ outputs $I_{\scriptscriptstyle j}=\bot$. So, due to the security of SAP and correctness of VSID,  we would have $y_{\scriptscriptstyle\mathcal{C}}[j]=y_{\scriptscriptstyle\mathcal{S}}[j]=0$ with a high probability.  That means, according to the Equation \ref{equ::what-server-recives-in-j}, the server is credited $\frac{coin_{\scriptscriptstyle\mathcal S}^{\scriptscriptstyle*}}{z}+o$ coins for that verification, with a  high probability. Note,  due to the security of SAP, the client cannot convince the contract to accept any accepting statement $\ddot{x}'_{\scriptscriptstyle cp}$ other than what was initially agreed  between the client and server (i.e. $\ddot{x}'_{\scriptscriptstyle cp}\neq \ddot{x}_{\scriptscriptstyle cp}$), except with   probability $\mu(\lambda)$.  
  
%  \begin{equation}\label{equ::what-server-recives-in-j}
%   \frac{coin^{\scriptscriptstyle *}_{\scriptscriptstyle\mathcal S}}{z}+o(1-y_{\scriptscriptstyle\mathcal S}[j])-l\cdot y_{\scriptscriptstyle\mathcal S}[j]
%     \end{equation}
%     
%    Since  we have  $y_{\scriptscriptstyle\mathcal{C}}[j]=y_{\scriptscriptstyle\mathcal{S}}[j]=0$ with a high probability,  the server is credited $\frac{coin_{\scriptscriptstyle\mathcal S}^{\scriptscriptstyle*}}{z}+o$ coins for that verification, with  high probability. 
%    
    
   
    
    We now turn our attention to $(M(u^{\scriptscriptstyle *},k)\neq \sigma \wedge a=1)\ \vee\ (Q(\text{aux},k)\neq \bm{q}_{\scriptscriptstyle j} \wedge b_{\scriptscriptstyle j}=1)$,   that captures the case where the server accepts an ill-formed metadata, or query. However, due to inputs well-formedness of VSID, the probability that either of the events happens is negligible, $\mu(\lambda)$. So, with a high probability $a=0$ or $b_{\scriptscriptstyle j}=0$.  Note, in the former case (i.e. $a=0$) the server does not raise any dispute, instead it avoids serving the client.  Next, we move on to      $Q(\text{aux},k)\neq \bm{q}_{\scriptscriptstyle j} \wedge
  (y_{\scriptscriptstyle\mathcal{C}}[j]=0 \ \vee  coin_{\scriptscriptstyle\mathcal{S},j}\neq \frac{coin_{\scriptscriptstyle\mathcal S}^{\scriptscriptstyle*}}{z}+o \ \vee  coin_{\scriptscriptstyle\mathcal{A}r,j}\neq l)$.  It considers the case where the client provides an invalid query, but  either convinces the arbiter to accept it, or  makes the server or arbiter  withdraw an incorrect amount of coins, i.e. $coin_{\scriptscriptstyle\mathcal{S},j}\neq \frac{coin_{\scriptscriptstyle\mathcal S}^{\scriptscriptstyle*}}{z}+o$ or  $coin_{\scriptscriptstyle\mathcal{A}r,j}\neq l$ respectively. Note, when the server rejects the query and raises a dispute, the client may try to convince the arbiter, and make it output $y_{\scriptscriptstyle \mathcal C}[j]=0$, e.g. by sending a complaint right after the server does so. However, for the adversary to win, either it has to provide a different accepting statement $\ddot{x}'_{\scriptscriptstyle qp}$, than what was initially agreed with the server (i.e. $\ddot{x}'_{\scriptscriptstyle qp}\neq \ddot{x}_{\scriptscriptstyle qp}$) and passes the verification. But,  due to the security of SAP, its probability of success is negligible, $\mu(\lambda)$. Or it has to make the arbiter  accept an invalid query, i.e. makes the arbiter output $y_{\scriptscriptstyle\mathcal{C}}[j]=0$. But, due to the detectable abort property of VSID, its probability of success is  $\mu(\lambda)$. Therefore, with a high probability we have $y_{\scriptscriptstyle\mathcal{C}}[j]=1$. Also, as discussed above, the client cannot make the arbiter recognise the  honest server as a misbehaving party with a probability significantly greater than $\mu(\lambda)$. That means with a high probability $y_{\scriptscriptstyle\mathcal{S}}[j]=0$. Hence, according to Equation \ref{equ::what-server-recives-in-j} the server is credited  $\frac{coin_{\scriptscriptstyle\mathcal S}^{\scriptscriptstyle*}}{z}+o$ coins for this verification; and also the arbiter is credited $l$ coins, according to Equation \ref{equ::arbiter}.  Note,  the adversary may still try to  make them withdraw an incorrect amount of coins (e.g. if the adversary does not succeed in convincing the server or arbiter). To this end, at the coin transfer phase, it has to send a  different accepting statement  than what was initially agreed with the server. However, due to the security of SAP, its  success probability is  $\mu(\lambda)$. Also, due to the security of SAP, the adversary cannot block an honest server's  messages, ``pay'' and $\ddot{x}_{\scriptscriptstyle cp}$, to the contract in the coin transfer phase.
 \hfill\(\Box\)\end{proof}
  
  
  Prior to proving  RC-S-P's privacy, we provide a lemma that will be used in the privacy's proof. Informally, the lemma states that encoded coins leaks no information about the actual amount of coins $(o,l)$, agreed between the client and server. 


\begin{lemma}\label{lemma::encoded-coins} Let $\beta\stackrel{\scriptscriptstyle\$}\leftarrow\{0,1\}$, price list be $\{(o_{\scriptscriptstyle 0},l_{\scriptscriptstyle 0}),(o_{\scriptscriptstyle 1},l_{\scriptscriptstyle 1})\}$, and encoded coin amounts be $coin^{\scriptscriptstyle *}_{\scriptscriptstyle\mathcal{C}}= z\cdot(Max(o_{\scriptscriptstyle\beta},o_{\scriptscriptstyle|\beta-1|})+Max(l_{\scriptscriptstyle\beta},l_{\scriptscriptstyle|\beta-1|}))$ and $coin^{\scriptscriptstyle *}_{\scriptscriptstyle\mathcal{S}}=z\cdot(Max(l_{\scriptscriptstyle\beta}, l_{\scriptscriptstyle |\beta-1|}))$. Then, given the price list, $z$, $coin^{\scriptscriptstyle *}_{\scriptscriptstyle\mathcal{C}}$, and $coin^{\scriptscriptstyle *}_{\scriptscriptstyle\mathcal{S}}$, an adversary $\mathcal{A}$ cannot tell the value of $\beta$ with probability significantly greater than $\frac{1}{2}$ (where the probability is taken over the choice of $\beta$ and the randomness of $\mathcal{A}$).
\end{lemma}



\begin{proof}
As it is evident, the list and $z$ contains no information about $\beta$. Also, since  $z$ is a public value, we could have   $coin'^{\scriptscriptstyle *}_{\scriptscriptstyle\mathcal C}=\frac{coin^{\scriptscriptstyle *}_{\scriptscriptstyle\mathcal C}}{z} = Max(o_{\scriptscriptstyle\beta},o_{\scriptscriptstyle|\beta-1|})+Max(l_{\scriptscriptstyle\beta}, l_{\scriptscriptstyle|\beta-1|})$. It is not hard to see $coin'^{\scriptscriptstyle *}_{\scriptscriptstyle\mathcal C}$ is a function of maximum value of $(o_{\scriptscriptstyle 0},o_{\scriptscriptstyle 1})$, and maximum value of $(l_{\scriptscriptstyle 0},l_{\scriptscriptstyle 1})$. It is also  independent of $\beta$. Therefore, given the list, $z$ and $coin'^{\scriptscriptstyle *}_{\scriptscriptstyle\mathcal C}$ the adversary learns nothing about $\beta$, unless it guesses the value,  with  success probability $\frac{1}{2}$. The same also holds for $coin^{\scriptscriptstyle *}_{\scriptscriptstyle\mathcal S}$. 
 \hfill\(\Box\)\end{proof}


\begin{lemma} If SAP is secure and the encryption scheme is semantically secure, then RC-S-P preserves privacy, w.r.t. Definition \ref{deff::RC-S-P-Privacy}. 
\end{lemma}

\begin{proof}[sketch]
Due to the privacy property of SAP, that stems from the  hiding property of the commitment scheme, given the commitments $g_{\scriptscriptstyle qp}$ and $g_{\scriptscriptstyle cp}$, (that are stored in the blockchain as result of running SAP) the adversary learns no information about the committed values (e.g. $o, l, pad_{\scriptscriptstyle\pi}, pad_{\scriptscriptstyle q},$ and $\bar{k}$), except with negligible probability, $\mu_{\scriptscriptstyle 1}(\lambda)$.   Also, given encoded coins $coin^{\scriptscriptstyle *}_{\scriptscriptstyle\mathcal{C}}=z\cdot (o_{\scriptscriptstyle max}+l_{\scriptscriptstyle max})$ and  $coin^{\scriptscriptstyle *}_{\scriptscriptstyle\mathcal{S}}=z\cdot l_{\scriptscriptstyle max}$, the adversary learns nothing about the actual price agreed between the server and client,  $(o,l)$, for each verification, due to Lemma \ref{lemma::encoded-coins}. Next we analyse the privacy of padded encrypted query vector $\bm{c}^{\scriptscriptstyle *}$. For  the sake of simplicity, we focus on   $\bm{q}^{\scriptscriptstyle *}_{\scriptscriptstyle j}\in c^{\scriptscriptstyle *}_{\scriptscriptstyle j}\in\bm{c}^{\scriptscriptstyle *}$, that is a padded encrypted query vector for $j$-th verification. Let $\bm{q}_{\scriptscriptstyle j,0}$ and $\bm{q}_{\scriptscriptstyle j,1}$  be query vectors, for $j$-th verification, related to the service inputs $u_{\scriptscriptstyle 0}$ and $u_{\scriptscriptstyle 1}$ that    are picked by the adversary according to  Definition  \ref{deff::RC-S-P-Privacy} which lets  the environment pick $\beta\stackrel{\scriptscriptstyle\$}\leftarrow \{0,1\}$. Also, let $\{\bm{q}_{\scriptscriptstyle j,0},...,\bm{q}_{\scriptscriptstyle j,m}\}$ be a  list of all queries of different sizes. In the experiment, if $\bm{q}_{\scriptscriptstyle j,\beta}$  is only   encrypted (but not padded), then given the ciphertext, due to semantical security of the encryption, an adversary cannot tell if the ciphertext corresponds to $\bm{q}_{\scriptscriptstyle j,0}$ or $\bm{q}_{\scriptscriptstyle j,1}$  (accordingly to $u_{\scriptscriptstyle 0}$ or $u_{\scriptscriptstyle 1}$) with probability greater than $\frac{1}{2}+\mu_{\scriptscriptstyle 1}(\lambda)$,  under the assumption  that $Max(|\bm{q}_{\scriptscriptstyle j,0}|,...,|\bm{q}_{\scriptscriptstyle j,m}|)=|\bm{q}_{\scriptscriptstyle j,\beta}|$. The  assumption is  relaxed with the use of a pad; as each encrypted query is padded  to the queries' maximum size, i.e. $Max(|\bm{q}_{\scriptscriptstyle j,0}|,...,|\bm{q}_{\scriptscriptstyle j,m}|)$, the adversary cannot tell with probability greater than $\frac{1}{2}+\mu_{\scriptscriptstyle 1}(\lambda)$ if the  padded encrypted proof corresponds to  $\bm{q}_{\scriptscriptstyle j,0}$ or $\bm{q}_{\scriptscriptstyle j,1}$, as the padded encrypted query \emph{always has the same size} and the pad values are picked from the same range as the encryption's ciphertext are defined. The same argument holds for $\bm{w}^{\scriptscriptstyle *}_{\scriptscriptstyle q_{_{\scaleto{j}{\size}}}}\in c^{\scriptscriptstyle *}_{\scriptscriptstyle j}\in\bm{c}^{\scriptscriptstyle *}$. 

Next we analyse the privacy of padded encrypted proof vector $\bm{\pi}^{\scriptscriptstyle *}$. The argument is similar to the one presented above; however, for  the sake of completeness we provide it.  Again, we focus on an element of the vector,  $\pi_{\scriptscriptstyle j}^{\scriptscriptstyle *}\in\bm{\pi}^{\scriptscriptstyle *}$, that is a padded encrypted proof for $j$-th verification. Let $\pi_{\scriptscriptstyle j,0}$ and $\pi_{\scriptscriptstyle j,1}$  be proofs, for $j$-th verification, related to the service inputs $u_{\scriptscriptstyle 0}$ and $u_{\scriptscriptstyle 1}$, where the inputs   are picked by the adversary, w.r.t. Definition  \ref{deff::RC-S-P-Privacy} in which  the environment picks $\beta\stackrel{\scriptscriptstyle\$}\leftarrow \{0,1\}$.  Let $\{\pi_{\scriptscriptstyle j,0},...,\pi_{\scriptscriptstyle j,m}\}$ be proof list including all proofs of different sizes. In the experiment, if $\pi_{\scriptscriptstyle j,\beta}$  is only   encrypted, then given the ciphertext, due to semantical security of the encryption, an adversary cannot tell if the ciphertext corresponds to $\pi_{\scriptscriptstyle j,0}$ or $\pi_{\scriptscriptstyle j,1}$  (accordingly to $u_{\scriptscriptstyle 0}$ or $u_{\scriptscriptstyle 1}$) with probability greater than $\frac{1}{2}+\mu_{\scriptscriptstyle 2}(\lambda)$,  if $Max(|\pi_{\scriptscriptstyle j,0}|,...,|\pi_{\scriptscriptstyle j,m}|)=|\pi_{\scriptscriptstyle j,\beta}|$. However, the  assumption is  relaxed with the use of a pad. In particular, since each encrypted proof is padded  to the proofs' maximum size, the adversary cannot tell with probability greater than $\frac{1}{2}+\mu_{\scriptscriptstyle 2}(\lambda)$ if the  padded encrypted proof corresponds to  $\pi_{\scriptscriptstyle j,0}$ or $\pi_{\scriptscriptstyle j,1}$. Also, since the value of $a$ is independent of $u_{\scriptscriptstyle 0}$  or $u_{\scriptscriptstyle 1}$, and only depends on whether the metadata is well-formed, it leaks nothing about the service input $u_{\scriptscriptstyle\beta}$ and $\beta$. Moreover, since each padded encrypted query and proof leak no information and always contains a fixed number of elements, an adversary cannot tell the status of a proof for each $j$-th verification (i.e. whether it is accepted or rejected) with the probability greater than $\frac{1}{2}+\mu_{\scriptscriptstyle 2}(\lambda)$, given  $\bm{c}^{\scriptscriptstyle *},coin^{\scriptscriptstyle *}_{\scriptscriptstyle \mathcal S},coin^{\scriptscriptstyle *}_{\scriptscriptstyle\mathcal C}, g_{\scriptscriptstyle cp},  g_{\scriptscriptstyle qp},$  $\bm{\pi}^{\scriptscriptstyle *}$, and $a$.
  \hfill\(\Box\)\end{proof}



 
 



