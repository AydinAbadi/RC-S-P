% !TEX root =R-zkCSP.tex

\subsection{Recurring Contingent PoR Payment (RC-PoR-P) Protocol}
This section presents recurring contingent PoR payment (RC-PoR-P) protocol. It is built upon PoRID protocol and is in the same security model as RC-S-P is. RC-PoR-P inherits the features of PoRID and RC-PoR-P; however, unlike RC-PoR-P, it does not use any zero-knowledge proofs and there is no third-party arbiter involved. Even though RC-PoR-P and RC-S-P have some overlaps, they have many differences too. Therefore, we provide the protocol’s overview and its detailed description below.
At a high level the protocol works as follows. The client and server utilise SAP to provably agree on two private statements, one statement includes payment details, and another one specifies a secret key, $k$, and a pad's length. Moreover, they  agree on public parameters such as the private time bubble's length (that is the total number of billing cycles: $z$, plus a waiting period, $\texttt{H}$) and a smart contract that specifies $z$ and the total amount of masked coins each party should deposit. They deploy the contract. Each party deposits its masked coins in the contract within a fixed time. If any party does not deposit enough coins on time, then the parties have a chance to withdraw their coins and terminate the contract after a certain time. To start using/providing the service, the client invokes $\mathtt{PoRID.setup}(.)$ to encode the file and generate  metadata and public parameters. It sends  encryption of the metadata and public parameters to the smart contract and sends the encoded file to the server who decrypts them and using the encoded file checks their correctness by calling $\mathtt{PoRID.server}(.)$. If the server decides not to serve, it sends to the contract $0$ within a fixed time; in this case, the parties can withdraw their deposit and terminate the contract. Otherwise, the server sends $1$ to the contract. 
At the end of each billing cycle, the client generates an encrypted query, by calling $\mathtt{PoRID.genQuery(.)}$ and encrypting its output using the key, $k$. It sends the result to the contract. In the same cycle, the server retrieves the query, and decrypts it. Then, it locally checks its correctness, by calling $\mathtt{PoRID.checkQuery(.)}$. If the query is rejected, the server locally stores the index of that billing cycle and generates a dummy PoR proofs. However, if the server accepts the query, it generates PoR proofs by calling $\mathtt{PoRID.prove(.)}$. Then, in either case, the server encrypts the proofs, pads them and sends the result to the contract. After that, the client removes the pads, decrypts the proofs and locally verifies them, by calling $\mathtt{PoRID.verify(.)}$. If the verification is passed, then the client knows the file is retrievable with a high probability. But, if the proof is rejected, then it locally stores the index of that billing cycle and waits until the private time bubble passes and dispute resolution time arrives. 

During the dispute resolution period, in case the server rejects the query or the client rejects the PoR proofs, that party sends to the contract (a) the indices of the billing cycles in which its counterparty provided invalid values, and (b)  the statement that contains the decryption key and padding detail. The contract checks the validity of the statement first. If it accepts the statement, then it removes the pads and decrypts the values whose indices were provided by the parties. Then, the contract checks the party’s claim by calling $\mathtt{PoRID.checkQuery (.)}$ and $\mathtt{PoRID.identify (.)}$ if the server or client calls the contract respectively. The contract also keeps track of the number of times each party provided invalid queries or PoR proofs.  
In the next phase, to distribute the coins, either client or server sends to the contract: (a) “pay” message, (b) the agreed statement that specifies the payment details, and (c) the statement’s proof. The contract verifies the statement and if it is approved, then the contract distributes the coins according to the statement’s detail, and the number of times each party misbehaved. 



\begin{enumerate}
\item \textbf{Key Generation}. $\mathtt{RCPoRP}.\mathtt{keyGen}(1^{\scriptscriptstyle\lambda})$
\begin{enumerate}

\item  $\mathcal C$  picks a random secret key  $\bar{k}$ for a symmetric key encryption. It also sets  parameter $pad_{\scriptscriptstyle \pi}$ which is the number of  dummy values that will be used to pad encrypted proofs,  let $qp:=(pad_{\scriptscriptstyle \pi}, \bar{k})$. The key's size is part of the security parameter. Let  ${k}:=(sk',pk')$, where  $sk'=qp$ and $pk':=(adr_{\scriptscriptstyle\mathcal{C}},adr_{\scriptscriptstyle\mathcal{S}})$. 




%\item  \textcolor{green}{$\mathcal C$ sets coin parameters as   follows,   $o$: the amount of coins for each accepting proof,  as well as $l$ and $l'$ which are the amount of coins $\mathcal C$ and $\mathcal S$ respectively need to send to a smart contract to resolve a potential dispute.  Let $k':=\{o,l,l',qp\}$. }


\end{enumerate}

\

\item \textbf{Client-side Initiation}. $\mathtt{RCPoRP}.\mathtt{cInit}(1^\lambda, u,{k}, z,pl)$
\begin{enumerate}

\item Calls $\mathtt{PoRID.setup}(1^{\lambda}, u)\rightarrow (u^{\scriptscriptstyle *},pp)$ to encode service input. It appends $pp$ to $qp$.


\item  Calls $\mathtt{SAP.init}(1^{\scriptscriptstyle\lambda}, adr_{\scriptscriptstyle\mathcal{C}}, adr_{\scriptscriptstyle\mathcal{S}},qp )\rightarrow(r_{\scriptscriptstyle qp}, g_{\scriptscriptstyle qp}, adr_{\scriptscriptstyle\text{SAP}_{\scriptscriptstyle 1}})$, to initiate an agreement (with $\mathcal{S}$) on $qp$.  Let $T_{\scriptscriptstyle qp}:=(\ddot{x}_{\scriptscriptstyle qp}, g_{\scriptscriptstyle qp})$ be proof/query encoding token,  where  $\ddot{x}_{\scriptscriptstyle qp}:=(qp,r_{\scriptscriptstyle qp})$ is the opening and $g_{\scriptscriptstyle qp}$ is the commitment stored on the contract as a result of running   SAP. 


\item   Sets coin parameters (given  price list $pl$) as  follows,  $o$: the amount of coins for each accepting proof,  and $l$ the amount of coin $\mathcal C$  or $\mathcal S$ needs to send to a smart contract to resolve a potential dispute. 


%
%\item   Sets coin parameters (given  price list $pl$) as  follows,  $o$: the amount of coins for each accepting proof,  as well as $l$ and $l'$ which are the amount of coins $\mathcal C$ and $\mathcal S$ respectively need to send to a smart contract to resolve a potential dispute. 






\item  Sets  $cp:=(o,o_{\scriptscriptstyle max}, l,  l_{\scriptscriptstyle max},z)$, where $o_{\scriptscriptstyle max}$ is the maximum amount of coins  for an accepting service proof,  $l_{\scriptscriptstyle max}$  is the maximum amount of coins to resolve a potential dispute, and $z$ is the number of  service proofs/verifications. Then, $\mathcal C$ calls $\mathtt{SAP.init}(1^{\scriptscriptstyle\lambda}, adr_{\scriptscriptstyle\mathcal{C}}, adr_{\scriptscriptstyle\mathcal{S}},cp )\rightarrow(r_{\scriptscriptstyle cp},g_{\scriptscriptstyle cp},adr_{\scriptscriptstyle\text{SAP}_{\scriptscriptstyle 2}})$, to initiate an agreement (with $\mathcal{S}$) on $cp$. Let $T_{\scriptscriptstyle cp}:=(\ddot{x}_{\scriptscriptstyle cp},g_{\scriptscriptstyle cp})$ be coin encoding token,  where  $\ddot{x}_{\scriptscriptstyle cp}:=(cp,r_{\scriptscriptstyle cp})$ is the opening and $g_{\scriptscriptstyle cp}$ is the commitment stored on the contract as a result of executing   SAP.  





\item Set parameters  $coin^{\scriptscriptstyle *}_{\scriptscriptstyle\mathcal C}=z\cdot (o_{\scriptscriptstyle max}+l_{\scriptscriptstyle max})$ and $p_{\scriptscriptstyle\mathcal{S}}=z\cdot l_{\scriptscriptstyle max}$, where $coin^{\scriptscriptstyle *}_{\scriptscriptstyle\mathcal C}$ and $p_{\scriptscriptstyle\mathcal{S}}$ are   the total number of masked coins $\mathcal C$ and $\mathcal S$ should deposit respectively. It signs and deploys a smart contract, SC, that explicitly specifies  parameters $z$,  $coin^{\scriptscriptstyle *}_{\scriptscriptstyle\mathcal C}$, $p_{\scriptscriptstyle\mathcal{S}}$,  $adr_{\scriptscriptstyle\text{SAP}_{\scriptscriptstyle 1}}$, and $adr_{\scriptscriptstyle\text{SAP}_{\scriptscriptstyle 2}}$. Let $adr_{\scriptscriptstyle \text{SC}}$ be the address of the deployed SC, and $T:=\{T_{\scriptscriptstyle qp},T_{\scriptscriptstyle cp}\}$. It sets a set of time points/windows, $\texttt{Time}:\{ \texttt{T}_{\scriptscriptstyle 0},..., \texttt{T}_{\scriptscriptstyle 3},\texttt{G}_{\scriptscriptstyle 1,1},...,\texttt{G}_{\scriptscriptstyle z,2},\texttt{H}, \texttt{K}_{\scriptscriptstyle 1},...,\texttt{K}_{\scriptscriptstyle 4}, \texttt{L}\}$, that are explicitly specified in SC who will accept a certain party's message only in a specified  time point/window. Time allocations will become clear in the next phases. 


\item Sets two counters $[y_{\scriptscriptstyle\mathcal C}, y_{\scriptscriptstyle\mathcal S}]$ in the contract, where their initial value is $0$. Let $\bm{y}: [y_{\scriptscriptstyle\mathcal C},y_{\scriptscriptstyle\mathcal S},  \texttt{Time},adr_{\scriptscriptstyle \text{SC}}]$

\item Deposits $coin^{\scriptscriptstyle *}_{\scriptscriptstyle\mathcal C}$ coins in the contract. It  sends $u^{\scriptscriptstyle *}, e, \ddot{x}_{\scriptscriptstyle qp},\ddot{x}_{\scriptscriptstyle cp}$, and $p_{\scriptscriptstyle\mathcal{S}}$ (along with $adr_{\scriptscriptstyle \text{SC}}$) to $\mathcal S$. It also sends  $pk'$ to SC. Let $\texttt{T}_{\scriptscriptstyle 0}$ be the time that the above process finishes. 



%%%%%%%%%%%%%%%%%%%%%%%%%%%%%%%%%%%%%
%\textcolor{red}{
%\item For $\mathcal C$ and $\mathcal S$ to provably agree on $qp$, $\mathcal C$ sends $qp$ to  $\mathcal S$.  Next (if the $\mathcal S$ agrees on the parameters) they take the steps in the Setup and Agreement phases in the SAP, at time $T_{\scriptscriptstyle 0}$. Let $t_{\scriptscriptstyle qp}:=(\ddot{x}_{\scriptscriptstyle qp},g_{\scriptscriptstyle qp})$ be proof encoding token,  where  $\ddot{x}_{\scriptscriptstyle qp}$ is the opening and $g_{\scriptscriptstyle qp}$ is the commitment stored on the contract as a result of running   SAP. 
%\item Let  $cp:=(o,o_{\scriptscriptstyle max},l,l',l_{\scriptscriptstyle max},z)$, where $o_{\scriptscriptstyle max}$ is the maximum amount of coins  for an accepting service proof,  $l_{\scriptscriptstyle max}$  is the maximum amount of coins to resolve a potential dispute, and $z$ is the number of  service proofs/verifications. For $\mathcal C$ and $\mathcal S$ to provably agree on $cp$, similar to the previous step, they invoke SAP, at time $T_{\scriptscriptstyle 1}$.  Let $t_{\scriptscriptstyle cp}:=(\ddot{x}_{\scriptscriptstyle cp},g_{\scriptscriptstyle cp})$ be coin encoding token,  where  $\ddot{x}_{\scriptscriptstyle cp}$ is the opening and $g_{\scriptscriptstyle cp}$ is the commitment stored on the contract as a result of executing the  SAP. 
%\item $\mathcal C$ sets   $coin^{\scriptscriptstyle *}_{\scriptscriptstyle\mathcal C}=z\cdot (o_{\scriptscriptstyle max}+l_{\scriptscriptstyle max})$ and $coin^{\scriptscriptstyle *}_{\scriptscriptstyle\mathcal S}=z\cdot l_{\scriptscriptstyle max}$, where $coin^{\scriptscriptstyle *}_{\scriptscriptstyle\mathcal C}$ and $coin^{\scriptscriptstyle *}_{\scriptscriptstyle\mathcal S}$ are   the total number of masked coins that $\mathcal C$ and $\mathcal S$ should deposit respectively. $\mathcal C$ signs and deploys  a smart contract, SC, that explicitly specifies  parameters $z$,  $coin^{\scriptscriptstyle *}_{\scriptscriptstyle\mathcal C}$ and $coin^{\scriptscriptstyle *}_{\scriptscriptstyle\mathcal S}$. It deposits $coin^{\scriptscriptstyle *}_{\scriptscriptstyle\mathcal C}$ coins in the contract.}
%\textcolor{green}{\item $\mathcal C$ constructs  vector $\bm{w}_{\scriptscriptstyle\mathcal C}$, also $\mathcal S$  constructs $\bm{v}_{\scriptscriptstyle\mathcal S}$,  where the vectors are initially empty. }
%
%\item\textcolor{red}{ $\mathcal C$ runs $\mathtt{PoRID.setup}(1^{\lambda}, u)\rightarrow (u^{\scriptscriptstyle *},pp)$.  It sends  encoded file $u^{\scriptscriptstyle *}$ to $\mathcal S$, and sends the public key's encryption: $\mathtt{Enc}(\bar{k},pk)$ to SC at time $T_{\scriptscriptstyle 2}$
%\item $\mathcal S$ checks the above parameters, and ensures sufficient amount of coins has been deposited.  If any check is rejected, then it sets $a=0$. Otherwise, it decrypts the public key, $\mathtt{Dec}(\bar{k},\mathtt{Enc}(\bar{k},pk))=pk$. It runs $\mathtt{PoRID.serve}(u^{\scriptscriptstyle *},pp)\rightarrow a$. Next, it sends $a$ and $coin^{\scriptscriptstyle *}_{\scriptscriptstyle\mathcal S}$ coins to SC at time $T_{\scriptscriptstyle 3}$, where  $coin^{\scriptscriptstyle *}_{\scriptscriptstyle\mathcal S}=\bot$ if $a=0$
%\item $\mathcal C$  and  $\mathcal S$ can withdraw their coins at time $T_{\scriptscriptstyle 4}$, if the server sends $a=0$, fewer coins than $coin^{\scriptscriptstyle *}_{\scriptscriptstyle\mathcal S}$, or nothing to the SC
%}

%%%%%%%%%%%%%%%%%%%%%%%%%%%%%%%%%%%%%

%
%
%\item $\mathcal C$ constructs  vector  $ {\bm{w}}_{\scriptscriptstyle\mathcal{C}}$ and $\mathcal S$  constructs $ {\bm{v}}_{\scriptscriptstyle\mathcal S}$,  where the vectors are initially empty.   
%
%\item $\mathcal C$ and $\mathcal S$ agree on the following parameters, $z$: the number of the service proofs/verifications, $o$: the amount of coins for each accepting proof, $l$ and $l'$ which are the amount of coins $\mathcal C$ and $\mathcal S$ respectively need to send to a smart contract to resolve a dispute. Furthermore, they agree on $f_{\scriptscriptstyle\mathcal C}$ and $f_{\scriptscriptstyle\mathcal S}$ that are the amount of masking coins $\mathcal C$ and $\mathcal S$ will use respectively. 


%where $l_{\scriptscriptstyle\mathcal C}$ and $l_{\scriptscriptstyle\mathcal S}$ cover the costs of a light verification of the client's and server's inputs respectively, while $l'_{\scriptscriptstyle\mathcal S}$ covers the cost of a more involved verification of the server's input.  

%\item $\mathcal C$ and $\mathcal S$ construct  a statement of the  form: $x=(o,l,l', f_{\scriptscriptstyle\mathcal C},f_{\scriptscriptstyle\mathcal S},z)$   
%\item For $\mathcal C$ and $\mathcal S$ to provably agree on $x$, they take the steps in the Setup and Agreement phases in the SAP, at time $T_0$. Let $\mu$ be the statement's proof. 
%


%\item $\mathcal S$ picks and sends to $\mathcal C$ a random key, $\bar{k}$

%\item  $\mathcal C$ sends to $\mathcal S$ a statement of the form $x'=({pad},\bar{k})$ where  ${pad}$  specifies the number of  dummy values that will be used to pad encrypted proofs and    $\bar{k}$ is a  secret key.
%
%
%
%\item For $\mathcal C$ and $\mathcal S$ to provably agree on $x'$, they take the steps in the Setup and Agreement phases in the SAP again, at time $T_{\scriptscriptstyle 1}$. Let $\mu'$ be the proof. 



%\item For $\mathcal C$ and $\mathcal S$ to provably agree on $\bar{k}$, they take the steps in the Setup and Agreement phases in the SAP again, at time $T_{\scriptscriptstyle 1}$. Let $\mu'$ be the proof. 

%
%\item $\mathcal C$ and $\mathcal S$ agree on a smart contract: SC. Let $p_{\scriptscriptstyle\mathcal C}=z(o+l')+f_{\scriptscriptstyle\mathcal C}$ and $p_{\scriptscriptstyle\mathcal S}=zl+f_{\scriptscriptstyle\mathcal S}$ be the total number of coins the client and server should deposit respectively. SC  explicitly specifies  parameters $z$, $p_{\scriptscriptstyle\mathcal C}$ and $p_{\scriptscriptstyle\mathcal S}$. They sign and deploy the contract. 
%
%\item $\mathcal C$ and $\mathcal S$ deposit their masked coins in SC at time $T_{\scriptscriptstyle 2}$
%\item At time $T_{\scriptscriptstyle 3}$, if the deposit amount that either party has put is less than the amount stated in SC, then SC allows each party to withdraw its
%coins and terminate the contract.
%
%
%
%\item $\mathcal S$ decrypts the public key, $\mathtt{Dec}(\bar{k},\mathtt{Enc}(\bar{k},pk))=pk$. Next, it runs $\mathtt{PoRID.serve}(u^{\scriptscriptstyle *},pk)\rightarrow a$ and then sends $a$  to SC at time $T_{\scriptscriptstyle 4}$
%
%\item $\mathcal C$ and $\mathcal S$ can withdraw their coins at time $T_{\scriptscriptstyle 5}$, if the server sends $a=0$ or nothing to the SC 

\end{enumerate}

\
\item \textbf{Server-side Initiation}\label{RCPoRP::Server-side-Initiation}. $\mathtt{RCPoRP}.\mathtt{sInit}(u^{\scriptscriptstyle *},  z, T, p_{\scriptscriptstyle\mathcal{S}} ,\bm{y})$

\begin{enumerate}

\item Checks the parameters  in $T$ (e.g. $qp$ and $cp$) and  in SC (e.g. $p_{\scriptscriptstyle\mathcal{S}}, \bm{y}$) and ensures sufficient amount of coins has been deposited by $\mathcal C$.  

\item Calls $\mathtt{SAP.agree}(qp,r_{\scriptscriptstyle qp},g_{\scriptscriptstyle qp},adr_{\scriptscriptstyle\mathcal{C}},adr_{\scriptscriptstyle\text{SAP}_{1}})\rightarrow (g'_{\scriptscriptstyle qp},b_{\scriptscriptstyle 1})$ and $\mathtt{SAP.agree}(cp,r_{\scriptscriptstyle cp},g_{\scriptscriptstyle cp},adr_{\scriptscriptstyle\mathcal{C}},adr_{\scriptscriptstyle\text{SAP}_{2}})\rightarrow (g'_{\scriptscriptstyle cp},b_{\scriptscriptstyle 2})$, to verify the correctness of tokens in $T$ and to agree on the tokens' parameters, where $qp,r_{\scriptscriptstyle qp}\in \ddot{x}_{\scriptscriptstyle qp}, \text{and }  cp,r_{\scriptscriptstyle cp} \in  \ddot{x}_{\scriptscriptstyle cp}$. Recall, if both $\mathcal{C}$ and $\mathcal{S}$ are honest, then $g_{\scriptscriptstyle qp}=g'_{\scriptscriptstyle qp}$ and $g_{\scriptscriptstyle cp}=g'_{\scriptscriptstyle cp}$. 




\item If any above check is rejected, then it sets $a=0$. Otherwise, it  calls $\mathtt{PoRID.serve}(u^{\scriptscriptstyle *}, {pp})\rightarrow a$. 

\item Sends $a$ and $coin^{\scriptscriptstyle *}_{\scriptscriptstyle\mathcal S}=p_{\scriptscriptstyle\mathcal{S}}$ coins to SC at time $\texttt{T}_{\scriptscriptstyle 1}$, where  $coin^{\scriptscriptstyle *}_{\scriptscriptstyle\mathcal S}=\bot$ if $a=0$
\end{enumerate}
Note, $\mathcal S$  and  $\mathcal C$ can withdraw their coins at time $\texttt{T}_{\scriptscriptstyle 2}$, if  $\mathcal S$ sends $a=0$, fewer coins than $p_{\scriptscriptstyle\mathcal{S}}$, or nothing to the SC. To withdraw, $\mathcal S$  or  $\mathcal C$  simply sends a ``pay'' message to $\mathtt{RCPoRP}.\mathtt{pay}(.)$ algorithm only at time $\texttt{T}_{\scriptscriptstyle 2}$.

\





%\item \textbf{Billing-cycles}. At each time $G_{\scriptscriptstyle j}$, (where $1\leq j\leq z$ and $G_{\scriptscriptstyle 1}>T_{\scriptscriptstyle 5}$), that a proof is generated, the parties perform as follows. 

\textbf{\textit{Billing-cycles Onset}}. $\mathcal{C}$ and $\mathcal{S}$ engage in the following three phases, i.e. phase \ref{Billing-cycles-genQuery}-\ref{Client-sideProofVerification},  at the end of every $j$-th billing cycle, where $1\leq j\leq z$. Each $j$-th cycle includes two  time points, $\texttt{G}_{\scriptscriptstyle j,1}$ and $\texttt{G}_{\scriptscriptstyle j,2}$, where $\texttt{G}_{\scriptscriptstyle j,2}>\texttt{G}_{\scriptscriptstyle j,1}$, and $\texttt{G}_{\scriptscriptstyle 1,1}>\texttt{T}_{\scriptscriptstyle 2}$ 


\

\item\textbf{Client-side Query Generation}.\label{Billing-cycles-genQuery} $\mathtt{RCPoRP}.\mathtt{genQuery}(1^\lambda,  T_{\scriptscriptstyle qp})$

\begin{enumerate}

\item Calls $\mathtt{PoRID.genQuery}(1^{\lambda},  pp)\rightarrow \hat{k}_{\scriptscriptstyle j}$ to generate a query, where  $ pp\in T_{\scriptscriptstyle qp}$
\item Sends   $\hat{k}_{\scriptscriptstyle j}$ to SC at time $\texttt{G}_{\scriptscriptstyle j,1}$


%\item Calls $\mathtt{VSID.genQuery}(1^{\lambda},  \text{aux},k,Q, \textcolor{purple}{pp})\rightarrow c_{\scriptscriptstyle j}:=(\bm{q}_{\scriptscriptstyle j},\bm{w}_{\scriptscriptstyle q_{_{\scaleto{j}{\size}}}})$, to generate a query-proof pair. 
%\item Encrypts the pair, $\mathtt{Enc}(\bar{k},c_{\scriptscriptstyle j})=c'_{\scriptscriptstyle j}$, where $\bar{k}\in T_{\scriptscriptstyle qp}$. Then, it pads (each element of) the result with ${pad}_{\scriptscriptstyle q}\in T_{\scriptscriptstyle qp}$ random values that are picked from the encryption's output range, $U$. Let $c^{\scriptscriptstyle *}_{\scriptscriptstyle j}$ be the result. 
%\item Sends the padded encrypted query-proof pair, $c^{\scriptscriptstyle *}_{\scriptscriptstyle j}$, to SC at time $\texttt{G}_{\scriptscriptstyle j,1}$
\end{enumerate}

\

\item\textbf{Server-side Proof Generation}. $\mathtt{RCPoRP}.\mathtt{prove}(u^{\scriptscriptstyle *},  \hat{k}_{\scriptscriptstyle j}, T_{\scriptscriptstyle qp})$


\begin{enumerate}

\item Constructs an empty  vector, $\bm{m}_{\scriptscriptstyle\mathcal S}=\bot$, if $j=1$.  



\item Calls $\mathtt{PoRID.checkQuery}(\hat{k}_{\scriptscriptstyle j}, pp)\rightarrow b_{\scriptscriptstyle j}$ to check the query's correctness.

\begin{itemize}
\item[$\bullet$] If it accepts the query, then it calls $\mathtt{PoRID.prove}(u^{\scriptscriptstyle *},\hat{k}_{\scriptscriptstyle j},pp)\rightarrow  {\bm{\pi}}_{\scriptscriptstyle j}$, to generate a PoR  proof. In this case,  $\mathcal S$ encrypts every  proof in the proof vector, i.e. $\forall g, 1\leq g\leq| {\bm{\pi}}_{\scriptscriptstyle j}|: \mathtt{Enc}(\bar{k}, {\bm{\pi}}_{\scriptscriptstyle j}[g])= {\bm{\pi}}'_{\scriptscriptstyle j}[g]$, where $\bar{k}\in T_{\scriptscriptstyle qp}$. Let vector $ {\bm{\pi}}'_{\scriptscriptstyle j}$ contain the encryption of all proofs. It pads every encrypted proof in $ {\bm{\pi}}'_{\scriptscriptstyle j}$ with ${pad}_{\scriptscriptstyle \pi}\in T_{\scriptscriptstyle qp}$ random values that are picked from the encryption's output range $U$, (by appending the random values to the encrypted proofs vector). Let $\bm{\pi}^{\scriptscriptstyle *}_{\scriptscriptstyle j}$ be the result. It sends the padded encrypted proofs  to SC at time $\texttt{G}_{\scriptscriptstyle j,2}$ 
\item[$\bullet$] Otherwise (if $\mathcal S$ rejects the query), it appends $j$  to $ {\bm{m}}_{\scriptscriptstyle\mathcal{S}}$,  constructs a dummy proof $ {\bm{\pi}}'_{\scriptscriptstyle j}$ whose elements are randomly picked from $U$,  pads the result as above, and sends the result, $\bm{\pi}^{\scriptscriptstyle *}_{\scriptscriptstyle j}$, to SC at time $\texttt{G}_{\scriptscriptstyle j,2}$

\end{itemize}


%
%\item Removes the pads from $c^{\scriptscriptstyle *}_{\scriptscriptstyle j}$, using  parameters of $T_{\scriptscriptstyle qp}$. Let  $c'_{\scriptscriptstyle j}$ be the result. Next, it  decrypts  the result, $\mathtt{Dec(}\bar{k},c'_{\scriptscriptstyle j})=c_{\scriptscriptstyle j}$. Then,  it   runs $\mathtt{VSID.checkQuery}(c_{\scriptscriptstyle j}, pk, \textcolor{purple}{pp})\rightarrow b_{\scriptscriptstyle j}$, to check the correctness of the queries.

%\item  $\mathcal S$ appends $j$  to $\vv{\bm{v}}_{\scriptscriptstyle\mathcal{S}}$, if the above proofs are rejected, i.e. $b_{\scriptscriptstyle j}=0$. % or were not posted on time. In this case, it does not proceed to the next step for this $j$-th verification.



%\item  $\mathcal S$ appends $j$  to $\vv{\bm{v}}_{\scriptscriptstyle\mathcal{S}}$, if the above proofs are rejected, i.e. $b_{\scriptscriptstyle j}=0$, or were not posted on time. In this case, it does not proceed to the next step for this $j$-th verification.

%\begin{itemize}
%\item[$\bullet$] If $\mathcal S$ accepts the query, i.e. $b_{\scriptscriptstyle j}=1$, then  calls $\mathtt{VSID.prove}(u^{\scriptscriptstyle *},\sigma,c_{\scriptscriptstyle j},pk,\textcolor{purple}{pp})\rightarrow \pi_{\scriptscriptstyle j}$, to generate the service proof. In this case, $\mathcal S$ encrypts it, $\mathtt{Enc}(\bar{k},\pi_{\scriptscriptstyle j})=\pi'_{\scriptscriptstyle j}$. Next, it   pads (every element of) the encrypted proof with ${pad}_{\scriptscriptstyle \pi}\in T_{\scriptscriptstyle qp}$ random values  picked from $U$. Let $\pi^{\scriptscriptstyle *}_{\scriptscriptstyle j}$ be the result. It sends the padded encrypted proof  to SC at time $\texttt{G}_{\scriptscriptstyle j,2}$ 
%
%\item[$\bullet$] Otherwise (if $\mathcal S$ rejects the query), it appends  $j$  to $\bm{m}_{\scriptscriptstyle\mathcal{S}}$,  constructs a dummy proof $\pi'_{\scriptscriptstyle j}\in U$,  pads the result as above, and sends the padded dummy proof, $\pi^{\scriptscriptstyle *}_{\scriptscriptstyle j}$, to SC at time $\texttt{G}_{\scriptscriptstyle j,2}$ %, if the above proofs are rejected, i.e. $b_{\scriptscriptstyle j}=0$.
%\end{itemize}
When $j=z$  and $\bm{m}_{\scriptscriptstyle\mathcal{S}}\neq\bot$, it sets    $m_{\scriptscriptstyle\mathcal S}:=\bm{m}_{\scriptscriptstyle\mathcal{S}}$.

\end{enumerate}













%\item \textbf{Billing-cycles}. The parties do the following, at the end of  every $j$-th billing cycle, where $1\leq j\leq z$. Each $j$-th cycle includes two  time points, $\texttt{G}_{\scriptscriptstyle j,1}$ and $\texttt{G}_{\scriptscriptstyle j,2}$, where $\texttt{G}_{\scriptscriptstyle j,2}>\texttt{G}_{\scriptscriptstyle j,1}$, and $\texttt{G}_{\scriptscriptstyle 1,1}>\texttt{T}_{\scriptscriptstyle 2}$ 

%At each time $G_{\scriptscriptstyle j}$, (where $1\leq j\leq z$ and $G_{\scriptscriptstyle 1}>T_{\scriptscriptstyle 5}$), that a proof is generated, the parties perform as follows. 



%\item $\mathcal C$ calls $\mathtt{PoRID.genQuery}(1^{\lambda},  pp)\rightarrow \hat{k}_{\scriptscriptstyle j}$, to generate a query. It sends   $\hat{k}_{\scriptscriptstyle j}$ to SC at time $\texttt{G}_{\scriptscriptstyle j,1}$

%\item $\mathcal S$  appends $j$ to $ {\bm{v}}_{\scriptscriptstyle\mathcal{S}}$, if the client's query was not posted on time. In this case, it does not take the next steps, for  $j$-th verification. 


%\item If $\mathcal S$ accepts the query, i.e. $b_{\scriptscriptstyle j}=1$, then it calls $\mathtt{VSID.prove}(u^{\scriptscriptstyle *},\sigma,c_{\scriptscriptstyle j},pk)\rightarrow \pi_{\scriptscriptstyle j}$, to generate the service proof. In this case, $\mathcal S$ encrypts the proof, $\mathtt{Enc}(\bar{k},\pi_{\scriptscriptstyle j})=\pi'_{\scriptscriptstyle j}$. Next, it   pads the encrypted proof with ${pad}_{\scriptscriptstyle 2}$ random values that are picked from $U$. It sends the padded encrypted proof  to SC at time $G_{\scriptscriptstyle j,2}$. Otherwise (if $\mathcal S$ rejects the query), it appends $j$  to $ {\bm{v}}_{\scriptscriptstyle\mathcal{S}}$,  constructs a dummy proof $\pi'_{\scriptscriptstyle j}\in U$,  pads the result as above, and sends the padded dummy proof to SC at time $G_{\scriptscriptstyle j,2}$.


%\item $\mathcal S$ decrypts the query,  $\mathtt{Dec(}\bar{k},\mathtt{Enc}(\bar{k},\hat{k}_{\scriptscriptstyle j}))=\hat{k}_{\scriptscriptstyle j}$.   Then, it calls $\mathtt{PoRID.checkQuery}(\hat{k}_{\scriptscriptstyle j}, pk)\rightarrow b_{\scriptscriptstyle j}$ to check the query's correctness. 

%\begin{enumerate}
%\item $\mathcal S$ calls $\mathtt{PoRID.checkQuery}(\hat{k}_{\scriptscriptstyle j}, pp)\rightarrow b_{\scriptscriptstyle j}$ to check the query's correctness. 
%
%\begin{itemize}
%\item[$\bullet$] If it accepts the query, then it calls $\mathtt{PoRID.prove}(u^{\scriptscriptstyle *},\hat{k}_{\scriptscriptstyle j},pp)\rightarrow  {\bm{\pi}}_{\scriptscriptstyle j}$, to generate a PoR  proof. In this case,  $\mathcal S$ encrypts every  proof in the proof vector, i.e. $\forall g, 1\leq g\leq| {\bm{\pi}}_{\scriptscriptstyle j}|: \mathtt{Enc}(\bar{k}, {\bm{\pi}}_{\scriptscriptstyle j}[g])= {\bm{\pi}}'_{\scriptscriptstyle j}[g]$. Let vector $ {\bm{\pi}}'_{\scriptscriptstyle j}$ contain the encryption of all proofs. It pads every encrypted proof in $ {\bm{\pi}}'_{\scriptscriptstyle j}$ with $pad_{\scriptscriptstyle \pi}$ random values that are picked from the encryption's output range $U$, (by appending the random values to the encrypted proofs vector). It sends the padded encrypted proofs  to SC at time $G_{\scriptscriptstyle j,2}$ 
%\item[$\bullet$] Otherwise (if $\mathcal S$ rejects the query), it appends $j$  to $ {\bm{v}}_{\scriptscriptstyle\mathcal{S}}$,  constructs a dummy proof $ {\bm{\pi}}'_{\scriptscriptstyle j}$ whose elements are randomly picked from $U$,  pads the result as above, and sends the padded dummy proof to SC at time $G_{\scriptscriptstyle j,2}$
%
%\end{itemize}

%\textcolor{green}{When $j=z$  and $\bm{v}_{\scriptscriptstyle\mathcal{S}}\neq\bot$, $\mathcal S$ sets    $\bm{m}_{\scriptscriptstyle\mathcal S}:[\bm{v}_{\scriptscriptstyle\mathcal{S}},\ddot{x}_{\scriptscriptstyle qp}, \text{``dispute''}]$.}


\


\item \textbf{Client-side Proof Verification}.\label{Client-sideProofVerification} $\mathtt{RCPoRP}.\mathtt{verify}(\bm{\pi}^{\scriptscriptstyle *}_{\scriptscriptstyle j}, \hat{k}_{\scriptscriptstyle j},T_{\scriptscriptstyle qp})$
\begin{enumerate}


\item Constructs an empty vector, $\bm{m}_{\scriptscriptstyle\mathcal C}=\bot$, if $j=1$. 

\item Removes the pads from $\bm{\pi}^{\scriptscriptstyle *}_{\scriptscriptstyle j}$, utilising  parameters of $T_{\scriptscriptstyle qp}$. Let $\bm{\pi}'_{\scriptscriptstyle j}$ be the result. It decrypts the service proofs $\mathtt{Dec}(\bar{k},\bm{\pi}'_{\scriptscriptstyle j})=\bm{\pi}_{\scriptscriptstyle j}$ and then calls $\mathtt{PoRID.verify}(\bm{\pi}_{\scriptscriptstyle j}, \hat{k}_{\scriptscriptstyle j}, {pp})\rightarrow \bm{d}_{\scriptscriptstyle j}$, to verify the proof.  

\begin{itemize}
\item[$\bullet$] If $ {\bm{\pi}}_{\scriptscriptstyle j}$ passes the verification, i.e. $ {\bm{d}}_{\scriptscriptstyle j}[0]=1$, then $\mathcal C$  concludes that the service for this verification has been delivered successfully.  


\item[$\bullet$] Otherwise (if proof $ {\bm{\pi}}_{\scriptscriptstyle j}$ is rejected, i.e. $ {\bm{d}}_{\scriptscriptstyle j}[0]=0$),  it sets $g= {\bm{d}}_{\scriptscriptstyle j}[1]$ and appends vector $[j,g]$  to $ {\bm{m}}_{\scriptscriptstyle\mathcal{C}}$. Recall, $ {\bm{d}}_{\scriptscriptstyle j}[1]$ refers to a rejected proof's index in proof vector $ {\bm{\pi}}_{\scriptscriptstyle j}$. 

\end{itemize}
\end{enumerate}
When $j=z$  and $\bm{m}_{\scriptscriptstyle\mathcal{C}}\neq\bot$, $\mathcal C$ sets    ${m}_{\scriptscriptstyle\mathcal C}:=\bm{m}_{\scriptscriptstyle\mathcal{C}}$.


%\item  $\mathcal C$ appends $j$ to $ {\bm{v}}_{\scriptscriptstyle\mathcal{C}}$, if  $\mathcal S$ does not post the proof on time. In this case, it does not take the next steps, for  $j$-th verification. 


%In this phase,  $\mathcal C$ verifies the service proof. It first constructs two empty vectors, $\bm{m}_{\scriptscriptstyle\mathcal C}=\bot$ and $\bm{w}_{\scriptscriptstyle\mathcal{C}}=\bot$. Next, it removes the pads from the padded encrypted proofs and then
% decrypts the encrypted proofs: $\mathtt{Dec}(\bar{k}, {\bm{\pi}}'_{\scriptscriptstyle j})= {\bm{\pi}}_{\scriptscriptstyle j}$. Then, it   calls $\mathtt{PoRID.verify}( {\bm{\pi}}_{\scriptscriptstyle j},\hat{k}_{\scriptscriptstyle j},pp)\rightarrow  {\bm{d}}_{\scriptscriptstyle j}$, to verify them. If $ {\bm{\pi}}_{\scriptscriptstyle j}$ passes the verification, i.e. $ {\bm{d}}_{\scriptscriptstyle j}[0]=1$, then $\mathcal C$  concludes that the service for this verification has been delivered successfully. Otherwise (if proof $ {\bm{\pi}}_{\scriptscriptstyle j}$ is rejected, i.e. $ {\bm{d}}_{\scriptscriptstyle j}[0]=0$), then it sets $g= {\bm{d}}_{\scriptscriptstyle j}[1]$ and appends vector $[j,g]$  to $ {\bm{w}}_{\scriptscriptstyle\mathcal{C}}$. Recall, $ {\bm{d}}_{\scriptscriptstyle j}[1]$ refers to a rejected proof's index in proof vector $ {\bm{\pi}}_{\scriptscriptstyle j}$. 
 
 
 %\textcolor{green}{If $j=z$  and $\bm{w}_{\scriptscriptstyle\mathcal{C}}\neq\bot$, $\mathcal C$ sets    $\bm{m}_{\scriptscriptstyle\mathcal C}:[\bm{w}_{\scriptscriptstyle\mathcal{C}},\ddot{x}_{\scriptscriptstyle qp}, \text{``dispute''}]$.}
  

%\end{enumerate}

\

% \textbf{Dispute Resolution}.\label{RCSP::Dispute-Resolution} $\mathtt{RCSP}.\mathtt{resolve}(m_{\scriptscriptstyle \mathcal{C}},m_{\scriptscriptstyle \mathcal{S}}, z, {\bm{\pi}}^{\scriptscriptstyle *}, {\bm{c}}^{\scriptscriptstyle *}, pk, T_{\scriptscriptstyle qp},enc)$


\item \textbf{Dispute Resolution}. \label{RCPoRP::Dispute-Resolution} $\mathtt{RCPoRP}. \mathtt{resolve}(m_{\scriptscriptstyle \mathcal{C}},m_{\scriptscriptstyle \mathcal{S}}, z, {\bm{\pi}}^{\scriptscriptstyle *}, {\bm{q}},  T_{\scriptscriptstyle qp})$



The phase takes place only in  case of dispute, i.e. when $\mathcal C$ rejects service proofs or $\mathcal S$ rejects the queries. %or either party does not receive its counterparty's message on time. %A party who accepts all proofs does not need to get involved in this phase. 

\begin{enumerate}

%\item $\mathcal C$  at time $K_{\scriptscriptstyle 1}>G_{\scriptscriptstyle z,2}+H$ sends ``dispute'' message to SC 

%\item $\mathcal S$ sends $ {\bm{v}}_{\scriptscriptstyle\mathcal{S}}$ to SC, at time $K_{\scriptscriptstyle 2}$ 


%\item SC  ensures $ {\bm{v}}_{\scriptscriptstyle\mathcal{S}}$ is well-formed. In particular, it removes from  $ {\bm{v}}_{\scriptscriptstyle\mathcal{S}}$ duplicated elements or  elements that are not in the range $[1,z]$. SC for every  element $i\in  {\bm{v}}_{\scriptscriptstyle\mathcal{S}}$ checks if $\mathcal C$  sent to SC an encrypted query on time, in step \ref{Billing-cycles-genQuery};  if the client did not do so, then it appends $i$ to $ {\bm{v}}_{\scriptscriptstyle{SC}}$ and   increments $u_{\scriptscriptstyle\mathcal C}$ and $y_{\scriptscriptstyle\mathcal C}$ by $1$
 
 
 % and retakes step \ref{the-arbiter-loop} with the next element of $ {\bm{v}}$. Otherwise, it proceeds to the next step. 

\item $\mathcal S$  sends ${m}_{\scriptscriptstyle\mathcal S}$ and $\ddot{x}_{\scriptscriptstyle qp}$ to SC, at time $\texttt{K}_{\scriptscriptstyle 1}$, where $\texttt{K}_{\scriptscriptstyle 1}>\texttt{G}_{\scriptscriptstyle z,2} + \texttt{H}$

\item SC upon receiving ${m}_{\scriptscriptstyle\mathcal S}$ does the following. 
\begin{enumerate}

%\item Sets two  counters: $y_{\scriptscriptstyle\mathcal C}$ and $y_{\scriptscriptstyle\mathcal S}$  that are initially set to $0$. Also,


\item Checks the validity of  statement $\ddot{x}_{\scriptscriptstyle qp}$, by  sending it to SAP contract which  returns   $1$ or $0$. If the output is $0$, then SC discards the server's complaint, $m_{\scriptscriptstyle\mathcal S}$, and does not take steps \ref{RCPoRP::check-v} and \ref{sc-check-server-claim}. Otherwise, it proceeds to the next step.

%\item\label{sc-dec-server-key} Uses secret key $\bar{k}\in \ddot{x}_{\scriptscriptstyle qp}$ to decrypt the encrypted public key, $\mathtt{Dec}(\bar{k},\mathtt{Enc}(\bar{k},pk))=pk$

\item\label{RCPoRP::check-v} Removes from $ {\bm{v}}_{\scriptscriptstyle\mathcal{S}}$ any element that is duplicated or is not in the range $[1,z]$. It also constructs an empty vector $ {\bm{v}}$.

\item\label{sc-check-server-claim}  For any element $i\in {\bm{v}}_{\scriptscriptstyle\mathcal{S}}$:
\begin{itemize}
\item[$\bullet$] Fetches the related query, $\hat{k}_{\scriptscriptstyle i}\in \bm{q}$,   from SC 
\item[$\bullet$]Checks if  the query is well-formed, by calling $\mathtt{PoRID.checkQuery}(\hat{k}_{\scriptscriptstyle i}, pp)\rightarrow b_{\scriptscriptstyle i}$. If the query is rejected, i.e. $ b_{\scriptscriptstyle i}=0$, then it increments $y_{\scriptscriptstyle\mathcal C}$ by 1 and appends $i$ to $ {\bm{v}}$
\end{itemize}
Let $\texttt{K}_{\scriptscriptstyle 2}$ be the time SC finishes the above checks.
\end{enumerate}





\item $\mathcal C$  sends $m_{\scriptscriptstyle\mathcal C}$ and $\ddot{x}_{\scriptscriptstyle qp}$ to SC, at time $\texttt{K}_{\scriptscriptstyle 3}$
 %Also,  $\mathcal S$ sends $m_{\scriptscriptstyle\mathcal S}:( {\bm{v}}_{\scriptscriptstyle\mathcal{S}},\bar{k},\mu')$ to SC %,   where $| {\bm{v}}_{\scriptscriptstyle\mathcal{C}}|\neq 0$ and $| {\bm{v}}_{\scriptscriptstyle\mathcal{S}}|\neq 0$. 

%\item\label{check-client-claim} SC  ensures $ {\bm{v}}_{\scriptscriptstyle\mathcal{C}}$ is well-formed. Specifically, it  removes from  $ {\bm{v}}_{\scriptscriptstyle\mathcal{C}}$ any element that is duplicated, or   not in the range $[1,z]$.  Also, it removes from $ {\bm{v}}_{\scriptscriptstyle\mathcal{C}}$ the elements that are in $ {\bm{v}}_{\scriptscriptstyle{SC}}$. Next, SC  for every  element $i\in  {\bm{v}}_{\scriptscriptstyle\mathcal{C}}$ checks if $\mathcal S$  sent to SC an encrypted query on time, in step \ref{Billing-cycles-genQuery} and increments $u_{\scriptscriptstyle\mathcal S}$ and $y_{\scriptscriptstyle\mathcal S}$ by $1$  if the server did not do that.  It appends those $i$'s that pass the check to $ {\bm{v}}'_{\scriptscriptstyle{SC}}$. Note, $ {\bm{v}}'_{\scriptscriptstyle{SC}}$ contains the indices of verifications whose PoR proofs  were rejected by the client.


\item SC upon receiving $m_{\scriptscriptstyle\mathcal C}$, does the following. 
\begin{enumerate}
\item Checks the validity of  statement $\ddot{x}_{\scriptscriptstyle qp}$, by   sending $\ddot{x}_{\scriptscriptstyle qp}$ to SAP contract which  returns  either $1$ or $0$. If the output is $0$, then SC discards the client's complaint, ${m}_{\scriptscriptstyle\mathcal C}$, and does not take steps \ref{sc-checks-client-claim}-\ref{the-contract-loop}. Otherwise, it proceeds to the next step. 

 \item\label{sc-checks-client-claim} Ensures each vector   $ {\bm{m}}\in  {\bm{m}}_{\scriptscriptstyle\mathcal{C}}$ is well-formed. In particular, it verifies there exist no two vectors: $ {\bm{m}}, {\bm{m}}'\in  {\bm{m}}_{\scriptscriptstyle\mathcal{C}}$ such that $ {\bm{m}}[0]= {\bm{m}}'[0]$.  If such vectors  exist, it deletes the redundant ones from $ {\bm{m}}_{\scriptscriptstyle\mathcal{C}}$. This ensures no two claims refer to the same verification.  Also, it removes any vector $ {\bm{m}}$ from $ {\bm{m}}_{\scriptscriptstyle\mathcal{C}}$ if $ {\bm{m}}[0]$ is not in the range $[1,z]$ or if $ {\bm{m}}[0]\in  {\bm{v}}$. Note the latter check  (i.e. $ {\bm{m}}[0]\in  {\bm{v}}$) ensures $\mathcal C$ cannot hold $\mathcal S$ accountable if $\mathcal C$ has generated an ill-formed query for the same verification.   

%\item Uses secret key $\bar{k}\in \ddot{x}_{\scriptscriptstyle qp}$ to decrypt the encrypted public key, $\mathtt{Dec}(\bar{k},\mathtt{Enc}(\bar{k},pk))=pk$

\item\label{the-contract-loop} For every  vector $ {\bm{m}}\in {\bm{m}}_{\scriptscriptstyle \mathcal{C}}$:
\begin{itemize}
\item[$\bullet$]  Retrieves details of a proof that was rejected in each $i$-th verification.  In particular, it sets   $i= {\bm{m}}[0]$ and $g= {\bm{m}}[1]$. Recall that $g$ refers to the index of a rejected proof in the proof vector which was generated for $i$-th verification, i.e. $ {\bm{\pi}}_{\scriptscriptstyle i}$ 

\item[$\bullet$]  Fetches the related  query, $\hat{k}_{\scriptscriptstyle i}$, from SC.

\item[$\bullet$]  Removes the pads only from $g$-th padded encrypted proof. Let $ {\bm{\pi}}'_{\scriptscriptstyle i}[g]$ be the result. Next, it decrypts the  encrypted proof, $\mathtt{Dec}(\bar{k}, {\bm{\pi}}'_{\scriptscriptstyle i}[g]))= {\bm{\pi}}_{\scriptscriptstyle i}[g]$

\item[$\bullet$] \label{dummy-vector} Constructs a fresh vector: $ {\bm{\pi}}''_{\scriptscriptstyle i}$, such that  its $g$-th element equals $ {\bm{\pi}}_{\scriptscriptstyle i}[g]$ (i.e. $ {\bm{\pi}}''_{\scriptscriptstyle i}[g]=  {\bm{\pi}}_{\scriptscriptstyle i}[g] \text{ and } | {\bm{\pi}}''_{\scriptscriptstyle i}|= | {\bm{\pi}}_{\scriptscriptstyle i}|$) and the rest of its elements are dummy values. 

\item[$\bullet$] Calls   $\mathtt{PoRID.identify}( {\bm{\pi}}''_{\scriptscriptstyle i},g,\hat{k}_{\scriptscriptstyle i},pp)\rightarrow I_{\scriptscriptstyle i}$. If $I_{\scriptscriptstyle i}=\mathcal S$, then it increments $y_{\scriptscriptstyle\mathcal S}$  by $1$. Otherwise, it does nothing. 

\end{itemize}
\end{enumerate}
Let $\texttt{K}_{\scriptscriptstyle 4}$ be the time that SC finishes all the above checks.% in step \ref{the-contract-loop}

\end{enumerate} %%%%%%%%%

\

\item \textbf{Coin Transfer}. \label{RCPoRP::CoinTransfer} $\mathtt{RCPoRP}.\mathtt{pay}(\bm{y},T_{\scriptscriptstyle cp},a,p_{\scriptscriptstyle\mathcal S}, coin^{\scriptscriptstyle *}_{\scriptscriptstyle\mathcal C},coin^{\scriptscriptstyle *}_{\scriptscriptstyle\mathcal S})$
\begin{enumerate}


 \item If SC  receives  ``pay'' message  at time $\texttt{T}_{\scriptscriptstyle 2}$, where $a=0$ or $coins^{\scriptscriptstyle *}_{\scriptscriptstyle\mathcal{S}}<p_{\scriptscriptstyle\mathcal{S}}$, then it sends $coin^{\scriptscriptstyle *}_{\scriptscriptstyle\mathcal C}$ coins to $\mathcal C$ and $coin^{\scriptscriptstyle *}_{\scriptscriptstyle\mathcal S}$ coins to $\mathcal S$. Otherwise (i.e. they reach an agreement), they take the following step. 

\item Either $\mathcal C$ or $\mathcal S$ sends ``pay'' message and  statement $\ddot{x}_{\scriptscriptstyle cp}\in T_{\scriptscriptstyle cp}$ to SC at time $\texttt{L}>\texttt{K}_{\scriptscriptstyle 4}$
\item SC checks the validity of the statement by sending it  to SAP contract that  returns  either $1$ or $0$. SC only proceeds to the next step if the output is $1$
\item SC distributes the coins to the parties as follows:

\begin{itemize}
\item[$\bullet$]   $coin^{\scriptscriptstyle *}_{\scriptscriptstyle\mathcal C}-o(z-y_{\scriptscriptstyle\mathcal S})+l(y_{\scriptscriptstyle\mathcal S}-y_{\scriptscriptstyle\mathcal C})$ coins  to $\mathcal C$

\item[$\bullet$] $coin^{\scriptscriptstyle *}_{\scriptscriptstyle\mathcal S}+o(z-y_{\scriptscriptstyle\mathcal S})+l( y_{\scriptscriptstyle\mathcal C}-y_{\scriptscriptstyle\mathcal S})$ coins to $\mathcal S$
%\item[$\bullet$] $l(u_{\scriptscriptstyle\mathcal S}+u_{\scriptscriptstyle\mathcal C}+y_{\scriptscriptstyle\mathcal S}+y_{\scriptscriptstyle\mathcal C})$ coins to the arbiter.  
\end{itemize}

\end{enumerate}
\end{enumerate}


\begin{remark}
The reason in step \ref{dummy-vector} vector $ {\bm{\pi}}''_{\scriptscriptstyle i}$ is constructed  is to let SC make \emph{black-box}  use of $\mathtt{PoRID.identify}(.)$. Alternatively, SC could decrypt all proofs in $\mathtt{Enc}(\bar{k}, {\bm{\pi}}_{\scriptscriptstyle i})$ and pass them to $\mathtt{PoRID.identify}(.)$. However, this approach would impose a high cost, as all proofs have to be decrypted. 
\end{remark}

\begin{remark}
In general, a transaction that is sent   to a smart contract should  cover the cost of the contract's execution. Therefore, in the above protocol, if a party unnecessarily invokes a contract for an accepting proof, it has to pay the execution cost in advance. This is the reason the above protocol (unlike RC-S-P protocol) does not need to track the number of times a party unnecessarily invokes the contract.    
\end{remark}


\begin{remark}
\textcolor{purple}{Explain how we can slight modify the protocol to capture the fact  that the amounts of coin the client pays for a dispute resolution is different to the amount paid by the server to resolve a dispute. The idea is we introduce another value $l'$, we let $l$ and $l'$ be the amount of coin the client and server pay to resolve a dispute.  We add $l'$ to $cp$. Accordingly, we would have $coin^{\scriptscriptstyle *}_{\scriptscriptstyle\mathcal C}-o(z-y_{\scriptscriptstyle\mathcal S})+l'\cdot y_{\scriptscriptstyle\mathcal S}-l\cdot y_{\scriptscriptstyle\mathcal C}$ coins  to $\mathcal C$ and  $coin^{\scriptscriptstyle *}_{\scriptscriptstyle\mathcal S}+o(z-y_{\scriptscriptstyle\mathcal S})+l\cdot y_{\scriptscriptstyle\mathcal C}-l'\cdot y_{\scriptscriptstyle\mathcal S}$ coins to $\mathcal S$}
\end{remark}


%\in[ut{RC-PoR-P-Definition}

%% !TEX root =R-zkCSP.tex


\begin{theorem}
The RC-PoR-P protocol is secure, w.r.t. Definition \ref{def::RC-S-P-Security}, if PoRID and SAP are secure and the encryption scheme is semantically secure. 
\end{theorem}


 In order to prove the above theorem, we show that RC-PoR-P meets all the security properties defined in Section \ref{sec::RC-S-P-Definition}.  We first prove that RC-PoR-P satisfies security against a malicious server.  

 \begin{lemma}
 If SAP is secure and  PoRID scheme supports correctness, soundness, and detectable abort, then RC-PoR-P is secure against malicious server, w.r.t. Definition \ref{deff::RC-S-P-SecurityAgainstMaliciousServer}. 
 \end{lemma}
 

 \begin{proof}[sketch]
 We first consider event $F(u^{\scriptscriptstyle *},\bm{q}_{\scriptscriptstyle j})= h_{\scriptscriptstyle j}\ \wedge coin_{\scriptscriptstyle\mathcal{C},j}\neq  \frac{coin_{\scriptscriptstyle\mathcal C}^{\scriptscriptstyle*}}{z}-o$ that captures the case where the server provides an accepting service proof but makes the client withdraw an incorrect amount of coins, i.e. $coin_{\scriptscriptstyle\mathcal{C},j}\neq  \frac{coin_{\scriptscriptstyle\mathcal C}^{\scriptscriptstyle*}}{z}-o$. In this case, since the proof is valid, an honest client accepts it and does not raise any dispute. 
  However, the server  would be able to make the client  withdraw incorrect amounts of coins, if it manages to either convince the arbiter that the client has misbehaved (through dispute resolution phase), or submit to the contract, at the coin transfer phase,  an accepting  statement $\ddot{x}'_{\scriptscriptstyle cp}$ other than what was agreed at the initiation phase, i.e. $\ddot{x}_{\scriptscriptstyle cp}$. Nevertheless, it cannot falsely accuse the client of misbehaviour. Because,  due to the security of SAP, it cannot  convince the arbiter to accept different decryption key or pads other than what was agreed with the client in the initiation phase; specifically, it cannot persuade the arbiter to accept  $\ddot{x}'_{\scriptscriptstyle qp}$, where $\ddot{x}'_{\scriptscriptstyle qp}\neq \ddot{x}_{\scriptscriptstyle qp}$, except with a negligible probability, $\mu(\lambda)$. This ensures that the honest client's message is accessed by  the arbiter with a high probability, as the arbiter can extract the client's message using valid pad information and decryption key. Moreover,  due to the security of SAP, the server cannot persuade the contract to accept any statement other than what was agreed initially between the client and server, except with a negligible probability $\mu(\lambda)$ when it finds the hash function's collision. Also, due to the correctness of VSID, the arbiter always accepts the honest client's accepting proof. 

We now move on to event $F(u^{\scriptscriptstyle *}, \bm{q}_{\scriptscriptstyle j})\neq h_{\scriptscriptstyle j} \ \wedge (d_{\scriptscriptstyle j}=1	\vee y_{\scriptscriptstyle \mathcal S}[j]=0 \ \vee (coin_{\scriptscriptstyle\mathcal{C},j}\neq \frac{coin_{\scriptscriptstyle\mathcal C}^{\scriptscriptstyle*}}{z} \ \vee  coin_{\scriptscriptstyle\mathcal{A}r,j}\neq l))$ which captures the case where the server provides an invalid service proof but either persuades the client to accept the proof, or  (when the client raises a dispute)  persuades the arbiter to accept the proof or makes the client or arbiter withdraw an incorrect amount of coins, i.e.    $coin_{\scriptscriptstyle\mathcal{C},j}\neq \frac{coin_{\scriptscriptstyle\mathcal C}^{\scriptscriptstyle*}}{z}$ or $coin_{\scriptscriptstyle\mathcal{A}r,j}\neq l$ respectively.   Nevertheless, due to the soundness of VSID, the probability that a corrupt server can convince an honest client to accept invalid proof (i.e. outputs $d_{\scriptscriptstyle j}=1$), is negligible, $\mu(\lambda)$. On the other hand, in the case where the client rejects the proof and raises a dispute, the server may try to convince the arbiter and make it output $y_{\scriptscriptstyle \mathcal S}[j]=0$, e.g. by sending a complaint right after the client does. But, for the adversary to win, it has to either provide a different accepting statement $\ddot{x}'_{\scriptscriptstyle qp}$, than what was initially agreed with the client (i.e. $\ddot{x}'_{\scriptscriptstyle qp}\neq \ddot{x}_{\scriptscriptstyle qp}$) and passes the verification, which   requires finding the hash function's collision, and its probability of success is negligible, $\mu(\lambda)$. Or it makes the arbiter  accept an invalid proof, but due to the detectable abort property of VSID, its probability of success is also negligible, $\mu(\lambda)$. In the case where the adversary does not succeed in convincing the client or arbiter, it may still try to  make them withdraw an incorrect amount of coins. To this end, at the coin transfer phase, it has to send a  different accepting statement  than what was initially agreed with the client. But, it would succeed only with a negligible probability, $\mu(\lambda)$, due to the security of SAP. 

Furthermore, in   both  events above, due to the security of SAP, the adversary cannot block an honest client's  messages, ``pay'' and $\ddot{x}_{\scriptscriptstyle cp}$, to the contract in the coin transfer phase.
  \hfill\(\Box\)\end{proof}



\begin{lemma}
 If SAP is secure and  VSID scheme supports correctness, inputs well-formedness, and detectable abort, then RC-S-P is secure against malicious client, w.r.t. Definition \ref{deff::RC-S-P-Security-Against Malicious-Client}. 
\end{lemma}


\begin{proof}[sketch]
  First, we consider event  $(M(u^{\scriptscriptstyle *},k)= \sigma  \wedge Q(\text{aux},k)= \bm{q}_{\scriptscriptstyle j}) \wedge
  (coin_{\scriptscriptstyle\mathcal{S},j}\neq  \frac{coin_{\scriptscriptstyle\mathcal S}^{\scriptscriptstyle*}}{z}+o)$ which captures the case where the client provides  accepting metadata and query  but makes the server withdraw an incorrect amount of coins, i.e. $coin_{\scriptscriptstyle\mathcal{S},j}\neq  \frac{coin_{\scriptscriptstyle\mathcal S}^{\scriptscriptstyle*}}{z}+o$. In short, for the event to  happen with a high probability, a malicious client has to break the security of SAP. In particular, since the metadata and query's proofs are valid, an honest server accepts them and does not raise any dispute.  But, the client may want to  make the server  withdraw incorrect amounts of coins, if it manages to either convince the arbiter, in phase \ref{RCSP::Dispute-Resolution}, that the server has misbehaved, or submit to the contract  an accepting  statement $\ddot{x}'_{\scriptscriptstyle cp}$ other than what was agreed at the initiation phase, i.e. $\ddot{x}_{\scriptscriptstyle cp}$, in phase \ref{RCSP::CoinTransfer}. However, it cannot falsely accuse the server of misbehaviour. As,  due to the security of SAP, it cannot  convince the arbiter to accept different decryption key and pads, by providing a different accepting statement $\ddot{x}'_{\scriptscriptstyle qp}$  (where $\ddot{x}'_{\scriptscriptstyle qp}\neq \ddot{x}_{\scriptscriptstyle qp}$), than what was initially agreed with the server, except with  probability $\mu(\lambda)$. This ensures that  the arbiter is given the honest server's messages (with a high probability). Moreover,  due to the security of SAP, the client cannot convince the contract to accept any accepting statement other than what was initially agreed  between the client and server, except with   probability $\mu(\lambda)$. Furthermore,   the correctness of VSID guarantees that  the arbiter always accepts the honest server's accepting proof. 
  
    We now turn our attention to $(M(u^{\scriptscriptstyle *},k)\neq \sigma \wedge a=1)\ \vee\ (Q(\text{aux},k)\neq \bm{q}_{\scriptscriptstyle j} \wedge b_{\scriptscriptstyle j}=1)$,   that captures the case where the server accepts an ill-formed metadata, or query. However, due to inputs well-formedness of VSID, the probability that either of the events happens is negligible, $\mu(\lambda)$. Next, we move of to      $Q(\text{aux},k)\neq \bm{q}_{\scriptscriptstyle j} \wedge
  (y_{\scriptscriptstyle\mathcal{C}}[j]=0 \ \vee  coin_{\scriptscriptstyle\mathcal{S},j}\neq \frac{coin_{\scriptscriptstyle\mathcal S}^{\scriptscriptstyle*}}{z}+o \ \vee  coin_{\scriptscriptstyle\mathcal{A}r,j}\neq l)$.  It considers the case where the client provides an invalid query, but  either convinces the arbiter to accept it, or  makes the server or arbiter  withdraw an incorrect amount of coins, i.e. $coin_{\scriptscriptstyle\mathcal{S},j}\neq \frac{coin_{\scriptscriptstyle\mathcal S}^{\scriptscriptstyle*}}{z}+o$ or  $coin_{\scriptscriptstyle\mathcal{A}r,j}\neq l$ respectively. Note, when the server rejects the query and raises a dispute, the client may try to convince the arbiter, and make it output $y_{\scriptscriptstyle \mathcal C}[j]=0$, e.g. by sending a complaint right after the server does so. However, for the adversary to win, either it has to provide a different accepting statement $\ddot{x}'_{\scriptscriptstyle qp}$, than what was initially agreed with the server (i.e. $\ddot{x}'_{\scriptscriptstyle qp}\neq \ddot{x}_{\scriptscriptstyle qp}$) and passes the verification. But,  due to the security of SAP, its probability of success is negligible, $\mu(\lambda)$. Or it has to make the arbiter  accept an invalid query, i.e. makes the arbiter output $y_{\scriptscriptstyle\mathcal{C}}[j]=0$. Nevertheless, due to the detectable abort property of VSID, its probability of success is  negligible, $\mu(\lambda)$. If the adversary does not succeed in convincing the server or arbiter, it may still try to  make them withdraw an incorrect amount of coins. To this end, at the coin transfer phase, it has to send a  different accepting statement  than what was initially agreed with the server. However, due to the security of SAP, its  success probability is negligible, $\mu(\lambda)$. Also, due to the security of SAP, the adversary cannot block an honest server's  messages, ``pay'' and $\ddot{x}_{\scriptscriptstyle cp}$, to the contract in the coin transfer phase.
 \hfill\(\Box\)\end{proof}
  
  
  Prior to proving  RC-S-P's privacy, we provide a lemma that will be used in the privacy's proof. Informally, the lemma states that encoded coins leaks no information about the actual amount of coins $(o,l)$, agreed between the client and server. 


\begin{lemma}\label{lemma::encoded-coins} Let $\beta\stackrel{\scriptscriptstyle\$}\leftarrow\{0,1\}$, price list be $\{(o_{\scriptscriptstyle 0},l_{\scriptscriptstyle 0}),(o_{\scriptscriptstyle 1},l_{\scriptscriptstyle 1})\}$, and encoded coin amounts be $coin^{\scriptscriptstyle *}_{\scriptscriptstyle\mathcal{C}}= z\cdot(Max(o_{\scriptscriptstyle\beta},o_{\scriptscriptstyle|\beta-1|})+Max(l_{\scriptscriptstyle\beta},l_{\scriptscriptstyle|\beta-1|}))$ and $coin^{\scriptscriptstyle *}_{\scriptscriptstyle\mathcal{S}}=z\cdot(Max(l_{\scriptscriptstyle\beta}, l_{\scriptscriptstyle |\beta-1|}))$. Then, given the price list, $z$, $coin^{\scriptscriptstyle *}_{\scriptscriptstyle\mathcal{C}}$, and $coin^{\scriptscriptstyle *}_{\scriptscriptstyle\mathcal{S}}$, an adversary $\mathcal{A}$ cannot tell the value of $\beta$ with probability significantly greater than $\frac{1}{2}$ (where the probability is taken over the choice of $\beta$ and the randomness of $\mathcal{A}$).
\end{lemma}



\begin{proof}
As it is evident, the list and $z$ contains no information about $\beta$. Also, since  $z$ is a public value, we could have   $coin'^{\scriptscriptstyle *}_{\scriptscriptstyle\mathcal C}=\frac{coin^{\scriptscriptstyle *}_{\scriptscriptstyle\mathcal C}}{z} = Max(o_{\scriptscriptstyle\beta},o_{\scriptscriptstyle|\beta-1|})+Max(l_{\scriptscriptstyle\beta}, l_{\scriptscriptstyle|\beta-1|})$. It is not hard to see $coin'^{\scriptscriptstyle *}_{\scriptscriptstyle\mathcal C}$ is a function of maximum value of $(o_{\scriptscriptstyle 0},o_{\scriptscriptstyle 1})$, and maximum value of $(l_{\scriptscriptstyle 0},l_{\scriptscriptstyle 1})$. It is also  independent of $\beta$. Therefore, given the list, $z$ and $coin'^{\scriptscriptstyle *}_{\scriptscriptstyle\mathcal C}$ the adversary learns nothing about $\beta$, unless it guesses the value,  with  success probability $\frac{1}{2}$. The same also holds for $coin^{\scriptscriptstyle *}_{\scriptscriptstyle\mathcal S}$. 
 \hfill\(\Box\)\end{proof}


\begin{lemma} If SAP is secure and the encryption scheme is semantically secure, then RC-S-P preserves privacy, w.r.t. Definition \ref{deff::RC-S-P-Privacy}. 
\end{lemma}

\begin{proof}[sketch]
Due to the privacy property of SAP, that stems from the  hiding property of the commitment scheme, given the commitments $g_{\scriptscriptstyle qp}$ and $g_{\scriptscriptstyle cp}$, (that are stored in the blockchain as result of running SAP) the adversary learns no information about the committed values (e.g. $o, l, pad_{\scriptscriptstyle\pi}, pad_{\scriptscriptstyle q},$ and $\bar{k}$), except with negligible probability, $\mu_{\scriptscriptstyle 1}(\lambda)$.   Also, given encoded coins $coin^{\scriptscriptstyle *}_{\scriptscriptstyle\mathcal{C}}=z\cdot (o_{\scriptscriptstyle max}+l_{\scriptscriptstyle max})$ and  $coin^{\scriptscriptstyle *}_{\scriptscriptstyle\mathcal{S}}=z\cdot l_{\scriptscriptstyle max}$, the adversary learns nothing about the actual price agreed between the server and client,  $(o,l)$, for each verification, due to Lemma \ref{lemma::encoded-coins}. Next we analyse the privacy of padded encrypted query vector $\bm{c}^{\scriptscriptstyle *}$. For  the sake of simplicity, we focus on   $\bm{q}^{\scriptscriptstyle *}_{\scriptscriptstyle j}\in c^{\scriptscriptstyle *}_{\scriptscriptstyle j}\in\bm{c}^{\scriptscriptstyle *}$, that is a padded encrypted query vector for $j$-th verification. Let $\bm{q}_{\scriptscriptstyle j,0}$ and $\bm{q}_{\scriptscriptstyle j,1}$  be query vectors, for $j$-th verification, related to the service inputs $u_{\scriptscriptstyle 0}$ and $u_{\scriptscriptstyle 1}$ that    are picked by the adversary according to  Definition  \ref{deff::RC-S-P-Privacy} which lets  the environment pick $\beta\stackrel{\scriptscriptstyle\$}\leftarrow \{0,1\}$. Also, let $\{\bm{q}_{\scriptscriptstyle j,0},...,\bm{q}_{\scriptscriptstyle j,m}\}$ be a  list of all queries of different sizes. In the experiment, if $\bm{q}_{\scriptscriptstyle j,\beta}$  is only   encrypted (but not padded), then given the ciphertext, due to semantical security of the encryption, an adversary cannot tell if the ciphertext corresponds to $\bm{q}_{\scriptscriptstyle j,0}$ or $\bm{q}_{\scriptscriptstyle j,1}$  (accordingly to $u_{\scriptscriptstyle 0}$ or $u_{\scriptscriptstyle 1}$) with probability greater than $\frac{1}{2}+\mu_{\scriptscriptstyle 1}(\lambda)$,  under the assumption  that $Max(|\bm{q}_{\scriptscriptstyle j,0}|,...,|\bm{q}_{\scriptscriptstyle j,m}|)=|\bm{q}_{\scriptscriptstyle j,\beta}|$. The  assumption is  relaxed with the use of a pad; as each encrypted query is padded  to the queries' maximum size, i.e. $Max(|\bm{q}_{\scriptscriptstyle j,0}|,...,|\bm{q}_{\scriptscriptstyle j,m}|)$, the adversary cannot tell with probability greater than $\frac{1}{2}+\mu_{\scriptscriptstyle 1}(\lambda)$ if the  padded encrypted proof corresponds to  $\bm{q}_{\scriptscriptstyle j,0}$ or $\bm{q}_{\scriptscriptstyle j,1}$, as the padded encrypted query \emph{always has the same size} and the pad values are picked from the same range as the encryption's ciphertext are defined. The same argument holds for $\bm{w}^{\scriptscriptstyle *}_{\scriptscriptstyle q_{_{\scaleto{j}{\size}}}}\in c^{\scriptscriptstyle *}_{\scriptscriptstyle j}\in\bm{c}^{\scriptscriptstyle *}$. 

Next we analyse the privacy of padded encrypted proof vector $\bm{\pi}^{\scriptscriptstyle *}$. The argument is similar to the one presented above; however, for  the sake of completeness we provide it.  Again, we focus on an element of the vector,  $\pi_{\scriptscriptstyle j}^{\scriptscriptstyle *}\in\bm{\pi}^{\scriptscriptstyle *}$, that is a padded encrypted proof for $j$-th verification. Let $\pi_{\scriptscriptstyle j,0}$ and $\pi_{\scriptscriptstyle j,1}$  be proofs, for $j$-th verification, related to the service inputs $u_{\scriptscriptstyle 0}$ and $u_{\scriptscriptstyle 1}$, where the inputs   are picked by the adversary, w.r.t. Definition  \ref{deff::RC-S-P-Privacy} in which  the environment picks $\beta\stackrel{\scriptscriptstyle\$}\leftarrow \{0,1\}$.  Let $\{\pi_{\scriptscriptstyle j,0},...,\pi_{\scriptscriptstyle j,m}\}$ be proof list including all proofs of different sizes. In the experiment, if $\pi_{\scriptscriptstyle j,\beta}$  is only   encrypted, then given the ciphertext, due to semantical security of the encryption, an adversary cannot tell if the ciphertext corresponds to $\pi_{\scriptscriptstyle j,0}$ or $\pi_{\scriptscriptstyle j,1}$  (accordingly to $u_{\scriptscriptstyle 0}$ or $u_{\scriptscriptstyle 1}$) with probability greater than $\frac{1}{2}+\mu_{\scriptscriptstyle 2}(\lambda)$,  if $Max(|\pi_{\scriptscriptstyle j,0}|,...,|\pi_{\scriptscriptstyle j,m}|)=|\pi_{\scriptscriptstyle j,\beta}|$. However, the  assumption is  relaxed with the use of a pad. In particular, since each encrypted proof is padded  to the proofs' maximum size, the adversary cannot tell with probability greater than $\frac{1}{2}+\mu_{\scriptscriptstyle 2}(\lambda)$ if the  padded encrypted proof corresponds to  $\pi_{\scriptscriptstyle j,0}$ or $\pi_{\scriptscriptstyle j,1}$. Also, since the value of $a$ is independent of $u_{\scriptscriptstyle 0}$  or $u_{\scriptscriptstyle 1}$, and only depends on whether the metadata is well-formed, it leaks nothing about the service input $u_{\scriptscriptstyle\beta}$ and $\beta$. Moreover, since each padded encrypted query and proof leak no information and always contains a fixed number of elements, an adversary cannot tell the status of a proof for each $j$-th verification (i.e. whether it is accepted or rejected) with the probability greater than $\frac{1}{2}+\mu_{\scriptscriptstyle 2}(\lambda)$, given  $\bm{c}^{\scriptscriptstyle *},coin^{\scriptscriptstyle *}_{\scriptscriptstyle \mathcal S},coin^{\scriptscriptstyle *}_{\scriptscriptstyle\mathcal C}, g_{\scriptscriptstyle cp},  g_{\scriptscriptstyle qp},$  $\bm{\pi}^{\scriptscriptstyle *}$, and $a$.
  \hfill\(\Box\)\end{proof}






