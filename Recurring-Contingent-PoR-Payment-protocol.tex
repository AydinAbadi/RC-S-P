% !TEX root =R-zkCSP.tex

\section{Protocol For RC-PoR-P Without Arbiter's Involvement}\label{sec::R-PoR-P-protocol-Withohut-Arbiter}

In this section, we elaborate on how the original recurring contingent PoR payment (RC-PoR-P) protocol, presented in Section \ref{sec::RC-PoR-P-protocol}, can be adjusted such that the third-party arbiter's role, i.e. resolving disputes, is totally delegated to the smart contract, SC.  The new variant is denoted by  ${\text{RC-}\overline{\text{PoR}}\text{-P}}$. Briefly, Phases \ref{RC-PoR-P::KeyGen}-\ref{Client-sideProofVerification} remain  unchanged, with an exception. Namely,   in step \ref{setcounters}, only two counters $y_{\scriptscriptstyle\mathcal{C}}$ and $y_{\scriptscriptstyle\mathcal{S}}$ are created, instead of four counters; accordingly, in the same step,  vector $\bm{y}$ is now   $\bm{y}: [y_{\scriptscriptstyle\mathcal C},y_{\scriptscriptstyle\mathcal S},  \texttt{Time},adr_{\scriptscriptstyle \text{SC}}]$, so counters $y'_{\scriptscriptstyle\mathcal{C}}$ and $y'_{\scriptscriptstyle\mathcal{S}}$  are excluded from the vector. At a high level,  the changes applied to phase \ref{RCPoRP::Dispute-Resolution} include that the parties send their complaints to SC now, SC does not maintain $y'_{\scriptscriptstyle\mathcal{C}}$ and $y'_{\scriptscriptstyle\mathcal{S}}$ anymore, SC takes the related steps  (on the arbiter's behalf), and it reads its internal state any time it needs to read data already stored on the contract. Moreover, the main adjustment to phase  \ref{RCPoRP::CoinTransfer} is that the amounts of coin each party receives  changes. For the sake of clarity, we  present  the modified version of  phases \ref{RCPoRP::Dispute-Resolution} and \ref{RCPoRP::CoinTransfer}, below. 




\begin{enumerate}
\item[\ref{RCPoRP::Dispute-Resolution}.] \textbf{Dispute Resolution}. \label{RCPoRP2::Dispute-Resolution2} $\mathtt{RCPoRP}. \mathtt{resolve}(m_{\scriptscriptstyle \mathcal{C}},m_{\scriptscriptstyle \mathcal{S}}, z, {\bm{\pi}}^{\scriptscriptstyle *}, \bm{q}^{\scriptscriptstyle *},  T_{\scriptscriptstyle qp})$


The phase takes place only in  case of dispute, i.e. when $\mathcal C$ rejects service proofs or $\mathcal S$ rejects the queries. %or either party does 

\begin{enumerate}

\item $\mathcal S$  sends ${m}_{\scriptscriptstyle\mathcal S}$ and $\ddot{x}_{\scriptscriptstyle qp}$ to SC, at time $\texttt{K}_{\scriptscriptstyle 1}$, where $\texttt{K}_{\scriptscriptstyle 1}>\texttt{G}_{\scriptscriptstyle z,2} + \texttt{H}$

\item SC upon receiving ${m}_{\scriptscriptstyle\mathcal S}$ does the following a time $\texttt{K}_{\scriptscriptstyle 2}$. 
\begin{enumerate}

\item Checks the validity of  statement $\ddot{x}_{\scriptscriptstyle qp}$, by  sending it to SAP contract which  returns   $1$ or $0$. If the output is $0$, then SC discards the server's complaint, $m_{\scriptscriptstyle\mathcal S}$, and does not take steps \ref{RCPoRP::check-v} and \ref{sc-check-server-claim}. Otherwise, it proceeds to the next step.


\item\label{RCPoRP2::check-v} Removes from $ {\bm{v}}_{\scriptscriptstyle\mathcal{S}}$ any element that is duplicated or is not in the range $[1,z]$. It also constructs an empty vector $ {\bm{v}}$.

\item\label{sc-check-server-claim2}  For any element $i\in {\bm{v}}_{\scriptscriptstyle\mathcal{S}}$:
\begin{itemize}

\item[$\bullet$] Fetches the related encrypted query $\hat{k}^{\scriptscriptstyle *}_{\scriptscriptstyle i}\in \bm{q}^{\scriptscriptstyle *}$, and   decrypts it, $\hat{k}_{\scriptscriptstyle i}=\mathtt{Dec}(\bar{k},\hat{k}^{\scriptscriptstyle *}_{\scriptscriptstyle i})$
 
\item[$\bullet$]Checks if  the query is well-formed, by calling $\mathtt{PoRID.checkQuery}(\hat{k}_{\scriptscriptstyle i}, pp)\rightarrow b_{\scriptscriptstyle i}$. If the query is rejected, i.e. $ b_{\scriptscriptstyle i}=0$, then it increments $y_{\scriptscriptstyle\mathcal C}$ by 1 and appends $i$ to $ {\bm{v}}$
\end{itemize}
Let $\texttt{K}_{\scriptscriptstyle 3}$ be the time SC finishes the above checks.
\end{enumerate}


\item $\mathcal C$  sends $m_{\scriptscriptstyle\mathcal C}$ and $\ddot{x}_{\scriptscriptstyle qp}$ to SC, at time $\texttt{K}_{\scriptscriptstyle 4}$


\item SC upon receiving $m_{\scriptscriptstyle\mathcal C}$, does the following at time $\texttt{K}_{\scriptscriptstyle 5}$. 
\begin{enumerate}
\item Checks the validity of  statement $\ddot{x}_{\scriptscriptstyle qp}$, by   sending $\ddot{x}_{\scriptscriptstyle qp}$ to SAP contract which  returns  either $1$ or $0$. If the output is $0$, then SC discards the client's complaint, ${m}_{\scriptscriptstyle\mathcal C}$, and does not take steps \ref{sc-checks-client-claim}-\ref{the-contract-loop}. Otherwise, it proceeds to the next step. 

 \item\label{sc-checks-client-claim2} Ensures each vector   $ {\bm{m}}\in  {\bm{m}}_{\scriptscriptstyle\mathcal{C}}$ is well-formed. In particular, it verifies there exist no two vectors: $ {\bm{m}}, {\bm{m}}'\in  {\bm{m}}_{\scriptscriptstyle\mathcal{C}}$ such that $ {\bm{m}}[0]= {\bm{m}}'[0]$.  If such vectors  exist, it deletes the redundant ones from $ {\bm{m}}_{\scriptscriptstyle\mathcal{C}}$. This ensures no two claims refer to the same verification.  Also, it removes any vector $ {\bm{m}}$ from $ {\bm{m}}_{\scriptscriptstyle\mathcal{C}}$ if $ {\bm{m}}[0]$ is not in the range $[1,z]$ or if $ {\bm{m}}[0]\in  {\bm{v}}$. Note the latter check  (i.e. $ {\bm{m}}[0]\in  {\bm{v}}$) ensures $\mathcal C$ cannot hold $\mathcal S$ accountable if $\mathcal C$ has generated an ill-formed query for the same verification.   

%\item Uses secret key $\bar{k}\in \ddot{x}_{\scriptscriptstyle qp}$ to decrypt the encrypted public key, $\mathtt{Dec}(\bar{k},\mathtt{Enc}(\bar{k},pk))=pk$

\item\label{the-contract-loop2} For every  vector $ {\bm{m}}\in {\bm{m}}_{\scriptscriptstyle \mathcal{C}}$:
\begin{itemize}
\item[$\bullet$]  Retrieves details of a proof that was rejected in each $i$-th verification.  In particular, it sets   $i= {\bm{m}}[0]$ and $g= {\bm{m}}[1]$. Recall that $g$ refers to the index of a rejected proof in the proof vector which was generated for $i$-th verification, i.e. $ {\bm{\pi}}_{\scriptscriptstyle i}$ 


\item[$\bullet$]  Fetches the related encrypted query $\hat{k}^{\scriptscriptstyle *}_{\scriptscriptstyle i}\in \bm{q}^{\scriptscriptstyle *}$, and   decrypts it, $\hat{k}_{\scriptscriptstyle i}=\mathtt{Dec}(\bar{k},\hat{k}^{\scriptscriptstyle *}_{\scriptscriptstyle i})$


\item[$\bullet$]  Removes the pads only from $g$-th padded encrypted proof. Let $ {\bm{\pi}}'_{\scriptscriptstyle i}[g]$ be the result. Next, it decrypts the  encrypted proof, $\mathtt{Dec}(\bar{k}, {\bm{\pi}}'_{\scriptscriptstyle i}[g]))= {\bm{\pi}}_{\scriptscriptstyle i}[g]$

\item[$\bullet$] \label{dummy-vector2} Constructs a fresh vector: $ {\bm{\pi}}''_{\scriptscriptstyle i}$, such that  its $g$-th element equals $ {\bm{\pi}}_{\scriptscriptstyle i}[g]$ (i.e. $ {\bm{\pi}}''_{\scriptscriptstyle i}[g]=  {\bm{\pi}}_{\scriptscriptstyle i}[g] \text{ and } | {\bm{\pi}}''_{\scriptscriptstyle i}|= | {\bm{\pi}}_{\scriptscriptstyle i}|$) and the rest of its elements are dummy values. 

\item[$\bullet$] Calls   $\mathtt{PoRID.identify}( {\bm{\pi}}''_{\scriptscriptstyle i},g,\hat{k}_{\scriptscriptstyle i},pp)\rightarrow I_{\scriptscriptstyle i}$. If $I_{\scriptscriptstyle i}=\mathcal S$, then it increments $y_{\scriptscriptstyle\mathcal S}$  by $1$. Otherwise, it does nothing. 

\end{itemize}
\end{enumerate}
Let $\texttt{K}_{\scriptscriptstyle 6}$ be the time that SC finishes all the above checks.% in step \ref{the-contract-loop}

\end{enumerate} %%%%%%%%%

\

\item[\ref{RCPoRP::CoinTransfer}.] \textbf{Coin Transfer}. \label{RCPoRP::CoinTransfer2} $\mathtt{\mathtt{RCPoRP}}.\mathtt{pay}(\bm{y},T_{\scriptscriptstyle cp},a,p_{\scriptscriptstyle\mathcal S}, coin^{\scriptscriptstyle *}_{\scriptscriptstyle\mathcal C},coin^{\scriptscriptstyle *}_{\scriptscriptstyle\mathcal S})$
\begin{enumerate}


 \item If SC  receives  ``pay'' message  at time $\texttt{T}_{\scriptscriptstyle 2}$, where $a=0$ or $coins^{\scriptscriptstyle *}_{\scriptscriptstyle\mathcal{S}}<p_{\scriptscriptstyle\mathcal{S}}$, then it sends $coin^{\scriptscriptstyle *}_{\scriptscriptstyle\mathcal C}$ coins to $\mathcal C$ and $coin^{\scriptscriptstyle *}_{\scriptscriptstyle\mathcal S}$ coins to $\mathcal S$. Otherwise (i.e. they reach an agreement), they take the following step. 

\item Either $\mathcal C$ or $\mathcal S$ sends ``pay'' message and  statement $\ddot{x}_{\scriptscriptstyle cp}\in T_{\scriptscriptstyle cp}$ to SC at time $\texttt{L}>\texttt{K}_{\scriptscriptstyle 6}$
\item SC checks the validity of the statement by sending it  to SAP contract that  returns  either $1$ or $0$. SC only proceeds to the next step if the output is $1$
\item SC distributes the coins to the parties as follows:

\begin{itemize}
\item[$\bullet$]   $coin^{\scriptscriptstyle *}_{\scriptscriptstyle\mathcal C}-o(z-y_{\scriptscriptstyle\mathcal S})+l(y_{\scriptscriptstyle\mathcal S}-y_{\scriptscriptstyle\mathcal C})$ coins  to $\mathcal C$

\item[$\bullet$] $coin^{\scriptscriptstyle *}_{\scriptscriptstyle\mathcal S}+o(z-y_{\scriptscriptstyle\mathcal S})+l( y_{\scriptscriptstyle\mathcal C}-y_{\scriptscriptstyle\mathcal S})$ coins to $\mathcal S$

 
\end{itemize}

\end{enumerate}
\end{enumerate}




\begin{theorem}
The ${\text{RC-}\overline{\text{PoR}}\text{-P}}$ protocol is secure, w.r.t. Definition \ref{def::RC-S-P-Security-2}, if PoRID, SAP, and blockchain are secure and the encryption scheme is semantically secure. 
\end{theorem}



 To prove the above theorem, we show that ${\text{RC-}\overline{\text{PoR}}\text{-P}}$ meets all security properties defined in Appendix \ref{sec::RC-S-P-Definition2}.  We start by proving that ${\text{RC-}\overline{\text{PoR}}\text{-P}}$ meets security against a malicious server.   The proof to some extent is  simpler to that  of RC-PoR-P against a malicious server (i.e. proof of Lemma \ref{lemma::RC-PoR-P-SecurityAgainstMaliciousServer})  as it does not involve any third-party arbiter. 

 
 % !TEX root =R-zkCSP.tex

 \begin{lemma}
 If SAP and blockchain are secure and  PoRID scheme supports correctness, soundness, and detectable abort, then ${\text{RC-}\overline{\text{PoR}}\text{-P}}$ is secure against malicious server, w.r.t. Definition \ref{deff::RC-S-P-SecurityAgainstMaliciousServer-2}. 
 \end{lemma}
 

 \begin{proof}[sketch]
 First, we  consider event  
 
  $$\Big(F(u^{\scriptscriptstyle *}, \bm{q}_{\scriptscriptstyle j},{pp})= h_{\scriptscriptstyle j}\ \wedge (coin_{\scriptscriptstyle\mathcal{C},j}\neq  \frac{coin_{\scriptscriptstyle\mathcal C}^{\scriptscriptstyle*}}{z}-o)\Big)$$
   that captures the case where the server provides an accepting  proof, i.e. PoR, but makes an honest client withdraw  incorrect amounts of coin, i.e. $coin_{\scriptscriptstyle\mathcal{C},j}\neq  \frac{coin_{\scriptscriptstyle\mathcal C}^{\scriptscriptstyle*}}{z}-o$. Note, in ${\text{RC-}\overline{\text{PoR}}\text{-P}}$ protocol, the total coins the client should receive after $z$ verifications is   $coin^{\scriptscriptstyle *}_{\scriptscriptstyle\mathcal C}-o(z-y_{\scriptscriptstyle\mathcal S})+l(y_{\scriptscriptstyle\mathcal S}-y_{\scriptscriptstyle\mathcal C})$. Since we focus on  $j$-th verification, the amounts of  coin that should be credited to the client for $j$-th verification is
 
 
% 
% $F(u^{\scriptscriptstyle *},\bm{q}_{\scriptscriptstyle j})= h_{\scriptscriptstyle j}\ \wedge coin_{\scriptscriptstyle\mathcal{C},j}\neq  \frac{coin_{\scriptscriptstyle\mathcal C}^{\scriptscriptstyle*}}{z}-o$ that captures the case where the server provides an accepting service proof but makes an honest client withdraw an incorrect amounts of coin, i.e. $coin_{\scriptscriptstyle\mathcal{C},j}\neq  \frac{coin_{\scriptscriptstyle\mathcal C}^{\scriptscriptstyle*}}{z}-o$. Recall, in the protocol, the total coins the client should receive after $z$ verifications is $coin^{\scriptscriptstyle *}_{\scriptscriptstyle\mathcal C}-o(z-y_{\scriptscriptstyle\mathcal S})-l\cdot y_{\scriptscriptstyle\mathcal C}$. Since we focus on  $j$-th verification, the amount of  coins that should be credited to the client for that verification is
  
  \begin{equation}\label{equ::what-client-recives-in-j-2}
   coin_{\scriptscriptstyle\mathcal{C},j}=\frac{coin^{\scriptscriptstyle *}_{\scriptscriptstyle\mathcal C}}{z}-o(1-y_{\scriptscriptstyle\mathcal {S},j})+l(y_{\scriptscriptstyle\mathcal {S},j}-y_{\scriptscriptstyle\mathcal {C},j})
     \end{equation}

    
As the  proof is valid, an honest client accepts it and does not raise any dispute. But, the server  would be able to make the client  withdraw incorrect amounts of coin, if it manages to either convince the contract that the client has misbehaved, by making the contract output  $y_{\scriptscriptstyle\mathcal {C},j}=1$ through dispute resolution phase, or submit to the contract, at the coin transfer phase,  an accepting  statement $\ddot{x}'_{\scriptscriptstyle cp}$ other than what was agreed at the initiation phase, i.e. $\ddot{x}'_{\scriptscriptstyle cp}\neq\ddot{x}_{\scriptscriptstyle cp}$, so it can change the payments' parameters, e.g. $l$ or $o$. Nevertheless, it cannot falsely accuse the client of misbehaviour. As,  due to the security of SAP, it cannot  convince the contract to accept different query's parameters other than what was agreed with the client in the initiation phase. In particular, it cannot persuade the contract to accept  $\ddot{x}'_{\scriptscriptstyle qp}$ such that $\ddot{x}'_{\scriptscriptstyle qp}\neq \ddot{x}_{\scriptscriptstyle qp}$, except with a negligible probability, $\mu(\lambda)$.  Furthermore, if the adversary provides a valid statement then, due to the correctness of PoRID, values  $y_{\scriptscriptstyle\mathcal {C}}$ and $y_{\scriptscriptstyle\mathcal {S}}$ are not incremented by $1$  in $j$-th verification, i.e. $y_{\scriptscriptstyle\mathcal{C},j}=y_{\scriptscriptstyle\mathcal{S},j}=0$. Also,  due to the security of SAP, the server cannot change the payment parameters by persuading the contract to accept any statement $\ddot{x}'_{\scriptscriptstyle cp}$ other than what was agreed initially between the client and server, except with a negligible probability $\mu(\lambda)$.  Therefore, according to Equation \ref{equ::what-client-recives-in-j-2}, the client is credited $\frac{coin_{\scriptscriptstyle\mathcal C}^{\scriptscriptstyle*}}{z}-o$ coins for $j$-th verification, with a high probability.  We now move on to event 
     
    
    
  $$  \Big(F(u^{\scriptscriptstyle *},\bm{q}_{\scriptscriptstyle j}, {pp})\neq h_{\scriptscriptstyle j} \ \wedge (d_{\scriptscriptstyle j}=1	\vee y_{\scriptscriptstyle \mathcal{S},j}=0 \ \vee \  {coin_{\scriptscriptstyle\mathcal{C},j}\neq \frac{coin_{\scriptscriptstyle\mathcal C}^{\scriptscriptstyle*}}{z}+l} ) \Big)$$
    
    
    
     
     It captures the case where the server provides an invalid  proof but either persuades the client to accept the proof, or  persuades the contract to set $y_{\scriptscriptstyle \mathcal{S},j}=0$  or makes the client  withdraw incorrect amounts of coin, i.e.    $coin_{\scriptscriptstyle\mathcal{C},j}\neq \frac{coin_{\scriptscriptstyle\mathcal C}^{\scriptscriptstyle*}}{z}+l$.   Nevertheless, due to the soundness of PoRID, the probability that a corrupt server can convince an honest client to accept invalid proof, i.e. outputs $d_{\scriptscriptstyle j}=1$, is negligible, $\mu(\lambda)$. So, the client detects it with a high probability and raises a dispute.  Also, the server may try to make the contract keep $y_{\scriptscriptstyle \mathcal {S},j}=0$. For $y_{\scriptscriptstyle \mathcal {S},j}=0$ to happen, it has to make the contract  recognise the client as the misbehaving party, i.e. makes the contract output $y_{\scriptscriptstyle \mathcal {C},j}=1$. In this case, the client's complaint would not be processed by the contract; therefore,  $y_{\scriptscriptstyle \mathcal {S},j}$ remains $0$.   Nevertheless, as we discussed above, the probability that the adversary makes the contract  recognise the client as misbehaving is negligible, $\mu(\lambda)$. Therefore, with a high probability  $y_{\scriptscriptstyle \mathcal {S},j}=1$ and $y_{\scriptscriptstyle \mathcal {C},j}=0$, after the contract  is invoked by the client or  server. The adversary may  try to  make the client withdraw  incorrect amounts of coin, e.g. in the case where it does not succeed in convincing the client or contract. To do so,  at the coin transfer phase, it has to send a  different accepting statement  than what was initially agreed with the client. But, it would succeed only with a negligible   probability, $\mu(\lambda)$, due to the security of SAP.   So, according to Equation \ref{equ::what-client-recives-in-j-2}, the client is credited $\frac{coin_{\scriptscriptstyle\mathcal C}^{\scriptscriptstyle*}}{z}+l$ coins for $j$-th verification, with a high probability. Furthermore, in   general, due to the security of SAP, the adversary cannot block an honest client's  messages, ``pay'' and $\ddot{x}_{\scriptscriptstyle cp}$, to the contract in the coin transfer phase.
  \hfill\(\Box\)\end{proof}

%We first consider event $F(u^{\scriptscriptstyle *},\bm{q}_{\scriptscriptstyle j})= h_{\scriptscriptstyle j}\ \wedge coin_{\scriptscriptstyle\mathcal{C},j}\neq  \frac{coin_{\scriptscriptstyle\mathcal C}^{\scriptscriptstyle*}}{z}-o$ that captures the case where the server provides an accepting service proof but makes an honest client withdraw an incorrect amounts of coin, i.e. $coin_{\scriptscriptstyle\mathcal{C},j}\neq  \frac{coin_{\scriptscriptstyle\mathcal C}^{\scriptscriptstyle*}}{z}-o$. Since the proof is valid, an honest client accepts it and does not raise any dispute. 
%  However, the server  would be able to make the client  withdraw incorrect amounts of coins, if it manages to either convince the arbiter that the client has misbehaved (by making the arbiter increment $y_{\scriptscriptstyle\mathcal {C}}$  by $1$ through dispute resolution phase), or submit to the contract, at the coin transfer phase,  an accepting  statement $\ddot{x}'_{\scriptscriptstyle cp}$ other than what was agreed at the initiation phase, i.e. $\ddot{x}_{\scriptscriptstyle cp}$. Nevertheless, it cannot falsely accuse the client of misbehaviour. Because,  due to the security of SAP, it cannot  convince the arbiter to accept different decryption key or pads other than what was agreed with the client in the initiation phase; specifically, it cannot persuade the arbiter to accept  $\ddot{x}'_{\scriptscriptstyle qp}$, where $\ddot{x}'_{\scriptscriptstyle qp}\neq \ddot{x}_{\scriptscriptstyle qp}$, except with a negligible probability, $\mu(\lambda)$. This ensures that the honest client's message is accessed by  the arbiter with a high probability, as the arbiter can extract the client's message using valid pad information and decryption key.  On the other hand, if the adversary provides a valid statement, i.e. $\ddot{x}_{\scriptscriptstyle qp}$, then due to the correctness of VSID, algorithm $\mathtt{VSID.identify}()$ outputs $I_{\scriptscriptstyle j}=\bot$. Therefore, due to the security of SAP and correctness of VSID,   $y_{\scriptscriptstyle\mathcal {C}}$ and $y_{\scriptscriptstyle\mathcal {S}}$ are not incremented by $1$  in $j$-th verification, i.e. $y_{\scriptscriptstyle\mathcal{C}}[j]=y_{\scriptscriptstyle\mathcal{S}}[j]=0$.  Moreover,  due to the security of SAP, the server cannot persuade the contract to accept any statement other than what was agreed initially between the client and server, except with a negligible probability $\mu(\lambda)$ when it finds the hash function's collision. Recall, in the protocol, the total coins the client receives after $z$ verifications is $coin^{\scriptscriptstyle *}_{\scriptscriptstyle\mathcal C}-o(z-y_{\scriptscriptstyle\mathcal S})-l\cdot y_{\scriptscriptstyle\mathcal C}$. Since we focus on each $j$-th verification, the amount of  coins credited to the client for that verification is
%  
%  \begin{equation}\label{equ::what-client-recives-in-j}
%   \frac{coin^{\scriptscriptstyle *}_{\scriptscriptstyle\mathcal C}}{z}-o(1-y_{\scriptscriptstyle\mathcal S}[j])-l\cdot y_{\scriptscriptstyle\mathcal C}[j]
%     \end{equation}
%     
%     In the above event, as we discussed,  with a high probability $y_{\scriptscriptstyle\mathcal{C}}[j]=y_{\scriptscriptstyle\mathcal{S}}[j]=0$ , so the client is credited $\frac{coin_{\scriptscriptstyle\mathcal C}^{\scriptscriptstyle*}}{z}-o$ coins for that verification. 
%     
   

Next, we prove that ${\text{RC-}\overline{\text{PoR}}\text{-P}}$ satisfies security against a malicious client. The proof is also slightly simpler than that  of RC-PoR-P against a malicious client (i.e. proof of Lemma \ref{lemma::RC-PoR-P-SecurityAgainstMaliciousClient}) as it does not involve any third-party arbiter. 


\begin{lemma}
 If SAP and blockchain are secure and  PoRID scheme supports correctness, inputs well-formedness, and detectable abort, then ${\text{RC-}\overline{\text{PoR}}\text{-P}}$ is secure against malicious client, w.r.t. Definition \ref{deff::RC-S-P-Security-Against Malicious-Client-2}. 
\end{lemma}


\begin{proof}[sketch]
  First, we consider event 
  
   $$\Big((M(u^{\scriptscriptstyle *},k,{pp})= \sigma \  \wedge \ Q(\text{aux},k, {pp})= \bm{q}_{\scriptscriptstyle j})\ \wedge \
(coin_{\scriptscriptstyle\mathcal{S},j}\neq  \frac{coin_{\scriptscriptstyle\mathcal S}^{\scriptscriptstyle*}}{z}+o)\Big)$$
  
  
  It captures the case where the client provides  accepting metadata and query  but makes the server withdraw an incorrect amounts of coin, i.e. $coin_{\scriptscriptstyle\mathcal{S},j}\neq  \frac{coin_{\scriptscriptstyle\mathcal S}^{\scriptscriptstyle*}}{z}+o$. According to  ${\text{RC-}\overline{\text{PoR}}\text{-P}}$ protocol, the total coins the server should receive after $z$ verifications is $coin^{\scriptscriptstyle *}_{\scriptscriptstyle\mathcal S}+o(z-y_{\scriptscriptstyle\mathcal S})+l(y_{\scriptscriptstyle\mathcal C}-y_{\scriptscriptstyle\mathcal S})$. As we focus on  $j$-th verification, the amount of  coins that should be credited to the server for $j$-th verification is
  
  \begin{equation}\label{equ::what-server-recives-in-j-2}
   coin_{\scriptscriptstyle\mathcal{S},j}=\frac{coin^{\scriptscriptstyle *}_{\scriptscriptstyle\mathcal S}}{z}+o(1-y_{\scriptscriptstyle\mathcal {S},j})+l(y_{\scriptscriptstyle\mathcal {C},j}-y_{\scriptscriptstyle\mathcal {S},j})
     \end{equation}
  
  
  
  Since the metadata and query are valid, an honest server accepts them and does not raise any dispute, so we have $y_{\scriptscriptstyle\mathcal{C},j}=0$.  The client however could  make the server  withdraw incorrect amounts of coin, if it manages to either convince the contract, in the dispute resolution phase,  that the server has misbehaved, i.e. makes the contract output $y_{\scriptscriptstyle\mathcal{S},j}=1$, or submit to the contract  an accepting  statement $\ddot{x}'_{\scriptscriptstyle cp}$ other than what was agreed at the initiation phase, i.e. $\ddot{x}_{\scriptscriptstyle cp}$, in the coin transfer phase. But, it cannot falsely accuse the server of misbehaviour, because due to the security of SAP, it cannot  convince the contract to accept different decryption key and pads' detail, by providing a different accepting statement $\ddot{x}'_{\scriptscriptstyle qp}$  (where $\ddot{x}'_{\scriptscriptstyle qp}\neq \ddot{x}_{\scriptscriptstyle qp}$), than what was initially agreed with the server, except with a negligible  probability, $\mu(\lambda)$.  So, with a high probability $y_{\scriptscriptstyle\mathcal{S},j}=0$. On the other hand, if the adversary provides a valid statement, i.e. $\ddot{x}_{\scriptscriptstyle qp}$, then due to the correctness of PoRID, algorithm $\mathtt{PoRID.identify}(.)$ outputs $I_{\scriptscriptstyle j}=\bot$. Thus, due to the security of SAP and correctness of PoRID,  we would have $y_{\scriptscriptstyle\mathcal{C},j}=y_{\scriptscriptstyle\mathcal{S},j}=0$ with a high probability.    Also,  due to the security of SAP, the client cannot change the payment parameters by convincing the contract to accept any accepting statement $\ddot{x}'_{\scriptscriptstyle cp}$ other than what was initially agreed  between the client and server (i.e. $\ddot{x}'_{\scriptscriptstyle cp}\neq \ddot{x}_{\scriptscriptstyle cp}$), except with a negligible  probability, $\mu(\lambda)$. That means, according to  Equation \ref{equ::what-server-recives-in-j-2}, the server is credited $\frac{coin_{\scriptscriptstyle\mathcal S}^{\scriptscriptstyle*}}{z}+o$ coins for that verification, with a  high probability.  We now move on  to 
  
  $$\Big(M(u^{\scriptscriptstyle *},k,{pp})\neq \sigma\ \wedge \ a=1\Big)$$
     
     
    It captures the case where the server accepts  ill-formed metadata. But, due to PoRID's inputs well-formedness, the probability the event happens is negligible,  $\mu(\lambda)$. So, with a high probability $a=0$; in this case  the server does not raise any dispute, instead it avoids serving the client.    Next, we turn our attention   to 
     
     $$\Big(Q(\text{aux},k,{pp})\neq \bm{q}_{\scriptscriptstyle j})\ \wedge \ (b_{\scriptscriptstyle j}=1\ \vee \ y_{\scriptscriptstyle\mathcal{C},j}=0 \ \vee \ coin_{\scriptscriptstyle\mathcal{S},j}\neq \frac{coin_{\scriptscriptstyle\mathcal S}^{\scriptscriptstyle*}}{z}+o+l)\Big)$$
     
       It considers the case where the client provides an invalid query, but  either convinces the server or contract to accept it, or  makes the server   withdraw  incorrect amounts of coin, i.e. $coin_{\scriptscriptstyle\mathcal{S},j}\neq \frac{coin_{\scriptscriptstyle\mathcal S}^{\scriptscriptstyle*}}{z}+o+l$. Due to inputs well-formedness of PoRID, the probability that the server outputs $b_{\scriptscriptstyle j}=1$ is negligible, $\mu(\lambda)$.  When the honest server rejects the query and raises a dispute, the contract checks the server's query and sets $y_{\scriptscriptstyle\mathcal{C},j}=1$. After that, due to the security of blockchain the client cannot make the contract to set $y_{\scriptscriptstyle\mathcal{C},j}=0$ except with probability $\mu(\lambda)$. Also, as discussed above, the client cannot make the contract recognise the  honest server as a misbehaving party with a probability significantly greater than $\mu(\lambda)$. That means with a high probability $y_{\scriptscriptstyle\mathcal{S},j}=0$.  The adversary may still try to  make the server withdraw   incorrect amounts of coin (e.g. if the adversary does not succeed in convincing the server). To this end, at the coin transfer phase, it has to convince the contract to accept a  different  statement  than what was initially agreed with the server. However, due to the security of SAP, its  success probability is negligible, $\mu(\lambda)$.     Hence, according to Equation \ref{equ::what-server-recives-in-j-2}, the server is credited  $\frac{coin_{\scriptscriptstyle\mathcal S}^{\scriptscriptstyle*}}{z}+o+l$ coins for $j$-th verification.   Also, due to the security of SAP, the adversary cannot block an honest server's  messages, ``pay'' and $\ddot{x}_{\scriptscriptstyle cp}$, to the contract in the coin transfer phase.
 \hfill\(\Box\)\end{proof}
  
  





In the following, we provide a lemma for  ${\text{RC-}\overline{\text{PoR}}\text{-P}}$'s privacy. For the lemma's  proof, we refer readers to the proof of Lemma \ref{lemma::RC-PoR-P-privacy}. 


\begin{lemma} If SAP is secure and the encryption scheme is semantically secure, then ${\text{RC-}\overline{\text{PoR}}\text{-P}}$ preserves privacy, w.r.t. Definition \ref{deff::RC-S-P-Privacy}. 
\end{lemma}

%
%\begin{proof}[sketch] Briefly, due to SAP's privacy property, given  commitments $g_{\scriptscriptstyle qp}$ and $g_{\scriptscriptstyle cp}$ (stored in the blockchain as a result of running SAP) the adversary learns no information about the committed values (e.g. $o, l, pad_{\scriptscriptstyle\pi}$ and $\bar{k}$), except with negligible probability $\mu(\lambda)$. Moreover, given  price list $pl$, and the parties' encoded coins $coin^{\scriptscriptstyle *}_{\scriptscriptstyle\mathcal{C}}$ and  $coin^{\scriptscriptstyle *}_{\scriptscriptstyle\mathcal{S}}$, the adversary learns nothing about the actual price agreed between the server and client, i.e. $(o,l)$, for each verification, due to Lemma \ref{lemma::encoded-coins}.  Also, since each proof $\bm{\pi}^{\scriptscriptstyle *}_{\scriptscriptstyle j}$ is encrypted and then padded, given $\bm{\pi}^{\scriptscriptstyle *}_{\scriptscriptstyle j}$ the adversary cannot tell  whether $\bm{\pi}^{\scriptscriptstyle *}_{\scriptscriptstyle j}$ is associated with $u_{\scriptscriptstyle 0}$ or with $u_{\scriptscriptstyle 1}$ (i.e. where $u_{\scriptscriptstyle 0}$ and $u_{\scriptscriptstyle 1}$ are the adversary's choice of files), with probability significantly greater than $\frac{1}{2}+\mu(\lambda)$. As each ciphertext $\hat{k}^{\scriptscriptstyle *}_{\scriptscriptstyle j}$ is an output a semantically secure symmetric key encryption scheme, it leaks noting to the adversary.  The value of $a$ is also independent of $u_{\scriptscriptstyle 0}$  or $u_{\scriptscriptstyle 1}$, and only depends on whether the metadata is well-formed, so it leaks nothing about the service input $u_{\scriptscriptstyle\beta}$ and $\beta\in\{0,1\}$. Hence, the adversary cannot tell with a probability significantly greater than $\frac{1}{2}+\mu(\lambda)$ which file of its choice has been used as the server input. 
%
%Furthermore, since each query and padded encrypted proof leak no information and always contain a fixed number of elements, an adversary cannot tell the status of  proof for each $j$-th verification (i.e. whether it is accepted or rejected) with the probability greater than $\frac{1}{2}+\mu(\lambda)$, given  $\hat{k}^{\scriptscriptstyle *}_{\scriptscriptstyle j},coin^{\scriptscriptstyle *}_{\scriptscriptstyle \mathcal S},coin^{\scriptscriptstyle *}_{\scriptscriptstyle\mathcal C}, g_{\scriptscriptstyle cp},  g_{\scriptscriptstyle qp},$  $\bm{\pi}^{\scriptscriptstyle *},pl$, and $a$.
%  \hfill\(\Box\)\end{proof}
%




%\in[ut{RC-PoR-P-Definition}

%% !TEX root =R-zkCSP.tex


\begin{theorem}
The RC-PoR-P protocol is secure, w.r.t. Definition \ref{def::RC-S-P-Security}, if PoRID and SAP are secure and the encryption scheme is semantically secure. 
\end{theorem}


 In order to prove the above theorem, we show that RC-PoR-P meets all the security properties defined in Section \ref{sec::RC-S-P-Definition}.  We first prove that RC-PoR-P satisfies security against a malicious server.  

 \begin{lemma}
 If SAP is secure and  PoRID scheme supports correctness, soundness, and detectable abort, then RC-PoR-P is secure against malicious server, w.r.t. Definition \ref{deff::RC-S-P-SecurityAgainstMaliciousServer}. 
 \end{lemma}
 

 \begin{proof}[sketch]
 We first consider event $F(u^{\scriptscriptstyle *},\bm{q}_{\scriptscriptstyle j})= h_{\scriptscriptstyle j}\ \wedge coin_{\scriptscriptstyle\mathcal{C},j}\neq  \frac{coin_{\scriptscriptstyle\mathcal C}^{\scriptscriptstyle*}}{z}-o$ that captures the case where the server provides an accepting service proof but makes the client withdraw an incorrect amount of coins, i.e. $coin_{\scriptscriptstyle\mathcal{C},j}\neq  \frac{coin_{\scriptscriptstyle\mathcal C}^{\scriptscriptstyle*}}{z}-o$. In this case, since the proof is valid, an honest client accepts it and does not raise any dispute. 
  However, the server  would be able to make the client  withdraw incorrect amounts of coins, if it manages to either convince the arbiter that the client has misbehaved (through dispute resolution phase), or submit to the contract, at the coin transfer phase,  an accepting  statement $\ddot{x}'_{\scriptscriptstyle cp}$ other than what was agreed at the initiation phase, i.e. $\ddot{x}_{\scriptscriptstyle cp}$. Nevertheless, it cannot falsely accuse the client of misbehaviour. Because,  due to the security of SAP, it cannot  convince the arbiter to accept different decryption key or pads other than what was agreed with the client in the initiation phase; specifically, it cannot persuade the arbiter to accept  $\ddot{x}'_{\scriptscriptstyle qp}$, where $\ddot{x}'_{\scriptscriptstyle qp}\neq \ddot{x}_{\scriptscriptstyle qp}$, except with a negligible probability, $\mu(\lambda)$. This ensures that the honest client's message is accessed by  the arbiter with a high probability, as the arbiter can extract the client's message using valid pad information and decryption key. Moreover,  due to the security of SAP, the server cannot persuade the contract to accept any statement other than what was agreed initially between the client and server, except with a negligible probability $\mu(\lambda)$ when it finds the hash function's collision. Also, due to the correctness of VSID, the arbiter always accepts the honest client's accepting proof. 

We now move on to event $F(u^{\scriptscriptstyle *}, \bm{q}_{\scriptscriptstyle j})\neq h_{\scriptscriptstyle j} \ \wedge (d_{\scriptscriptstyle j}=1	\vee y_{\scriptscriptstyle \mathcal S}[j]=0 \ \vee (coin_{\scriptscriptstyle\mathcal{C},j}\neq \frac{coin_{\scriptscriptstyle\mathcal C}^{\scriptscriptstyle*}}{z} \ \vee  coin_{\scriptscriptstyle\mathcal{A}r,j}\neq l))$ which captures the case where the server provides an invalid service proof but either persuades the client to accept the proof, or  (when the client raises a dispute)  persuades the arbiter to accept the proof or makes the client or arbiter withdraw an incorrect amount of coins, i.e.    $coin_{\scriptscriptstyle\mathcal{C},j}\neq \frac{coin_{\scriptscriptstyle\mathcal C}^{\scriptscriptstyle*}}{z}$ or $coin_{\scriptscriptstyle\mathcal{A}r,j}\neq l$ respectively.   Nevertheless, due to the soundness of VSID, the probability that a corrupt server can convince an honest client to accept invalid proof (i.e. outputs $d_{\scriptscriptstyle j}=1$), is negligible, $\mu(\lambda)$. On the other hand, in the case where the client rejects the proof and raises a dispute, the server may try to convince the arbiter and make it output $y_{\scriptscriptstyle \mathcal S}[j]=0$, e.g. by sending a complaint right after the client does. But, for the adversary to win, it has to either provide a different accepting statement $\ddot{x}'_{\scriptscriptstyle qp}$, than what was initially agreed with the client (i.e. $\ddot{x}'_{\scriptscriptstyle qp}\neq \ddot{x}_{\scriptscriptstyle qp}$) and passes the verification, which   requires finding the hash function's collision, and its probability of success is negligible, $\mu(\lambda)$. Or it makes the arbiter  accept an invalid proof, but due to the detectable abort property of VSID, its probability of success is also negligible, $\mu(\lambda)$. In the case where the adversary does not succeed in convincing the client or arbiter, it may still try to  make them withdraw an incorrect amount of coins. To this end, at the coin transfer phase, it has to send a  different accepting statement  than what was initially agreed with the client. But, it would succeed only with a negligible probability, $\mu(\lambda)$, due to the security of SAP. 

Furthermore, in   both  events above, due to the security of SAP, the adversary cannot block an honest client's  messages, ``pay'' and $\ddot{x}_{\scriptscriptstyle cp}$, to the contract in the coin transfer phase.
  \hfill\(\Box\)\end{proof}



\begin{lemma}
 If SAP is secure and  VSID scheme supports correctness, inputs well-formedness, and detectable abort, then RC-S-P is secure against malicious client, w.r.t. Definition \ref{deff::RC-S-P-Security-Against Malicious-Client}. 
\end{lemma}


\begin{proof}[sketch]
  First, we consider event  $(M(u^{\scriptscriptstyle *},k)= \sigma  \wedge Q(\text{aux},k)= \bm{q}_{\scriptscriptstyle j}) \wedge
  (coin_{\scriptscriptstyle\mathcal{S},j}\neq  \frac{coin_{\scriptscriptstyle\mathcal S}^{\scriptscriptstyle*}}{z}+o)$ which captures the case where the client provides  accepting metadata and query  but makes the server withdraw an incorrect amount of coins, i.e. $coin_{\scriptscriptstyle\mathcal{S},j}\neq  \frac{coin_{\scriptscriptstyle\mathcal S}^{\scriptscriptstyle*}}{z}+o$. In short, for the event to  happen with a high probability, a malicious client has to break the security of SAP. In particular, since the metadata and query's proofs are valid, an honest server accepts them and does not raise any dispute.  But, the client may want to  make the server  withdraw incorrect amounts of coins, if it manages to either convince the arbiter, in phase \ref{RCSP::Dispute-Resolution}, that the server has misbehaved, or submit to the contract  an accepting  statement $\ddot{x}'_{\scriptscriptstyle cp}$ other than what was agreed at the initiation phase, i.e. $\ddot{x}_{\scriptscriptstyle cp}$, in phase \ref{RCSP::CoinTransfer}. However, it cannot falsely accuse the server of misbehaviour. As,  due to the security of SAP, it cannot  convince the arbiter to accept different decryption key and pads, by providing a different accepting statement $\ddot{x}'_{\scriptscriptstyle qp}$  (where $\ddot{x}'_{\scriptscriptstyle qp}\neq \ddot{x}_{\scriptscriptstyle qp}$), than what was initially agreed with the server, except with  probability $\mu(\lambda)$. This ensures that  the arbiter is given the honest server's messages (with a high probability). Moreover,  due to the security of SAP, the client cannot convince the contract to accept any accepting statement other than what was initially agreed  between the client and server, except with   probability $\mu(\lambda)$. Furthermore,   the correctness of VSID guarantees that  the arbiter always accepts the honest server's accepting proof. 
  
    We now turn our attention to $(M(u^{\scriptscriptstyle *},k)\neq \sigma \wedge a=1)\ \vee\ (Q(\text{aux},k)\neq \bm{q}_{\scriptscriptstyle j} \wedge b_{\scriptscriptstyle j}=1)$,   that captures the case where the server accepts an ill-formed metadata, or query. However, due to inputs well-formedness of VSID, the probability that either of the events happens is negligible, $\mu(\lambda)$. Next, we move of to      $Q(\text{aux},k)\neq \bm{q}_{\scriptscriptstyle j} \wedge
  (y_{\scriptscriptstyle\mathcal{C}}[j]=0 \ \vee  coin_{\scriptscriptstyle\mathcal{S},j}\neq \frac{coin_{\scriptscriptstyle\mathcal S}^{\scriptscriptstyle*}}{z}+o \ \vee  coin_{\scriptscriptstyle\mathcal{A}r,j}\neq l)$.  It considers the case where the client provides an invalid query, but  either convinces the arbiter to accept it, or  makes the server or arbiter  withdraw an incorrect amount of coins, i.e. $coin_{\scriptscriptstyle\mathcal{S},j}\neq \frac{coin_{\scriptscriptstyle\mathcal S}^{\scriptscriptstyle*}}{z}+o$ or  $coin_{\scriptscriptstyle\mathcal{A}r,j}\neq l$ respectively. Note, when the server rejects the query and raises a dispute, the client may try to convince the arbiter, and make it output $y_{\scriptscriptstyle \mathcal C}[j]=0$, e.g. by sending a complaint right after the server does so. However, for the adversary to win, either it has to provide a different accepting statement $\ddot{x}'_{\scriptscriptstyle qp}$, than what was initially agreed with the server (i.e. $\ddot{x}'_{\scriptscriptstyle qp}\neq \ddot{x}_{\scriptscriptstyle qp}$) and passes the verification. But,  due to the security of SAP, its probability of success is negligible, $\mu(\lambda)$. Or it has to make the arbiter  accept an invalid query, i.e. makes the arbiter output $y_{\scriptscriptstyle\mathcal{C}}[j]=0$. Nevertheless, due to the detectable abort property of VSID, its probability of success is  negligible, $\mu(\lambda)$. If the adversary does not succeed in convincing the server or arbiter, it may still try to  make them withdraw an incorrect amount of coins. To this end, at the coin transfer phase, it has to send a  different accepting statement  than what was initially agreed with the server. However, due to the security of SAP, its  success probability is negligible, $\mu(\lambda)$. Also, due to the security of SAP, the adversary cannot block an honest server's  messages, ``pay'' and $\ddot{x}_{\scriptscriptstyle cp}$, to the contract in the coin transfer phase.
 \hfill\(\Box\)\end{proof}
  
  
  Prior to proving  RC-S-P's privacy, we provide a lemma that will be used in the privacy's proof. Informally, the lemma states that encoded coins leaks no information about the actual amount of coins $(o,l)$, agreed between the client and server. 


\begin{lemma}\label{lemma::encoded-coins} Let $\beta\stackrel{\scriptscriptstyle\$}\leftarrow\{0,1\}$, price list be $\{(o_{\scriptscriptstyle 0},l_{\scriptscriptstyle 0}),(o_{\scriptscriptstyle 1},l_{\scriptscriptstyle 1})\}$, and encoded coin amounts be $coin^{\scriptscriptstyle *}_{\scriptscriptstyle\mathcal{C}}= z\cdot(Max(o_{\scriptscriptstyle\beta},o_{\scriptscriptstyle|\beta-1|})+Max(l_{\scriptscriptstyle\beta},l_{\scriptscriptstyle|\beta-1|}))$ and $coin^{\scriptscriptstyle *}_{\scriptscriptstyle\mathcal{S}}=z\cdot(Max(l_{\scriptscriptstyle\beta}, l_{\scriptscriptstyle |\beta-1|}))$. Then, given the price list, $z$, $coin^{\scriptscriptstyle *}_{\scriptscriptstyle\mathcal{C}}$, and $coin^{\scriptscriptstyle *}_{\scriptscriptstyle\mathcal{S}}$, an adversary $\mathcal{A}$ cannot tell the value of $\beta$ with probability significantly greater than $\frac{1}{2}$ (where the probability is taken over the choice of $\beta$ and the randomness of $\mathcal{A}$).
\end{lemma}



\begin{proof}
As it is evident, the list and $z$ contains no information about $\beta$. Also, since  $z$ is a public value, we could have   $coin'^{\scriptscriptstyle *}_{\scriptscriptstyle\mathcal C}=\frac{coin^{\scriptscriptstyle *}_{\scriptscriptstyle\mathcal C}}{z} = Max(o_{\scriptscriptstyle\beta},o_{\scriptscriptstyle|\beta-1|})+Max(l_{\scriptscriptstyle\beta}, l_{\scriptscriptstyle|\beta-1|})$. It is not hard to see $coin'^{\scriptscriptstyle *}_{\scriptscriptstyle\mathcal C}$ is a function of maximum value of $(o_{\scriptscriptstyle 0},o_{\scriptscriptstyle 1})$, and maximum value of $(l_{\scriptscriptstyle 0},l_{\scriptscriptstyle 1})$. It is also  independent of $\beta$. Therefore, given the list, $z$ and $coin'^{\scriptscriptstyle *}_{\scriptscriptstyle\mathcal C}$ the adversary learns nothing about $\beta$, unless it guesses the value,  with  success probability $\frac{1}{2}$. The same also holds for $coin^{\scriptscriptstyle *}_{\scriptscriptstyle\mathcal S}$. 
 \hfill\(\Box\)\end{proof}


\begin{lemma} If SAP is secure and the encryption scheme is semantically secure, then RC-S-P preserves privacy, w.r.t. Definition \ref{deff::RC-S-P-Privacy}. 
\end{lemma}

\begin{proof}[sketch]
Due to the privacy property of SAP, that stems from the  hiding property of the commitment scheme, given the commitments $g_{\scriptscriptstyle qp}$ and $g_{\scriptscriptstyle cp}$, (that are stored in the blockchain as result of running SAP) the adversary learns no information about the committed values (e.g. $o, l, pad_{\scriptscriptstyle\pi}, pad_{\scriptscriptstyle q},$ and $\bar{k}$), except with negligible probability, $\mu_{\scriptscriptstyle 1}(\lambda)$.   Also, given encoded coins $coin^{\scriptscriptstyle *}_{\scriptscriptstyle\mathcal{C}}=z\cdot (o_{\scriptscriptstyle max}+l_{\scriptscriptstyle max})$ and  $coin^{\scriptscriptstyle *}_{\scriptscriptstyle\mathcal{S}}=z\cdot l_{\scriptscriptstyle max}$, the adversary learns nothing about the actual price agreed between the server and client,  $(o,l)$, for each verification, due to Lemma \ref{lemma::encoded-coins}. Next we analyse the privacy of padded encrypted query vector $\bm{c}^{\scriptscriptstyle *}$. For  the sake of simplicity, we focus on   $\bm{q}^{\scriptscriptstyle *}_{\scriptscriptstyle j}\in c^{\scriptscriptstyle *}_{\scriptscriptstyle j}\in\bm{c}^{\scriptscriptstyle *}$, that is a padded encrypted query vector for $j$-th verification. Let $\bm{q}_{\scriptscriptstyle j,0}$ and $\bm{q}_{\scriptscriptstyle j,1}$  be query vectors, for $j$-th verification, related to the service inputs $u_{\scriptscriptstyle 0}$ and $u_{\scriptscriptstyle 1}$ that    are picked by the adversary according to  Definition  \ref{deff::RC-S-P-Privacy} which lets  the environment pick $\beta\stackrel{\scriptscriptstyle\$}\leftarrow \{0,1\}$. Also, let $\{\bm{q}_{\scriptscriptstyle j,0},...,\bm{q}_{\scriptscriptstyle j,m}\}$ be a  list of all queries of different sizes. In the experiment, if $\bm{q}_{\scriptscriptstyle j,\beta}$  is only   encrypted (but not padded), then given the ciphertext, due to semantical security of the encryption, an adversary cannot tell if the ciphertext corresponds to $\bm{q}_{\scriptscriptstyle j,0}$ or $\bm{q}_{\scriptscriptstyle j,1}$  (accordingly to $u_{\scriptscriptstyle 0}$ or $u_{\scriptscriptstyle 1}$) with probability greater than $\frac{1}{2}+\mu_{\scriptscriptstyle 1}(\lambda)$,  under the assumption  that $Max(|\bm{q}_{\scriptscriptstyle j,0}|,...,|\bm{q}_{\scriptscriptstyle j,m}|)=|\bm{q}_{\scriptscriptstyle j,\beta}|$. The  assumption is  relaxed with the use of a pad; as each encrypted query is padded  to the queries' maximum size, i.e. $Max(|\bm{q}_{\scriptscriptstyle j,0}|,...,|\bm{q}_{\scriptscriptstyle j,m}|)$, the adversary cannot tell with probability greater than $\frac{1}{2}+\mu_{\scriptscriptstyle 1}(\lambda)$ if the  padded encrypted proof corresponds to  $\bm{q}_{\scriptscriptstyle j,0}$ or $\bm{q}_{\scriptscriptstyle j,1}$, as the padded encrypted query \emph{always has the same size} and the pad values are picked from the same range as the encryption's ciphertext are defined. The same argument holds for $\bm{w}^{\scriptscriptstyle *}_{\scriptscriptstyle q_{_{\scaleto{j}{\size}}}}\in c^{\scriptscriptstyle *}_{\scriptscriptstyle j}\in\bm{c}^{\scriptscriptstyle *}$. 

Next we analyse the privacy of padded encrypted proof vector $\bm{\pi}^{\scriptscriptstyle *}$. The argument is similar to the one presented above; however, for  the sake of completeness we provide it.  Again, we focus on an element of the vector,  $\pi_{\scriptscriptstyle j}^{\scriptscriptstyle *}\in\bm{\pi}^{\scriptscriptstyle *}$, that is a padded encrypted proof for $j$-th verification. Let $\pi_{\scriptscriptstyle j,0}$ and $\pi_{\scriptscriptstyle j,1}$  be proofs, for $j$-th verification, related to the service inputs $u_{\scriptscriptstyle 0}$ and $u_{\scriptscriptstyle 1}$, where the inputs   are picked by the adversary, w.r.t. Definition  \ref{deff::RC-S-P-Privacy} in which  the environment picks $\beta\stackrel{\scriptscriptstyle\$}\leftarrow \{0,1\}$.  Let $\{\pi_{\scriptscriptstyle j,0},...,\pi_{\scriptscriptstyle j,m}\}$ be proof list including all proofs of different sizes. In the experiment, if $\pi_{\scriptscriptstyle j,\beta}$  is only   encrypted, then given the ciphertext, due to semantical security of the encryption, an adversary cannot tell if the ciphertext corresponds to $\pi_{\scriptscriptstyle j,0}$ or $\pi_{\scriptscriptstyle j,1}$  (accordingly to $u_{\scriptscriptstyle 0}$ or $u_{\scriptscriptstyle 1}$) with probability greater than $\frac{1}{2}+\mu_{\scriptscriptstyle 2}(\lambda)$,  if $Max(|\pi_{\scriptscriptstyle j,0}|,...,|\pi_{\scriptscriptstyle j,m}|)=|\pi_{\scriptscriptstyle j,\beta}|$. However, the  assumption is  relaxed with the use of a pad. In particular, since each encrypted proof is padded  to the proofs' maximum size, the adversary cannot tell with probability greater than $\frac{1}{2}+\mu_{\scriptscriptstyle 2}(\lambda)$ if the  padded encrypted proof corresponds to  $\pi_{\scriptscriptstyle j,0}$ or $\pi_{\scriptscriptstyle j,1}$. Also, since the value of $a$ is independent of $u_{\scriptscriptstyle 0}$  or $u_{\scriptscriptstyle 1}$, and only depends on whether the metadata is well-formed, it leaks nothing about the service input $u_{\scriptscriptstyle\beta}$ and $\beta$. Moreover, since each padded encrypted query and proof leak no information and always contains a fixed number of elements, an adversary cannot tell the status of a proof for each $j$-th verification (i.e. whether it is accepted or rejected) with the probability greater than $\frac{1}{2}+\mu_{\scriptscriptstyle 2}(\lambda)$, given  $\bm{c}^{\scriptscriptstyle *},coin^{\scriptscriptstyle *}_{\scriptscriptstyle \mathcal S},coin^{\scriptscriptstyle *}_{\scriptscriptstyle\mathcal C}, g_{\scriptscriptstyle cp},  g_{\scriptscriptstyle qp},$  $\bm{\pi}^{\scriptscriptstyle *}$, and $a$.
  \hfill\(\Box\)\end{proof}






