% !TEX root =R-zkCSP.tex


\subsection{Recurring Contingent Service Payment (RC-S-P) Protocol}


%In this section, we present a generic ``recurring contingent service payment'' (R-CSP) protocol. It utilises a novel combination of $\text{VSID}_{\scriptscriptstyle\text{light}}$, SAP, and symmetric key encryption schemes along with the coin masking technique and the private time bubble notion. At a high level the protocol works as follows.  The  client and server use SAP to provably agree on (a) a secret key, $k$, for a symmetric key encryption and (b) a statement that includes payment details. They also  agree on (a) the private time bubble's length, that is the total number of billing cycles: $z$, plus a waiting period, $H$, and (b) a smart contract which  specifies $z$ and  the total amount of masked coins each party should deposit. They deploy the contract.  Each party deposits its masked coins in the contract. If the deposit amount is less than what is stated in the contract, each party has a chance to withdraw its coins and terminate the contract. To start using/providing the service, they invoke $\text{VSID}_{\scriptscriptstyle\text{light}}$ protocol. In particular, they engage in   $\mathtt{VSID.KeyGen}()$, $\mathtt{VSID.setup}()$, and $\mathtt{VSID.serve}()$ algorithms. If the server decides not to serve, e.g. it detects the client's misbehaviour,  it sends $0$  within a fixed time; in this case, the parties can withdraw their deposit and terminate the contract. Otherwise, the server sends $1$ to the contract. 
 
 
 In this section, we present  ``recurring contingent service payment'' (RC-S-P) protocol for a generic service. It utilises a novel combination of $\text{VSID}_{\scriptscriptstyle\text{light}}$, SAP, the private time bubble notion, and symmetric key encryption schemes along with the coin masking and padding techniques. At a high level the protocol works as follows.  The  client and server use SAP to provably agree on two private statements; first statement includes  payment details, while another one specifies a secret key, $k$, and a pad's length. They also  agree on  public parameters such as (a) the private time bubble's length, that is the total number of billing cycles: $z$, plus a waiting period, $H$, and (b) a smart contract which  specifies $z$ and  the total amount of masked coins each party should deposit. They deploy the contract.  Each party deposits its masked coins in the contract. If either party does not deposit enough coins on time,   later each party has a chance to withdraw its coins and terminate the contract. To start using/providing the service, they invoke $\text{VSID}_{\scriptscriptstyle\text{light}}$ protocol. In particular, they engage in   $\mathtt{VSID.keyGen}()$, $\mathtt{VSID.setup}()$, and $\mathtt{VSID.serve}()$ algorithms. If the server decides not to serve, e.g. it detects the client's misbehaviour,  it sends $0$  within a fixed time; in this case, the parties can withdraw their deposit and terminate the contract. Otherwise, the server sends $1$ to the contract. 
 
 
 
At the end of each billing cycle, the client generates an encrypted query, by calling $\mathtt{VSID.genQuery}()$ and encrypting its output using the key, $k$. It pads the encrypted query and  sends the result  to the contract. The encryption and pads ensure  nothing about the client's input (e.g. outsourced file) is revealed to the public within the private time bubble.  In the same cycle, the server retrieves the query, removes the pads and decrypts the result. Then, it locally checks its validity, by calling $\mathtt{VSID.checkQuery}()$. If the query is rejected, the server locally stores the index of the billing cycle and then  generates a dummy proof.  Otherwise, if the server accepts the query, it generates a proof of service by calling $\mathtt{VSID.prove}()$. In either case, the server encrypts the proof, pads it and sends the result to the contract. Note that sending (padded encrypted) dummy proof ensures that the public, during the private time bubble, does not learn if the client generates invalid queries. 

 After the server sends the messages to the contract, the client removes the pads, decrypts the proof and locally verifies it, by calling $\mathtt{VSID.verify}()$. If the verification is passed, then the client knows the server has delivered the service honestly. However, if the proof is rejected,  it waits until the private time bubble passes and dispute resolution time arrives. During the dispute resolution period, in the case the client or server  rejects any proofs, it sends a ``dispute'' message to the contract. The party also invokes the arbiter,  refers it to the invalid encrypted proofs in the contract, and  sends to it the decryption key and the pads' detail. The arbiter checks the validity of the key and pads, by using SAP. If they are accepted,  then the arbiter locally removes the pads from the encrypted proofs,   decrypts the related proofs, and runs $\mathtt{VSID.identify}()$ to check the validity of the party's claim. The arbiter sends to the contract a report of its findings that includes  the total number of times  the server and client provided invalid proofs. In the next phase, to distribute the coins, either client or server  sends: (a) ``pay'' message, (b) the agreed statement that specifies the payment details, and (c) the statement's proof to the contract which verifies the statement and if approved it distributes the coins according to the statement's detail, and the arbiter's report. 

Now we outline why RC-S-P addresses the issues.    In the setup, if the client provides ill-formed inputs (so later it can  accuse the server) then the server can detect and avoid serving it.  After the setup, if the client avoids sending any input,  then the server still gets paid for the service it provided. Also, in the case of a dispute between the parties, their claim is checked, and  the corrupt party is identified. The corrupt party has to pay the arbiter and if that is the client, then it has to pay the server as well.  These  features not only do guarantee the server's resource is not wasted, but also ensures fairness (i.e.  if a potentially malicious server is paid, then it must have provided the service and if a potentially malicious client does not pay, then it will learn nothing). Furthermore, as during the private time bubble  (a) no plaintext proof is given to the contract, and (b) no dispute resolution and coin transfer take place on contract, the public cannot figure out the outcome of each verification. This  preserves the server's privacy. Also, because the  deposited coins are masked and the agreed statement is kept private,   nothing about the detail of the service  is leaked to the public before the bubble bursts. This preserves the client's privacy. Moreover, as either party can prove to the contract the validity of the agreed statement, and ask the contract to distribute the coins, the coins will be not be locked forever. The RC-S-P protocol is presented below. It is assumed that (a) each party $\mathcal{R} \in\{\mathcal{C,S,A}r\}$ already has a blockchain public address which is known to all parties, (b) it uses that (authorised) address to send transactions to the smart contract, and (c) the contract   before recording a transaction, ensures the transaction is  originated from an authorised address.

\begin{enumerate}
\item \textbf{Key Generation}. $\mathtt{RCSP}.\mathtt{keyGen}(1^{\scriptscriptstyle\lambda},F)$
\begin{enumerate}

\item {\color{purple}Aydin---I'm modifying/improving the RC-S-P protocol.}

\item  $\mathcal C$ runs  $\mathtt{VSID.keyGen}(1^{\lambda},F)\rightarrow k:=(sk,pk)$. It picks a random secret key  $\bar{k}$ for a symmetric key encryption. Also, it sets two parameters: $pad_{\scriptscriptstyle \pi}$ and $pad_{\scriptscriptstyle q}$, where  $pad_{\scriptscriptstyle \pi}$ and $pad_{\scriptscriptstyle q}$  refer to the number of  dummy values that will be used to pad encrypted proofs and encrypted queries respectively \footnote{\scriptsize The values of  $pad_{\scriptscriptstyle \pi}$ and $pad_{\scriptscriptstyle q}$ is determined as follows, $pad_{\scriptscriptstyle \pi}=\pi_{\scriptscriptstyle max}-\pi_{\scriptscriptstyle act}$ and $pad_{\scriptscriptstyle q}=q_{\scriptscriptstyle max}-q_{\scriptscriptstyle act}$, where $\pi_{\scriptscriptstyle max}$ and $\pi_{\scriptscriptstyle act}$ refer to the maximum and actual proof size  while $q_{\scriptscriptstyle max}$ and $q_{\scriptscriptstyle act}$ refer to the maximum and actual query size, respectively.}.  Let $qp:=(pad_{\scriptscriptstyle \pi},pad_{\scriptscriptstyle q}, \bar{k})$. The keys' size is part of the security parameter. 


\item  $\mathcal C$ sets coin parameters as   follows,   $o$: the amount of coins for each accepting proof, and $l$: the amount of coins  to cover the cost of each potential dispute resolution.    Let $k':\{o,l,qp\}$ and  let $\bm{k}=[k,k']$. 
\end{enumerate}

\

\item \textbf{Client-side Initiation}. $\mathtt{RCSP}.\mathtt{cInit}(1^\lambda, u,\bm{k}, M,z,cp,enc)$
\begin{enumerate}

\item Calls  $\mathtt{VSID.setup}(1^{\lambda}, u,k,M)\rightarrow (u^{\scriptscriptstyle *},e)$, to encode  service input, and generate metadata. 


\item  Calls $\mathtt{SAP.init}(1^{\scriptscriptstyle\lambda}, adr_{\scriptscriptstyle\mathcal{C}}, adr_{\scriptscriptstyle\mathcal{S}},qp )\rightarrow(r_{\scriptscriptstyle qp}, g_{\scriptscriptstyle qp}, adr_{\scriptscriptstyle\text{SAP}_{\scriptscriptstyle 1}})$, to initiate an agreement (with $\mathcal{S}$) on $qp$.  Let $T_{\scriptscriptstyle qp}:=(\ddot{x}_{\scriptscriptstyle qp}, g_{\scriptscriptstyle qp})$ be proof/query encoding token,  where  $\ddot{x}_{\scriptscriptstyle qp}:=(qp,r_{\scriptscriptstyle qp})$ is the opening and $g_{\scriptscriptstyle qp}$ is the commitment stored on the contract as a result of running   SAP. 


\item  Sets  $cp:=(o,o_{\scriptscriptstyle max},l,l_{\scriptscriptstyle max},z)$, where $o_{\scriptscriptstyle max}$ is the maximum amount of coins  for an accepting service proof,  $l_{\scriptscriptstyle max}$  is the maximum amount of coins to resolve a potential dispute, and $z$ is the number of  service proofs/verifications. Then, $\mathcal C$ calls $\mathtt{SAP.init}(1^{\scriptscriptstyle\lambda}, adr_{\scriptscriptstyle\mathcal{C}}, adr_{\scriptscriptstyle\mathcal{S}},cp )\rightarrow(r_{\scriptscriptstyle cp},g_{\scriptscriptstyle cp},adr_{\scriptscriptstyle\text{SAP}_{\scriptscriptstyle 2}})$, to initiate an agreement (with $\mathcal{S}$) on $cp$. Let $T_{\scriptscriptstyle cp}:=(\ddot{x}_{\scriptscriptstyle cp},g_{\scriptscriptstyle cp})$ be coin encoding token,  where  $\ddot{x}_{\scriptscriptstyle cp}:=(cp,r_{\scriptscriptstyle cp})$ is the opening and $g_{\scriptscriptstyle cp}$ is the commitment stored on the contract as a result of running the  SAP.  

\item Set parameters  $coin^{\scriptscriptstyle *}_{\scriptscriptstyle\mathcal C}=z\cdot (o_{\scriptscriptstyle max}+l_{\scriptscriptstyle max})$ and $coin^{\scriptscriptstyle *}_{\scriptscriptstyle\mathcal S}=z\cdot l_{\scriptscriptstyle max}$, where $coin^{\scriptscriptstyle *}_{\scriptscriptstyle\mathcal C}$ and $coin^{\scriptscriptstyle *}_{\scriptscriptstyle\mathcal S}$ are   the total number of masked coins $\mathcal C$ and $\mathcal S$ should deposit respectively. $\mathcal C$ signs and deploys a smart contract, SC, that explicitly specifies  parameters $z$,  $coin^{\scriptscriptstyle *}_{\scriptscriptstyle\mathcal C}$ and $coin^{\scriptscriptstyle *}_{\scriptscriptstyle\mathcal S}$. Let $adr_{\scriptscriptstyle \text{SC}}$ be the address of the deployed SC, and $T:=\{T_{\scriptscriptstyle qp},T_{\scriptscriptstyle cp}\}$. 

\item Sets four counters $\bm{y}=[y_{\scriptscriptstyle\mathcal C},y'_{\scriptscriptstyle\mathcal C},y_{\scriptscriptstyle\mathcal S}, y'_{\scriptscriptstyle\mathcal S}]$ in the contract, where their initial value is $0$. 

\item Deposits $coin^{\scriptscriptstyle *}_{\scriptscriptstyle\mathcal C}$ coins in the contract.

\item Sends $u^{\scriptscriptstyle *}, e, \ddot{x}_{\scriptscriptstyle qp},\ddot{x}_{\scriptscriptstyle cp}$, and $coin^{\scriptscriptstyle *}_{\scriptscriptstyle\mathcal S}$ (along with $adr_{\scriptscriptstyle \text{SC}}, adr_{\scriptscriptstyle\text{SAP}_{\scriptscriptstyle 1}},adr_{\scriptscriptstyle\text{SAP}_{\scriptscriptstyle 2}}$) to $\mathcal S$. It sends  the public key's encryption: $pk^{\scriptscriptstyle *}=\mathtt{Enc}(\bar{k},pk)$ to SC. Let the above process finish at time $\texttt{T}_{\scriptscriptstyle 0}$. 





%\item For $\mathcal C$ and $\mathcal S$ to provably agree on $qp$, $\mathcal C$ sends $qp$ to  $\mathcal S$.  Next (if the $\mathcal S$ agrees on the parameters) they take the steps in the Setup and Agreement phases in the SAP, at time $T_{\scriptscriptstyle 0}$. Let $t_{\scriptscriptstyle qp}:=(\ddot{x}_{\scriptscriptstyle qp},g_{\scriptscriptstyle qp})$ be proof/query encoding token,  where  $\ddot{x}_{\scriptscriptstyle qp}$ is the opening and $g_{\scriptscriptstyle qp}$ is the commitment stored on the contract as a result of running   SAP. 
%
%
%\item Let  $cp:=(o,o_{\scriptscriptstyle max},l,l_{\scriptscriptstyle max},z)$, where $o_{\scriptscriptstyle max}$ is the maximum amount of coins  for an accepting service proof,  $l_{\scriptscriptstyle max}$  is the maximum amount of coins to resolve a potential dispute, and $z$ is the number of  service proofs/verifications. For $\mathcal C$ and $\mathcal S$ to provably agree on $cp$, similar to the previous step, they invoke SAP again, at time $T_{\scriptscriptstyle 1}$.  Let $t_{\scriptscriptstyle cp}:=(\ddot{x}_{\scriptscriptstyle cp},g_{\scriptscriptstyle cp})$ be coin encoding token,  where  $\ddot{x}_{\scriptscriptstyle cp}$ is the opening and $g_{\scriptscriptstyle cp}$ the commitment stored on the contract as a result of running the  SAP. 
%
%
%
%\item $\mathcal C$ set parameters  $coin^{\scriptscriptstyle *}_{\scriptscriptstyle\mathcal C}=z\cdot (o_{\scriptscriptstyle max}+l_{\scriptscriptstyle max})$ and $coin^{\scriptscriptstyle *}_{\scriptscriptstyle\mathcal S}=z\cdot l_{\scriptscriptstyle max}$, where $coin^{\scriptscriptstyle *}_{\scriptscriptstyle\mathcal C}$ and $coin^{\scriptscriptstyle *}_{\scriptscriptstyle\mathcal S}$ are   the total number of masked coins $\mathcal C$ and $\mathcal S$ should deposit respectively. $\mathcal C$ signs and deploys  smart contract SC that explicitly specifies  parameters $z$,  $coin^{\scriptscriptstyle *}_{\scriptscriptstyle\mathcal C}$ and $coin^{\scriptscriptstyle *}_{\scriptscriptstyle\mathcal S}$. Moreover, it sets four counters $\bm{y}=[y_{\scriptscriptstyle\mathcal C},y'_{\scriptscriptstyle\mathcal C},y_{\scriptscriptstyle\mathcal S}, y'_{\scriptscriptstyle\mathcal S}]$ in the contract, where their initial value is $0$. It deposits $coin^{\scriptscriptstyle *}_{\scriptscriptstyle\mathcal C}$ coins in the contract.

% at time $T_{\scriptscriptstyle 2}$. $\mathcal S$ signs the contract too, if it agrees on the parameters.  

 

%\item $\mathcal C$ runs  $\mathtt{VSID.setup}(1^{\lambda}, u,k,M)\rightarrow (u^{\scriptscriptstyle *},e)$.  It sends  $e$ and $u^{\scriptscriptstyle *}$ to $\mathcal S$ and sends  the public key's encryption: $\mathtt{Enc}(\bar{k},pk)$ to SC at time $T_{\scriptscriptstyle 2}$.





%$\mathcal C$ and $\mathcal S$ respectively construct  vectors $\vv{\bm{v}}_{\scriptscriptstyle\mathcal C}$ and $\vv{\bm{v}}_{\scriptscriptstyle\mathcal S}$,  initially empty. 
%
%\item $\mathcal C$ and $\mathcal S$ agree on the following parameters,  $z$: the number of the service proofs/verifications, $o$: the amount of coins for each accepting proof, and $l$: the amount of coins  to cover the cost of each potential dispute resolution.  Also, they agree on $f_{\scriptscriptstyle\mathcal C}$ and $f_{\scriptscriptstyle\mathcal S}$ that are the amount of masking coins the client and server will use  respectively. 








%\item $\mathcal C$ constructs  vector $\vv{\bm{v}}_{\scriptscriptstyle\mathcal C}$, also $\mathcal S$  constructs $\vv{\bm{v}}_{\scriptscriptstyle\mathcal S}$,  where the vectors are initially empty. 
%
%%$\mathcal C$ and $\mathcal S$ respectively construct  vectors $\vv{\bm{v}}_{\scriptscriptstyle\mathcal C}$ and $\vv{\bm{v}}_{\scriptscriptstyle\mathcal S}$,  initially empty. 
%
%\item $\mathcal C$ and $\mathcal S$ agree on the following parameters,  $z$: the number of the service proofs/verifications, $o$: the amount of coins for each accepting proof, and $l$: the amount of coins  to cover the cost of each potential dispute resolution.  Also, they agree on $f_{\scriptscriptstyle\mathcal C}$ and $f_{\scriptscriptstyle\mathcal S}$ that are the amount of masking coins the client and server will use  respectively. 
%
%\item $\mathcal C$ and $\mathcal S$ construct  a statement of the  form: $x=(o,l,  f_{\scriptscriptstyle\mathcal C},f_{\scriptscriptstyle\mathcal S},z)$   
%\item For $\mathcal C$ and $\mathcal S$ to provably agree on $x$, they take the steps in the Setup and Agreement phases in the SAP, at time $T_{\scriptscriptstyle 0}$. Let $\mu$ be the statement's proof. 
%
%\item  $\mathcal C$ sends to $\mathcal S$ a statement of the form $x'=({pad}_{\scriptscriptstyle 1},{pad}_{\scriptscriptstyle 2},\bar{k})$ where  ${pad}_{\scriptscriptstyle 1}$ and ${pad}_{\scriptscriptstyle 2}$  refer to the number of  dummy values that will be used to pad encrypted queries and encrypted proofs respectively; and    $\bar{k}$ is a  secret key.
%
%
%
%\item For $\mathcal C$ and $\mathcal S$ to provably agree on $x'$, they take the steps in the Setup and Agreement phases in the SAP, at time $T_{\scriptscriptstyle 1}$. Let $\mu'$ be the proof. 
%
%
%\item $\mathcal C$ and $\mathcal S$ agree on a smart contract: SC. Let $p_{\scriptscriptstyle\mathcal C}=z(o+l)+f_{\scriptscriptstyle\mathcal C}$ and $p_{\scriptscriptstyle\mathcal S}=zl+f_{\scriptscriptstyle\mathcal S}$ be   the total number of coins the client and server should deposit respectively. SC  explicitly specifies  parameters $z$,  $p_{\scriptscriptstyle\mathcal C}$ and $p_{\scriptscriptstyle\mathcal S}$. They sign and deploy the contract. 


% that specifies the total number of the service's proofs: $z$ and total amount of masked coins each party should deposit, i.e. the client and server should deposit $p_{\scriptscriptstyle\mathcal C}=z(o+l)+f_{\scriptscriptstyle\mathcal C}$ and $p_{\scriptscriptstyle\mathcal S}=zl+f_{\scriptscriptstyle\mathcal S}$ coins, respectively. They sign and deploy the contract. 

%\item $\mathcal C$ and $\mathcal S$ deposit their masked coins in SC at time $T_{\scriptscriptstyle 2}$
%\item At time $T_{\scriptscriptstyle 3}$, if the deposit amount that either party has put is less than the amount stated in SC, then SC allows each party to withdraw its coins and terminate the contract.

\end{enumerate}

\
\item \textbf{Server-side Initiation}. $\mathtt{RCSP}.\mathtt{sInit}(u^{\scriptscriptstyle *},e, pk^{\scriptscriptstyle *},z, T, coin^{\scriptscriptstyle *}_{\scriptscriptstyle\mathcal S} ,\bm{y},enc)$

\begin{enumerate}

\item Checks the parameters  in $T$ (e.g. $qp$ and $cp$) and  in SC (e.g. $coin^{\scriptscriptstyle *}_{\scriptscriptstyle\mathcal S}, \bm{y}$) and ensures sufficient amount of coins has been deposited by $\mathcal C$.  

\item Calls $\mathtt{SAP.agree}(qp,r_{\scriptscriptstyle qp},g_{\scriptscriptstyle qp},adr_{\scriptscriptstyle\mathcal{C}},adr_{\scriptscriptstyle\text{SAP}_{1}})\rightarrow (g'_{\scriptscriptstyle qp},b_{\scriptscriptstyle 1})$ and $\mathtt{SAP.agree}(cp,r_{\scriptscriptstyle cp},g_{\scriptscriptstyle cp},adr_{\scriptscriptstyle\mathcal{C}},adr_{\scriptscriptstyle\text{SAP}_{2}})\rightarrow (g'_{\scriptscriptstyle cp},b_{\scriptscriptstyle 2})$, to verify the correctness of tokens in $T$ and to agree on the tokens' parameters, where $qp,r_{\scriptscriptstyle qp}\in \ddot{x}_{\scriptscriptstyle qp}, \text{and }  cp,r_{\scriptscriptstyle cp} \in  \ddot{x}_{\scriptscriptstyle cp}$. Recall that if both $\mathcal{C}$ and $\mathcal{S}$ are honest, then $g_{\scriptscriptstyle qp}=g'_{\scriptscriptstyle qp}$ and $g_{\scriptscriptstyle cp}=g'_{\scriptscriptstyle cp}$. 




\item If any above check is rejected, then it sets $a=0$. Otherwise, it decrypts the public key, $\mathtt{Dec}(\bar{k},pk^{\scriptscriptstyle *})=pk$; and then runs $\mathtt{VSID.serve}(u^{\scriptscriptstyle *},e,pk)\rightarrow a$. 

\item Sends $a$ and $coin^{\scriptscriptstyle *}_{\scriptscriptstyle\mathcal S}$ coins to SC at time $\texttt{T}_{\scriptscriptstyle 1}$, where  $coin^{\scriptscriptstyle *}_{\scriptscriptstyle\mathcal S}=\bot$ if $a=0$
\end{enumerate}
Note, $\mathcal S$  and  $\mathcal C$ can withdraw their coins at time $\texttt{T}_{\scriptscriptstyle 2}$, if  $\mathcal S$ sends $a=0$, fewer coins than $coin^{\scriptscriptstyle *}_{\scriptscriptstyle\mathcal S}$, or nothing to the SC. 

\

\textbf{\textit{Start of Billing-cycles}}. $\mathcal{C}$ and $\mathcal{S}$ engage in the following three phases, i.e. phase \ref{Billing-cycles-genQuery}-\ref{Client-sideProofVerification},  at the end of  every $j$-th billing cycle, where $1\leq j\leq z$. Each $j$-th cycle includes two  time points, $\texttt{G}_{\scriptscriptstyle j,1}$ and $\texttt{G}_{\scriptscriptstyle j,2}$, where $\texttt{G}_{\scriptscriptstyle j,2}>\texttt{G}_{\scriptscriptstyle j,1}$, and $\texttt{G}_{\scriptscriptstyle 1,1}>\texttt{T}_{\scriptscriptstyle 2}$ 

%At each time $G_{\scriptscriptstyle j}$, (where $1\leq j\leq z$ and $G_{\scriptscriptstyle 1}>T_{\scriptscriptstyle 5}$), that a proof is generated, the parties perform as follows. 
%\item \textbf{Billing-cycles}. At each time $G_{\scriptscriptstyle j}$, (where $1\leq j\leq z$ and $G_{\scriptscriptstyle 1}>T_{\scriptscriptstyle 5}$), that a proof is generated, the parties perform as follows. 



\

\item\textbf{Client-side Query Generation}.\label{Billing-cycles-genQuery} $\mathtt{RCSP}.\mathtt{genQuery}(1^\lambda, \text{aux},k,Q, T_{\scriptscriptstyle qp}, enc)$

\begin{enumerate}
\item Calls $\mathtt{VSID.genQuery}(1^{\lambda},  \text{aux},k,Q)\rightarrow c_{\scriptscriptstyle j}:=(\bm{q}_{\scriptscriptstyle j},\bm{w}_{\scriptscriptstyle q_{_{\scaleto{j}{\size}}}})$, to generate a query-proof pair. 
\item Encrypts the pair, $\mathtt{Enc}(\bar{k},c_{\scriptscriptstyle j})=c'_{\scriptscriptstyle j}$. Then, it pads (each element of) the result with ${pad}_{\scriptscriptstyle q}\in T_{\scriptscriptstyle qp}$ random values that are picked from the encryption's output range, $U$. Let $c^{\scriptscriptstyle *}_{\scriptscriptstyle j}$ be the result. 
\item Sends the padded encrypted query-proof pair, $c^{\scriptscriptstyle *}_{\scriptscriptstyle j}$, to SC at time $\texttt{G}_{\scriptscriptstyle j,1}$
\end{enumerate}

\

\item\textbf{Server-side Proof Generation}. $\mathtt{RCSP}.\mathtt{prove}(u^{\scriptscriptstyle *},  \sigma,  c^{\scriptscriptstyle *}_{\scriptscriptstyle j},pk^{\scriptscriptstyle *}, T_{\scriptscriptstyle qp},enc)$


\begin{enumerate}

\item Constructs an empty  vector, $\bm{m}_{\scriptscriptstyle\mathcal S}=\bot$, if $j=1$.  
\item Removes the pads from $c^{\scriptscriptstyle *}_{\scriptscriptstyle j}$, using  parameters of $T_{\scriptscriptstyle qp}$. Let  $c'_{\scriptscriptstyle j}$ be the result. Next, it  decrypts  the result, $\mathtt{Dec(}\bar{k},c'_{\scriptscriptstyle j})=c_{\scriptscriptstyle j}$. Then, using the decrypted public key, it   runs $\mathtt{VSID.checkQuery}(c_{\scriptscriptstyle j}, pk)\rightarrow b_{\scriptscriptstyle j}$, to check the correctness of the queries.

%\item  $\mathcal S$ appends $j$  to $\vv{\bm{v}}_{\scriptscriptstyle\mathcal{S}}$, if the above proofs are rejected, i.e. $b_{\scriptscriptstyle j}=0$. % or were not posted on time. In this case, it does not proceed to the next step for this $j$-th verification.



%\item  $\mathcal S$ appends $j$  to $\vv{\bm{v}}_{\scriptscriptstyle\mathcal{S}}$, if the above proofs are rejected, i.e. $b_{\scriptscriptstyle j}=0$, or were not posted on time. In this case, it does not proceed to the next step for this $j$-th verification.

\begin{itemize}
\item[$\bullet$] If $\mathcal S$ accepts the query, i.e. $b_{\scriptscriptstyle j}=1$, then  calls $\mathtt{VSID.prove}(u^{\scriptscriptstyle *},\sigma,c_{\scriptscriptstyle j},pk)\rightarrow \pi_{\scriptscriptstyle j}$, to generate the service proof. In this case, $\mathcal S$ encrypts it, $\mathtt{Enc}(\bar{k},\pi_{\scriptscriptstyle j})=\pi'_{\scriptscriptstyle j}$. Next, it   pads (every element of) the encrypted proof with ${pad}_{\scriptscriptstyle \pi}\in T_{\scriptscriptstyle qp}$ random values  picked from $U$. Let $\pi^{\scriptscriptstyle *}_{\scriptscriptstyle j}$ be the result. It sends the padded encrypted proof  to SC at time $\texttt{G}_{\scriptscriptstyle j,2}$ 

\item[$\bullet$] Otherwise (if $\mathcal S$ rejects the query), it appends  $j$  to $\bm{m}_{\scriptscriptstyle\mathcal{S}}$,  constructs a dummy proof $\pi'_{\scriptscriptstyle j}\in U$,  pads the result as above, and sends the padded dummy proof, $\pi^{\scriptscriptstyle *}_{\scriptscriptstyle j}$, to SC at time $\texttt{G}_{\scriptscriptstyle j,2}$ %, if the above proofs are rejected, i.e. $b_{\scriptscriptstyle j}=0$.

\end{itemize}
When $j=z$  and $\bm{m}_{\scriptscriptstyle\mathcal{S}}\neq\bot$, $\mathcal S$ sets    $m_{\scriptscriptstyle\mathcal S}:=(\bm{m}_{\scriptscriptstyle\mathcal{S}},\bot)$.

\end{enumerate}

\

\item \textbf{Client-side Proof Verification}.\label{Client-sideProofVerification} $\mathtt{RCSP}.\mathtt{verify}(\pi^{\scriptscriptstyle *}_{\scriptscriptstyle j}, \bm{q}_{\scriptscriptstyle j},k,T_{\scriptscriptstyle qp},enc)$
\begin{enumerate}


\item Constructs an empty vector, $\bm{m}_{\scriptscriptstyle\mathcal C}=\bot$, if $j=1$. 

\item Removes the pads from $\pi^{\scriptscriptstyle *}_{\scriptscriptstyle j}$, utilising  parameters of $T_{\scriptscriptstyle qp}$. Let $\pi'_{\scriptscriptstyle j}$ be the result. It decrypts the service proof: $\mathtt{Dec}(\bar{k},\pi'_{\scriptscriptstyle j})=\pi''_{\scriptscriptstyle j}$ and then calls $\mathtt{VSID.verify}(\pi''_{\scriptscriptstyle j}, \bm{q}_{\scriptscriptstyle j},k)\rightarrow d_{\scriptscriptstyle j}$, to verify the proof. Note that if $\pi'_{\scriptscriptstyle j}=\mathtt{Enc}(\bar{k},\pi_{\scriptscriptstyle j})$, then $\pi''_{\scriptscriptstyle j}=\pi_{\scriptscriptstyle j}$. 

\begin{itemize}
\item[$\bullet$] If $\pi''_{\scriptscriptstyle j}$ passes the verification (i.e. $d_{\scriptscriptstyle j}=1$), then $\mathcal C$ concludes that the service  for this verification has been delivered successfully. 

\item[$\bullet$]  Otherwise (when $\pi''_{\scriptscriptstyle j}$ is rejected), $\mathcal C$ appends $j$  to $\bm{m}_{\scriptscriptstyle\mathcal{C}}$. 


\end{itemize}
When $j=z$  and $\bm{m}_{\scriptscriptstyle\mathcal{C}}\neq\bot$, $\mathcal C$ sets    ${m}_{\scriptscriptstyle\mathcal C}:=(\bm{m}_{\scriptscriptstyle\mathcal{C}},e')$, where  $e'$  contains the opening of $\mathtt{Com}_{\scriptscriptstyle sk}$ or  $\bot$ as stated in Remark \ref{remark::light-VSID}.


\end{enumerate}

\

\item \textbf{Dispute Resolution}.\label{RCSP::Dispute-Resolution} $\mathtt{RCSP}.\mathtt{resolve}(m_{\scriptscriptstyle \mathcal{C}},m_{\scriptscriptstyle \mathcal{S}},z, {\bm{\pi}}^{\scriptscriptstyle *}, {\bm{c}}^{\scriptscriptstyle *},pk^{\scriptscriptstyle*}, T_{\scriptscriptstyle qp},enc)$



 The phase takes place only in  case of dispute, e.g. when $\mathcal C$ and/or $\mathcal S$ reject any proofs in the previous phases. %or  either party does not receive its counterparty's message on time. %A party who accepts all proofs does not need to get involved in this phase. 

\begin{enumerate}
%\item $\mathcal C$ and/or $\mathcal S$ at time $K_{\scriptscriptstyle 1}>G_{\scriptscriptstyle z,2}+H$ sends ``dispute'' message to the contract. 


\item The arbiter sets  counters: $y_{\scriptscriptstyle\mathcal C}, y'_{\scriptscriptstyle\mathcal C},y_{\scriptscriptstyle\mathcal S}$ and  $y'_{\scriptscriptstyle\mathcal S}$ and a vector: $\bm{des}$, that are initially set to $0$,  before time $\texttt{K}_{\scriptscriptstyle 1}$,  where $\texttt{K}_{\scriptscriptstyle 1}>\texttt{G}_{\scriptscriptstyle z,2}+\texttt{H}$. 
\item $\mathcal C$  sends $m_{\scriptscriptstyle\mathcal C}$ and $\ddot{x}_{\scriptscriptstyle qp}$ to the arbiter at time $\texttt{K}_{\scriptscriptstyle 1}$. Also,  $\mathcal S$ sends $m_{\scriptscriptstyle\mathcal S}$ and $\ddot{x}_{\scriptscriptstyle qp}$ to the arbiter at time $\texttt{K}_{\scriptscriptstyle 1}$ %,   where $|\vv{\bm{v}}_{\scriptscriptstyle\mathcal{C}}|\neq 0$ and $|\vv{\bm{v}}_{\scriptscriptstyle\mathcal{S}}|\neq 0$. 



\item\label{arbiter-checks-statement} The arbiter, checks the validity of  statement $\ddot{x}_{\scriptscriptstyle qp}$ sent by party  $\forall \mathcal R\in \{\mathcal {C,S}\}$. To do that, it sends  $\ddot{x}_{\scriptscriptstyle qp}$ to SAP contract which  returns  either $1$ or $0$. The arbiter  constructs an empty vector: $\bm{v}$.  If  party $\mathcal R$'s statement is accepted, then 
 it appends every  element of $\bm{m}_{\scriptscriptstyle\mathcal{R}}$ to $\bm{v}$ and $\bm{des}$  if that element is not already in $\bm{des}$. Also, it ensures $\bm{v}$ contains only distinct elements which are in the range $[1,z]$. Otherwise (if the party's statement is rejected) it discards the party's request, ${m}_{\scriptscriptstyle\mathcal{R}}$. It proceeds to the next step if $\bm{v}$ is not empty, otherwise it halts. 
 %, and increments $ y'_{\scriptscriptstyle\mathcal R}$ by 1

%\item The Arbiter, for each party, checks the validity of $k$ by sending $k$ and $\mu'$ to SAP contract which  returns  either $1$ or $0$. The arbiter proceeds to the next step if the output is 1; otherwise, it does nothing. 
%\item The Arbiter decrypts those proofs and challenges whose index are in $\vv{\bm{v}}$. In particular,  $\forall i\in \vv{\bm{v}}: \mathtt{Dec}(\bar{k},\mathtt{Enc}(c_{\scriptscriptstyle i}))=c_{\scriptscriptstyle i}$ and $ \mathtt{Dec}(\bar{k},\mathtt{Enc}(\pi_{\scriptscriptstyle i}))=\pi_{\scriptscriptstyle i}$

\item The arbiter uses $\bar{k}\in \ddot{x}_{\scriptscriptstyle qp}$ to decrypt the encrypted public key that was sent to SC, $\mathtt{Dec}(\bar{k},\mathtt{Enc}(\bar{k},pk))=pk$



\item\label{the-arbiter-loop} The arbiter  for every element $i\in\bm{v}$:

\begin{enumerate}
%\item Checks if $\mathcal C$  sent to SC an encrypted query on time, in step \ref{Billing-cycles-genQuery}. If the check fails, it increments $u_{\scriptscriptstyle\mathcal C}$ by $1$ and retakes step \ref{the-arbiter-loop} with the next element of $\vv{\bm{v}}$. Otherwise, it proceeds to the next step. 

%\item Checks if $\mathcal S$  sent to SC an encrypted proof on time, in step \ref{Billing-cycles-sendProof}. If the check fails, it increments $u_{\scriptscriptstyle\mathcal S}$ by $1$ and retakes step \ref{the-arbiter-loop} with the next element of $\vv{\bm{v}}$. Otherwise, it proceeds to the next step. 
\item removes the pads from the related encrypted    query-proof pair and encrypted service proof. Let $c'_{\scriptscriptstyle i}$ and $\pi'_{\scriptscriptstyle i}$ be the result. %unpadded encrypted  query-proof pair  and encrypted service proof  respectively. 
\item decrypts the encrypted query-proof pair and encrypted service proof as follows, $ \mathtt{Dec}(\bar{k},c'_{\scriptscriptstyle i})=c_{\scriptscriptstyle i}$ and 
$ \mathtt{Dec}(\bar{k},\pi'_{\scriptscriptstyle i})=\pi''_{\scriptscriptstyle i}$ 
\item\label{arbiter-identify} calls   $\mathtt{VSID.identify}(\pi''_{\scriptscriptstyle i},c_{\scriptscriptstyle i},pk,e')\rightarrow I_{\scriptscriptstyle i}$
\begin{enumerate}
\item[$\bullet$] if $I_{\scriptscriptstyle i}=\mathcal C$, then it increments $y_{\scriptscriptstyle\mathcal C}$ by $1$ 
\item[$\bullet$]  if $I_{\scriptscriptstyle i}=\mathcal S$, then it increments $y_{\scriptscriptstyle\mathcal S}$ by $1$ 
\item[$\bullet$]  if $I_{\scriptscriptstyle i}=\bot$, then it increments  $y'_{\scriptscriptstyle\mathcal C}$ or $y'_{\scriptscriptstyle\mathcal S}$ by $1$, if the arbiter is invoked by the $\mathcal {C}$ or $\mathcal {S}$ respectively. 
\end{enumerate}
\end{enumerate}

%\item The Arbiter for every  $i\in\vv{\bm{v}}$ calls   $\mathtt{VSID.identify}(\pi_{\scriptscriptstyle i},c_{\scriptscriptstyle i},pk,e')\rightarrow I_{\scriptscriptstyle i}$, to resolve the dispute. 
%
%
%
%
%Let $u_{\scriptscriptstyle\mathcal C}$ and $u_{\scriptscriptstyle\mathcal S}$  be the total number of times $\mathtt{VSID.identify}()$ returns $\mathcal{C}$ and $\mathcal{S}$ respectively.  Also, let $y_{\scriptscriptstyle\mathcal C}$ and $y_{\scriptscriptstyle\mathcal S}$ be the total number of times $\mathtt{VSID.identify}()$ returns $\bot$ when it is invoked by the client and server  respectively. 
Let $K_{\scriptscriptstyle 2}$ be the time that the arbiter finishes the above checks.

\item The arbiter sends $(y_{\scriptscriptstyle\mathcal C},y_{\scriptscriptstyle\mathcal S})$  to SC at time $K_{\scriptscriptstyle 3}$
\end{enumerate}
\item \textbf{Coin Transfer}\label{RCSP::CoinTransfer}.
\begin{enumerate}
\item Either $\mathcal C$ or $\mathcal S$ send ``pay'' message and the statement, $\ddot{x}_{\scriptscriptstyle cp}$,  to SC at time $L>K_{\scriptscriptstyle 3}$
\item SC checks the validity of the statement by sending $\ddot{x}_{\scriptscriptstyle cp}$ to SAP contract which  returns  either $1$ or $0$. SC only proceeds to the next step if the output is $1$. 
\item SC distributes the coins to the parties as follows:

\begin{itemize}
\item[$\bullet$]   $coin^{\scriptscriptstyle *}_{\scriptscriptstyle\mathcal C}-o(z-y_{\scriptscriptstyle\mathcal S})-l\cdot y_{\scriptscriptstyle\mathcal C}$ coins  to $\mathcal C$
\item[$\bullet$] $coin^{\scriptscriptstyle *}_{\scriptscriptstyle\mathcal S}+o(z-y_{\scriptscriptstyle\mathcal S})-l\cdot y_{\scriptscriptstyle\mathcal S}$ coins to $\mathcal S$
\item[$\bullet$] $l(y_{\scriptscriptstyle\mathcal S}+y_{\scriptscriptstyle\mathcal C}+y'_{\scriptscriptstyle\mathcal S}+y'_{\scriptscriptstyle\mathcal C})$ coins to the arbiter.  
\end{itemize}

\end{enumerate}
\end{enumerate}



\begin{remark} If all parties behave honestly,   then the server receives all its deposit back plus the amount of coins they initially agreed to pay the sever if it  delivers accepting proofs for all $z$ cycles, i.e. in total it receives $coin^{\scriptscriptstyle *}_{\scriptscriptstyle\mathcal S}+o\cdot z$ coins. Also, in this case an honest client receives all coins minus the amount of coins paid to the server for delivering accepting proofs for $z$ cycles, i.e. in total it receives $coin^{\scriptscriptstyle *}_{\scriptscriptstyle\mathcal C}-o\cdot z$ coins. However, the arbiter receives no coins, as it is never invoked. 


%\begin{remark}
%Keeping track of  $(y'_{\scriptscriptstyle\mathcal C},y'_{\scriptscriptstyle\mathcal S})$ enables the arbiter to make malicious parties, who \emph{unnecessarily} invoke  it for invalid statement in step \ref{arbiter-checks-statement} or  accepting proofs in step \ref{arbiter-identify}, pay  for the verifications it performs. 
%\end{remark}

%\begin{remark}
%If the client or server  receives no input from its counter-party, it still needs to invoke the arbiter, so that party can be paid (back) fairly. 
%\end{remark}




\end{remark}
\begin{remark} The VSID scheme  does not (need to) preserve the privacy of the proofs. However, in RC-S-P  protocol  each proof's privacy must be preserved, for a certain time; otherwise, the proof itself can leak its status, e.g. when it can be publicly verified. This is the reason in RC-S-P protocol,  \emph{encrypted} proofs are sent to the contract.   
\end{remark}


\begin{remark}
For the sake of simplicity, in the above protocol, we assumed that each arbiter's invocation has a fixed  cost regardless of the number of steps it takes. To define a fine grained costing, one can simply allocate to each step the arbiter takes a certain rate and also separate counter  for the client and server.  
\end{remark}

\begin{remark}
In the case where $\mathtt{VSID.verify}()$ is privately verifiable and    the server invokes the arbiter, the client needs to provide inputs to the arbiter too. Otherwise (when it is publicly verifiable and  the server invokes the arbiter), the client's involvement is not required in the dispute resolution phase.  In contrast, if the client invokes the arbiter, the server's involvement is not required in that phase, regardless of the type of verifiability $\mathtt{VSID.verify}()$ supports. 
\end{remark}

\begin{theorem}
The RC-S-P protocol is secure, w.r.t. Definition \ref{def::RC-S-P-Security}, if VSID and SAP are secure and the encryption scheme is semantically secure. 
\end{theorem}



 To prove the above theorem, we show that RC-S-P meets all security properties defined in Section \ref{sec::RC-S-P-Definition}.  We start by proving that RC-S-P satisfies security against a malicious server.  
 
 % !TEX root =R-zkCSP.tex

 \begin{lemma}
 If SAP is secure and  VSID scheme supports correctness, soundness, and detectable abort, then RC-S-P is secure against malicious server, w.r.t. Definition \ref{deff::RC-S-P-SecurityAgainstMaliciousServer}. 
 \end{lemma}
 

 \begin{proof}[sketch]
 We first consider event $F(u^{\scriptscriptstyle *},\bm{q}_{\scriptscriptstyle j})= h_{\scriptscriptstyle j}\ \wedge coin_{\scriptscriptstyle\mathcal{C},j}\neq  \frac{coin_{\scriptscriptstyle\mathcal C}^{\scriptscriptstyle*}}{z}-o$ that captures the case where the server provides an accepting service proof but makes an honest client withdraw an incorrect amount of coins, i.e. $coin_{\scriptscriptstyle\mathcal{C},j}\neq  \frac{coin_{\scriptscriptstyle\mathcal C}^{\scriptscriptstyle*}}{z}-o$. Recall, in the protocol, the total coins the client should receive after $z$ verifications is $coin^{\scriptscriptstyle *}_{\scriptscriptstyle\mathcal C}-o(z-y_{\scriptscriptstyle\mathcal S})-l\cdot y_{\scriptscriptstyle\mathcal C}$. Since we focus on  $j$-th verification, the amount of  coins that should be credited to the client for that verification is
  
  \begin{equation}\label{equ::what-client-recives-in-j}
   coin_{\scriptscriptstyle\mathcal{C},j}=\frac{coin^{\scriptscriptstyle *}_{\scriptscriptstyle\mathcal C}}{z}-o(1-y_{\scriptscriptstyle\mathcal S}[j])-l\cdot y_{\scriptscriptstyle\mathcal C}[j]
     \end{equation}

    
As the proof is valid, an honest client accepts it and does not raise any dispute. However, the server  would be able to make the client  withdraw incorrect amounts of coins, if it manages to either convince the arbiter that the client has misbehaved, by making the arbiter output  $y_{\scriptscriptstyle\mathcal {C}}[j]=1$ through dispute resolution phase, or submit to the contract, at the coin transfer phase,  an accepting  statement $\ddot{x}'_{\scriptscriptstyle cp}$ other than what was agreed at the initiation phase, i.e. $\ddot{x}'_{\scriptscriptstyle cp}\neq\ddot{x}_{\scriptscriptstyle cp}$, so it can change the payments parameters, e.g. $l$ or $o$. Nevertheless, it cannot falsely accuse the client of misbehaviour. Because,  due to the security of SAP, it cannot  convince the arbiter to accept different decryption key or pads other than what was agreed with the client in the initiation phase; specifically, it cannot persuade the arbiter to accept  $\ddot{x}'_{\scriptscriptstyle qp}$, where $\ddot{x}'_{\scriptscriptstyle qp}\neq \ddot{x}_{\scriptscriptstyle qp}$, except with a negligible probability, $\mu(\lambda)$. This ensures that the honest client's message is accessed by  the arbiter with a high probability, as the arbiter can extract the client's message using valid pad information and decryption key.  On the other hand, if the adversary provides a valid statement, i.e. $\ddot{x}_{\scriptscriptstyle qp}$, then due to the correctness of VSID, algorithm $\mathtt{VSID.identify}()$ outputs $I_{\scriptscriptstyle j}=\bot$. Therefore, due to the security of SAP and correctness of VSID,   $y_{\scriptscriptstyle\mathcal {C}}$ and $y_{\scriptscriptstyle\mathcal {S}}$ are not incremented by $1$  in $j$-th verification, i.e. $y_{\scriptscriptstyle\mathcal{C}}[j]=y_{\scriptscriptstyle\mathcal{S}}[j]=0$. So, according to Equation \ref{equ::what-client-recives-in-j}, the client is credited $\frac{coin_{\scriptscriptstyle\mathcal C}^{\scriptscriptstyle*}}{z}-o$ coins for that verification, with a high probability. Note,  due to the security of SAP, the server cannot change the payment parameters by persuading the contract to accept any statement $\ddot{x}'_{\scriptscriptstyle cp}$ other than what was agreed initially between the client and server, except with a negligible probability $\mu(\lambda)$ when it finds the hash function's collision. 
    
     We now move on to event $F(u^{\scriptscriptstyle *}, \bm{q}_{\scriptscriptstyle j})\neq h_{\scriptscriptstyle j} \ \wedge (d_{\scriptscriptstyle j}=1	\vee y_{\scriptscriptstyle \mathcal S}[j]=0 \ \vee (coin_{\scriptscriptstyle\mathcal{C},j}\neq \frac{coin_{\scriptscriptstyle\mathcal C}^{\scriptscriptstyle*}}{z} \ \vee  coin_{\scriptscriptstyle\mathcal{A}r,j}\neq l))$ which captures the case where the server provides an invalid service proof but either persuades the client to accept the proof, or  persuades the arbiter to accept the proof (e.g. when the client raises a dispute) or makes the client or arbiter withdraw an incorrect amount of coins, i.e.    $coin_{\scriptscriptstyle\mathcal{C},j}\neq \frac{coin_{\scriptscriptstyle\mathcal C}^{\scriptscriptstyle*}}{z}$ or $coin_{\scriptscriptstyle\mathcal{A}r,j}\neq l$ respectively.   Nevertheless, due to the soundness of VSID, the probability that a corrupt server can convince an honest client to accept invalid proof (i.e. outputs $d_{\scriptscriptstyle j}=1$), is negligible, $\mu(\lambda)$. So, the client detects it with a high probability and raises a dispute.  On the other hand, the server may try to convince the arbiter, and make it output $y_{\scriptscriptstyle \mathcal S}[j]=0$ or $y_{\scriptscriptstyle \mathcal C}[j]=1$, e.g. by sending a complaint. For $y_{\scriptscriptstyle \mathcal S}[j]=0$ to happen, it has to either provide a different accepting statement $\ddot{x}'_{\scriptscriptstyle qp}$, than what was initially agreed with the client (i.e. $\ddot{x}'_{\scriptscriptstyle qp}\neq \ddot{x}_{\scriptscriptstyle qp}$) and passes the verification, which   requires finding the hash function's collision, and its probability of success is   $\mu(\lambda)$. Or it makes the arbiter  accept an invalid proof, but due to the detectable abort property of VSID, its probability of success is also  $\mu(\lambda)$. As we discussed above, the probability that the adversary makes the arbiter output $y_{\scriptscriptstyle \mathcal C}[j]=1$ is  $\mu(\lambda)$ too. Therefore, the arbiter outputs $y_{\scriptscriptstyle \mathcal S}[j]=1$ and $y_{\scriptscriptstyle \mathcal C}[j]=0$  with a high probability, in both events when it is invoked by the client or  server. That means according to Equation \ref{equ::what-client-recives-in-j}, the client is credited $\frac{coin_{\scriptscriptstyle\mathcal C}^{\scriptscriptstyle*}}{z}$ coins for that verification, with a high probability. Recall, in the protocol, the total coins the arbiter should receive for $z$ verifications is $l(y_{\scriptscriptstyle\mathcal S}+y_{\scriptscriptstyle\mathcal C})$, so for the $j$-th the credited coins should be: 
     
     
     \begin{equation}\label{equ::arbiter}
    coin_{\scriptscriptstyle\mathcal{A}r,j}= l(y_{\scriptscriptstyle\mathcal S}[j]+y_{\scriptscriptstyle\mathcal C}[j])
     \end{equation}
     
   As we discussed above, $y_{\scriptscriptstyle \mathcal S}[j]=1$ and $y_{\scriptscriptstyle \mathcal C}[j]=0$. So, according to Equation \ref{equ::arbiter}, the arbiter is credited $l$ coins for that verification, with a high probability.  The adversary may  try to  make them withdraw an incorrect amount of coins, e.g. in the case where it does not succeed in convincing the client or arbiter. To this end,  at the coin transfer phase, it has to send a  different accepting statement  than what was initially agreed with the client. But, it would succeed only with a negligible probability, $\mu(\lambda)$, due to the security of SAP. 
     
  
Furthermore, in   both  events above, due to the security of SAP, the adversary cannot block an honest client's  messages, ``pay'' and $\ddot{x}_{\scriptscriptstyle cp}$, to the contract in the coin transfer phase.
  \hfill\(\Box\)\end{proof}

%We first consider event $F(u^{\scriptscriptstyle *},\bm{q}_{\scriptscriptstyle j})= h_{\scriptscriptstyle j}\ \wedge coin_{\scriptscriptstyle\mathcal{C},j}\neq  \frac{coin_{\scriptscriptstyle\mathcal C}^{\scriptscriptstyle*}}{z}-o$ that captures the case where the server provides an accepting service proof but makes an honest client withdraw an incorrect amount of coins, i.e. $coin_{\scriptscriptstyle\mathcal{C},j}\neq  \frac{coin_{\scriptscriptstyle\mathcal C}^{\scriptscriptstyle*}}{z}-o$. Since the proof is valid, an honest client accepts it and does not raise any dispute. 
%  However, the server  would be able to make the client  withdraw incorrect amounts of coins, if it manages to either convince the arbiter that the client has misbehaved (by making the arbiter increment $y_{\scriptscriptstyle\mathcal {C}}$  by $1$ through dispute resolution phase), or submit to the contract, at the coin transfer phase,  an accepting  statement $\ddot{x}'_{\scriptscriptstyle cp}$ other than what was agreed at the initiation phase, i.e. $\ddot{x}_{\scriptscriptstyle cp}$. Nevertheless, it cannot falsely accuse the client of misbehaviour. Because,  due to the security of SAP, it cannot  convince the arbiter to accept different decryption key or pads other than what was agreed with the client in the initiation phase; specifically, it cannot persuade the arbiter to accept  $\ddot{x}'_{\scriptscriptstyle qp}$, where $\ddot{x}'_{\scriptscriptstyle qp}\neq \ddot{x}_{\scriptscriptstyle qp}$, except with a negligible probability, $\mu(\lambda)$. This ensures that the honest client's message is accessed by  the arbiter with a high probability, as the arbiter can extract the client's message using valid pad information and decryption key.  On the other hand, if the adversary provides a valid statement, i.e. $\ddot{x}_{\scriptscriptstyle qp}$, then due to the correctness of VSID, algorithm $\mathtt{VSID.identify}()$ outputs $I_{\scriptscriptstyle j}=\bot$. Therefore, due to the security of SAP and correctness of VSID,   $y_{\scriptscriptstyle\mathcal {C}}$ and $y_{\scriptscriptstyle\mathcal {S}}$ are not incremented by $1$  in $j$-th verification, i.e. $y_{\scriptscriptstyle\mathcal{C}}[j]=y_{\scriptscriptstyle\mathcal{S}}[j]=0$.  Moreover,  due to the security of SAP, the server cannot persuade the contract to accept any statement other than what was agreed initially between the client and server, except with a negligible probability $\mu(\lambda)$ when it finds the hash function's collision. Recall, in the protocol, the total coins the client receives after $z$ verifications is $coin^{\scriptscriptstyle *}_{\scriptscriptstyle\mathcal C}-o(z-y_{\scriptscriptstyle\mathcal S})-l\cdot y_{\scriptscriptstyle\mathcal C}$. Since we focus on each $j$-th verification, the amount of  coins credited to the client for that verification is
%  
%  \begin{equation}\label{equ::what-client-recives-in-j}
%   \frac{coin^{\scriptscriptstyle *}_{\scriptscriptstyle\mathcal C}}{z}-o(1-y_{\scriptscriptstyle\mathcal S}[j])-l\cdot y_{\scriptscriptstyle\mathcal C}[j]
%     \end{equation}
%     
%     In the above event, as we discussed,  with a high probability $y_{\scriptscriptstyle\mathcal{C}}[j]=y_{\scriptscriptstyle\mathcal{S}}[j]=0$ , so the client is credited $\frac{coin_{\scriptscriptstyle\mathcal C}^{\scriptscriptstyle*}}{z}-o$ coins for that verification. 
%     
   




\begin{lemma}
 If SAP is secure and  VSID scheme supports correctness, inputs well-formedness, and detectable abort, then RC-S-P is secure against malicious client, w.r.t. Definition \ref{deff::RC-S-P-Security-Against Malicious-Client}. 
\end{lemma}


\begin{proof}[sketch]
  First, we consider event  $(M(u^{\scriptscriptstyle *},k)= \sigma  \wedge Q(\text{aux},k)= \bm{q}_{\scriptscriptstyle j}) \wedge
  (coin_{\scriptscriptstyle\mathcal{S},j}\neq  \frac{coin_{\scriptscriptstyle\mathcal S}^{\scriptscriptstyle*}}{z}+o)$ which captures the case where the client provides  accepting metadata and query  but makes the server withdraw an incorrect amount of coins, i.e. $coin_{\scriptscriptstyle\mathcal{S},j}\neq  \frac{coin_{\scriptscriptstyle\mathcal S}^{\scriptscriptstyle*}}{z}+o$. According to the RC-S-P protocol, the total coins the server should receive after $z$ verifications is $coin^{\scriptscriptstyle *}_{\scriptscriptstyle\mathcal S}+o(z-y_{\scriptscriptstyle\mathcal S})-l\cdot y_{\scriptscriptstyle\mathcal S}$. Since we focus on  $j$-th verification, the amount of  coins that should be credited to the server for $j$-th verification is
  
  \begin{equation}\label{equ::what-server-recives-in-j}
   coin_{\scriptscriptstyle\mathcal{S},j}=\frac{coin^{\scriptscriptstyle *}_{\scriptscriptstyle\mathcal S}}{z}+o(1-y_{\scriptscriptstyle\mathcal S}[j])-l\cdot y_{\scriptscriptstyle\mathcal S}[j]
     \end{equation}
  
  
  
  Since the metadata and query's proofs are valid, an honest server accepts them and does not raise any dispute, so we have $y_{\scriptscriptstyle\mathcal{C}}[j]=0$.  The client may want to  make the server  withdraw incorrect amounts of coins, if it manages to either convince the arbiter, in phase \ref{RCSP::Dispute-Resolution}, that the server has misbehaved, i.e. makes the arbiter output $y_{\scriptscriptstyle\mathcal{S}}[j]=1$, or submit to the contract  an accepting  statement $\ddot{x}'_{\scriptscriptstyle cp}$ other than what was agreed at the initiation phase, i.e. $\ddot{x}_{\scriptscriptstyle cp}$, in phase \ref{RCSP::CoinTransfer}. However, it cannot falsely accuse the server of misbehaviour. As,  due to the security of SAP, it cannot  convince the arbiter to accept different decryption key and pads, by providing a different accepting statement $\ddot{x}'_{\scriptscriptstyle qp}$  (where $\ddot{x}'_{\scriptscriptstyle qp}\neq \ddot{x}_{\scriptscriptstyle qp}$), than what was initially agreed with the server, except with  probability $\mu(\lambda)$. This ensures that  the arbiter is given the honest server's messages (with a high probability). On the other hand, if the adversary provides a valid statement, i.e. $\ddot{x}_{\scriptscriptstyle qp}$, then due to the correctness of VSID, algorithm $\mathtt{VSID.identify}()$ outputs $I_{\scriptscriptstyle j}=\bot$. So, due to the security of SAP and correctness of VSID,  we would have $y_{\scriptscriptstyle\mathcal{C}}[j]=y_{\scriptscriptstyle\mathcal{S}}[j]=0$ with a high probability.  That means, according to the Equation \ref{equ::what-server-recives-in-j}, the server is credited $\frac{coin_{\scriptscriptstyle\mathcal S}^{\scriptscriptstyle*}}{z}+o$ coins for that verification, with a  high probability. Note,  due to the security of SAP, the client cannot convince the contract to accept any accepting statement $\ddot{x}'_{\scriptscriptstyle cp}$ other than what was initially agreed  between the client and server (i.e. $\ddot{x}'_{\scriptscriptstyle cp}\neq \ddot{x}_{\scriptscriptstyle cp}$), except with   probability $\mu(\lambda)$.  
  
%  \begin{equation}\label{equ::what-server-recives-in-j}
%   \frac{coin^{\scriptscriptstyle *}_{\scriptscriptstyle\mathcal S}}{z}+o(1-y_{\scriptscriptstyle\mathcal S}[j])-l\cdot y_{\scriptscriptstyle\mathcal S}[j]
%     \end{equation}
%     
%    Since  we have  $y_{\scriptscriptstyle\mathcal{C}}[j]=y_{\scriptscriptstyle\mathcal{S}}[j]=0$ with a high probability,  the server is credited $\frac{coin_{\scriptscriptstyle\mathcal S}^{\scriptscriptstyle*}}{z}+o$ coins for that verification, with  high probability. 
%    
    
   
    
    We now turn our attention to $(M(u^{\scriptscriptstyle *},k)\neq \sigma \wedge a=1)\ \vee\ (Q(\text{aux},k)\neq \bm{q}_{\scriptscriptstyle j} \wedge b_{\scriptscriptstyle j}=1)$,   that captures the case where the server accepts an ill-formed metadata, or query. However, due to inputs well-formedness of VSID, the probability that either of the events happens is negligible, $\mu(\lambda)$. So, with a high probability $a=0$ or $b_{\scriptscriptstyle j}=0$.  Note, in the former case (i.e. $a=0$) the server does not raise any dispute, instead it avoids serving the client.  Next, we move on to      $Q(\text{aux},k)\neq \bm{q}_{\scriptscriptstyle j} \wedge
  (y_{\scriptscriptstyle\mathcal{C}}[j]=0 \ \vee  coin_{\scriptscriptstyle\mathcal{S},j}\neq \frac{coin_{\scriptscriptstyle\mathcal S}^{\scriptscriptstyle*}}{z}+o \ \vee  coin_{\scriptscriptstyle\mathcal{A}r,j}\neq l)$.  It considers the case where the client provides an invalid query, but  either convinces the arbiter to accept it, or  makes the server or arbiter  withdraw an incorrect amount of coins, i.e. $coin_{\scriptscriptstyle\mathcal{S},j}\neq \frac{coin_{\scriptscriptstyle\mathcal S}^{\scriptscriptstyle*}}{z}+o$ or  $coin_{\scriptscriptstyle\mathcal{A}r,j}\neq l$ respectively. Note, when the server rejects the query and raises a dispute, the client may try to convince the arbiter, and make it output $y_{\scriptscriptstyle \mathcal C}[j]=0$, e.g. by sending a complaint right after the server does so. However, for the adversary to win, either it has to provide a different accepting statement $\ddot{x}'_{\scriptscriptstyle qp}$, than what was initially agreed with the server (i.e. $\ddot{x}'_{\scriptscriptstyle qp}\neq \ddot{x}_{\scriptscriptstyle qp}$) and passes the verification. But,  due to the security of SAP, its probability of success is negligible, $\mu(\lambda)$. Or it has to make the arbiter  accept an invalid query, i.e. makes the arbiter output $y_{\scriptscriptstyle\mathcal{C}}[j]=0$. But, due to the detectable abort property of VSID, its probability of success is  $\mu(\lambda)$. Therefore, with a high probability we have $y_{\scriptscriptstyle\mathcal{C}}[j]=1$. Also, as discussed above, the client cannot make the arbiter recognise the  honest server as a misbehaving party with a probability significantly greater than $\mu(\lambda)$. That means with a high probability $y_{\scriptscriptstyle\mathcal{S}}[j]=0$. Hence, according to Equation \ref{equ::what-server-recives-in-j} the server is credited  $\frac{coin_{\scriptscriptstyle\mathcal S}^{\scriptscriptstyle*}}{z}+o$ coins for this verification; and also the arbiter is credited $l$ coins, according to Equation \ref{equ::arbiter}.  Note,  the adversary may still try to  make them withdraw an incorrect amount of coins (e.g. if the adversary does not succeed in convincing the server or arbiter). To this end, at the coin transfer phase, it has to send a  different accepting statement  than what was initially agreed with the server. However, due to the security of SAP, its  success probability is  $\mu(\lambda)$. Also, due to the security of SAP, the adversary cannot block an honest server's  messages, ``pay'' and $\ddot{x}_{\scriptscriptstyle cp}$, to the contract in the coin transfer phase.
 \hfill\(\Box\)\end{proof}
  
  
  Prior to proving  RC-S-P's privacy, we provide a lemma that will be used in the privacy's proof. Informally, the lemma states that encoded coins leaks no information about the actual amount of coins $(o,l)$, agreed between the client and server. 


\begin{lemma}\label{lemma::encoded-coins} Let $\beta\stackrel{\scriptscriptstyle\$}\leftarrow\{0,1\}$, price list be $\{(o_{\scriptscriptstyle 0},l_{\scriptscriptstyle 0}),(o_{\scriptscriptstyle 1},l_{\scriptscriptstyle 1})\}$, and encoded coin amounts be $coin^{\scriptscriptstyle *}_{\scriptscriptstyle\mathcal{C}}= z\cdot(Max(o_{\scriptscriptstyle\beta},o_{\scriptscriptstyle|\beta-1|})+Max(l_{\scriptscriptstyle\beta},l_{\scriptscriptstyle|\beta-1|}))$ and $coin^{\scriptscriptstyle *}_{\scriptscriptstyle\mathcal{S}}=z\cdot(Max(l_{\scriptscriptstyle\beta}, l_{\scriptscriptstyle |\beta-1|}))$. Then, given the price list, $z$, $coin^{\scriptscriptstyle *}_{\scriptscriptstyle\mathcal{C}}$, and $coin^{\scriptscriptstyle *}_{\scriptscriptstyle\mathcal{S}}$, an adversary $\mathcal{A}$ cannot tell the value of $\beta$ with probability significantly greater than $\frac{1}{2}$ (where the probability is taken over the choice of $\beta$ and the randomness of $\mathcal{A}$).
\end{lemma}



\begin{proof}
As it is evident, the list and $z$ contains no information about $\beta$. Also, since  $z$ is a public value, we could have   $coin'^{\scriptscriptstyle *}_{\scriptscriptstyle\mathcal C}=\frac{coin^{\scriptscriptstyle *}_{\scriptscriptstyle\mathcal C}}{z} = Max(o_{\scriptscriptstyle\beta},o_{\scriptscriptstyle|\beta-1|})+Max(l_{\scriptscriptstyle\beta}, l_{\scriptscriptstyle|\beta-1|})$. It is not hard to see $coin'^{\scriptscriptstyle *}_{\scriptscriptstyle\mathcal C}$ is a function of maximum value of $(o_{\scriptscriptstyle 0},o_{\scriptscriptstyle 1})$, and maximum value of $(l_{\scriptscriptstyle 0},l_{\scriptscriptstyle 1})$. It is also  independent of $\beta$. Therefore, given the list, $z$ and $coin'^{\scriptscriptstyle *}_{\scriptscriptstyle\mathcal C}$ the adversary learns nothing about $\beta$, unless it guesses the value,  with  success probability $\frac{1}{2}$. The same also holds for $coin^{\scriptscriptstyle *}_{\scriptscriptstyle\mathcal S}$. 
 \hfill\(\Box\)\end{proof}


\begin{lemma} If SAP is secure and the encryption scheme is semantically secure, then RC-S-P preserves privacy, w.r.t. Definition \ref{deff::RC-S-P-Privacy}. 
\end{lemma}

\begin{proof}[sketch]
Due to the privacy property of SAP, that stems from the  hiding property of the commitment scheme, given the commitments $g_{\scriptscriptstyle qp}$ and $g_{\scriptscriptstyle cp}$, (that are stored in the blockchain as result of running SAP) the adversary learns no information about the committed values (e.g. $o, l, pad_{\scriptscriptstyle\pi}, pad_{\scriptscriptstyle q},$ and $\bar{k}$), except with negligible probability, $\mu_{\scriptscriptstyle 1}(\lambda)$.   Also, given encoded coins $coin^{\scriptscriptstyle *}_{\scriptscriptstyle\mathcal{C}}=z\cdot (o_{\scriptscriptstyle max}+l_{\scriptscriptstyle max})$ and  $coin^{\scriptscriptstyle *}_{\scriptscriptstyle\mathcal{S}}=z\cdot l_{\scriptscriptstyle max}$, the adversary learns nothing about the actual price agreed between the server and client,  $(o,l)$, for each verification, due to Lemma \ref{lemma::encoded-coins}. Next we analyse the privacy of padded encrypted query vector $\bm{c}^{\scriptscriptstyle *}$. For  the sake of simplicity, we focus on   $\bm{q}^{\scriptscriptstyle *}_{\scriptscriptstyle j}\in c^{\scriptscriptstyle *}_{\scriptscriptstyle j}\in\bm{c}^{\scriptscriptstyle *}$, that is a padded encrypted query vector for $j$-th verification. Let $\bm{q}_{\scriptscriptstyle j,0}$ and $\bm{q}_{\scriptscriptstyle j,1}$  be query vectors, for $j$-th verification, related to the service inputs $u_{\scriptscriptstyle 0}$ and $u_{\scriptscriptstyle 1}$ that    are picked by the adversary according to  Definition  \ref{deff::RC-S-P-Privacy} which lets  the environment pick $\beta\stackrel{\scriptscriptstyle\$}\leftarrow \{0,1\}$. Also, let $\{\bm{q}_{\scriptscriptstyle j,0},...,\bm{q}_{\scriptscriptstyle j,m}\}$ be a  list of all queries of different sizes. In the experiment, if $\bm{q}_{\scriptscriptstyle j,\beta}$  is only   encrypted (but not padded), then given the ciphertext, due to semantical security of the encryption, an adversary cannot tell if the ciphertext corresponds to $\bm{q}_{\scriptscriptstyle j,0}$ or $\bm{q}_{\scriptscriptstyle j,1}$  (accordingly to $u_{\scriptscriptstyle 0}$ or $u_{\scriptscriptstyle 1}$) with probability greater than $\frac{1}{2}+\mu_{\scriptscriptstyle 1}(\lambda)$,  under the assumption  that $Max(|\bm{q}_{\scriptscriptstyle j,0}|,...,|\bm{q}_{\scriptscriptstyle j,m}|)=|\bm{q}_{\scriptscriptstyle j,\beta}|$. The  assumption is  relaxed with the use of a pad; as each encrypted query is padded  to the queries' maximum size, i.e. $Max(|\bm{q}_{\scriptscriptstyle j,0}|,...,|\bm{q}_{\scriptscriptstyle j,m}|)$, the adversary cannot tell with probability greater than $\frac{1}{2}+\mu_{\scriptscriptstyle 1}(\lambda)$ if the  padded encrypted proof corresponds to  $\bm{q}_{\scriptscriptstyle j,0}$ or $\bm{q}_{\scriptscriptstyle j,1}$, as the padded encrypted query \emph{always has the same size} and the pad values are picked from the same range as the encryption's ciphertext are defined. The same argument holds for $\bm{w}^{\scriptscriptstyle *}_{\scriptscriptstyle q_{_{\scaleto{j}{\size}}}}\in c^{\scriptscriptstyle *}_{\scriptscriptstyle j}\in\bm{c}^{\scriptscriptstyle *}$. 

Next we analyse the privacy of padded encrypted proof vector $\bm{\pi}^{\scriptscriptstyle *}$. The argument is similar to the one presented above; however, for  the sake of completeness we provide it.  Again, we focus on an element of the vector,  $\pi_{\scriptscriptstyle j}^{\scriptscriptstyle *}\in\bm{\pi}^{\scriptscriptstyle *}$, that is a padded encrypted proof for $j$-th verification. Let $\pi_{\scriptscriptstyle j,0}$ and $\pi_{\scriptscriptstyle j,1}$  be proofs, for $j$-th verification, related to the service inputs $u_{\scriptscriptstyle 0}$ and $u_{\scriptscriptstyle 1}$, where the inputs   are picked by the adversary, w.r.t. Definition  \ref{deff::RC-S-P-Privacy} in which  the environment picks $\beta\stackrel{\scriptscriptstyle\$}\leftarrow \{0,1\}$.  Let $\{\pi_{\scriptscriptstyle j,0},...,\pi_{\scriptscriptstyle j,m}\}$ be proof list including all proofs of different sizes. In the experiment, if $\pi_{\scriptscriptstyle j,\beta}$  is only   encrypted, then given the ciphertext, due to semantical security of the encryption, an adversary cannot tell if the ciphertext corresponds to $\pi_{\scriptscriptstyle j,0}$ or $\pi_{\scriptscriptstyle j,1}$  (accordingly to $u_{\scriptscriptstyle 0}$ or $u_{\scriptscriptstyle 1}$) with probability greater than $\frac{1}{2}+\mu_{\scriptscriptstyle 2}(\lambda)$,  if $Max(|\pi_{\scriptscriptstyle j,0}|,...,|\pi_{\scriptscriptstyle j,m}|)=|\pi_{\scriptscriptstyle j,\beta}|$. However, the  assumption is  relaxed with the use of a pad. In particular, since each encrypted proof is padded  to the proofs' maximum size, the adversary cannot tell with probability greater than $\frac{1}{2}+\mu_{\scriptscriptstyle 2}(\lambda)$ if the  padded encrypted proof corresponds to  $\pi_{\scriptscriptstyle j,0}$ or $\pi_{\scriptscriptstyle j,1}$. Also, since the value of $a$ is independent of $u_{\scriptscriptstyle 0}$  or $u_{\scriptscriptstyle 1}$, and only depends on whether the metadata is well-formed, it leaks nothing about the service input $u_{\scriptscriptstyle\beta}$ and $\beta$. Moreover, since each padded encrypted query and proof leak no information and always contains a fixed number of elements, an adversary cannot tell the status of a proof for each $j$-th verification (i.e. whether it is accepted or rejected) with the probability greater than $\frac{1}{2}+\mu_{\scriptscriptstyle 2}(\lambda)$, given  $\bm{c}^{\scriptscriptstyle *},coin^{\scriptscriptstyle *}_{\scriptscriptstyle \mathcal S},coin^{\scriptscriptstyle *}_{\scriptscriptstyle\mathcal C}, g_{\scriptscriptstyle cp},  g_{\scriptscriptstyle qp},$  $\bm{\pi}^{\scriptscriptstyle *}$, and $a$.
  \hfill\(\Box\)\end{proof}



 
 
\subsection{Reducing the Arbiter involvement}

{\color{blue}Explain how we can use the smart contracts of \cite{DongWAMM17} to delegate the arbiter task to smart contracts. In this case, only if the smart contracts do not reach a consensus the arbiter is involved. }
% !TEX root =R-zkCSP.tex



\section{Recurring Contingent PoR Payment Protocol}
In this section, we present recurring contingent PoR payment (RC-PoR-P). Since proofs of  retrievability (PoR), whose definition was given in Section \ref{PoR-def}, is a concrete instantiation of the generic verifiable service, RC-PoR-P is a concrete instantiation of the generic recurring contingent service payment (RC-S-P), when the service is PoR. Nevertheless, RC-PoR-P offers  two primary added features. Specifically, unlike RC-S-P, it (a) does not use any zero-knowledge proofs (even though either client or server can still be  malicious) which significantly improves  costs, and (b)  does not involve a third-party arbiter, which ultimately minimises trust assumption and costs.  In the following, first we  explain how the features are satisfied.


\noindent\textit{\textbf{Avoiding the Use of Zero-knowledge Proofs}}. In general, the majority of  PoR's are in the security model where a client is honest while the server is potentially malicious. They rely on metadata that is either a set of tags (e.g. MAC's or signatures) or a root of a Merkle tree,  constructed on  file blocks to ensure the file's availability. In the case where a client can  be malicious too,  if tags are used then using zero-knowledge proofs seem  an obvious choice, as it allows the client to ensure the server that the tags have been constructed correctly without leaking verification keys. However, this imposes significant computation and communication costs. We observed that using a Merkle tree would benefit our protocol from a couple of perspectives; in short, it removes the need for zero-knowledge proofs and it supports proof of misbehaviour. Our first observation is that if a Merkle tree is used to generate a metadata, then   there would be no need for the client to use zero-knowledge proofs to prove the correctness of the metadata to the server. Instead, the server can efficiently check the metadata's correctness, by reconstructing the Merkle tree on top of the file blocks. 


\noindent\textit{\textbf{Efficiently Eliminating Arbiter's Involvement}}. To totally eliminate the involvement of a third-party arbiter, one could use the Merkle tree-based PoR in RC-S-P and let a smart contract play the arbiter's entire role. In this case, there will be two  naive approaches.  One approach is that for each verification, the client verifies the server's proofs\footnote{In a Merkle tree-based PoR, the number of proofs that are sent to a verifier for each verification is linear with the number of challenges, e.g. 460 challenges  to ensure 99\% of file blocks is retrievable. In contrast, in a tag-based PoR, in each verification, the verifier receives only a few proofs.} and if it rejects them, then it sends the proofs (after the private time bubble)  to the contract who verifies the proofs again. However, this requires the smart contract to verify \emph{all} proofs which imposes a high cost. Another approach is that for each verification, the server's proofs  are always checked only by the contract, without the client's involvement. But, this requires the contract to always perform the verification, even if both parties  behaved honestly, which imposes unnecessary hight computation cost too. To  achieve our goal, we use a Merkle tree-based PoR  and let the client  verify the server's proofs locally. However, we use the idea of proof of misbehaviour, put forth in \cite{}. In particular, if the client detects invalid proofs for each verification  it only sends one invalid proof to the contract  who  checks the validity of that single proof (as apposed to checking all proofs). Thus, this eliminates the involvement of the arbiter and has a much lower cost. 



To present RC-PoR-P protocol, we will use the same approach we used to present RC-S-P. In particular, first we present the verifiable service, that is a (modified) Merkle tree-based PoR. Then, we upgrade it to the one that supports  identifiable abort, denoted by PoRID. Next,   we use PoRID to build RC-PoR-P. 

\subsection{Modified Merkle tree-based PoR}\label{protocol::PoR}

In this section, we present a modified version of the standard Merkle tree-based PoR, detonated by PoR. At a high level, the protocol works as follows. The client encodes its file using an error-correcting code, splits the encoded file into blocks, and constructs a Merkle tree on top of the blocks. It locally keeps the tree's root and sends the blocks to the server who rebuilds the tree on the blocks. At a verification time, the client sends a pseudorandom function's key to the server who derives a predetermined number of pseudorandom  indices of the blocks, that indicates which blocks have been challenged. The server for each challenged block generates a Merkle tree proof and sends all proofs to the client. The client, given the root and   key, verifies all  proofs. If all proofs are accepted, then the client outputs $1$ and concludes that its file is retrievable (with a high probability). However, if  it rejects a set of proofs, it outputs $0$ along with an index of the challenged block whose proof was rejected. In the following, we first present the protocol and then elaborate on the  modifications we have applied. 

%, i.e. a verifier outputs (the index of) a rejected proof.  Recall, that in the standard PoR protocol the client is honest and the server is potentially malicious. 

\begin{enumerate}

\item\textbf{Client-side Setup}. $\mathtt{PoR.setup}(1^{\lambda},u)$  
\begin{enumerate}
\item Uses an error correcting code, e.g. Reed-Solomon codes, to encode  the file: $u$. Let $u'$ be the encoded file. It splits $u'$  into  blocks as follows, $u^{\scriptscriptstyle *}=u^{\scriptscriptstyle '}_{\scriptscriptstyle 0}||0,...,u^{\scriptscriptstyle '}_{\scriptscriptstyle m}||m$
\item Generates  metadata: $\sigma$, by constructing Merkle tree on blocks of $u^{\scriptscriptstyle *}$, i.e. $\mathtt{MT.genTree}(u^{\scriptscriptstyle *})$. Let $\sigma$ be the root of the resulting tree, and $\beta$ be a security parameter. It sets public  parameters as $pk:= (\sigma,\beta,m,\zeta)$, where $\zeta:=(\psi,\eta, \iota)$ is a $\mathtt{PRF}$'s description, as it was defined in Section \ref{preliminaries}
\item Sends $pk$ and $u^{\scriptscriptstyle *}$ to $\mathcal{S}$
\end{enumerate}

\item\textbf{Client-side Query Generation}. $\mathtt{PoR.genQuery}(1^{\scriptscriptstyle\lambda}, pk)$ 
\begin{enumerate}
\item\label{key-chalenge} Picks a random key $\hat{k}$ of a pseudorandom function $\mathtt{PRF}$, i.e. $\hat{k}\stackrel{\scriptscriptstyle\$}\leftarrow\{0,1\}^{\scriptscriptstyle\psi}$. It ensures the function outputs distinct values, i.e. $\forall i,j\in [0,m]: (\mathtt{PRF}(\hat{k},i)\bmod m+1)\neq(\mathtt{PRF}(\hat{k},j)\bmod m+1)$, where $i\neq j$

\item It sends $\hat{k}$ to $\mathcal{S}$
 
 \end{enumerate}
\item\label{PoR-server-prove}\textbf{Server-side Proof Generation} $\mathtt{PoR.prove}(u^{\scriptscriptstyle *},\hat{k},pk)$ 
 \begin{enumerate}
\item Derives $\beta$ pseudorandom indices from $\hat{k}$ as follows. $\forall i,1\leq i\leq \beta: q_{\scriptscriptstyle i}=\mathtt{PRF}(\hat{k},i)\bmod m+1$. Let ${\bm{q}}=[q_{\scriptscriptstyle 1},..., q_{\scriptscriptstyle \beta}]$

\item For each random index $q_{\scriptscriptstyle i}$, generates a Merkle tree proof: $\pi_{\scriptscriptstyle{q_{_{\scriptscriptstyle i}}}}$, by  running Merkle tree proof generator function on $u^{\scriptscriptstyle *}$, i.e. $\mathtt{MT.prove}(u^{*},q_{\scriptscriptstyle i})$. The final result  is $ {\bm{\pi}}=[(u^{\scriptscriptstyle *}_{\scriptscriptstyle q_{_{\scriptscriptstyle i}}},\pi_{\scriptscriptstyle q_{_{\scriptscriptstyle i}}})]_{\scriptscriptstyle q_{\scriptscriptstyle i}\in {\bm{q}}}$, where $i$-th element in $ {\bm{\pi}}$ corresponds to $i$-th pseudorandom value: $q_{\scriptscriptstyle i}$ and   each $\pi_{\scriptscriptstyle{q_{_{\scriptscriptstyle i}}}}$ is path in the tree that proves its corresponding block: $u^{\scriptscriptstyle *}_{\scriptscriptstyle q_{_{\scriptscriptstyle i}}}$ is a leaf node of the  tree. 

\item Sends $ {\bm{\pi}}$ to $\mathcal{C}$
 \end{enumerate}
\item\label{PoR-verify}\textbf{Client-side Proof Verification}. $\mathtt{PoR.verify}( {\bm{\pi}},\hat{k},pk)$ 
\begin{enumerate}
\item Regenerates $\beta$ pseudorandom indices that are derived from $\hat{k}$ as follows. $\forall i,1\leq i\leq \beta: q_{\scriptscriptstyle i}=\mathtt{PRF}(\hat{k},i)\bmod m+1$. Let $ {\bm{q}}=[q_{\scriptscriptstyle 1},..., q_{\scriptscriptstyle \beta}]$
\item Checks if the server has sent proofs related to all challenged file blocks. To do that, for all $i$ (where $1\leq i\leq \beta$),  it first parses every  element of $\pi$ as follows, $\mathtt{parse}(u^{\scriptscriptstyle *}_{\scriptscriptstyle q_{_{\scriptscriptstyle i}}})=u^{\scriptscriptstyle '}_{\scriptscriptstyle q_{_{\scriptscriptstyle i}}}||q_{_{\scriptscriptstyle i}}$, and then checks if its index: $q_{_{\scriptscriptstyle i}}$ equals $i$-th element of $ {\bm{q}}$. If all checks pass, then it proceeds to the next step. Otherwise,  it outputs $ {\bm{d}}:[0,i]$, where $i$ refers to the index of the  element in $ {\bm{\pi}}$ that does not pass the check. 


\item Checks if every path in  $ {\bm{\pi}}$ is valid and corresponds to the root, by calling $\mathtt{MT.verify}(u^{\scriptscriptstyle *}_{\scriptscriptstyle q_{_{\scriptscriptstyle i}}},\pi_{\scriptscriptstyle q_{_{\scriptscriptstyle i}}},\sigma)$. If all checks pass, it outputs $ {\bm{d}}=[1,\bot]$ (where $\bot$ denotes empty); otherwise, it outputs $ {\bm{d}}:[0,i]$, where $i$ refers to the index of the  element in $ {\bm{\pi}}$ that does not pass the check. 
\end{enumerate}
\end{enumerate}



%------------------------------------------------------------
%\begin{enumerate}
%%\item \textbf{keyGen}. $\mathtt{PoR.keyGen}(1^{\lambda})$. $\mathcal{C}$ sets $pk$ as public key, initially empty.    
%\item\textbf{Client-side Setup}. $\mathtt{PoR.setup}(1^{\lambda},u,M)$  
%\begin{enumerate}
%\item Uses an error correcting code, e.g. Reed-Solomon codes, to encode  the file: $u$. Let $u'$ be the encoded file. It splits $u'$  into  blocks as follows, $u^{\scriptscriptstyle *}=u^{\scriptscriptstyle '}_{\scriptscriptstyle 0}||0,...,u^{\scriptscriptstyle '}_{\scriptscriptstyle m}||m$.
%\item Generates  metadata: $\sigma$, by constructing Merkle tree on blocks of $u^{\scriptscriptstyle *}$, i.e. $\mathtt{MT.genTree}(u^{\scriptscriptstyle *})$. Let $\sigma$ be the root of the resulting tree, and $\beta$ be a security parameter. It sets public  parameters as $pk: (\sigma,\beta,m)$.
%\item Sends $pk$ and $u^{\scriptscriptstyle *}$ to $\mathcal{S}$. 
%\end{enumerate}
%
%\item\textbf{Client-side Query Generation}. $\mathtt{PoR.genQuery}(1^{\scriptscriptstyle\lambda}, pk)$ 
%\begin{enumerate}
%\item Given the maximum block's index number: $m\in pk$, generates $\beta$ distinct random indices, where each random index is in the range $[0,m]$. 
%Let vector $ {\bm{q}}$ contain the  random indices, $q_{\scriptscriptstyle i}$
%\item It sends $ {\bm{q}}$ to $\mathcal{S}$
% 
% \end{enumerate}
%\item\label{PoR-server-prove}\textbf{Server-side Proof Generation} $\mathtt{PoR.prove}(u^{\scriptscriptstyle *}, {\bm{q}},pk)$ 
%
%For each random index $q_{\scriptscriptstyle i}$, generates a Merkle tree proof: $\pi_{\scriptscriptstyle{q_{_{\scriptscriptstyle i}}}}$, by  running Merkle tree proof generator function on $u^{\scriptscriptstyle *}$, i.e. $\mathtt{MT.prove}(u^{*},q_{\scriptscriptstyle i})$. The final result  is $ {\bm{\pi}}=[(u^{\scriptscriptstyle *}_{\scriptscriptstyle q_{_{\scriptscriptstyle i}}},\pi_{\scriptscriptstyle q_{_{\scriptscriptstyle i}}})]_{\scriptscriptstyle q_{\scriptscriptstyle i}\in {\bm{q}}}$, where $i$-th element in $ {\bm{\pi}}$ corresponds to $i$-th element in $ {\bm{q}}$; also   each $\pi_{\scriptscriptstyle{q_{_{\scriptscriptstyle i}}}}$ is path in the tree and proves its corresponding block: $u^{\scriptscriptstyle *}_{\scriptscriptstyle q_{_{\scriptscriptstyle i}}}$ is a leaf node of the  tree. $\mathcal{S}$ sends $ {\bm{\pi}}$ to $\mathcal{C}$.
%
%\item\label{PoR-verify}\textbf{Client-side Proof Verification} $\mathtt{PoR.verify}( {\bm{\pi}}, {\bm{q}},pk)$. 
%\begin{enumerate}
%\item Checks if the server has sent proofs related to all challenged file blocks. To do that,  it parses every $i$-th element of $\pi$ as: $\mathtt{parse}(u^{\scriptscriptstyle *}_{\scriptscriptstyle q_{_{\scriptscriptstyle i}}})=u^{\scriptscriptstyle '}_{\scriptscriptstyle q_{_{\scriptscriptstyle i}}}||q_{_{\scriptscriptstyle i}}$, and then checks if its index: $q_{_{\scriptscriptstyle i}}$ equals $i$-th element in $ {\bm{q}}$, where $0\leq i\leq \beta$. If all checks pass, then it proceeds to the next step. Otherwise,  it outputs $d:(0,j)$, where $j$ refers to the index of the  element in $ {\bm{\pi}}$ that does not pass the check. 
%
%
%\item Checks if every path in  $ {\bm{\pi}}$ is valid and corresponds to the root, by calling $\mathtt{MT.verify}(u^{\scriptscriptstyle *}_{\scriptscriptstyle q_{_{\scriptscriptstyle i}}},\pi_{\scriptscriptstyle q_{_{\scriptscriptstyle i}}},\sigma)$. If all checks pass, it outputs $d=1$; otherwise, it outputs $d:(0,j)$, where $j$ refers to the index of the  element in $ {\bm{\pi}}$ that does not pass the check. 
%\end{enumerate}
%\end{enumerate}
%------------------------------------------------------------

\begin{theorem}
The PoR scheme, presented in Section \ref{protocol::PoR}, is $\epsilon$-sound, w.r.t. Definitions \ref{extractable}, if Merkle tree and pseudorandom function $\mathtt{PRF}$, are secure. 

\end{theorem}


 The above protocol differs from the standard Merkle tree-based PoR from two perspectives; First and far most,  in step \ref{PoR-verify} in addition to outputting a binary value, the client outputs only one index of a rejected proof. This will enable any third-party who is given that index (and vectors of proofs and challenges) to verify the client's claim by checking only that proof, i.e. proof of misbehaviour. Second,   in step \ref{key-chalenge} instead of sending $\beta$ challenges, we allow the client to send only a key of a pseudorandom function to the server who can derive a set of challenges form it. This will ultimately  lead to a decrease in costs too, i.e. the client's communication and a smart contract's storage costs.   
 
 \begin{proof}[sketch]
 As stated above, the proposed PoR differs from the standard Merkle tree-based PoR from a couple of perspectives. However, the changes do not affect the security and soundness of the proposed PoR and its security proof is similar to the existing Merkle tree-based PoR schemes, e.g. \cite{DBLP:conf/ccs/HaleviHPS11,MillerPermacoin,DBLP:journals/iacr/JuelsK07}. Alternatively, our protocol can be proven based on the security analysis of the PoR schemes that use  MACs or BLS signatures, e.g. \cite{DBLP:conf/asiacrypt/ShachamW08}. In this case, the extractor design (in the Merkle tree-based PoR)  would be simpler as it does not need to extract blocks from a linear combination of MAC's or signatures, as the  blocks are included in PoR proofs, i.e. they are part of the Merkle tree proofs. 
 
 Intuitively, in either case, the extractor interacts with any adversarial prover that passes non-negligible $\epsilon$ fraction of audits. It initialises an empty array. Then it  challenges a subset of file blocks and asks the prover to generate a proof.  If the received proof passes the verification, then it adds the related block (in the proof) to the array. It then rewinds the prover and challenges a fresh set of blocks, and repeats the process  many times. Since, the prover has a good chance of passing the audit, it is easy to show that the extractor can eventually extract a large fraction of the entire file. Due to the security, i.e. authenticity, of the Merkle tree, the retrieved values are the valid and correct file blocks and due to security of the pseudorandom function, the challenges (or the function's outputs) are not predictable. After collecting sufficient number of blocks, the extractor can use the error correcting code to decode and recover the entire file blocks, given the retrieved ones. 
 \hfill\(\Box\)\end{proof}
   
   \begin{remark}
   Recall,  the generic definition of a verifiable service scheme (i.e. Definition \ref{service-def}) involves three algorithms: $F$, $M$, and $Q$. However, the three algorithms are implicit in the original definition of  PoR and accordingly in  PoR protocols. In the following, we explain how each algorithm is defined in PoR context.  $M$ is an algorithm that processes a file and generates metadata. For instance, when PoR uses a Merkle tree (to ensure the file's integrity and availability), then $M$ refers to the Merkle tree's algorithm that constructs a tree on top of the file blocks. Also,  $F$ is an algorithm that, during generating a PoR proof, processes a subset of the outsourced file, given the client's query (or challenged file blocks). For instance, if a PoR  utilises a Merkle tree, then $F$ refers to the algorithm that generates Merkle tree's proofs, i.e.  membership of the challenged file blocks. Furthermore, $Q$ can be  a pseudorandom function that generates a set of pseudorandom strings in a certain range, e.g. file block's indices.   
   \end{remark}


% !TEX root =R-zkCSP.tex


\subsection{PoRID Protocol}

In this section, we propose ``PoR with identifiable abort'' (PoRID) that is a concrete instantiation of $\text{VSID}_{\scriptscriptstyle\text{light}}$. It is built upon the PoR protocol, presented in the previous section and is in the same security model as $\text{VSID}_{\scriptscriptstyle\text{light}}$ is, i.e. either $\mathcal C$ or $\mathcal S$ can be malicious. In PoRID similar to $\text{VSID}_{\scriptscriptstyle\text{light}}$, $\mathcal C$ and $\mathcal S$ use a bulletin board to exchange signed messages.  In the protocol, at setup $\mathcal C$ encodes its file and generates public parameters and  metadata.  It posts the public parameters and metadata to the bulletin board and   sends the encoded file to $\mathcal S$ who runs a few lightweight checks to ensure the correctness of the public parameters and metadata. It agrees to serve, if it is convinced of their correctness. Later, when $\mathcal C$ wants to ensure the availability of its outsourced file, it  generates and posts a query  to the board.   $\mathcal S$  checks the correctness of the query, by performing a couple of highly efficient verifications. 

The server-side prove and client-side verify algorithms are similar to those in PoR with a difference that $\mathcal S$ posts the PoR proofs (i.e. output of prove algorithm) to the board. In  case of any dispute, $\mathcal C$ or $\mathcal S$ invokes the arbiter who, given the signed posted messages, checks the proofs  to identify a corrupt party. In particular,  it first checks the validity of the query (regardless of the party who invokes it). However, if  is invoked by  $\mathcal C$, it also checks  only one of the PoR proofs that the client claims it is invalid. Thus, it is much more efficient than $\text{VSID}_{\scriptscriptstyle\text{light}}$ as it does not need any zero-knowledge proofs (mainly due to the use of Merkle tree) and requires the  arbiter to check only one of the proofs (due  to the idea of proof of misbehaviour). PoRID protocol is presented below.




%In this section, we propose PoR with identifiable abort: PoRID. Here we consider the light version of  PoRID due to its efficiency. 

%However, if $M$ takes as inputs private parameters, then the client commits to the parameters and  


\begin{enumerate}
%\item\textbf{Key Generation}. $\mathtt{VSID.KeyGen}(1^{\lambda},F)$ 
%\begin{enumerate}
%\item Calls  $\mathtt{VS.KeyGen}(1^{\lambda},F)$ to generate a pair of secret and public keys, $k:(sk,pk)$.
%%\item Commits to the secret keys: $sk\in k$. This yields: $\mathtt{Com}(sk)$.
%\item Posts $pk$ to a bulletin board. 
%\end{enumerate}
%\item \textbf{keyGen}. $\mathtt{PoR.keyGen}(1^{\lambda})$. $\mathcal{C}$ sets $pk$ as public key, initially empty.    
\item\textbf{Client-side Setup}. $\mathtt{PoRID.setup}(1^{\lambda}, u)$
\begin{enumerate}
\item Calls  $\mathtt{PoR.setup}(1^{\lambda}, u)\rightarrow (u^{\scriptscriptstyle *},pk)$, that results in public parameters $pk:=(\sigma,\beta,m,\zeta)$ and encoded file: $u^{\scriptscriptstyle *}=u^{\scriptscriptstyle '}_{\scriptscriptstyle 0}||0,...,u^{\scriptscriptstyle '}_{\scriptscriptstyle m}||m$. Recall,  $\zeta:=(\psi,\eta, \iota)$ is the $\mathtt{PRF}$'s description.

\item Posts $pk$ to the bulletin board and sends $u^{\scriptscriptstyle *}$ to $\mathcal{S}$ 
\end{enumerate}

\item\textbf{Server-side Setup}. $\mathtt{PoRID.serve}(u^{\scriptscriptstyle *},pk)$
%\begin{enumerate}

Verifies the correctness of public parameters:
\begin{enumerate}
\item rebuilds  the Merkle tree on $u^{\scriptscriptstyle *}$ and checks the resulting root equals $\sigma$
\item checks $|u^{*}|=m$ and  $\beta\leq m$
\end{enumerate}
 If the proofs are accepted, then it proceeds to the next step; otherwise, it halts.
 
%\item Posts $1$ to the board,  if the above checks pass; otherwise, it posts $0$ to the board.  

%\end{enumerate}

\item\textbf{Client-side Query Generation}. $\mathtt{PoRID.genQuery}(1^{\lambda},  pk)$ 
\begin{enumerate}
\item Calls $\mathtt{PoR.genQuery}(1^{\lambda}, pk)\rightarrow \hat{k}$, to generate a key, $\hat{k}$ %whose element have been sorted in ascending order. 
%\item Sorts the queries in ascending order. Let vector $ {\bm{q}}'$ contain the ordered random indices (i.e.  $ {\bm{q}}'[i]< {\bm{q}}'[i+1]$, where $i\in [0,\beta-1]$).
\item Posts $\hat{k}$ to the board. 
\end{enumerate}




\item\textbf{Server-side Query Verification}. $\mathtt{PoRID.checkQuery}(\hat{k}, pk)$
\begin{enumerate}
\item  Checks if  $\hat{k}$ is not empty, i.e. $\hat{k}\neq \bot$, and is in the key's universe, i.e. $\hat{k}\in \{0,1\}^{\scriptscriptstyle\psi}$%If it is empty, then it outputs $I=\mathcal C$
\item  If the checks pass, then it outputs $b=1$; otherwise, it outputs $b=0$
% If the check passes, then it proceeds to the next step; otherwise, it aborts.

%\item Verifies if  the query: $ {\bm{q}}$ has been constructed correctly as follows. 
\end{enumerate}
%
%\begin{enumerate}
%\item  checks the query vector size: $| {\bm{q}}|\stackrel{\scriptscriptstyle ?}=\beta$. If the equation does not hold, then it posts  $b=0$ to the board and aborts. Otherwise, it proceeds to the next step. 
%\item verifies that the elements of $ {\bm{q}}$ are in ascending order, by checking if there are two consecutive elements of the vector: $(q_{\scriptscriptstyle i},q_{\scriptscriptstyle i+1})\in  {\bm{q}}$, such that: $q_{\scriptscriptstyle i}\not<q_{\scriptscriptstyle i+1}$.  If the pair is found, then it posts  $b:(0,i)$ to the board and aborts. Otherwise, it proceeds to the next step. 
%\end{enumerate}
%\item Posts $b=1$ to the board, if the above two checks pass.  
%\end{enumerate}

\item\textbf{Server-side Service Proof Generation}. $\mathtt{PoRID.prove}(u^{\scriptscriptstyle *},\hat{k},pk)$ 
\begin{enumerate}
\item Calls $\mathtt{PoR.prove}(u^{\scriptscriptstyle *},\hat{k},pk)\rightarrow  {\bm{\pi}}$, to generate proof vector: $ {\bm{\pi}}$ 
\item Posts $ {\bm{\pi}}$ to the board. 
\end{enumerate}

\item\textbf{Client-side Proof Verification}. $\mathtt{PoRID.verify}( {\bm{\pi}},\hat{k} ,pk)$
%\begin{enumerate}

 Calls $\mathtt{PoR.verify}( {\bm{\pi}},\hat{k} ,pk)\rightarrow  {\bm{d}}$, to verify the proof. If $ {\bm{d}}[0] = 1$,  it accepts the proof; otherwise, it rejects it.
%\item Posts $  {\bm{d}}$ to the board. 
%\end{enumerate}

\item\textbf{Arbiter-side Identification}. $\mathtt{PoRID.identify}( {\bm{\pi}}, g,\hat{k},pk)$ 

This algorithm can be invoked by $\mathcal{C}$ or $\mathcal{S}$, in the case of dispute. If it is invoked by $\mathcal{C}$, then $g$ refers to a rejected proof's  index; however, if it is invoked by $\mathcal{S}$, then $g$ is null, i.e.  $g=\bot$. The arbiter performs as follows. 


\begin{enumerate}

%\item Calls $\mathtt{VSID.checkQuery}(c, pk)\rightarrow b$. If $b=1$, then it proceeds to the next step. Otherwise, it outputs $I=\mathcal C$ and aborts. 
%\item   Ensures $ {\bm{d}}[0]=0$; otherwise, it aborts.

\item Ensures  query $\hat{k}$ is well-structured by calling $\mathtt{PoRID.checkQuery}(\hat{k}, pk)$. If it returns $b=0$, then it outputs $I=\mathcal C$ and halts; otherwise, it proceeds to the next step. 


%not empty (i.e. $\hat{k}\neq \bot$). If it is empty, then it outputs $I=\mathcal C$

\item Derives the related challenged block's index from $\hat{k}$, by computing $q_{\scriptscriptstyle g}=\mathtt{PRF}(\hat{k},g)\bmod m+1$ 



\item If $g\neq \bot$, then verifies only $g$-th proof, by calling $\mathtt{PoR.verify}( {\bm{\pi}}[g],q_{\scriptscriptstyle g},pk)\rightarrow  {\bm{d}}'$. If $ {\bm{d}}'[0]=0$, then it outputs $I=\mathcal S$. Otherwise, it outputs $I=\bot$
\end{enumerate}
\end{enumerate}

\begin{theorem}
The PoRID protocol satisfies the $\epsilon$-soundness, inputs well-formedness, and detectable abort properties, w.r.t. Definitions \ref{extractable}, \ref{deff::VSID-Inputs-Well-formedess}, and \ref{def::VSID-Identifiable-Abort}, if PoR is $\epsilon$-sound and the Merkle tree and signature schemes are secure.
\end{theorem}


\begin{proof}[sketch]
The $\epsilon$-soundness of PoRID directly stems from the security of PoR scheme, i.e. $\epsilon$-soundness. Specifically,
in PoRID the (honest) client makes block-box calls to the algorithms of PoR, to ensure the soundness. The latter scheme's soundness ensures that an extractor can recover the entire file interacting with a corrupt server who passes  $\epsilon$ fraction of challenges.




 \hfill\(\Box\)\end{proof}


% !TEX root =R-zkCSP.tex

\subsection{Recurring Contingent PoR Payment (RC-PoR-P) Protocol}
This section presents recurring contingent PoR payment (RC-PoR-P) protocol. It is built upon PoRID protocol and is in the same security model as RC-S-P is. RC-PoR-P inherits the features of PoRID and RC-PoR-P; however, unlike RC-PoR-P, it does not use any zero-knowledge proofs and there is no third-party arbiter involved. Even though RC-PoR-P and RC-S-P have some overlaps, they have many differences too. Therefore, we provide the protocol’s overview and its detailed description below.
At a high level the protocol works as follows. The client and server utilise SAP to provably agree on two private statements, one statement includes payment details, and another one specifies a secret key, $k$, and a pad's length. Moreover, they  agree on public parameters such as the private time bubble's length (that is the total number of billing cycles: $z$, plus a waiting period, $H$) and a smart contract that specifies $z$ and the total amount of masked coins each party should deposit. They deploy the contract. Each party deposits its masked coins in the contract within a fixed time. If any party does not deposit enough coins on time, then the parties have a chance to withdraw their coins and terminate the contract after a certain time. To start using/providing the service, the client invokes $\mathtt{PoRID.setup}()$ to encode the file and generate  metadata and public parameters. It sends  encryption of the metadata and public parameters to the smart contract and sends the encoded file to the server who decrypts them and using the encoded file checks their correctness by calling $\mathtt{PoRID.server}()$. If the server decides not to serve, it sends to the contract $0$ within a fixed time; in this case, the parties can withdraw their deposit and terminate the contract. Otherwise, the server sends $1$ to the contract. 
At the end of each billing cycle, the client generates an encrypted query, by calling $\mathtt{PoRID.genQuery()}$ and encrypting its output using the key, $k$. It sends the result to the contract. In the same cycle, the server retrieves the query, and decrypts it. Then, it locally checks its correctness, by calling $\mathtt{PoRID.checkQuery()}$. If the query is rejected, the server locally stores the index of that billing cycle and generates a dummy PoR proofs. However, if the server accepts the query, it generates PoR proofs by calling $\mathtt{PoRID.prove()}$. Then, in either case, the server encrypts the proofs, pads them and sends the result to the contract. After that, the client removes the pads, decrypts the proofs and locally verifies them, by calling $\mathtt{PoRID.verify()}$. If the verification is passed, then the client knows the file is retrievable with a high probability. But, if the proof is rejected, then it locally stores the index of that billing cycle and waits until the private time bubble passes and dispute resolution time arrives. 

During the dispute resolution period, in case the server rejects the query or the client rejects the PoR proofs, that party sends to the contract (a) the indices of the billing cycles in which its counterparty provided invalid values, and (b)  the statement that contains the decryption key and padding detail. The contract checks the validity of the statement first. If it accepts the statement, then it removes the pads and decrypts the values whose indices were provided by the parties. Then, the contract checks the party’s claim by calling $\mathtt{PoRID.checkQuery ()}$ and $\mathtt{PoRID.identify ()}$ if the server or client calls the contract respectively. The contract also keeps track of the number of times each party provided invalid queries or PoR proofs.  
In the next phase, to distribute the coins, either client or server sends to the contract: (a) “pay” message, (b) the agreed statement that specifies the payment details, and (c) the statement’s proof. The contract verifies the statement and if it is approved, then the contract distributes the coins according to the statement’s detail, and the number of times each party misbehaved. 





\begin{enumerate}
\item \textbf{Key Generation}. 
\begin{enumerate}

\item  $\mathcal C$  picks a random secret key  $\bar{k}$ for a symmetric key encryption. Also, it sets  parameter $pad_{\scriptscriptstyle \pi}$ which is the number of  dummy values that will be used to pad encrypted proofs,  let $qp:(pad_{\scriptscriptstyle \pi}, \bar{k})$. The key's size is part of the security parameter. 




\item  $\mathcal C$ sets coin parameters as   follows,   $o$: the amount of coins for each accepting proof,  as well as $l$ and $l'$ which are the amount of coins $\mathcal C$ and $\mathcal S$ respectively need to send to a smart contract to resolve a potential dispute.  Let $k':\{o,l,l',qp\}$. 


\end{enumerate}

\item \textbf{Initiation}. 
\begin{enumerate}

\item For $\mathcal C$ and $\mathcal S$ to provably agree on $qp$, $\mathcal C$ sends $qp$ to  $\mathcal S$.  Next (if the $\mathcal S$ agrees on the parameters) they take the steps in the Setup and Agreement phases in the SAP, at time $T_{\scriptscriptstyle 0}$. Let $t_{\scriptscriptstyle qp}=(\ddot{x}_{\scriptscriptstyle qp},g_{\scriptscriptstyle qp})$ be proof encoding token,  where  $\ddot{x}_{\scriptscriptstyle qp}$ is the opening and $g_{\scriptscriptstyle qp}$ is the commitment stored on the contract as a result of running   SAP. 


\item Let  $cp:(o,o_{\scriptscriptstyle max},l,l',l_{\scriptscriptstyle max},z)$, where $o_{\scriptscriptstyle max}$ is the maximum amount of coins  for an accepting service proof,  $l_{\scriptscriptstyle max}$  is the maximum amount of coins to resolve a potential dispute, and $z$ is the number of  service proofs/verifications. For $\mathcal C$ and $\mathcal S$ to provably agree on $cp$, similar to the previous step, they invoke SAP, at time $T_{\scriptscriptstyle 1}$.  Let $t_{\scriptscriptstyle cp}=(\ddot{x}_{\scriptscriptstyle cp},g_{\scriptscriptstyle cp})$ be coin encoding token,  where  $\ddot{x}_{\scriptscriptstyle cp}$ is the opening and $g_{\scriptscriptstyle cp}$ is the commitment stored on the contract as a result of executing the  SAP. 


\item $\mathcal C$ sets   $coin^{\scriptscriptstyle *}_{\scriptscriptstyle\mathcal C}=z\cdot (o_{\scriptscriptstyle max}+l_{\scriptscriptstyle max})$ and $coin^{\scriptscriptstyle *}_{\scriptscriptstyle\mathcal S}=z\cdot l_{\scriptscriptstyle max}$, where $coin^{\scriptscriptstyle *}_{\scriptscriptstyle\mathcal C}$ and $coin^{\scriptscriptstyle *}_{\scriptscriptstyle\mathcal S}$ are   the total number of masked coins that $\mathcal C$ and $\mathcal S$ should deposit respectively. $\mathcal C$ signs and deploys  a smart contract, SC, that explicitly specifies  parameters $z$,  $coin^{\scriptscriptstyle *}_{\scriptscriptstyle\mathcal C}$ and $coin^{\scriptscriptstyle *}_{\scriptscriptstyle\mathcal S}$. It deposits $coin^{\scriptscriptstyle *}_{\scriptscriptstyle\mathcal C}$ coins in the contract.

\item $\mathcal C$ constructs  vector $\bm{w}_{\scriptscriptstyle\mathcal C}$, also $\mathcal S$  constructs $\bm{v}_{\scriptscriptstyle\mathcal S}$,  where the vectors are initially empty. 


\item $\mathcal C$ runs $\mathtt{PoRID.setup}(1^{\lambda}, u)\rightarrow (u^{\scriptscriptstyle *},pk)$.  It sends  encoded file $u^{\scriptscriptstyle *}$ to $\mathcal S$, and sends the public key's encryption: $\mathtt{Enc}(\bar{k},pk)$ to SC at time $T_{\scriptscriptstyle 2}$

\item $\mathcal S$ checks the above parameters, and ensures sufficient amount of coins has been deposited.  If any check is rejected, then it sets $a=0$. Otherwise, it decrypts the public key, $\mathtt{Dec}(\bar{k},\mathtt{Enc}(\bar{k},pk))=pk$. It runs $\mathtt{PoRID.serve}(u^{\scriptscriptstyle *},pk)\rightarrow a$. Next, it sends $a$ and $coin^{\scriptscriptstyle *}_{\scriptscriptstyle\mathcal S}$ coins to SC at time $T_{\scriptscriptstyle 3}$, where  $coin^{\scriptscriptstyle *}_{\scriptscriptstyle\mathcal S}=\bot$ if $a=0$



\item $\mathcal C$  and  $\mathcal S$ can withdraw their coins at time $T_{\scriptscriptstyle 4}$, if the server sends $a=0$, fewer coins than $coin^{\scriptscriptstyle *}_{\scriptscriptstyle\mathcal S}$, or nothing to the SC


%
%
%\item $\mathcal C$ constructs  vector  $ {\bm{w}}_{\scriptscriptstyle\mathcal{C}}$ and $\mathcal S$  constructs $ {\bm{v}}_{\scriptscriptstyle\mathcal S}$,  where the vectors are initially empty.   
%
%\item $\mathcal C$ and $\mathcal S$ agree on the following parameters, $z$: the number of the service proofs/verifications, $o$: the amount of coins for each accepting proof, $l$ and $l'$ which are the amount of coins $\mathcal C$ and $\mathcal S$ respectively need to send to a smart contract to resolve a dispute. Furthermore, they agree on $f_{\scriptscriptstyle\mathcal C}$ and $f_{\scriptscriptstyle\mathcal S}$ that are the amount of masking coins $\mathcal C$ and $\mathcal S$ will use respectively. 


%where $l_{\scriptscriptstyle\mathcal C}$ and $l_{\scriptscriptstyle\mathcal S}$ cover the costs of a light verification of the client's and server's inputs respectively, while $l'_{\scriptscriptstyle\mathcal S}$ covers the cost of a more involved verification of the server's input.  

%\item $\mathcal C$ and $\mathcal S$ construct  a statement of the  form: $x=(o,l,l', f_{\scriptscriptstyle\mathcal C},f_{\scriptscriptstyle\mathcal S},z)$   
%\item For $\mathcal C$ and $\mathcal S$ to provably agree on $x$, they take the steps in the Setup and Agreement phases in the SAP, at time $T_0$. Let $\mu$ be the statement's proof. 
%


%\item $\mathcal S$ picks and sends to $\mathcal C$ a random key, $\bar{k}$

%\item  $\mathcal C$ sends to $\mathcal S$ a statement of the form $x'=({pad},\bar{k})$ where  ${pad}$  specifies the number of  dummy values that will be used to pad encrypted proofs and    $\bar{k}$ is a  secret key.
%
%
%
%\item For $\mathcal C$ and $\mathcal S$ to provably agree on $x'$, they take the steps in the Setup and Agreement phases in the SAP again, at time $T_{\scriptscriptstyle 1}$. Let $\mu'$ be the proof. 



%\item For $\mathcal C$ and $\mathcal S$ to provably agree on $\bar{k}$, they take the steps in the Setup and Agreement phases in the SAP again, at time $T_{\scriptscriptstyle 1}$. Let $\mu'$ be the proof. 

%
%\item $\mathcal C$ and $\mathcal S$ agree on a smart contract: SC. Let $p_{\scriptscriptstyle\mathcal C}=z(o+l')+f_{\scriptscriptstyle\mathcal C}$ and $p_{\scriptscriptstyle\mathcal S}=zl+f_{\scriptscriptstyle\mathcal S}$ be the total number of coins the client and server should deposit respectively. SC  explicitly specifies  parameters $z$, $p_{\scriptscriptstyle\mathcal C}$ and $p_{\scriptscriptstyle\mathcal S}$. They sign and deploy the contract. 
%
%\item $\mathcal C$ and $\mathcal S$ deposit their masked coins in SC at time $T_{\scriptscriptstyle 2}$
%\item At time $T_{\scriptscriptstyle 3}$, if the deposit amount that either party has put is less than the amount stated in SC, then SC allows each party to withdraw its
%coins and terminate the contract.
%
%
%
%\item $\mathcal S$ decrypts the public key, $\mathtt{Dec}(\bar{k},\mathtt{Enc}(\bar{k},pk))=pk$. Next, it runs $\mathtt{PoRID.serve}(u^{\scriptscriptstyle *},pk)\rightarrow a$ and then sends $a$  to SC at time $T_{\scriptscriptstyle 4}$
%
%\item $\mathcal C$ and $\mathcal S$ can withdraw their coins at time $T_{\scriptscriptstyle 5}$, if the server sends $a=0$ or nothing to the SC 

\end{enumerate}
%\item \textbf{Billing-cycles}. At each time $G_{\scriptscriptstyle j}$, (where $1\leq j\leq z$ and $G_{\scriptscriptstyle 1}>T_{\scriptscriptstyle 5}$), that a proof is generated, the parties perform as follows. 

\item \textbf{Billing-cycles}. The parties do the following, at the end of  every $j$-th billing cycle, where $1\leq j\leq z$. Each $j$-th cycle includes two  time points, $G_{\scriptscriptstyle j,1}$ and $G_{\scriptscriptstyle j,2}$, where $G_{\scriptscriptstyle j,2}>G_{\scriptscriptstyle j,1}$, and $G_{\scriptscriptstyle 1,1}>T_{\scriptscriptstyle 4}$ 

%At each time $G_{\scriptscriptstyle j}$, (where $1\leq j\leq z$ and $G_{\scriptscriptstyle 1}>T_{\scriptscriptstyle 5}$), that a proof is generated, the parties perform as follows. 
\begin{enumerate}

\item $\mathcal C$ calls $\mathtt{PoRID.genQuery}(1^{\lambda},  pk)\rightarrow \hat{k}_{\scriptscriptstyle j}$, to generate a query. It sends   $\hat{k}_{\scriptscriptstyle j}$ to SC at time $G_{\scriptscriptstyle j,1}$

%\item $\mathcal S$  appends $j$ to $ {\bm{v}}_{\scriptscriptstyle\mathcal{S}}$, if the client's query was not posted on time. In this case, it does not take the next steps, for  $j$-th verification. 


%\item If $\mathcal S$ accepts the query, i.e. $b_{\scriptscriptstyle j}=1$, then it calls $\mathtt{VSID.prove}(u^{\scriptscriptstyle *},\sigma,c_{\scriptscriptstyle j},pk)\rightarrow \pi_{\scriptscriptstyle j}$, to generate the service proof. In this case, $\mathcal S$ encrypts the proof, $\mathtt{Enc}(\bar{k},\pi_{\scriptscriptstyle j})=\pi'_{\scriptscriptstyle j}$. Next, it   pads the encrypted proof with ${pad}_{\scriptscriptstyle 2}$ random values that are picked from $U$. It sends the padded encrypted proof  to SC at time $G_{\scriptscriptstyle j,2}$. Otherwise (if $\mathcal S$ rejects the query), it appends $j$  to $ {\bm{v}}_{\scriptscriptstyle\mathcal{S}}$,  constructs a dummy proof $\pi'_{\scriptscriptstyle j}\in U$,  pads the result as above, and sends the padded dummy proof to SC at time $G_{\scriptscriptstyle j,2}$.


%\item $\mathcal S$ decrypts the query,  $\mathtt{Dec(}\bar{k},\mathtt{Enc}(\bar{k},\hat{k}_{\scriptscriptstyle j}))=\hat{k}_{\scriptscriptstyle j}$.   Then, it calls $\mathtt{PoRID.checkQuery}(\hat{k}_{\scriptscriptstyle j}, pk)\rightarrow b_{\scriptscriptstyle j}$ to check the query's correctness. 


\item $\mathcal S$ calls $\mathtt{PoRID.checkQuery}(\hat{k}_{\scriptscriptstyle j}, pk)\rightarrow b_{\scriptscriptstyle j}$ to check the query's correctness. 

\begin{itemize}
\item[$\bullet$] If it accepts the query, then it calls $\mathtt{PoRID.prove}(u^{\scriptscriptstyle *},\hat{k}_{\scriptscriptstyle j},pk)\rightarrow  {\bm{\pi}}_{\scriptscriptstyle j}$, to generate a PoR  proof. In this case,  $\mathcal S$ encrypts every  proof in the proof vector, i.e. $\forall g, 1\leq g\leq| {\bm{\pi}}_{\scriptscriptstyle j}|: \mathtt{Enc}(\bar{k}, {\bm{\pi}}_{\scriptscriptstyle j}[g])= {\bm{\pi}}'_{\scriptscriptstyle j}[g]$. Let vector $ {\bm{\pi}}'_{\scriptscriptstyle j}$ contain the encryption of all proofs. It pads every encrypted proof in $ {\bm{\pi}}'_{\scriptscriptstyle j}$ with $pad_{\scriptscriptstyle \pi}$ random values that are picked from the encryption's output range $U$, (by appending the random values to the encrypted proofs vector). It sends the padded encrypted proofs  to SC at time $G_{\scriptscriptstyle j,2}$ 
\item[$\bullet$] Otherwise (if $\mathcal S$ rejects the query), it appends $j$  to $ {\bm{v}}_{\scriptscriptstyle\mathcal{S}}$,  constructs a dummy proof $ {\bm{\pi}}'_{\scriptscriptstyle j}$ whose elements are randomly picked from $U$,  pads the result as above, and sends the padded dummy proof to SC at time $G_{\scriptscriptstyle j,2}$

\end{itemize}

When $j=z$  and $\bm{v}_{\scriptscriptstyle\mathcal{S}}\neq\bot$, $\mathcal S$ sets    $\bm{m}_{\scriptscriptstyle\mathcal S}:[\bm{v}_{\scriptscriptstyle\mathcal{S}},\ddot{x}_{\scriptscriptstyle qp}, \text{``dispute''}]$.




%\item  $\mathcal C$ appends $j$ to $ {\bm{v}}_{\scriptscriptstyle\mathcal{C}}$, if  $\mathcal S$ does not post the proof on time. In this case, it does not take the next steps, for  $j$-th verification. 


\item In this phase,  $\mathcal C$ verifies the service proof. It first constructs two empty vectors, $\bm{m}_{\scriptscriptstyle\mathcal C}=\bot$ and $\bm{w}_{\scriptscriptstyle\mathcal{C}}=\bot$. Next, it removes the pads from the padded encrypted proofs and then
 decrypts the encrypted proofs: $\mathtt{Dec}(\bar{k}, {\bm{\pi}}'_{\scriptscriptstyle j})= {\bm{\pi}}_{\scriptscriptstyle j}$. Then, it   calls $\mathtt{PoRID.verify}( {\bm{\pi}}_{\scriptscriptstyle j},\hat{k}_{\scriptscriptstyle j},pk)\rightarrow  {\bm{d}}_{\scriptscriptstyle j}$, to verify them. If $ {\bm{\pi}}_{\scriptscriptstyle j}$ passes the verification, i.e. $ {\bm{d}}_{\scriptscriptstyle j}[0]=1$, then $\mathcal C$  concludes that the service for this verification has been delivered successfully. Otherwise (if proof $ {\bm{\pi}}_{\scriptscriptstyle j}$ is rejected, i.e. $ {\bm{d}}_{\scriptscriptstyle j}[0]=0$), then it sets $g= {\bm{d}}_{\scriptscriptstyle j}[1]$ and appends vector $[j,g]$  to $ {\bm{w}}_{\scriptscriptstyle\mathcal{C}}$. Recall, $ {\bm{d}}_{\scriptscriptstyle j}[1]$ refers to a rejected proof's index in proof vector $ {\bm{\pi}}_{\scriptscriptstyle j}$. If $j=z$  and $\bm{w}_{\scriptscriptstyle\mathcal{C}}\neq\bot$, $\mathcal C$ sets    $\bm{m}_{\scriptscriptstyle\mathcal C}:[\bm{w}_{\scriptscriptstyle\mathcal{C}},\ddot{x}_{\scriptscriptstyle qp}, \text{``dispute''}]$.
  

\end{enumerate}
\item \textbf{Dispute Resolution}. The phase takes place only in  case of dispute, i.e. when $\mathcal C$ rejects service proofs or $\mathcal S$ rejects the queries. %or either party does not receive its counterparty's message on time. %A party who accepts all proofs does not need to get involved in this phase. 

\begin{enumerate}

%\item $\mathcal C$  at time $K_{\scriptscriptstyle 1}>G_{\scriptscriptstyle z,2}+H$ sends ``dispute'' message to SC 

%\item $\mathcal S$ sends $ {\bm{v}}_{\scriptscriptstyle\mathcal{S}}$ to SC, at time $K_{\scriptscriptstyle 2}$ 


%\item SC  ensures $ {\bm{v}}_{\scriptscriptstyle\mathcal{S}}$ is well-formed. In particular, it removes from  $ {\bm{v}}_{\scriptscriptstyle\mathcal{S}}$ duplicated elements or  elements that are not in the range $[1,z]$. SC for every  element $i\in  {\bm{v}}_{\scriptscriptstyle\mathcal{S}}$ checks if $\mathcal C$  sent to SC an encrypted query on time, in step \ref{Billing-cycles-genQuery};  if the client did not do so, then it appends $i$ to $ {\bm{v}}_{\scriptscriptstyle{SC}}$ and   increments $u_{\scriptscriptstyle\mathcal C}$ and $y_{\scriptscriptstyle\mathcal C}$ by $1$
 
 
 % and retakes step \ref{the-arbiter-loop} with the next element of $ {\bm{v}}$. Otherwise, it proceeds to the next step. 

\item $\mathcal S$  sends $\bm{m}_{\scriptscriptstyle\mathcal S}$ to SC, at time $K_{\scriptscriptstyle 1}$, where $K_{\scriptscriptstyle 1}>G_{\scriptscriptstyle z,2}+H$

\item SC upon receiving $\bm{m}_{\scriptscriptstyle\mathcal S}$ does the following. 
\begin{enumerate}

\item Sets two  counters: $y_{\scriptscriptstyle\mathcal C}$ and $y_{\scriptscriptstyle\mathcal S}$  that are initially set to $0$. Also, it constructs an empty vector $ {\bm{v}}$ 
\item Checks the validity of  statement $\ddot{x}_{\scriptscriptstyle qp}\in \bm{m}_{\scriptscriptstyle\mathcal S}$, by  sending it to SAP contract which  returns   $1$ or $0$. If the output is $0$, then SC discards the server's complaint, $\bm{m}_{\scriptscriptstyle\mathcal S}$, and does not take steps \ref{sc-dec-server-key}-\ref{sc-check-server-claim}. Otherwise, it proceeds to the next step.

\item\label{sc-dec-server-key} Uses secret key $\bar{k}\in \ddot{x}_{\scriptscriptstyle qp}$ to decrypt the encrypted public key, $\mathtt{Dec}(\bar{k},\mathtt{Enc}(\bar{k},pk))=pk$

\item Removes from $ {\bm{v}}_{\scriptscriptstyle\mathcal{S}}$ any element that is duplicated or is not in the range $[1,z]$ 

\item\label{sc-check-server-claim}  For any element $i\in {\bm{v}}_{\scriptscriptstyle\mathcal{S}}$:
\begin{itemize}
\item[$\bullet$] Fetches the related query, $\hat{k}_{\scriptscriptstyle i}$,   from SC 
\item[$\bullet$]Checks if  the query is well-formed, by calling $\mathtt{PoRID.checkQuery}(\hat{k}_{\scriptscriptstyle i}, pk)\rightarrow b_{\scriptscriptstyle i}$. If the query is rejected, i.e. $ b_{\scriptscriptstyle i}=0$, then it increments $y_{\scriptscriptstyle\mathcal C}$ by 1 and appends $i$ to $ {\bm{v}}$
\end{itemize}
Let $K_{\scriptscriptstyle 2}$ be the time SC finishes the above checks.
\end{enumerate}





\item $\mathcal C$  sends $\bm{m}_{\scriptscriptstyle\mathcal C}$ to SC, at time $K_{\scriptscriptstyle 3}$
 %Also,  $\mathcal S$ sends $m_{\scriptscriptstyle\mathcal S}:( {\bm{v}}_{\scriptscriptstyle\mathcal{S}},\bar{k},\mu')$ to SC %,   where $| {\bm{v}}_{\scriptscriptstyle\mathcal{C}}|\neq 0$ and $| {\bm{v}}_{\scriptscriptstyle\mathcal{S}}|\neq 0$. 

%\item\label{check-client-claim} SC  ensures $ {\bm{v}}_{\scriptscriptstyle\mathcal{C}}$ is well-formed. Specifically, it  removes from  $ {\bm{v}}_{\scriptscriptstyle\mathcal{C}}$ any element that is duplicated, or   not in the range $[1,z]$.  Also, it removes from $ {\bm{v}}_{\scriptscriptstyle\mathcal{C}}$ the elements that are in $ {\bm{v}}_{\scriptscriptstyle{SC}}$. Next, SC  for every  element $i\in  {\bm{v}}_{\scriptscriptstyle\mathcal{C}}$ checks if $\mathcal S$  sent to SC an encrypted query on time, in step \ref{Billing-cycles-genQuery} and increments $u_{\scriptscriptstyle\mathcal S}$ and $y_{\scriptscriptstyle\mathcal S}$ by $1$  if the server did not do that.  It appends those $i$'s that pass the check to $ {\bm{v}}'_{\scriptscriptstyle{SC}}$. Note, $ {\bm{v}}'_{\scriptscriptstyle{SC}}$ contains the indices of verifications whose PoR proofs  were rejected by the client.


\item SC upon receiving $\bm{m}_{\scriptscriptstyle\mathcal C}$, does the following. 
\begin{enumerate}
\item Checks the validity of  statement $\ddot{x}_{\scriptscriptstyle qp}\in \bm{m}_{\scriptscriptstyle\mathcal C}$.  To do that, it sends $\ddot{x}_{\scriptscriptstyle qp}$ to SAP contract which  returns  either $1$ or $0$. If the output is $0$, then SC discards the client's complaint, $\bm{m}_{\scriptscriptstyle\mathcal C}$, and does not take steps \ref{sc-checks-client-claim}-\ref{the-contract-loop}. Otherwise, it proceeds to the next step. 

 \item\label{sc-checks-client-claim} Ensures each vector   $ {\bm{w}}\in  {\bm{w}}_{\scriptscriptstyle\mathcal{C}}$ is well-formed. In particular, it verifies there exist no two vectors: $ {\bm{w}}, {\bm{w}}'\in  {\bm{w}}_{\scriptscriptstyle\mathcal{C}}$ such that $ {\bm{w}}[0]= {\bm{w}}'[0]$.  If such vectors  exist, it deletes the redundant ones from $ {\bm{w}}_{\scriptscriptstyle\mathcal{C}}$. This ensures no two claims refer to the same verification.  Also, it removes any vector $ {\bm{w}}$ from $ {\bm{w}}_{\scriptscriptstyle\mathcal{C}}$ if $ {\bm{w}}[0]$ is not in the range $[1,z]$ or if $ {\bm{w}}[0]\in  {\bm{v}}$. Note the latter check  (i.e. $ {\bm{w}}[0]\in  {\bm{v}}$) ensures $\mathcal C$ cannot hold $\mathcal S$ accountable if $\mathcal C$ has generated an ill-formed query for the same verification.   

\item Uses secret key $\bar{k}\in \ddot{x}_{\scriptscriptstyle qp}$ to decrypt the encrypted public key, $\mathtt{Dec}(\bar{k},\mathtt{Enc}(\bar{k},pk))=pk$

\item\label{the-contract-loop} For every  vector $ {\bm{w}}\in {\bm{w}}_{\scriptscriptstyle{C}}$:
\begin{itemize}
\item[$\bullet$]  Retrieves details of a proof that was rejected in each $i$-th verification.  In particular, it sets   $i= {\bm{w}}[0]$ and $g= {\bm{w}}[1]$. Recall that $g$ refers to the index of a rejected proof in the proof vector which was generated for $i$-th verification, i.e. $ {\bm{\pi}}_{\scriptscriptstyle i}$ 

\item[$\bullet$]  Fetches the related  query, $\hat{k}_{\scriptscriptstyle i}$, from SC.

\item[$\bullet$]  Removes the pads only from $g$-th padded encrypted proof. Let $ {\bm{\pi}}'_{\scriptscriptstyle i}[g]$ be the result. Next, it decrypts the  encrypted proof, $\mathtt{Dec}(\bar{k}, {\bm{\pi}}'_{\scriptscriptstyle i}[g]))= {\bm{\pi}}_{\scriptscriptstyle i}[g]$

\item[$\bullet$] \label{dummy-vector} Constructs a fresh vector: $ {\bm{\pi}}''_{\scriptscriptstyle i}$, such that  its $g$-th element equals $ {\bm{\pi}}_{\scriptscriptstyle i}[g]$ (i.e. $ {\bm{\pi}}''_{\scriptscriptstyle i}[g]=  {\bm{\pi}}_{\scriptscriptstyle i}[g] \text{ and } | {\bm{\pi}}''_{\scriptscriptstyle i}|= | {\bm{\pi}}_{\scriptscriptstyle i}|$) and the rest of its elements are dummy values. 

\item[$\bullet$] Calls   $\mathtt{PoRID.identify}( {\bm{\pi}}''_{\scriptscriptstyle i},g,\hat{k}_{\scriptscriptstyle i},pk)\rightarrow I_{\scriptscriptstyle i}$. If $I_{\scriptscriptstyle i}=\mathcal S$, then it increments $y_{\scriptscriptstyle\mathcal S}$  by $1$. Otherwise, it does nothing. 

\end{itemize}
\end{enumerate}
Let $K_{\scriptscriptstyle 4}$ be the time that SC finishes all the above checks.% in step \ref{the-contract-loop}

\end{enumerate} %%%%%%%%%

\item \textbf{Coin Transfer}.
\begin{enumerate}
\item Either $\mathcal C$ or $\mathcal S$ sends ``pay'' message and the statement: $\ddot{x}_{\scriptscriptstyle cp}$ to SC at time $L>K_{\scriptscriptstyle 4}$
\item SC checks the validity of the statement by sending it  to SAP contract that  returns  either $1$ or $0$. SC only proceeds to the next step if the output is $1$
\item SC distributes the coins to the parties as follows:

\begin{itemize}
\item[$\bullet$]   $coin^{\scriptscriptstyle *}_{\scriptscriptstyle\mathcal C}-o(z-y_{\scriptscriptstyle\mathcal S})+l\cdot y_{\scriptscriptstyle\mathcal S}-l'\cdot y_{\scriptscriptstyle\mathcal C}$ coins  to $\mathcal C$

\item[$\bullet$] $coin^{\scriptscriptstyle *}_{\scriptscriptstyle\mathcal S}+o(z-y_{\scriptscriptstyle\mathcal S})-l\cdot y_{\scriptscriptstyle\mathcal S}+l'\cdot y_{\scriptscriptstyle\mathcal C}$ coins to $\mathcal S$
%\item[$\bullet$] $l(u_{\scriptscriptstyle\mathcal S}+u_{\scriptscriptstyle\mathcal C}+y_{\scriptscriptstyle\mathcal S}+y_{\scriptscriptstyle\mathcal C})$ coins to the arbiter.  
\end{itemize}

\end{enumerate}
\end{enumerate}


\begin{remark}
The reason in step \ref{dummy-vector} vector $ {\bm{\pi}}''_{\scriptscriptstyle i}$ is constructed  is to let SC make \emph{black-box}  use of $\mathtt{PoRID.identify}()$. Alternatively, SC could decrypt all proofs in $\mathtt{Enc}(\bar{k}, {\bm{\pi}}_{\scriptscriptstyle i})$ and pass them to $\mathtt{PoRID.identify}()$. However, this approach would impose a high cost, as all proofs have to be decrypted. 
\end{remark}

\begin{remark}
In general, a transaction that is sent   to a smart contract should  cover the cost of the contract's execution. Therefore, in the above protocol, if a party unnecessarily invokes a contract for an accepting proof, it has to pay the execution cost in advance. This is the reason the above protocol (unlike RC-S-P protocol) does not need to track the number of times a party unnecessarily invokes the contract.    
\end{remark}







%In the case where the party's outgoing message input is of large size, the parties can post a hash of the data to the board and send the data directly to its counter party. This requires, the other party to simply check if the data matches the representation and then sends to the board a short string declaring it has received it. 








%In the above definition,  $\mathtt{Service.verify}()$ algorithm  allows a verifier to detect only a misbehaving server; nevertheless, it is not suitable to  detect a misbehaving client who may try to falsely accuse the server.  Thus, $\mathtt{Service.resolve()}$ algorithm has also been incorporated in the definition, to allow an arbiter to detect either party's misbehaviour and resolve any dispute between them.  Informally, a service scheme has two main properties: correctness and soundness. The correctness requires that for any key, the verification algorithm accepts a proof generated by an honest prover. The soundness requires that if a prover convinces the verifier (i.e. client or arbiter), with a high probability, then the service has been provided by the prover. {\color {blue} it seems undeniability and accountability/liability, i.e. a malicious party can be identified and held accountable, are needed too. Thin if it's needed here on in the main payment protocol.}
%The exact formalisation of soundness however totally depends on the kind of service provided. 



%Thus, since the two party are mutually distrusted, the above definition has also incorporated Service.resolve() algorithm that  allows an arbiter to resolve any dispute between client and server. 
























