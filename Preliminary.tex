% !TEX root =R-zkCSP.tex



\section{Preliminaries and Notations} \label{preliminaries}




\subsection{Smart Contract} Cryptocurrencies, such as Bitcoin and Ethereum, in addition to offering a decentralised currency,  support  computations on  transactions. In this setting, often a certain computation logic is encoded in a computer program, called \emph{``smart contract''}. To date, Ethereum is the most predominant cryptocurrency framework that enables users to define arbitrary smart contracts. In this framework,  contract code is stored on the blockchain and  executed by all parties (i.e. miners) maintaining the cryptocurrency,  when the program inputs are provided by transactions. The program execution's  correctness  is  guaranteed by the security of the underlying blockchain components. To prevent  a denial of service attack, the framework requires a transaction creator to pay a  fee, called \emph{``gas''}, depending on the complexity of the contract running on  it.  Nonetheless,  Ethereum smart contracts suffer from an important   issue; namely, the \emph{lack of privacy}, as it requires  every contract's data to be public, which is a major impediment  to  the broad adoption of  smart contracts when a certain level of privacy is desired. To address the issue, researchers/users may either (a)  utilise existing decentralised frameworks  which support privacy-preserving smart contracts, e.g. \cite{KosbaMSWP16}. But, due to the use of generic and computationally expensive cryptographic tools,  they impose a significant cost to their users. Or (b)  design  efficient tailored cryptographic protocols  that preserve (contracts) data privacy, even though non-private smart contracts are used. We take the latter approach in this work. 


%Nonetheless,  Ethereum smart contracts suffer from an important   issue; namely, the \emph{lack of privacy}, as it requires  every contract's data to be public, which is a major impediment  to  the broad adoption of  smart contracts when a certain level of privacy is desired. To address the issue, researchers/users may either (a)  utilise existing decentralised frameworks  which support privacy-preserving smart contracts, e.g. \cite{KosbaMSWP16}. But, due to the use of generic and computationally expensive cryptographic tools,  they impose a significant cost to their users. Or (b)  design  efficient tailored cryptographic protocols  that preserve (contracts) data privacy, even though non-private smart contracts are used. We take the latter approach in this work. 


%\subsection{Commitment Scheme} A commitment scheme involves two parties:  \emph{sender} and  \emph{receiver}, and includes  two phases: \emph{commit} and  \emph{open}. In the commit phase, the sender  commits to a message: $m$ as $\mathtt{Com}(m,d)=h$, that involves a secret value: $d$. At the end of the commit phase,  the commitment: $h$ is sent to the receiver. In the open phase, the sender sends the opening: $\ddot{p}=(m,d)$ to the receiver who verifies its correctness: $\mathtt{Ver}(h,\ddot{p})\stackrel{\scriptscriptstyle ?}=1$ and accepts if the output is $1$.  A commitment scheme must satisfy two properties: (a) \textit{hiding}: infeasible for an adversary (i.e. the receiver) to learn any information about the committed  message: $m$, until the commitment: $h$ is opened, and (b) \textit{binding}:   infeasible for an adversary (i.e. the sender) to open a commitment: $h$ to different values: $\ddot{p}'=(m',d')$ than that used in the commit phase, i.e. infeasible to find  $\ddot{p}'$, \textit{s.t.} $\mathtt{Ver}(h,\ddot{p})=\mathtt{Ver}(h,\ddot{p}')=1$, where $\ddot{p}\neq \ddot{p}'$.  There exist efficient non-interactive  commitment schemes both in (a) the random oracle model using the well-known hash-based scheme such that $\mathtt{Com}(m,d)$ involves computing: $\mathtt{H}(m||d)=h$ and $\mathtt{Ver}(h,\ddot{p})$ requires checking: $\mathtt{H}(m||d)\stackrel{\scriptscriptstyle ?}=h$, where $\mathtt{H}$ is a hash function, and (b)  the standard model, e.g. Pedersen scheme \cite{Pedersen91}. 





\subsection{Pseudorandom Function}

Informally, a pseudorandom function ($\mathtt{PRF}$) is a deterministic function that takes a key and an input; and outputs a value  indistinguishable from that of  a truly random function with the same input.   A $\mathtt{PRF}$ is formally defined as follows \cite{DBLP:books/crc/KatzLindell2007}. 
\begin{definition} Let $W:\{0,1\}^{\scriptscriptstyle\psi}\times \{0,1\}^{\scriptscriptstyle \eta}\rightarrow \{0,1\}^{\scriptscriptstyle  \iota}$ be an efficient  keyed function. It is said $W$ is a pseudorandom function if for all probabilistic polynomial-time distinguishers $B$, there is a negligible function, $\mu(.)$, such that:

\begin{equation*}
\bigg | Pr[B^{\scriptscriptstyle W_{\hat{k}}(.)}(1^{\scriptscriptstyle \psi})=1]- Pr[B^{\scriptscriptstyle \omega(.)}(1^{\scriptscriptstyle \psi})=1] \bigg |\leq \mu(\psi)
\end{equation*}
where  the key, $\hat{k}\stackrel{\scriptscriptstyle\$}\leftarrow\{0,1\}^{\scriptscriptstyle\psi}$, is chosen uniformly at random and $\omega$ is chosen uniformly at random from the set of functions mapping $\eta$-bit strings to $\iota$-bit strings. We let public parameters $\zeta:(\psi,\eta, \iota)$ be the description of $\mathtt{PRF}$
\end{definition}



\subsection{Commitment Scheme}


 A commitment scheme involves two parties:  \emph{sender} and  \emph{receiver}, and includes  two phases: \emph{commit} and  \emph{open}. In the commit phase, the sender  commits to a message: $x$ as $\mathtt{Com}(x,r)=\mathtt{Com}_{\scriptscriptstyle x}$, that involves a secret value: $r\stackrel{\scriptscriptstyle\$}\leftarrow \{0,1\}^{\scriptscriptstyle\lambda}$. At the end of the commit phase,  the commitment: $\mathtt{Com}_{\scriptscriptstyle x}$ is sent to the receiver. In the open phase, the sender sends the opening: $\ddot{x}:=(x,r)$ to the receiver who verifies its correctness: $\mathtt{Ver}(\mathtt{Com}_{\scriptscriptstyle x},\ddot{x})\stackrel{\scriptscriptstyle ?}=1$ and accepts if the output is $1$.  A commitment scheme must satisfy two properties: (a) \textit{hiding}: infeasible for an adversary (i.e. the receiver) to learn any information about the committed  message: $x$, until the commitment: $\mathtt{Com}_{\scriptscriptstyle x}$ is opened, and (b) \textit{binding}:   infeasible for an adversary (i.e. the sender) to open a commitment: $\mathtt{Com}_{\scriptscriptstyle x}$ to different values: $\ddot{x}':=(x',r')$ than that was  used in the commit phase, i.e. infeasible to find  $\ddot{x}'$, \textit{s.t.} $\mathtt{Ver}(\mathtt{Com}_{\scriptscriptstyle x},\ddot{x})=\mathtt{Ver}(\mathtt{Com}_{\scriptscriptstyle x},\ddot{x}')=1$, where $\ddot{x}\neq \ddot{x}'$.  There exist efficient non-interactive  commitment schemes both in (a) the standard model, e.g. Pedersen scheme \cite{Pedersen91}, and (b)  the random oracle model using the well-known hash-based scheme such that committing  is : $\mathtt{H}(x||r)=\mathtt{Com}_{\scriptscriptstyle x}$ and $\mathtt{Ver}(\mathtt{Com}_{\scriptscriptstyle x},\ddot{x})$ requires checking: $\mathtt{H}(x||r)\stackrel{\scriptscriptstyle ?}=\mathtt{Com}_{\scriptscriptstyle x}$, where $\mathtt{H}:\{0,1\}^{\scriptscriptstyle *}\rightarrow \{0,1\}^{\scriptscriptstyle \lambda}$ is a collision resistance hash function, i.e. the probability to find $x$ and $x'$ such that $\mathtt{H}(x)=\mathtt{H}(x')$ is negligible, $\mu(\lambda)$.
%
% A commitment scheme involves two parties:  \emph{sender} and  \emph{receiver}, and includes  two phases: \emph{commit} and  \emph{open}. In the commit phase, the sender  commits to a message: $x$ as $\mathtt{Com}(x,r)=g$, that involves a secret value: $r$. At the end of the commit phase,  the commitment: $g$ is sent to the receiver. In the open phase, the sender sends the opening: $\ddot{p}=(x,r)$ to the receiver who verifies its correctness: $\mathtt{Ver}(g,\ddot{p})\stackrel{\scriptscriptstyle ?}=1$ and accepts if the output is $1$.  A commitment scheme must satisfy two properties: (a) \textit{hiding}: infeasible for an adversary (i.e. the receiver) to learn any information about the committed  message: $x$, until the commitment: $g$ is opened, and (b) \textit{binding}:   infeasible for an adversary (i.e. the sender) to open a commitment: $g$ to different values: $\ddot{p}'=(x',r')$ than that was  used in the commit phase, i.e. infeasible to find  $\ddot{p}'$, \textit{s.t.} $\mathtt{Ver}(g,\ddot{p})=\mathtt{Ver}(g,\ddot{p}')=1$, where $\ddot{p}\neq \ddot{p}'$.  There exist efficient non-interactive  commitment schemes both in (a) the random oracle model using the well-known hash-based scheme such that $\mathtt{Com}(x,r)$ involves computing: $\mathtt{H}(x||r)=g$ and $\mathtt{Ver}(g,\ddot{p})$ requires checking: $\mathtt{H}(x||r)\stackrel{\scriptscriptstyle ?}=g$, where $\mathtt{H}$ is a hash function, and (b)  the standard model, e.g. Pedersen scheme \cite{Pedersen91}. 




\subsection{Non-interactive Publicly Verifiable Zero-knowledge Proofs}

\subsection{Digital Signatures}

Some of our protocols utilise digital signatures to commit a party to messages it sends. 

\subsection{Merkle Tree}

A Merkle tree scheme introduced by Merkle \cite{DBLP:conf/sp/Merkle80,DBLP:conf/crypto/Merkle89} allows   committing to data blocks, such that it is  possible later to open the commitment and verify individual blocks of the file without the need to have the entire file to verify the opening. To construct a Merkle tree a file is split into blocks, then the blocks are grouped in pairs. Next,   a collision-resistant hash function is used to hash each pair. After that, the hash values are grouped in pairs and each pair is further hashed, and this process is repeated until only a single hash value remains. This results in a  tree with the leaves corresponding to the blocks of the input file and the root corresponding to the last remaining hash value.  {\color{blue} Add the Merkle tree algorithms, e.g. build, prove, verify}






\subsection{Proofs of Retrievability (PoR)}\label{PoR-def}
% !TEX root =R-zkCSP.tex

In general, a PoR scheme considers the case where an honest client wants to store its file(s) on a  potentially malicious server, i.e active adversary. It is a challenge-response interactive protocol, where the server proves to the client that its file is intact and retrievable. Below, we restate PoR's formal definition (and security property) originally provided in \cite{DBLP:journals/iacr/JuelsK07,DBLP:conf/asiacrypt/ShachamW08}. PoR scheme comprises five algorithms: 

\begin{itemize}
\item[$\bullet$] $\mathtt{PoR.keyGen}(1^{\scriptscriptstyle\lambda})\rightarrow k:=(sk,pk)$.  A probabilistic algorithm, run by a client. It takes as input the security parameter  $1^{\scriptscriptstyle\lambda}$ and outputs  private and public verification keys $k:=(sk,pk)$.

\

\item[$\bullet$] $\mathtt{PoR.setup}(1^{\scriptscriptstyle\lambda}, u,k)\rightarrow (u^{\scriptscriptstyle *}, \sigma, {pp})$. A probabilistic algorithm, run by the client. It takes as input the security parameter $1^{\scriptscriptstyle\lambda}$,  a file $u$, and key $k$. It encodes $u$, denoted by $u^{\scriptscriptstyle *}$ and generates  a metadata, $\sigma$.  The client outputs encoded file $u^{\scriptscriptstyle *}$, metadata $\sigma$, and  { (possibly file dependent) public parameters $pp$}. It sends to the server   $u^{\scriptscriptstyle *},\sigma$, and { $pp$}.

\

\item[$\bullet$] $\mathtt{PoR.genQuery}(1^{\scriptscriptstyle\lambda},k,{ pp})\rightarrow \bm{q}$. A probabilistic algorithm, run by the client. It takes as input the security parameter  $1^{\scriptscriptstyle\lambda}$,  key $k$, and {  public parameters $pp$}. It outputs a  query vector $\bm{q}$, possibly picked uniformly at random. The query is given to the server. 

\

\item[$\bullet$] $\mathtt{PoR.prove}(u^{\scriptscriptstyle *}, \sigma,\bm{q},pk,{ pp})\rightarrow \pi$. It is run by a server.  It takes as input the encoded file $u^{\scriptscriptstyle *}$, metadata $\sigma$, query $\bm{q}$,  public key $pk$, and {  public parameters $pp$}. It  outputs a proof, $\pi$, given to the client. 

\

\item[$\bullet$] $\mathtt{PoR.verify}(\pi,\bm{q},k,{ pp})\rightarrow d\in\{0,1\}$. It is run by the client. It takes as input the proof $\pi$,  query $\bm{q}$,  key pair $k$, and {  public parameters $pp$}. It outputs either $0$ if it rejects,  or $1$ if it accepts the proof. 

\end{itemize}


Informally, a PoR scheme has two main properties: \emph{correctness} and \emph{soundness}. Correctness requires that the verification algorithm accepts  proofs generated by an honest verifier. Formally, it requires that for any key $k$, any file $u\in\{0,1\}^{\scriptscriptstyle *}$, and any  pair $(u^{*},\sigma)$ output by $\mathtt{PoR.setup}(1^{\scriptscriptstyle\lambda}, u,k)$, and any query $\bm{q}$, the verifier  accepts when it interacts with an honest prover.

 Soundness requires that if a prover convinces the verifier (with high probability) then the file is stored by the prover. This is formalized via the notion of an extractor algorithm, that is able to extract the file in interaction with the adversary using a polynomial number of  rounds. Before we define  soundness, we restate the  experiment, defined in \cite{DBLP:conf/asiacrypt/ShachamW08}, that takes place between an environment $\mathcal{E}$ and  adversary $\mathcal{A}$. In this experiment, $\mathcal{A}$ plays the role of a corrupt party and $\mathcal{E}$ simulates an honest party's role. 


\begin{enumerate}
\item $\mathcal{E}$ executes $\mathtt{PoR.keyGen}(1^{\scriptscriptstyle\lambda})$ algorithm and provides public key, $pk$, to $\mathcal{A}$.   
\item $\mathcal{A}$ can pick  arbitrary file $u$, and  uses it to make queries to  $\mathcal{E}$ who runs  $\mathtt{PoR.setup}(1^{\scriptscriptstyle\lambda}, u,k)\rightarrow (u^{\scriptscriptstyle *}, \sigma, { pp})$   and returns the output to $\mathcal{A}$. Also, upon receiving the output of $\mathtt{PoR.setup}(1^{\scriptscriptstyle\lambda}, u,k)$, $\mathcal{A}$ can ask $\mathcal{E}$  to run   $\mathtt{PoR.genQuery}(1^{\scriptscriptstyle\lambda}$ $, k,{ pp})\rightarrow \bm{q}$ and give the output to it. $\mathcal{A}$ can locally run $\mathtt{PoR.prove}(u^{\scriptscriptstyle *}, \sigma, \bm{q},pk,{ pp})\rightarrow \pi$ to get its outputs as well. 
\item $\mathcal{A}$ can request $\mathcal{E}$ the execution of $\mathtt{PoR.verify}(\pi,\bm{q},k,{ pp})$ for any $u$ used to query $\mathtt{PoR.setup}(.)$. Accordingly, $\mathcal{E}$ informs  $\mathcal{A}$ about the verification output. The adversary can send a polynomial number of queries to $\mathcal{E}$. Finally, $\mathcal{A}$ outputs metadata $\sigma$ returned from a setup query and the description of a prover, $\mathcal{A}'$, for any file it has already chosen above. 
\end{enumerate}

It is said that a cheating prover, $\mathcal{A}'$, is \emph{$\epsilon$-admissible} if it convincingly answers $\epsilon$ fraction of verification challenges. Informally, a PoR scheme supports extractability, if there is an extractor algorithm $\mathtt{Ext}(k,\sigma, \mathtt{P}')$, that takes as input the key $k$, metadata $\sigma$, and the description $\mathtt{P}'$ of the  machine implementing the prover's role $\mathcal{A}'$ and outputs the file, $u$. The extractor has the ability to reset the adversary to the beginning of the challenge phase and repeat this step polynomially many times for the purpose of extraction, i.e. the extractor can rewind $\mathtt{P}'$.


%XXXXThomas: \emph{What is actually the entity that can be rewinded (i.e., the adversary)? Is it $\mathtt{P}'$ or $\mathcal{A}'$?}XXXX. Aydin: done


\begin{definition}[$\epsilon$-soundness]\label{extractable} A PoR scheme is $\epsilon$-sound if   there exists an extraction algorithm $\mathtt{Ext}(.)$ such that, for every adversary $\mathcal{A}$ who plays the experiment above, and outputs an $\epsilon$-admissible cheating prover $\mathcal{A}'$ for a file $u$,  the extraction algorithm  recovers $u$ from $\mathcal{A}'$, given honest parties private key, public parameters, metadata and the description of $\mathcal{A}'$,  i.e. $\mathtt{Ext}(k,{ pp},\sigma, \mathtt{P}')\rightarrow u$, except with a negligible probability. 
\end{definition}






%\begin{definition}[$\epsilon$-sound]\label{extractable} A PoR scheme is $\epsilon$-sound if  for every adversary, $\mathcal{A}$, who plays the experiment above, and outputs an $\epsilon$-admissible cheating prover $\mathcal{A}'$, for a file $u$,  there exists an extraction algorithm $\mathtt{Ext}()$, that recovers $u$ from $\mathcal{A}'$, given honest parties public-private keys, metadata and the description of $\mathcal{A}'$,  i.e. $\mathtt{Ext}(k,\sigma, \mathtt{P}')\rightarrow u$, except with a negligible probability. 
%\end{definition}


In contrast to the PoR definition in \cite{DBLP:journals/iacr/JuelsK07,DBLP:conf/asiacrypt/ShachamW08} where $\mathtt{PoR.genQuery}(.)$ is implicit, in the above we have explicitly stated  it, as it    plays an important role in this paper. Also, there are PoR protocols, e.g. \cite{MillerPermacoin}, that do not involve $\mathtt{PoR.keyGen}(.)$. Instead a set of public parameters/keys (e.g. file size or a root of Merkle tree) are output by $\mathtt{PoR.setup}(.)$. To make the PoR definition generic to capture both cases, we have explicitly included the public parameters $pp$ in the algorithms definitions too.  

\subsection{Notations} In the formal definitions in this paper, we often use bar symbol, i.e. `` $|$ '', for the sake of readability and to separate events from an experiment. It should not be confused with conditional probability's symbol. 





%In practice, we are interested in  pseudorandom functions that can be efficiently built on smart contracts given the tools  that a smart contract framework (e.g. Ethereum) offers. HMAC \cite{DBLP:conf/crypto/BellareCK96} satisfies the requirements above.


 

%\input{notation-Table.tex}

