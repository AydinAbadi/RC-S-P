% !TEX root =R-zkCSP.tex

\begin{proof}[sketch] The soundness  of VSID  stems  from the hiding property of the commitment, zero-knowledge property of the publicly verifiable zero-knowledge proofs, and soundness of  the  verifiable service (VS) schemes. In particular, in VSID the verifier (i.e. in this case the client) makes block-box calls to the algorithms of   VS, to ensure soundness. However, the prover (i.e.  the server) is given  additional messages, i.e.   $\mathtt{Com}_{\scriptscriptstyle sk}$,  $\mathtt{Com}_{\scriptscriptstyle \text{aux}}$, $w_{\scriptscriptstyle\sigma}$ and $w_{\scriptscriptstyle q}$. The hiding property of the commitment scheme and zero-knowledge property of the zero-knowledge system ensure, given the messages, the prover learns nothing about the verification key and auxiliary information, except with a negligible probability $\mu(\lambda)$. Moreover, the soundness of VS scheme ensures a corrupt prover cannot convince an honest verifier, except with  probability $\mu(\lambda)$.   Inputs well-formedness property boils down to the security of the commitment and publicly verifiable non-interactive zero-knowledge proofs   schemes that are used in steps \ref{VSID::keygen}, \ref{VSID::Client-side-Setup} and \ref{VSID::Client-side-QueryGeneration} in VSID protocol. Specifically,  the binding property of the commitment and the soundness of the publicly verifiable non-interactive zero-knowledge proofs schemes  ensure that a corrupt prover (i.e. in this case the client) cannot convince a verifier (i.e. the server) to accept metadata proofs, $w_{\scriptscriptstyle\sigma}$ and $\mathtt{Com}_{\scriptscriptstyle sk}\in pk$, while  $M(u^{\scriptscriptstyle *},k,{\color{purple}pp})\neq \sigma $ or to accept query proofs, $\bm{w}_{\scriptscriptstyle q}$ and $\mathtt{Com}_{\scriptscriptstyle \text{aux}}$, while $Q(\text{aux},k,{\color{purple}pp})\neq \bm{q}$, except with  probability $\mu(\lambda)$.  Moreover, the detectable abort property holds as long as both previous properties (i.e. soundness and inputs well-formedness) hold, the commitment is secure, the zero-knowledge proofs are publicly verifiable and the signature scheme is secure. The reason is that algorithm $\mathtt{VSID.identify}(.)$, which ensures detectable abort, is a wrapper function that is invoked by the arbiter, and  sequentially makes subroutine calls to algorithms  $\mathtt{VSID.serve}(.)$,  $\mathtt{VSID.checkQuery}(.)$ and $\mathtt{VS.verify}(.)$,   where the first two ensure input well-formedness, and the last one ensures soundness. Also, due to the security of the commitment (i.e. binding) the malicious client cannot provide to the arbiter other secret verification key than what  was initially committed. Moreover, due to the public verifiability of the zero-knowledge proofs, the arbiter can verify all proofs input to $\mathtt{VSID.serve}(.)$ and $\mathtt{VSID.checkQuery}(.)$. The signature's security ensures if a proof is not signed correctly, then it can also be rejected by the arbiter; on the other hand, if a proof is signed correctly, then  it cannot be repudiated by the signer later on (due to signature's unforgeability); this guarantees that the signer is held accountable for a rejected proof it provides.
  \hfill\(\Box\)\end{proof}
  
  
  
  
  
  