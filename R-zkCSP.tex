% This is LLNCS.DEM the demonstration file of
% the LaTeX macro package from Springer-Verlag
% for Lecture Notes in Computer Science,
% version 2.4 for LaTeX2e as of 16. April 2010
%
%\documentclass[A4paper,11pt,orivec]{llncs}
\documentclass[orivec]{llncs}
%
\usepackage{color, colortbl}
\usepackage{capt-of}
\usepackage[table]{xcolor}
%----------------------------
\usepackage[makeroom]{cancel}
%----------------------------
\usepackage{enumitem}
\usepackage{makeidx}  % allows for indexgeneration
\usepackage{amsfonts,amsmath,amssymb,graphicx,setspace,times,tipx}


%\usepackage[ruled,linesnumbered]{algorithm2e}
%\usepackage{subfig}
\usepackage{framed}
\usepackage{esvect}
\usepackage{tikz}
\usepackage{latexsym}
\usepackage{multirow}
\let\oldReturn\Return
%\renewcommand{\Return}{\State\oldReturn}
%\usepackage{algpseudocode}
\usepackage[mathscr]{eucal}
\usepackage{algpseudocode}

\newcommand{\size}{3.3pt}


\usepackage{arydshln}



\usepackage{algorithm}

\usepackage{scalerel}
\usepackage{listings}
\pagestyle{plain}
\usepackage{fullpage}
\usepackage{array}
\usepackage{subcaption}
\captionsetup{compatibility=false}
\newcommand{\vars}{\texttt}
\newcommand{\func}{\textrm}
\newcommand{\resizeT}{\scalebox{1}}
\newcommand{\resizeS}{\scalebox{.56}}
\newcommand{\resizeSS}{\scalebox{.38}}

\usepackage{longtable}

\usepackage{tikz}
\usepackage{pgfplots}
\usepackage{pgfplots, pgfplotstable}

%\usepackage{kbordermatrix}% http://www.hss.caltech.edu/~kcb/TeX/kbordermatrix.sty
\usepackage{blkarray}% http://ctan.org/pkg/blkarra
\usepackage{mathtools}
\usepackage{amsmath}
\usepackage{bm}
\usepackage{adjustbox}
\usepackage{blindtext}
\usepackage{multicol}

\definecolor{bubblegum}{rgb}{0.99, 0.76, 0.8}




\usepackage{makecell, cellspace, caption}



\definecolor{Gray}{gray}{0.9}
%\usepackage[left=3cm,right=3cm,top=3cm,bottom=3cm]{geometry}
%

\usepackage[first=0,last=9]{lcg}
\newcommand{\ra}{\rand0.\arabic{rand}}

\usepackage[title]{appendix}

\usepackage{boxedminipage}

\usepackage{multirow}


%\usepackage{makeidx}  % allows for indexgeneration

%\usepackage[ruled,linesnumbered]{algorithm2e}
%\usepackage{enumitem}
%\usepackage[empty]{fullpage}% http://ctan.org/pkg/fullpage
%\usepackage{amsfonts,amsmath,amssymb,graphicx,setspace,times,tipx}
%\usepackage{kbordermatrix}% http://www.hss.caltech.edu/~kcb/TeX/kbordermatrix.sty
%\usepackage{blkarray}% http://ctan.org/pkg/blkarra


\usepackage{adjustbox}
\usepackage{blindtext}
\usepackage{multicol}
%\usepackage{lipsum}
%\usepackage{amsthm}
%\newtheorem*{assumption*}{\assumptionnumber}
\usepackage{tablefootnote}

\newtheorem{assumption}{Assumption}


\newcommand\NameEntry[1]{%
  \multirow{3}*{%
    \begin{varwidth}{10em}% --- or minipage, if you prefer a fixed width
    \flushright #1%
    \end{varwidth}}}




\newcommand{\specialcell}[2][c]{%
  \begin{tabular}[#1]{@{}c@{}}#2\end{tabular}}


















\begin{document}
  \setlength\abovedisplayskip{0pt}
  \setlength\belowdisplayskip{0pt}


\newenvironment{packed_item}{
\begin{itemize}
	\setlength{\topsep}{0pt}
	\setlength{\partopsep}{0pt}
  \setlength{\itemsep}{0pt}
  \setlength{\parskip}{0pt}
  \setlength{\parsep}{0pt}
}{\end{itemize}}

\newenvironment{packed_enum}{
\begin{enumerate}
	\setlength{\topsep}{0pt}
	\setlength{\partopsep}{0pt}
  \setlength{\itemsep}{0pt}
  \setlength{\parskip}{0pt}
  \setlength{\parsep}{0pt}
}{\end{enumerate}}




%\newtheorem{definition}{Definition}
%\newtheorem{theorem}{Theorem}
%\newtheorem{lemma}{Lemma}
%\newtheorem{claim}{Claim}

\title{Recurring Contingent Service Payments}
%
%\titlerunning{Hamiltonian Mechanics}  % abbreviated title (for running head)
%                                     also used for the TOC unless
%                                     \toctitle is used
%
\author{}
%%
\institute{}

%\email{....}\\
%}

\maketitle              % typeset the title of the contribution
%
%\vspace{-6mm}
\begin{abstract}



\end{abstract}
 % !TEX root =R-zkCSP.tex


% !TEX root =R-zkCSP.tex



\section{Introduction}

\section{Related Work}

\section{Preliminaries and Notations} \label{preliminaries}




\subsection{Smart Contract} Cryptocurrencies, such as Bitcoin and Ethereum, in addition to offering a decentralised currency,  support  computations on  transactions. In this setting, often a certain computation logic is encoded in a computer program, called \emph{``smart contract''}. To date, Ethereum is the most predominant cryptocurrency framework that enables users to define arbitrary smart contracts. In this framework,  contract code is stored on the blockchain and  executed by all parties (i.e. miners) maintaining the cryptocurrency,  when the program inputs are provided by transactions. The program execution's  correctness  is  guaranteed by the security of the underlying blockchain components. To prevent  a denial of service attack, the framework requires a transaction creator to pay a  fee, called \emph{``gas''}, depending on the complexity of the contract running on  it.  Nonetheless,  Ethereum smart contracts suffer from an important   issue; namely, the \emph{lack of privacy}, as it requires  every contract's data to be public, which is a major impediment  to  the broad adoption of  smart contracts when a certain level of privacy is desired. To address the issue, researchers/users may either (a)  utilise existing decentralised frameworks  which support privacy-preserving smart contracts, e.g. \cite{KosbaMSWP16}. But, due to the use of generic and computationally expensive cryptographic tools,  they impose a significant cost to their users. Or (b)  design  efficient tailored cryptographic protocols  that preserve (contracts) data privacy, even though non-private smart contracts are used. We take the latter approach in this work. 


%Nonetheless,  Ethereum smart contracts suffer from an important   issue; namely, the \emph{lack of privacy}, as it requires  every contract's data to be public, which is a major impediment  to  the broad adoption of  smart contracts when a certain level of privacy is desired. To address the issue, researchers/users may either (a)  utilise existing decentralised frameworks  which support privacy-preserving smart contracts, e.g. \cite{KosbaMSWP16}. But, due to the use of generic and computationally expensive cryptographic tools,  they impose a significant cost to their users. Or (b)  design  efficient tailored cryptographic protocols  that preserve (contracts) data privacy, even though non-private smart contracts are used. We take the latter approach in this work. 


%\subsection{Commitment Scheme} A commitment scheme involves two parties:  \emph{sender} and  \emph{receiver}, and includes  two phases: \emph{commit} and  \emph{open}. In the commit phase, the sender  commits to a message: $m$ as $\mathtt{Com}(m,d)=h$, that involves a secret value: $d$. At the end of the commit phase,  the commitment: $h$ is sent to the receiver. In the open phase, the sender sends the opening: $\ddot{p}=(m,d)$ to the receiver who verifies its correctness: $\mathtt{Ver}(h,\ddot{p})\stackrel{\scriptscriptstyle ?}=1$ and accepts if the output is $1$.  A commitment scheme must satisfy two properties: (a) \textit{hiding}: infeasible for an adversary (i.e. the receiver) to learn any information about the committed  message: $m$, until the commitment: $h$ is opened, and (b) \textit{binding}:   infeasible for an adversary (i.e. the sender) to open a commitment: $h$ to different values: $\ddot{p}'=(m',d')$ than that used in the commit phase, i.e. infeasible to find  $\ddot{p}'$, \textit{s.t.} $\mathtt{Ver}(h,\ddot{p})=\mathtt{Ver}(h,\ddot{p}')=1$, where $\ddot{p}\neq \ddot{p}'$.  There exist efficient non-interactive  commitment schemes both in (a) the random oracle model using the well-known hash-based scheme such that $\mathtt{Com}(m,d)$ involves computing: $\mathtt{H}(m||d)=h$ and $\mathtt{Ver}(h,\ddot{p})$ requires checking: $\mathtt{H}(m||d)\stackrel{\scriptscriptstyle ?}=h$, where $\mathtt{H}$ is a hash function, and (b)  the standard model, e.g. Pedersen scheme \cite{Pedersen91}. 





\subsection{Pseudorandom Function}

Informally, a pseudorandom function ($\mathtt{PRF}$) is a deterministic function that takes a key and an input; and outputs a value  indistinguishable from that of  a truly random function with the same input.   A $\mathtt{PRF}$ is formally defined as follows \cite{DBLP:books/crc/KatzLindell2007}. 
\begin{definition} Let $W:\{0,1\}^{\scriptscriptstyle\psi}\times \{0,1\}^{\scriptscriptstyle \eta}\rightarrow \{0,1\}^{\scriptscriptstyle  \iota}$ be an efficient  keyed function. It is said $W$ is a pseudorandom function if for all probabilistic polynomial-time distinguishers $B$, there is a negligible function, $\mu(.)$, such that:

\begin{equation*}
\bigg | Pr[B^{\scriptscriptstyle W_{\hat{k}}(.)}(1^{\scriptscriptstyle \psi})=1]- Pr[B^{\scriptscriptstyle \omega(.)}(1^{\scriptscriptstyle \psi})=1] \bigg |\leq \mu(\psi)
\end{equation*}
where  the key, $\hat{k}\stackrel{\scriptscriptstyle\$}\leftarrow\{0,1\}^{\scriptscriptstyle\psi}$, is chosen uniformly at random and $\omega$ is chosen uniformly at random from the set of functions mapping $\eta$-bit strings to $\iota$-bit strings. We let public parameters $\zeta:(\psi,\eta, \iota)$ be the description of $\mathtt{PRF}$
\end{definition}



\subsection{Commitment Scheme}


 A commitment scheme involves two parties:  \emph{sender} and  \emph{receiver}, and includes  two phases: \emph{commit} and  \emph{open}. In the commit phase, the sender  commits to a message: $x$ as $\mathtt{Com}(x,r)=\mathtt{Com}_{\scriptscriptstyle x}$, that involves a secret value: $r\stackrel{\scriptscriptstyle\$}\leftarrow \{0,1\}^{\scriptscriptstyle\lambda}$. At the end of the commit phase,  the commitment: $\mathtt{Com}_{\scriptscriptstyle x}$ is sent to the receiver. In the open phase, the sender sends the opening: $\ddot{x}:=(x,r)$ to the receiver who verifies its correctness: $\mathtt{Ver}(\mathtt{Com}_{\scriptscriptstyle x},\ddot{x})\stackrel{\scriptscriptstyle ?}=1$ and accepts if the output is $1$.  A commitment scheme must satisfy two properties: (a) \textit{hiding}: infeasible for an adversary (i.e. the receiver) to learn any information about the committed  message: $x$, until the commitment: $\mathtt{Com}_{\scriptscriptstyle x}$ is opened, and (b) \textit{binding}:   infeasible for an adversary (i.e. the sender) to open a commitment: $\mathtt{Com}_{\scriptscriptstyle x}$ to different values: $\ddot{x}':=(x',r')$ than that was  used in the commit phase, i.e. infeasible to find  $\ddot{x}'$, \textit{s.t.} $\mathtt{Ver}(\mathtt{Com}_{\scriptscriptstyle x},\ddot{x})=\mathtt{Ver}(\mathtt{Com}_{\scriptscriptstyle x},\ddot{x}')=1$, where $\ddot{x}\neq \ddot{x}'$.  There exist efficient non-interactive  commitment schemes both in (a) the standard model, e.g. Pedersen scheme \cite{Pedersen91}, and (b)  the random oracle model using the well-known hash-based scheme such that committing  is : $\mathtt{H}(x||r)=\mathtt{Com}_{\scriptscriptstyle x}$ and $\mathtt{Ver}(\mathtt{Com}_{\scriptscriptstyle x},\ddot{x})$ requires checking: $\mathtt{H}(x||r)\stackrel{\scriptscriptstyle ?}=\mathtt{Com}_{\scriptscriptstyle x}$, where $\mathtt{H}:\{0,1\}^{\scriptscriptstyle *}\rightarrow \{0,1\}^{\scriptscriptstyle \lambda}$ is a collision resistance hash function, i.e. the probability to find $x$ and $x'$ such that $\mathtt{H}(x)=\mathtt{H}(x')$ is negligible, $\mu(\lambda)$.
%
% A commitment scheme involves two parties:  \emph{sender} and  \emph{receiver}, and includes  two phases: \emph{commit} and  \emph{open}. In the commit phase, the sender  commits to a message: $x$ as $\mathtt{Com}(x,r)=g$, that involves a secret value: $r$. At the end of the commit phase,  the commitment: $g$ is sent to the receiver. In the open phase, the sender sends the opening: $\ddot{p}=(x,r)$ to the receiver who verifies its correctness: $\mathtt{Ver}(g,\ddot{p})\stackrel{\scriptscriptstyle ?}=1$ and accepts if the output is $1$.  A commitment scheme must satisfy two properties: (a) \textit{hiding}: infeasible for an adversary (i.e. the receiver) to learn any information about the committed  message: $x$, until the commitment: $g$ is opened, and (b) \textit{binding}:   infeasible for an adversary (i.e. the sender) to open a commitment: $g$ to different values: $\ddot{p}'=(x',r')$ than that was  used in the commit phase, i.e. infeasible to find  $\ddot{p}'$, \textit{s.t.} $\mathtt{Ver}(g,\ddot{p})=\mathtt{Ver}(g,\ddot{p}')=1$, where $\ddot{p}\neq \ddot{p}'$.  There exist efficient non-interactive  commitment schemes both in (a) the random oracle model using the well-known hash-based scheme such that $\mathtt{Com}(x,r)$ involves computing: $\mathtt{H}(x||r)=g$ and $\mathtt{Ver}(g,\ddot{p})$ requires checking: $\mathtt{H}(x||r)\stackrel{\scriptscriptstyle ?}=g$, where $\mathtt{H}$ is a hash function, and (b)  the standard model, e.g. Pedersen scheme \cite{Pedersen91}. 




\subsection{Non-interactive Publicly Verifiable Zero-knowledge Proofs}

\subsection{Digital Signatures}

Some of our protocols utilise digital signatures to commit a party to messages it sends. 

\subsection{Merkle Tree}

A Merkle tree scheme introduced by Merkle \cite{DBLP:conf/sp/Merkle80,DBLP:conf/crypto/Merkle89} allows   committing to data blocks, such that it is  possible later to open the commitment and verify individual blocks of the file without the need to have the entire file to verify the opening. To construct a Merkle tree a file is split into blocks, then the blocks are grouped in pairs. Next,   a collision-resistant hash function is used to hash each pair. After that, the hash values are grouped in pairs and each pair is further hashed, and this process is repeated until only a single hash value remains. This results in a  tree with the leaves corresponding to the blocks of the input file and the root corresponding to the last remaining hash value.  {\color{blue} Add the Merkle tree algorithms, e.g. build, prove, verify}






\subsection{Proofs of Retrievability (PoR)}\label{PoR-def}
% !TEX root =R-zkCSP.tex

In general, a PoR scheme considers the case where an honest client wants to store its file(s) on a  potentially malicious server, i.e active adversary. It is a challenge-response interactive protocol, where the server proves to the client that its file is intact and retrievable. Below, we restate PoR's formal definition (and security property) originally provided in \cite{DBLP:journals/iacr/JuelsK07,DBLP:conf/asiacrypt/ShachamW08}. PoR scheme comprises five algorithms: 

\begin{itemize}
\item[$\bullet$] $\mathtt{PoR.keyGen}(1^{\scriptscriptstyle\lambda})\rightarrow k:=(sk,pk)$.  A probabilistic algorithm, run by a client. It takes as input the security parameter  $1^{\scriptscriptstyle\lambda}$ and outputs  private and public verification keys $k:=(sk,pk)$.

\

\item[$\bullet$] $\mathtt{PoR.setup}(1^{\scriptscriptstyle\lambda}, u,k)\rightarrow (u^{\scriptscriptstyle *}, \sigma, {\color{purple}pp})$. A probabilistic algorithm, run by the client. It takes as input the security parameter $1^{\scriptscriptstyle\lambda}$,  a file $u$, and key $k$. It encodes $u$, denoted by $u^{\scriptscriptstyle *}$ and generates  a metadata, $\sigma$.  The client outputs encoded file $u^{\scriptscriptstyle *}$, metadata $\sigma$, and  {\color{purple}(possibly file dependent) public parameters $pp$}. It sends to the server   $u^{\scriptscriptstyle *},\sigma$, and {\color{purple}$pp$}.

\

\item[$\bullet$] $\mathtt{PoR.genQuery}(1^{\scriptscriptstyle\lambda},k,{\color{purple}pp})\rightarrow \bm{q}$. A probabilistic algorithm, run by the client. It takes as input the security parameter  $1^{\scriptscriptstyle\lambda}$,  key $k$, and {\color{purple} public parameters $pp$}. It outputs a  query vector $\bm{q}$, possibly picked uniformly at random. The query is given to the server. 

\

\item[$\bullet$] $\mathtt{PoR.prove}(u^{\scriptscriptstyle *}, \sigma,\bm{q},pk,{\color{purple}pp})\rightarrow \pi$. It is run by a server.  It takes as input the encoded file $u^{\scriptscriptstyle *}$, metadata $\sigma$, query $\bm{q}$,  public key $pk$, and {\color{purple} public parameters $pp$}. It  outputs a proof, $\pi$, given to the client. 

\

\item[$\bullet$] $\mathtt{PoR.verify}(\pi,\bm{q},k,{\color{purple}pp})\rightarrow d\in\{0,1\}$. It is run by the client. It takes as input the proof $\pi$,  query $\bm{q}$,  key pair $k$, and {\color{purple} public parameters $pp$}. It outputs either $0$ if it rejects,  or $1$ if it accepts the proof. 

\end{itemize}


Informally, a PoR scheme has two main properties: \emph{correctness} and \emph{soundness}. Correctness requires that the verification algorithm accepts  proofs generated by an honest verifier. Formally, it requires that for any key $k$, any file $u\in\{0,1\}^{\scriptscriptstyle *}$, and any  pair $(u^{*},\sigma)$ output by $\mathtt{PoR.setup}(1^{\scriptscriptstyle\lambda}, u,k)$, and any query $\bm{q}$, the verifier  accepts when it interacts with an honest prover.

 Soundness requires that if a prover convinces the verifier (with high probability) then the file is stored by the prover. This is formalized via the notion of an extractor algorithm, that is able to extract the file in interaction with the adversary using a polynomial number of  rounds. Before we define  soundness, we restate the  experiment, defined in \cite{DBLP:conf/asiacrypt/ShachamW08}, that takes place between an environment $\mathcal{E}$ and  adversary $\mathcal{A}$. In this experiment, $\mathcal{A}$ plays the role of a corrupt party and $\mathcal{E}$ simulates an honest party's role. 


\begin{enumerate}
\item $\mathcal{E}$ executes $\mathtt{PoR.keyGen}(1^{\scriptscriptstyle\lambda})$ algorithm and provides public key, $pk$, to $\mathcal{A}$.   
\item $\mathcal{A}$ can pick  arbitrary file $u$, and  uses it to make queries to  $\mathcal{E}$ who runs  $\mathtt{PoR.setup}(1^{\scriptscriptstyle\lambda}, u,k)\rightarrow (u^{\scriptscriptstyle *}, \sigma, {\color{purple}pp})$   and returns the output to $\mathcal{A}$. Also, upon receiving the output of $\mathtt{PoR.setup}(1^{\scriptscriptstyle\lambda}, u,k)$, $\mathcal{A}$ can ask $\mathcal{E}$  to run   $\mathtt{PoR.genQuery}(1^{\scriptscriptstyle\lambda},k,{\color{purple}pp})\rightarrow \bm{q}$ and give the output to it. $\mathcal{A}$ can locally run $\mathtt{PoR.prove}(u^{\scriptscriptstyle *}, \sigma, \bm{q},pk,{\color{purple}pp})\rightarrow \pi$ to get its outputs as well. 
\item $\mathcal{A}$ can request $\mathcal{E}$ the execution of $\mathtt{PoR.verify}(\pi,\bm{q},k,{\color{purple}pp})$ for any $u$ used to query $\mathtt{PoR.setup}()$. Accordingly, $\mathcal{E}$ informs  $\mathcal{A}$ about the verification output. The adversary can send a polynomial number of queries to $\mathcal{E}$. Finally, $\mathcal{A}$ outputs metadata $\sigma$ returned from a setup query and the description of a prover, $\mathcal{A}'$, for any file it has already chosen above. 
\end{enumerate}

It is said that a cheating prover, $\mathcal{A}'$, is \emph{$\epsilon$-admissible} if it convincingly answers $\epsilon$ fraction of verification challenges. Informally, a PoR scheme supports extractability, if there is an extractor algorithm $\mathtt{Ext}(k,\sigma, \mathtt{P}')$, that takes as input the key $k$, metadata $\sigma$, and the description $\mathtt{P}'$ of the  machine implementing the prover's role $\mathcal{A}'$ and outputs the file, $u$. The extractor has the ability to reset the adversary to the beginning of the challenge phase and repeat this step polynomially many times for the purpose of extraction, i.e. the extractor can rewind $\mathtt{P}'$.


%XXXXThomas: \emph{What is actually the entity that can be rewinded (i.e., the adversary)? Is it $\mathtt{P}'$ or $\mathcal{A}'$?}XXXX. Aydin: done


\begin{definition}[$\epsilon$-soundness]\label{extractable} A PoR scheme is $\epsilon$-sound if   there exists an extraction algorithm $\mathtt{Ext}()$ such that, for every adversary $\mathcal{A}$ who plays the experiment above, and outputs an $\epsilon$-admissible cheating prover $\mathcal{A}'$ for a file $u$,  the extraction algorithm  recovers $u$ from $\mathcal{A}'$, given honest parties private key, public parameters, metadata and the description of $\mathcal{A}'$,  i.e. $\mathtt{Ext}(k,{\color{purple}pp},\sigma, \mathtt{P}')\rightarrow u$, except with a negligible probability. 
\end{definition}






%\begin{definition}[$\epsilon$-sound]\label{extractable} A PoR scheme is $\epsilon$-sound if  for every adversary, $\mathcal{A}$, who plays the experiment above, and outputs an $\epsilon$-admissible cheating prover $\mathcal{A}'$, for a file $u$,  there exists an extraction algorithm $\mathtt{Ext}()$, that recovers $u$ from $\mathcal{A}'$, given honest parties public-private keys, metadata and the description of $\mathcal{A}'$,  i.e. $\mathtt{Ext}(k,\sigma, \mathtt{P}')\rightarrow u$, except with a negligible probability. 
%\end{definition}


In contrast to the PoR definition in \cite{DBLP:journals/iacr/JuelsK07,DBLP:conf/asiacrypt/ShachamW08} where $\mathtt{PoR.genQuery}(.)$ is implicit, in the above we have explicitly stated  it, as it    plays an important role in this paper. Also, there are PoR protocols, e.g. \cite{MillerPermacoin}, that do not involve $\mathtt{PoR.keyGen}(.)$. Instead a set of public parameters/keys (e.g. file size or a root of Merkle tree) are output by $\mathtt{PoR.setup}(.)$. To make the PoR definition generic to capture both cases, we have explicitly included the public parameters $pp$ in the algorithms definitions too. 

\subsection{Notations} In the formal definitions in this paper, we often use bar symbol, i.e. `` $|$ '', for the sake of readability and to separate events from an experiment. It should not be confused with conditional probability's symbol. 





%In practice, we are interested in  pseudorandom functions that can be efficiently built on smart contracts given the tools  that a smart contract framework (e.g. Ethereum) offers. HMAC \cite{DBLP:conf/crypto/BellareCK96} satisfies the requirements above.


 

%\input{notation-Table.tex}



% !TEX root =R-zkCSP.tex

\section{Previous Work's Limitations and Our Solution's Overview}\label{sec::previous-work-limitations}
In this section, we first elaborate on the limitations of previous  work, and then outline how we address the limitations.  

% !TEX root =R-zkCSP.tex

\subsection{Limitations of zkCSP}

Recall,  the main purpose of zero-knowledge contingent service payment (zkCSP) protocol \cite{CampanelliGGN17}  is to minimise the role of smart contract as much as possible, so  (a) the verification's cost would be much lower, if a Turing-complete smart contract framework (e.g. Ethereum) is used, or (b) it can be implemented on  a non-Turing-complete contract framework (e.g. Bitcoin).   


Nevertheless, as we will show,  zkCSP suffers from  major issues; namely, it allows a malicious client to waste the server resources and  it leaks non-trivial information in real-time to the public. Also, when the payment is recurring (i.e. the server interacts with a client multiple times and/or the server interact with multiple clients), a malicious client can get a free ride from the server, in the sense that it can collect enough fresh information convincing him that the server is behaving honestly, without paying the server. In the following, we elaborate on the  above issues: 

%shorts fall of  offering what could be offered if a service verification is performed by a Turing complete smart contract, in both one-off and recurring payment. To illustrate the problem, without loss of generality, we assume the service is proofs of retrievability (PoR), an application of zkCSP suggested by the authors of \cite{CampanelliGGN17}. 


%Nevertheless, as we will show,  zkCSP shorts fall of  offering what could be offered if a service verification is performed by a Turing complete smart contract, in both one-off and recurring payment. To illustrate the problem, without loss of generality, we assume the service is proofs of retrievability (PoR), an application of zkCSP suggested by the authors of \cite{CampanelliGGN17}. 

%\
%
%\noindent\textbf{{Problem Statement}}: In short, zkCSP is not suitable for the case where the payment is recurring and a fixed prover deals with multiple clients over time. In this setting,  a malicious client can get a free ride from an honest server  without paying it, by  observing the server's past behavior. In the following, we elaborate on that.  Consider the case where an \emph{honest} server deals with $n$ clients. In this case, according to zkCSP, all clients have already uploaded their file to the server and want to pay it when they get an accepting PoR proof. Let's assume the server invokes the zkCSP with $n-1$ clients and accordingly it gets paid. However, $n$-th client can observe that the server has behaved honestly all the time, as it received all coins from all $n-1$ clients.  The malicious client may either never engages in the zkCSP protocol and never pay, even though it has been using the storage, or to get further assurance it engages in zkCSP up to the point where it has to deposit the coins (i.e. as soon as it accepts the zk proof provided by the prover, it aborts and again never pays). However, such problems would never arise in the naive  solution where the contract performs the service verification. We elaborate on that below. 

 
%\
%
%\noindent\textbf{{Trivial solution using Ethereum smart contract}}: Consider the case, where  client and server sign and agree on a smart contract. Then, when a client uploads its data on the server, it deposits and locks a fixed amount of coins in a contract, such that the client cannot get its coins back before the verification time. Later on, when PoR proof is needed, the server generates and sends the proof to the contract, who (a) verifies the proof, and (b) transfers the coins to the server if the proof is accepted. As it is evident, this setting (unlike zkCSP)  does not suffer from the above issues caused by a malicious client. 
%
%
%Hence, the zkCSP  does not offer  the same security guarantees that  the above trivial scheme offers.  Instead, it leaves rooms for attacks and  misuse of the service. 



\begin{enumerate}

%\item\textit{Coins Transfer Requires Clients' Approval}. zkCSP does not enforce the client to commit to a locked deposit that can be redeemed if and only if an accepting PoR proof is provided at the end of a billing cycle. In other words,  the transfer of the deposit requires the client's engagement and approval, after it receives a proof from the server. This allows  a malicious client who has been   using the service,  to avoid sending its approval to the contract  at the payment time. Therefore, a server's resources (e.g. the storage space allocated to the client's data) are wasted. As we will show shortly, in recurring payments (when the server deals with multiple clients) a malicious client can collect  convincing information about an honest server  that allows it to conclude that it has been  served honestly; even though, it does not pay the server and does not check the proof. Thus, it can  get a free ride from the server. 


\item\textit{Discrepancies between the Security Guarantees of Service and Fair Exchange Schemes}. zkCSP combines a service scheme secure against only a malicious server (where a client is assumed to be fully trusted) with a fair exchange protocol that takes into the consideration that either server or client might be corrupt.  The mismatch leads to serious consequences when the client misbehaves. For instance, in zkCSP the transfer of the deposit requires the client's engagement and approval, after it receives a proof from the server. This allows  a malicious client who has been using the service,  to avoid sending its approval to the contract or falsely claims the proofs are invalid at the payment time. So, an honest server's resources (e.g. the storage space allocated to the client's data) are wasted. As we will show shortly, in recurring payments (when the server deals with multiple clients) a malicious client can collect  convincing information about an honest server  that allows it to conclude that it has been  served honestly; even though, it does not pay the server and does not check the proof. Thus, it can  get a free ride from the server. 



%\item\textit{Coins Transfer is Public and Immediate}. 

\item\textit{Real-time Leakage of Verification Outputs and Deposit Amount}. zkCSP leaks in real-time non-trivial fresh information,  about  the server and clients, to the public. The leakage includes:
\begin{enumerate}
 \item\textit{proof verification status}: it is visible in the real time to everyone that the proof has been accepted or reject, that reflects whether  the server has successfully delivered the agreed-upon service or failed to do so that has serious immediate consequences for both the server and clients, e.g. lost revenue, negative press, stock value drop, or opening doors for attackers to exploit such incident. As an example, observing  proof's verification outputs (when a server deals with multiple clients) allows a malicious client to immediately construct a comprehensive background knowledge on the server's current behaviour and status, e.g. the server has been  acting honestly.  Such auxiliary information can assist the  client to more wisely exploit the above deposit issue (that can avoid sending the deposit); for instance, when the sever always acts honestly towards its clients, the client refuses to send the deposit and still  has  high confidence that the server has delivered the service. As another example, in the case of PoR, a malicious observer can  simply find out that the service is suffering hardware/software failure and exploit such vulnerability to  mount   social engineering attacks on clients or penetrate to the system. 

 
\item \textit{deposit amount}: the amount of deposit placed on the contract,  swiftly leaks non-trivial  information about the client to the public. For instance, in the case of PoR, an observer can learn the size of data outsourced to the server, service type or level of data sensitivity. The situation gets even worse if the client updates its data (e.g. delete or append) or asks the server for additional service (e.g. S3 Glacier or S3 Glacier Deep Archive\footnote{https://aws.amazon.com/s3/pricing/}), as an observer can learn such changes immediately by just observing the amount of deposit put by the client for each payment.  




\end{enumerate}
%For instance, the deposit transfer is public and takes place as soon as the client declares to the contract that it accepted the proof. An adversary can take advantage of this data leakage in certain cases. For instance,  observing the outputs of zkCSP invoked by the server and other clients,  allows a malicious client to eventually construct a comprehensive background knowledge on the server's current behaviour, e.g. the server has been  acting honestly. Note, the server (or its address) is fixed and known to the public, so the background knowledge can be linked to it. The amount of deposit placed on the contract,  immediately leaks non-trivial  information to the public that includes, data size, service type or level of data sensitivity. Since usually proof of services (and a bill) is provided by a server on a regular basis (e.g. every month) an adversary can learn in the \emph{real time} what kinds of service are being used by the client. Furthermore, the immediate transfer of coins, leaks to the public that whether the service delivered the agreed service or failed doing so. This also can have serious consequences, for instance in the case of PoR, a malicious observer can  simply find out that the service is suffering hardware/software failure and exploit such vulnerability to  mount certain  attacks, e.g. social engineering attacks on clients or penetrating to the system. 


\end{enumerate}


\noindent\textbf{{Strawman Solutions for the Two Problems}}. To address  Problem 1, one may slightly adjust the zkCSP protocol such that it would require  the client to deposit coins (long) before the server provides the ZK to it, with the hope that the client cannot avoid depositing after the server  provides ZK proofs. Nevertheless, this would not work, as the client after accepting the ZK proof, needs to send a confirmation message/transaction to the contract. But a malicious client can avoid doing so and later on, gets its deposit back.  Moreover, one might want to let the server pick a fresh address for each verifier/verification, to  preserver its pseudonymity with the hope that an observer cannot link clients to a server. However, for this to work, we have to assume multiple service providers  use the same protocol on the blockchain and all of them are pseudonymous. But, this is a strong assumption and may  not be always feasible. Alternatively, one may let a smart contract to perform the verification on the client's behalf, such that the client deposits its coins in the contract when it starts using the service. Then, the server sends its proof to the contract who performs the verification and pays the server if the proof is accepted. Even though this approach would solve problem 1 above, it imposes a high cost and defeats the purpose of zkCSP design. The reason is that the contract has to always run the verification algorithm that  has to be a publicly verifiable one, which  usually involves costly public-key primitives. 
















\input{Our-Solution}

% !TEX root =R-zkCSP.tex

\section{Definitions}\label{Definition}

In this section, first we present a generic formal definition of a verifiable service (VS) scheme. Second, we provide a formal definition of an enhanced VS scheme that  allows: (a) either party to be malicious and (b) a trusted third-party arbiter, to identify a corrupt party. We call the  scheme that offers the above features verifiable service with identifiable abort (VSID).   Third, we present a formal definition of an upgraded VSID scheme that  addresses   the issues we identified in Section \ref{sec::previous-work-limitations}, i.e. wasting server's resources and real-time leakage of service input and proof's status.  We call the latter scheme recurring contingent service payment (RC-S-P). 

% !TEX root =R-zkCSP.tex

\subsection{Verifiable Service (VS) Definition}\label{subsec:VS}
At a high-level, a verifiable service scheme is a two-party protocol in which a client chooses a function, $F$, and provides (an encoding of) $F$, and its input $u$, and a query $\bm{q}$ to a server.  The server is expected to evaluate $F$ on $u$ and $q$ and respond with the output. Then, the client  verifies that the output is indeed the output of the function computed on the provided input. In verifiable services, either the computation (on the input) or both the computation and storage of the input are delegated to the server. A verifiable service is defined as follows. 


\begin{definition}[VS Scheme]\label{service-def}
A verifiable service scheme VS $:=(\mathtt{VS.keyGen}, \mathtt{VS.setup},\mathtt{VS.genQuery},$ $\mathtt{VS.prove},\mathtt{VS.verify})$ consists of five algorithms defined as follows.


\begin{itemize}

\item[$\bullet$] $\mathtt{VS.keyGen}(1^{\lambda},F)\rightarrow k:=(sk,pk)$.  A probabilistic algorithm run by the client. It takes as input the security parameter $1^{\lambda}$ and a function, $F$, that will be run on the client's input by the server. It outputs a secret/public verification key pair $k$. 



\

\item[$\bullet$] $\mathtt{VS.setup}(1^{\lambda}, u,k, M)\rightarrow (u^{\scriptscriptstyle *},\sigma,{pp})$. It is run by the client. It takes as input the security parameter $1^{\lambda}$,  the service input $u$,    key pair $k$ and metadata generator deterministic function $M$, where $M$ is publicly known. If an encoding is needed, then it encodes $u$, that results $u^{\scriptscriptstyle *}$; otherwise, $u^*=u$. It outputs encoded input $u^{\scriptscriptstyle *}$, (possibly input dependent) public parameters {$pp$}, metadata $\sigma=M(u^{\scriptscriptstyle *},k,{pp})$. Right after that, the server is  given $u^{\scriptscriptstyle *}$,  $\sigma$, {$pp$}, and $pk$.




\

\item[$\bullet$] $\mathtt{VS.genQuery}(1^{\lambda}, \text{aux},k,Q,{pp})\rightarrow \bm{q}$. A probabilistic algorithm run by the client. It takes as input the security parameter $1^{\lambda}$, auxiliary information $\text{aux}$,   the key pair $k$,   query generator deterministic function $Q$ (where $Q$ is publicly known) and {public parameters $pp$}. It outputs  a query vector $\bm{q}=Q( \text{aux},k,{pp})$.  Depending on service types, $\bm{q}$ may be empty or contain only random strings. The output  is given to the server. 

\




\item[$\bullet$] $\mathtt{VS.prove}(u^{\scriptscriptstyle *},\sigma, \bm{q}, pk,{pp})\rightarrow \pi$. It is run by the server. It  takes as input the service encoded input $u^{\scriptscriptstyle *}$, metadata $\sigma$,   queries $\bm{q}$, public key $pk$, and { public parameters  $pp$}. It outputs a proof pair, $\pi :=(F(u^{\scriptscriptstyle *}, \bm{q},{pp}),\delta)$ containing the function evaluation for service input $u$, {public parameters $pp$}, and query $\bm{q}$, i.e. $h=F(u^{\scriptscriptstyle *},\bm{q},{pp})$, and a proof $\delta$ asserting the evaluation is performed correctly, where generating $\delta$ may involve $\sigma$. The output is given to the client. 

\

\item[$\bullet$] $\mathtt{VS.verify}(\pi, \bm{q}, k, {pp})\rightarrow d\in\{0,1\}$. It is run by the client. It takes as input the proof $\pi$,   query vector $\bm{q}$,   key  $k$, and {public parameters $pp$}.   In the case where $\mathtt{VS.verify}(.)$ is publicly verifiable then $k:=(\bot,pk)$,  and when it is privately verifiable $k:=(sk,pk)$. The algorithm  outputs $d=1$, if the proof is accepted; otherwise, it outputs $d=0$. 



\end{itemize}
\end{definition}


A verifiable service scheme has two main properties, \emph{correctness} and \emph{soundness}. Correctness requires that  the verification algorithm always accepts a proof generated by an honest prover.  It is formally stated below.

\begin{definition}[VS Correctness] A verifiable service scheme, VS, is  correct,  if for functions $F,Q,M$, and an auxiliary information $aux$, the key generation algorithm produces keys $\mathtt{VS.keyGen}(1^{\lambda},F)\rightarrow k:=(sk,pk)$ s.t. for any service input $u$, if $\ \mathtt{VS.setup}(1^{\lambda}, u,k,M)$ $\rightarrow (u^{\scriptscriptstyle *},\sigma, {pp})$,  $\mathtt{VS.genQuery}(1^{\lambda}, \text{aux},k,Q, {pp})\rightarrow \bm{q}$ and $\mathtt{VS.prove}(u^{\scriptscriptstyle *},\sigma, \bm{q},pk, {pp})\rightarrow \pi$, then $\mathtt{VS.verify}(\pi, \bm{q}, k, {pp})\rightarrow 1$
\end{definition}

Intuitively, a verifiable service is sound  if a malicious server cannot convince the verification algorithm to accept an incorrect output of $F$ except with negligible probability. Soundness is formally stated as follows.



%%%%%%%%%%%%%%%%%%%%%%%%%%%%

\begin{definition}[VS Soundness] A verifiable service VS is sound  for  functions $F,Q,M$, and an auxiliary information $aux$, if for any probabilistic polynomial time adversaries $\mathcal{A}$, there exists a negligible function $\mu(\cdot)$, such that for any security parameter $\lambda$: 
%\footnotesize{
\small{
$$ \Pr\left[
  \begin{array}{l}
F(u^{\scriptscriptstyle *},\bm{q},{pp})\neq h \wedge d=1
\end{array} \middle | 
    \begin{array} {l}
 \mathtt{VS.keyGen}(1^{\lambda},F)\rightarrow k:=(sk,pk)\\
 \mathcal{A}(1^{\scriptscriptstyle\lambda},pk, F)\rightarrow u\\
\mathtt{VS.setup}(1^{\lambda}, u,k,M)\rightarrow (u^{\scriptscriptstyle *},\sigma, {pp})\\
 \mathtt{VS.genQuery}(1^{\lambda}, \text{aux},k,Q,{pp})\rightarrow \bm{q}\\
 \mathcal{A}(\bm{q},u^{\scriptscriptstyle *},\sigma,{pp})\rightarrow \pi:=(h,\delta)\\
 \mathtt{VS.verify}(\pi,\bm{q},k,{pp})\rightarrow d\\
\end{array}    \right]\leq \mu(\lambda).$$
}
%where the probability is taken over uniform  choice of $k\stackrel{\scriptscriptstyle\$}\leftarrow \{0,1\}^{\scriptscriptstyle\lambda}$  as well as the randomness of $\mathcal{A}$.
\end{definition}




The above generic definition captures the core requirements of a wide range of verifiable services such as verifiable outsourced storage, i.e. Proofs of Retrievability \cite{DBLP:journals/iacr/JuelsK07,DBLP:conf/asiacrypt/ShachamW08} or Provable Data Possession \cite{DBLP:conf/ccs/AtenieseBCHKPS07,ShenT11}, verifiable computation, verifiable searchable encryption, and verifiable information retrieval, to name a few. Other additional  security properties mandated by certain services can be added to the above definition. Depending on the properties,  they can be  plugged into  the above definition with minimal adjustment to the definition. Privacy is an example. Alternatively, the definition can be upgraded to capture the  additional requirements.  The verifiable service with identifiable abort (VSID) and recurring contingent service payment (RC-S-P) definitions presented in this paper are two examples. 


\begin{remark}
It is not hard to see that the original PoR definition (presented in Section \ref{PoR-def}) captures VS definition. In particular, PoR's $\epsilon$-soundness captures VS's soundness.  Because in the $\epsilon$-soundness, the extractor algorithm interacts (many times) with the cheating prover  who must not be able to persuade the extractor to accept an invalid proof with a high probability and should provide  accepting proofs for non-negligible $\epsilon$ fraction of verification challenges.  The former property is exactly what VS soundness states. Thus, any protocol that realises PoR definition, realises VS definition as well. 
\end{remark}





% !TEX root =R-zkCSP.tex



\subsection{Verifiable Service with Identifiable Abort (VSID) Definition}\label{subsec:VSID}


A protocol that realises only VS's definition, would be merely secure against a malicious server and  assumes the client is honest.  Although this  assumption would suffice in certain settings and has been used before (e.g. in \cite{}), it is rather strong and not suitable in the real world, especially when there are monetary incentives (e.g. service payment)  that encourage a client to misbehave. Therefore, in the following we enhance VS's definition to allow  (a) either party to be malicious and (b) a trusted third-party, \emph{arbiter}, to identify a corrupt party. We call an upgraded verifiable service scheme with that features ``verifiable service with identifiable abort'' (VSID), inspired by the notion of secure multi-party computation with identifiable abort \cite{DBLP:conf/crypto/IshaiOZ14}. 




\



\begin{definition}[VSID Scheme]\label{service-def} A verifiable service with identifiable abort  VSID $:=(\mathtt{VSID.keyGen}, $ $\mathtt{VSID.setup}, $ $\mathtt{VSID.serve},\mathtt{VSID.genQuery}$,$\mathtt{VSID.checkQuery},$ $\mathtt{VSID.prove},$ $\mathtt{VSID.verify}, \mathtt{VSID.identify})$ consists of eight algorithms defined below.


\begin{itemize}
\item[$\bullet$] $\mathtt{VSID.keyGen}(1^{\lambda},F)\rightarrow k:=(sk,pk)$.  A probabilistic algorithm run by the client. It takes as input the security parameter $1^{\lambda}$ and a function, $F$, that will be run on the client's input by the server. It outputs a secret/public verification key pair $k$. %$k$ containing a secret key $sk$ and a set of public parameters  $pk$.




\

\item[$\bullet$] $\mathtt{VSID.setup}(1^{\lambda}, u,k,M)\rightarrow (u^{\scriptscriptstyle *}, {pp},e)$. It is run by the client. It takes as input the security parameter $1^{\lambda}$,  the service  input $u$,  the key pair $k$, and metadata generator deterministic function $M$, where $M$ is publicly known. If an encoding is needed, then it encodes $u$, that results $u^{\scriptscriptstyle *}$; otherwise, $u^{\scriptscriptstyle *}=u$. It outputs  $u^{\scriptscriptstyle *}$, {(possibly file dependent) public parameters $pp$} and $e:=(\sigma,w_{\sigma})$, where $\sigma=M(u^{\scriptscriptstyle *},k, {pp})$ is a metadata and $w_{\sigma}$ is a proof asserting the metadata is well-structured.  

\

\item[$\bullet$] $\mathtt{VSID.serve}(u^{\scriptscriptstyle *},e,pk, {pp})\rightarrow a\in\{0,1\}$. It is run by the server. It takes as input the encoded service input $u^{\scriptscriptstyle *}$, the pair $e:=(\sigma,w_{\sigma})$, public key $pk$, and {public parameters $pp$}. It outputs $a=1$, if the proof $w_{\sigma}$ is accepted, i.e. if the metadata is well-formed. Otherwise, it outputs $a=0$. 

\

\item[$\bullet$] $\mathtt{VSID.genQuery}(1^{\lambda},  \text{aux}, k,Q,{pp})\rightarrow c:=(\bm{q},\bm{w}_{\scriptscriptstyle q})$. A probabilistic algorithm run by the client. It takes as input the security parameter $1^{\lambda}$, auxiliary information $ \text{aux}$,  the key pair $k$,    query generator deterministic function $Q$ (where $Q$ is publicly known), and { public parameters $pp$}. It outputs a pair $c$ containing a query vector, $\bm{q}=Q( \text{aux},k,{pp})$,  and proofs, $\bm{w}_{\scriptscriptstyle q}$, proving the queries are well-structured. Depending on service types, $c$ might be  empty or  contain only random strings.





\

\item[$\bullet$] $\mathtt{VSID.checkQuery}(c, pk,{pp})\rightarrow b\in\{0,1\}$. It is run by the server. It takes as input a pair $c:=(\bm{q}, \bm{w}_{\scriptscriptstyle q})$ including queries and their proofs, as well as public key $pk$, and {public parameters $pp$}. It outputs $b = 1$ if the proofs $\bm{w}_{\scriptscriptstyle\sigma}$ are accepted, i.e. the queries are well-structured. Otherwise, it outputs $b = 0$.

\

\item[$\bullet$] $\mathtt{VSID.prove}(u^{\scriptscriptstyle *},\sigma,c,pk, { pp})\rightarrow \pi$. It is run by the server. It takes as input the  encoded service input $u^{\scriptscriptstyle *}$, metadata $\sigma$, a pair $c:=(\bm{q},\bm{w}_{\scriptscriptstyle q})$,  public key $pk$, and { public parameters $pp$}. It outputs a proof pair, $\pi:=(F(u^{\scriptscriptstyle *},\bm{q},{ pp}),\delta)$ containing the function evaluation, i.e. $h=F(u^{\scriptscriptstyle *},\bm{q},{ pp})$, and a proof $\delta$ asserting the evaluation is performed correctly, where computing $h$ may involve $pk$ and computing $\delta$ may involve $\sigma$.



\

\item[$\bullet$] $\mathtt{VSID.verify}(\pi, \bm{q}, k, { pp})\rightarrow d\in\{0,1\}$. It is run by the client. It takes as input the proof $\pi$, queries $\bm{q}$,  key pair $k$, and  { public parameters pp}. If the proof is accepted, it outputs $d=1$; otherwise, it outputs $d=0$. 

\

\item[$\bullet$]  $\mathtt{VSID.identify}(\pi,c,k,e,u^{\scriptscriptstyle *}, { pp})\rightarrow I\in \{ \mathcal{C},\mathcal{S},\bot\}$. It is run by a third-party arbiter. It takes as input the proof $\pi$, query pair $c:=(\bm{q},\bm{w}_{\scriptscriptstyle q})$,  key pair $k$, metadata pair  $e:=(\sigma,w_{\sigma})$,  $u^{\scriptscriptstyle *}$, and { public parameters pp}. If   proof $w_{\scriptscriptstyle\sigma}$ or $\bm{w}_{\scriptscriptstyle q}$  is rejected, then it outputs $I=\mathcal C$; otherwise, if proof $\pi$ is rejected it outputs $I=\mathcal S$.  Otherwise, if  $w_{\scriptscriptstyle\sigma}, \bm{w}_{\scriptscriptstyle q}$, and $\pi$ are accepted, it outputs   $I=\bot$. 



\end{itemize}
\end{definition}


A VSID scheme  has four main properties; namely, it is (a) correct, (b) sound, (c)  inputs of clients are well-formed, and (d) a corrupt party can be identified by an arbiter. In the following, we formally define each of them.  Correctness requires that  the verification algorithm always accepts a proof generated by an honest prover and both parties are identified as honest.  It is formally stated as follows.

\begin{definition}[VSID Correctness] A verifiable service with identifiable abort scheme  is  correct   if for any functions $F,M,Q$, and any auxiliary information $\text{aux}$, the key generation algorithm produces keys $\mathtt{VSID.keyGen}(1^{\lambda},F)\rightarrow k:=(sk,pk)$ such that for any service input $u$, if $\mathtt{VSID.setup}(1^{\lambda}, u,k,M)$ $\rightarrow (u^{\scriptscriptstyle *},{pp},e)$, $\mathtt{VSID.serve}(u^{\scriptscriptstyle *},e,pk,{pp})\rightarrow a$, $\mathtt{VSID.genQuery}(1^{\lambda}, $ $ \text{aux},k,Q,{pp})$ $ \rightarrow$ $ c$, $\mathtt{VSID.checkQuery}$ $(c, $ $pk,{pp})\rightarrow b,$  $\mathtt{VSID.prove}(u^{\scriptscriptstyle *},\sigma,c,pk,{pp})\rightarrow \pi$, and \\$\mathtt{VSID.verify}(\pi,\bm{q},k,{pp})\rightarrow d$, then  $\mathtt{VSID.identify}(\pi,c,k,e,u^{\scriptscriptstyle *},{pp})\rightarrow I=\bot \  \  \wedge \ \ a=1 \ \ \wedge \  b=1 \ \ \wedge \  d=1$
\end{definition}



Intuitively, a VSID is sound  if a malicious server cannot convince the client to accept an incorrect output of $F$ except with negligible probability.  It is  formally stated as follows.

\begin{definition}[VSID Soundness]\label{deff::VSID-Soundness}  A VSID  is sound for functions $F,Q,M$, and an auxiliary information $aux$, if for any probabilistic polynomial time adversary $\mathcal{A}_{\scriptscriptstyle 1}$, there exists a negligible function $\mu(\cdot)$, such that for any security parameter $\lambda$: 
%\footnotesize{
\small{
$$ \Pr\left[
  \begin{array}{l}
F(u^{\scriptscriptstyle *}, \bm{q},{pp})\neq h \wedge d=1
\end{array} \middle |
    \begin{array}{l}
    \mathtt{VSID.keyGen}(1^{\lambda},F)\rightarrow k:=(sk,pk)\\
    \mathcal{A}_{\scriptscriptstyle 1}(1^{\scriptscriptstyle\lambda},pk, F)\rightarrow u\\
    \mathtt{VSID.setup}(1^{\lambda}, u, k, M)\rightarrow (u^{\scriptscriptstyle *},e,{pp})\\
    \mathtt{VSID.genQuery}(1^{\lambda},  \text{aux},k,Q,{pp})\rightarrow c:=(\bm{q}, \bm{w}_{q})\\
     \mathcal{A}_{\scriptscriptstyle 1}(c,e, u^{\scriptscriptstyle *},{pp})\rightarrow \pi:=(h,\delta)\\
     \mathtt{VSID.verify}(\pi,\bm{q},k,{pp})\rightarrow d\\
\end{array}    \right]\leq \mu(\lambda).$$
}
%where the probability is taken over uniform  choice of $k\stackrel{\scriptscriptstyle\$}\leftarrow \{0,1\}^{\scriptscriptstyle\lambda}$ as well as the randomness of $\mathcal{A}_{\scriptscriptstyle 1}$.
\end{definition}


A VSID has   well-formed inputs, if a malicious client cannot persuade a server to serve it on  ill-structured inputs (i.e. to accept  incorrect outputs of $M$ or $Q$). Below, we state the property formally.

\begin{definition}[VSID Inputs Well-formedness]\label{deff::VSID-Inputs-Well-formedess}  A  VSID  has  well-formed inputs for functions $F,Q,M$, and an auxiliary information $aux$,  if for  any probabilistic polynomial time adversary $\mathcal{A}_{\scriptscriptstyle 2}$, there exists a negligible function $\mu(\cdot)$, such that for any security parameter $\lambda$: 
%\footnotesize{
\small{
$$ \Pr\left[
  \begin{array}{l}
  (M(u^{\scriptscriptstyle *},k,{pp})\neq \sigma \wedge a=1) \vee\\ (Q(\text{aux},k,{pp})\neq \bm{q}) \wedge  b=1)\\
%(M(u^{\scriptscriptstyle *},k)\neq \sigma \lor Q(\text{aux},k)\neq q) \wedge\\ (a=1 \ \vee b=1)
\end{array} \middle |
    \begin{array}{l}
    
    \mathcal{A}_{\scriptscriptstyle 2}(1^{\scriptscriptstyle\lambda},F,M,Q)\rightarrow (u^{\scriptscriptstyle *},k:=(sk,pk),e:=(\sigma,w_{\sigma}),{pp})\\
    \mathtt{VSID.serve}(u^{\scriptscriptstyle *},e,pk,{pp})\rightarrow a\\
       \mathcal{A}_{\scriptscriptstyle 2}(\text{aux},k)\rightarrow c:=(\bm{q},\bm{w}_{\scriptscriptstyle q})\\
    \mathtt{VSID.checkQuery}(c, pk,{pp})\rightarrow b\\
\end{array}    \right]\leq \mu(\lambda).$$
}
%where the probability is taken over the randomness used in  proofs $w_{\scriptscriptstyle q}$ and $w_{\scriptscriptstyle\sigma}$  as well as  the randomness of $\mathcal{A}_{\scriptscriptstyle 2}$.

\end{definition}

The   above property ensures an honest server can detect  a malicious client if the client provides ill-structured inputs. It is further required that a malicious party to be identified by an honest third-party, arbiter. This ensures that in the case of dispute (or false accusation) a malicious party can be pinpointed. A VSID supports  detectable abort if a corrupt party can escape from being identified, by the arbiter,  with only negligible probability.  Formally:



\begin{definition}[VSID Detectable Abort]\label{def::VSID-Identifiable-Abort}  A  VSID  supports  detectable for functions $F,Q,M$, and an auxiliary information $aux$, if  the following hold: 

\begin{enumerate}
\item For any PPT adversary $\mathcal{A}_{\scriptscriptstyle 1}$ there exists a negligible function $\mu_{\scriptscriptstyle 1}(\cdot)$ such that for any  security parameter $\lambda$:
{\small
$$ \Pr\left[
  \begin{array}{l}
 d=0 \wedge I\neq\mathcal{S}
\end{array} \middle |
    \begin{array}{l}
    \mathtt{VSID.keyGen}(1^{\lambda},F)\rightarrow k:=(sk,pk)\\
    \mathcal{A}_{\scriptscriptstyle 1}(1^{\scriptscriptstyle\lambda},pk, F)\rightarrow u\\
    \mathtt{VSID.setup}(1^{\lambda}, u, k, M,{pp})\rightarrow (u^{\scriptscriptstyle *},e)\\
    \mathtt{VSID.genQuery}(1^{\lambda},  \text{aux},k,Q,{pp})\rightarrow c:=(\bm{q}, \bm{w}_{q})\\
     \mathcal{A}_{\scriptscriptstyle 1}(c,e, u^{\scriptscriptstyle *},{pp})\rightarrow \pi:=(h,\delta)\\
     \mathtt{VSID.verify}(\pi,\bm{q},k,{pp})\rightarrow d\\
     \mathtt{VSID.identify}(\pi,c,k,e,u^{\scriptscriptstyle *},{pp})\rightarrow I\\
\end{array}    \right]\leq \mu(\lambda).$$
}
%\normalsize

\item For any PPT adversary $\mathcal{A}_{\scriptscriptstyle 2}$ there exists a negligible function $\mu_{\scriptscriptstyle 2}(\cdot)$ such that for any  security parameter $\lambda$:
{\small
$$ \Pr\left[
  \begin{array}{l}
 (a=0 \vee b=0)\wedge I\neq\mathcal{C}
\end{array} \middle |
    \begin{array}{l}
    
    \mathcal{A}_{\scriptscriptstyle 2}(1^{\scriptscriptstyle\lambda},F,M,Q)\rightarrow (u^{\scriptscriptstyle *},k:=(sk,pk),e:=(\sigma,w_{\sigma}),{pp})\\
    \mathtt{VSID.serve}(u^{\scriptscriptstyle *},e,pk,{pp})\rightarrow a\\
       \mathcal{A}_{\scriptscriptstyle 2}(\text{aux},k)\rightarrow c:=(\bm{q},\bm{w}_{\scriptscriptstyle q})\\
    \mathtt{VSID.checkQuery}(c, pk,{pp})\rightarrow b\\
    \mathtt{VSID.prove}(u^{\scriptscriptstyle *},\sigma,c,pk,{pp})\rightarrow \pi\\
    \mathtt{VSID.identify}(\pi,c,k,e,u^{\scriptscriptstyle *},{pp})\rightarrow I\\
\end{array}    \right]\leq \mu(\lambda).$$
}

\end{enumerate}
\end{definition}


\subsubsection{Lighter VSID Scheme ($\text{VSID}_{\scriptscriptstyle\text{light}}$)}\label{remark::light-VSID}
In the VSID  definition, algorithm  $\mathtt{VSID.identify}(.)$    allows an arbiter  to identify a misbehaving party even in the setup phase. Nevertheless, often it is sufficient  to let the arbiter pinpoint a corrupt party \emph{after} the client and server agree to deal with each other, i.e. after the setup when the server runs  $\mathtt{VSID.serve}(.)$ and outputs $1$. A VSID protocol that meets the latter (lighter) requirements, denoted by $\text{VSID}_{\scriptscriptstyle \text{light}}$, would impose lower costs  especially when $u$ and elements of $e$ are of large size. In $\text{VSID}_{\scriptscriptstyle\text{light}}$  the arbiter algorithm, i.e. $\mathtt{VSID.identify}(.)$, needs to take only $(\pi, c, k, e',{pp})$ as input, where $e'\subset e$. Note also $u^{\scriptscriptstyle *}$ is not given to the arbiter. In the light version, the arbiter   skips  checking the correctness of metadata. So, this requires two changes to the VSID definition, (a) the arbiter algorithm would be   $\mathtt{VSID.identify}(\pi,c,k,e',{pp})\rightarrow I$, and (b) in case 2, in Definition \ref{def::VSID-Identifiable-Abort} we would have $b=0\wedge I\neq \mathcal C$, so event $a=0$ is excluded. In this paper, any time we refer to $\text{VSID}_{\scriptscriptstyle\text{light}}$, we assume the above minor adjustments are applied to the VSID definition. 








% !TEX root =R-zkCSP.tex

\subsection{RC-S-P Definition}\label{sec::RC-S-P-Definition}


\begin{definition}[RC-S-P Scheme]\label{RC-S-P-def}
A recurring contingent service payment scheme RC-S-P  $=(\mathtt{RCSP}.\mathtt{keyGen}, $ $\mathtt{RCSP}.\mathtt{cInit}, $ $\mathtt{RCSP}.\mathtt{sInit},  $ $\mathtt{RCSP}.\mathtt{genQuery},\mathtt{RCSP}.\mathtt{prove},\mathtt{RCSP}.\mathtt{verify},\mathtt{RCSP}.\mathtt{resolve},\mathtt{RCSP}.\mathtt{pay})$ involves four parties; namely, client, server, arbiter and smart contract, and consists of eight algorithms defined as follows.


\begin{itemize}

\item[$\bullet$] $\mathtt{RCSP}.\mathtt{keyGen}(1^{\scriptscriptstyle\lambda},F)\rightarrow \bm{k}$.  A probabilistic algorithm run by the client. It takes as input security parameter $1^{\scriptscriptstyle\lambda}$  and  function $F$ that will be run on the client's input by the server. It outputs $\bm{k}$ that contains a secret/public verification key $k:(sk,pk)$ and a set of secret parameters, $k'$. %The client gives $pk$ to other parties.

\

\item[$\bullet$] $\mathtt{RCSP}.\mathtt{cInit}(1^{\scriptscriptstyle\lambda}, u,\bm{k}, M,z,cp)\rightarrow (u^{\scriptscriptstyle *},e,t,coin^{\scriptscriptstyle*}_{\scriptscriptstyle\mathcal C}, \bm{y}_{\scriptscriptstyle \mathcal  C},\bm{y}_{\scriptscriptstyle \mathcal  S})$. It is run by the client. It takes as input $1^{\scriptscriptstyle\lambda}$,  the service input: $u$,  the keys  $\bm{k}:[k, k']$,  metadata generator function: $M$, total number of verifications: $z$, and coin secret parameters $cp$ that includes a subset of  $k'$ and the actual amount of coins for each accepting service proof: $o$ and  for covering each potential dispute resolution's cost: $l$.  It encodes $u$, that results $u^{\scriptscriptstyle *}$.  It computes a metadata, $\sigma=M(u^{\scriptscriptstyle *},k)$, and a  proof $w_{\scriptscriptstyle\sigma}$ asserting the metadata is well-structured.  It constructs coin encoding token $t_{\scriptscriptstyle cp}$ that contains   $cp$, total coins the server should deposit $p_{\scriptscriptstyle\mathcal S}$, and   $cp$'s witness, $g_{\scriptscriptstyle cp}$.  It also constructs proof/query encoding token $t_{\scriptscriptstyle pq}$ that contains  secret  parameters $qp\in k'$ (used to encode the service queries/proofs)  and     $qp$'s witness,  $g_{\scriptscriptstyle qp}$.   Given a valid value and its witness  anyone can check if they match. It constructs two binary vectors $\bm{y}_{\scriptscriptstyle \mathcal  C}$ and $\bm{y}_{\scriptscriptstyle \mathcal  S}$ where they are set to $0$ and each string's length is $z$.  It outputs encoded input $u^{\scriptscriptstyle *}$, metadata-proof pair $e:=(\sigma,w_{\scriptscriptstyle\sigma})$, the two sets $t:\{t_{\scriptscriptstyle cp},t_{\scriptscriptstyle qp}\}$, the two binary vectors $\bm{y}_{\scriptscriptstyle \mathcal  C}, \bm{y}_{\scriptscriptstyle \mathcal  S}$, and  the encoded coins amount  $coin^{\scriptscriptstyle*}_{\scriptscriptstyle\mathcal C}$ (i.e. contains $o$ and $l$ coins in an encoded form). The client sends  $u^{\scriptscriptstyle *}$, $z$, $pk$, $e$, and $t \setminus  \{g_{\scriptscriptstyle cp},g_{\scriptscriptstyle qp},p_{\scriptscriptstyle\mathcal S}\}$ to the server and  sends $\{g_{\scriptscriptstyle cp},  g_{\scriptscriptstyle qp},p_{\scriptscriptstyle\mathcal S}\}$, $\bm{y}_{\scriptscriptstyle \mathcal  C}$, $\bm{y}_{\scriptscriptstyle \mathcal  S}$  and $coin^{\scriptscriptstyle*}_{\scriptscriptstyle\mathcal C}$ coins to the smart contract. 
%\item[$\bullet$] $\mathtt{RCSP}.\mathtt{cInit}(1^{\scriptscriptstyle\lambda}, u,\bm{k}, M,z,o,l,E_{\scriptscriptstyle 1})\rightarrow (u^{\scriptscriptstyle *},e,t,coin^{\scriptscriptstyle*}_{\scriptscriptstyle\mathcal C})$. It is run by the client. It takes as input $1^{\scriptscriptstyle\lambda}$,  the service input: $u$,  the key pair: $(k, k')$,  metadata generator function: $M$, total number of verifications: $z$,  the amount of coins for each accepting service proof: $o$,  coins amount  to cover each potential dispute resolution's cost: $l$ and  encoding function $E_{\scriptscriptstyle 1}$ to encode the coins actual amount.  It encodes $u$, that results $u^{\scriptscriptstyle *}$.  It computes a metadata, $\sigma=M(u^{\scriptscriptstyle *},k)$, and a publicly verifiable zero-knowledge proof $w_{\scriptscriptstyle\sigma}$ asserting the metadata is well-structured.  It constructs coins' token $t_{\scriptscriptstyle cp}$ that contains  coin's secret  parameters $cp$ and   sealed    parameters $sld_{\scriptscriptstyle cp}$. Given $cp$ and $sld_{\scriptscriptstyle cp}$ anyone can check if $sld_{\scriptscriptstyle cp}$ corresponds to $cp$. It constructs  encoding token $t_{\scriptscriptstyle pq}$ that contains  secret  parameters $qp\in k'$ (used to encode the service queries/proofs) and  sealed   parameters $sld_{\scriptscriptstyle qp}$, where  given $qp$ and $sld_{\scriptscriptstyle qp}$ anyone can check if $sld_{\scriptscriptstyle qp}$ corresponds to $qp$.  It outputs encoded input $u^{\scriptscriptstyle *}$, metadata-proof pair $e:(\sigma,w_{\scriptscriptstyle\sigma})$, the two sets $t:\{t_{\scriptscriptstyle cp},t_{\scriptscriptstyle qp}\}$ and  the encoded coins amount  $coin^{\scriptscriptstyle*}_{\scriptscriptstyle\mathcal C}=E_{\scriptscriptstyle 1}(z,cp)$. The server is  given $u^{\scriptscriptstyle *}$,  $e$, and $t \setminus  \{sld_{\scriptscriptstyle cp},sld_{\scriptscriptstyle qp}\}$ and  the smart contract  is given $sld_{\scriptscriptstyle cp}$,  $sld_{\scriptscriptstyle qp}$ and $coin^{\scriptscriptstyle*}_{\scriptscriptstyle\mathcal C}$ coins by the client. 


%\item[$\bullet$] $\mathtt{VS.setup}(1^{\lambda}, u,k, M)\rightarrow \sigma$. A probabilistic algorithm run by the client. It takes as input security parameter,  the service input: $u$,  the key pair: $k$ and metadata generator deterministic function: $M$, publicly known. It runs $M(u,k)$ and outputs  metadata: $\sigma$. Right after that, the server might be given $u$,  $\sigma$ and $pk$. 

\

\item[$\bullet$] $\mathtt{RCSP}.\mathtt{sInit}(u^{\scriptscriptstyle *},e, pk,z, t,\bm{y}_{\scriptscriptstyle \mathcal  C},\bm{y}_{\scriptscriptstyle \mathcal  S})\rightarrow (coin^{\scriptscriptstyle *}_{\scriptscriptstyle\mathcal S},a)$. It is  run by the server. It takes as input the  service encoded input $u^{\scriptscriptstyle *}$, metadata-proof pair $e:(\sigma,w_{\scriptscriptstyle\sigma})$,  public key $pk$, the total number of verifications $z$, sets $\{t_{\scriptscriptstyle cp},t_{\scriptscriptstyle qp}\}\in t$ (where $\{g_{\scriptscriptstyle cp},  g_{\scriptscriptstyle qp},p_{\scriptscriptstyle\mathcal S}\}\in t$ are read from the contract), and reads the two binary strings $\bm{y}_{\scriptscriptstyle \mathcal  C}, \bm{y}_{\scriptscriptstyle \mathcal  S}$ from the contract.  It verifies the validity of the elements in $e$ and  $t$. Also, it ensures $\bm{y}_{\scriptscriptstyle \mathcal  C}$ and $\bm{y}_{\scriptscriptstyle \mathcal  S}$ have been set to $0$. If all approved, then it encodes the amount of its coins  $coin^{\scriptscriptstyle*}_{\scriptscriptstyle\mathcal S}$ and sets $a=1$. Otherwise, it sets $coin^{\scriptscriptstyle *}_{\scriptscriptstyle\mathcal S}=\bot$ and $a=0$. It outputs $coin^{\scriptscriptstyle *}_{\scriptscriptstyle\mathcal S}$ and $a$. The smart contract is given $coin^{\scriptscriptstyle *}_{\scriptscriptstyle\mathcal S}$ coins and $a$.  

\

\item[$\bullet$] $\mathtt{RCSP}.\mathtt{genQuery}(1^{\scriptscriptstyle\lambda}, \text{aux},k,Q, en)\rightarrow c^{\scriptscriptstyle *}_{\scriptscriptstyle j}$. A probabilistic algorithm run by the client. It takes as input  $1^{\scriptscriptstyle\lambda}$,  auxiliary information $\text{aux}$,  the key pair $k$,   query generator deterministic function $Q$,  and $en$ that contains the encoding/decoding token  $t_{\scriptscriptstyle qp}$ and  encoding/decoding functions $(E,D)$  used to encode/decode service's proofs and queries.  It computes a pair $c_{\scriptscriptstyle j}$ containing a query vector $\bm{q}_{\scriptscriptstyle j}:Q( \text{aux},k)$,  and proof $\bm{w}_{\scriptscriptstyle q_{_{\scaleto{j}{\size}}}}$ proving the query is well-structured. It outputs the encoding of the pair, $c^{\scriptscriptstyle *}_{\scriptscriptstyle j}=E(c_{\scriptscriptstyle j},t_{\scriptscriptstyle qp})$. The output is sent to the smart contract. 

%\scaleto{c}{1pt}

%a set of (random) challenges $c$ or empty set if the proving/verification  algorithms do not require any challenges. %The challenges set is published to a bulletin board.
\

\item[$\bullet$] $\mathtt{RCSP}.\mathtt{prove}(u^{\scriptscriptstyle *},  \sigma,  c^{\scriptscriptstyle *}_{\scriptscriptstyle j},pk, en)\rightarrow (b_{\scriptscriptstyle j},m_{\scriptscriptstyle \mathcal{S},j},\pi^{\scriptscriptstyle *}_{\scriptscriptstyle j})$. It is run by the server. It  takes service input: $u^{\scriptscriptstyle *}$, metadata: $\sigma$,   encoded query $c^{\scriptscriptstyle *}_{\scriptscriptstyle j}$, public key $pk$,  and $en:=(E,D,t_{\scriptscriptstyle qp})$.  It checks the validity of decoded query, $c_{\scriptscriptstyle j}=D(c^{\scriptscriptstyle *}_{\scriptscriptstyle j},t_{\scriptscriptstyle qp})$.  If it is rejected, then it sets $b_{\scriptscriptstyle j}=0$ and constructs a complaint $m_{\scriptscriptstyle \mathcal{S},j}$. Otherwise,  it   sets $b_{\scriptscriptstyle j}=1$ and $m_{\scriptscriptstyle \mathcal{S},j}=\bot$. It outputs $b_{\scriptscriptstyle j},m_{\scriptscriptstyle \mathcal{S},j}$,   and  encoded proof $\pi^{\scriptscriptstyle *}_{\scriptscriptstyle j}=E(\pi_{\scriptscriptstyle j},t_{\scriptscriptstyle qp})$, where  $\pi_{\scriptscriptstyle j}$ contains $h_{\scriptscriptstyle j}=F(u^{\scriptscriptstyle *},\bm{q}_{\scriptscriptstyle j})$ and a proof $\delta_{\scriptscriptstyle j}$  asserting the evaluation is performed correctly ($\pi_{\scriptscriptstyle j}$ may contain dummy values if $b_{\scriptscriptstyle j}=0$). The contract is given $\pi^{\scriptscriptstyle *}_{\scriptscriptstyle j}$.




\

\item[$\bullet$] $\mathtt{RCSP}.\mathtt{verify}(\pi^{\scriptscriptstyle *}_{\scriptscriptstyle j}, \bm{q}_{\scriptscriptstyle j},k,en)\rightarrow (d_{\scriptscriptstyle j},m_{\scriptscriptstyle \mathcal{C},j})$. A deterministic algorithm run by the client. It takes encoded proof $\pi^{\scriptscriptstyle *}_{\scriptscriptstyle j}$,  query $\bm{q}_{\scriptscriptstyle j}\in c_{\scriptscriptstyle j}$,  verification key  $k$,  and $en:=(E,D,t_{\scriptscriptstyle qp})$. If the decoded proof $\pi_{\scriptscriptstyle j}=D(\pi^{\scriptscriptstyle *}_{\scriptscriptstyle j},t_{\scriptscriptstyle qp})$ is rejected,  it outputs $d_{\scriptscriptstyle j}=0$ and a complaint $m_{\scriptscriptstyle \mathcal{C},j}$. Otherwise, it outputs  $d_{\scriptscriptstyle j}=1$ and $m_{\scriptscriptstyle \mathcal{C},j}=\bot$




%, where $k=pk$  in the case where $\mathtt{VS.verify}()$ is publicly verifiable, or $k:(sk,pk)$ when it is privately verifiable

\


\item[$\bullet$] $\mathtt{RCSP}.\mathtt{resolve}(\bm{m}_{\scriptscriptstyle \mathcal{C}},\bm{m}_{\scriptscriptstyle \mathcal{S}},z, {\bm{\pi}}^{\scriptscriptstyle *}, {\bm{c}}^{\scriptscriptstyle *},pk,en)\rightarrow (\bm{y}_{\scriptscriptstyle \mathcal C},\bm{y}_{\scriptscriptstyle \mathcal S})$. It is run by the arbiter. It takes client's complaints $\bm{m}_{\scriptscriptstyle \mathcal{C}}$,   server's complaints   $\bm{m}_{\scriptscriptstyle \mathcal{S}}$, total number of verifications $z$, all encoded proofs $ {\bm{\pi}}^{\scriptscriptstyle *}$, all encoded queries $ {\bm{c}}^{\scriptscriptstyle *}$, public key $pk$, and $en:=(E,D,t_{\scriptscriptstyle qp})$.  It verifies the token,   decoded queries, and proofs.  It outputs two binary strings $\bm{y}_{\scriptscriptstyle \mathcal C}$ and $\bm{y}_{\scriptscriptstyle \mathcal S}$ of length $z$, it sets each element of $\bm{y}_{\scriptscriptstyle \mathcal E}$ to one, i.e. $y_{\scriptscriptstyle \mathcal E,j}=1$, if party $\mathcal{E}\in\{\mathcal{C},\mathcal{S}\}$ has misbehaved in $j$-th verification (i.e. provided invalid query or service proof). 


%\item[$\bullet$] $\mathtt{RCSP}.\mathtt{resolve}(m_{\scriptscriptstyle \mathcal{C}},m_{\scriptscriptstyle \mathcal{S}},z, {\bm{\pi}}^{\scriptscriptstyle *}, {\bm{c}}^{\scriptscriptstyle *},pk,E^{\scriptscriptstyle -1}_{\scriptscriptstyle 2},t_{\scriptscriptstyle qp},a)\rightarrow (y_{\scriptscriptstyle \mathcal C},y_{\scriptscriptstyle \mathcal S})$. It is run by the arbiter. It takes client's complaint $m_{\scriptscriptstyle \mathcal{C}}$,   server's complaints   $m_{\scriptscriptstyle \mathcal{S}}$, total number of verifications $z$, all encoded proofs $ {\bm{\pi}}^{\scriptscriptstyle *}$, all encoded queries $ {\bm{c}}^{\scriptscriptstyle *}$, public key $pk$, decoding function $E^{\scriptscriptstyle -1}_{\scriptscriptstyle 2}$, encoding token $t_{\scriptscriptstyle qp}$ and  $a$.  It proceeds only if $a=1$. It verifies the token,   decoded queries, and proofs.  It outputs total number of times the client and server misbehaved, i.e. $y_{\scriptscriptstyle \mathcal C}$ and $y_{\scriptscriptstyle \mathcal S}$ respectively. 

\
	
\item[$\bullet$] $\mathtt{RCSP}.\mathtt{pay}(\bm{y}_{\scriptscriptstyle \mathcal C},\bm{y}_{\scriptscriptstyle \mathcal S},t_{\scriptscriptstyle cp},a,coin^{\scriptscriptstyle *}_{\scriptscriptstyle\mathcal C},coin^{\scriptscriptstyle *}_{\scriptscriptstyle\mathcal S})\rightarrow ({\bm{coin}}_{\scriptscriptstyle\mathcal C},{\bm{coin}}_{\scriptscriptstyle\mathcal S},{\bm{coin}}_{\scriptscriptstyle\mathcal Ar})$. It is run by the smart contract and can be invoked by the client or server. It takes  two binary strings $\bm{y}_{\scriptscriptstyle \mathcal C}$ and $\bm{y}_{\scriptscriptstyle \mathcal S}$ that indicates which party misbehaved in each verification, and coins' token $t_{\scriptscriptstyle cp}$. If $a=1$ and  $coin^{\scriptscriptstyle *}_{\scriptscriptstyle\mathcal S}= p_{\scriptscriptstyle\mathcal S}$, then it verifies the validity of $t_{\scriptscriptstyle cp}$. If $t_{\scriptscriptstyle cp}$ is rejected, then it aborts. 
If it is accepted, then it constructs  vector ${\bm{coin}}_{\scriptscriptstyle\mathcal I}$, where $\mathcal{I}\in\{\mathcal{C},\mathcal{S},\mathcal{A}r\}$; It sends ${{coin}}_{\scriptscriptstyle\mathcal{I},j}\in{\bm{coin}}_{\scriptscriptstyle\mathcal I}$ coins to party $\mathcal{I}$ for each $j$-th verification. Otherwise (i.e. $a=0$ or $coin^{\scriptscriptstyle *}_{\scriptscriptstyle\mathcal S}\neq p_{\scriptscriptstyle\mathcal S}\in t_{\scriptscriptstyle cp}$)  it only sends $coin^{\scriptscriptstyle *}_{\scriptscriptstyle\mathcal C}$ to $\mathcal C$. 

%sets $coin_{\scriptscriptstyle\mathcal C,0}=coin^{\scriptscriptstyle *}_{\scriptscriptstyle\mathcal C}, \bm{coin}_{\scriptscriptstyle\mathcal S}=\bot$, and $\bm{coin}_{\scriptscriptstyle\mathcal Ar}=\bot$.  %It outputs ${\bm{coin}}_{\scriptscriptstyle\mathcal C},{\bm{coin}}_{\scriptscriptstyle\mathcal S}$, and ${\bm{coin}}_{\scriptscriptstyle\mathcal Ar}$

%sends $coin_{\scriptscriptstyle\mathcal C}, coin_{\scriptscriptstyle\mathcal S}$, and $coin_{\scriptscriptstyle\mathcal Ar}$ coins to the client, server, and arbiter respectively. Otherwise, it does nothing. 

\end{itemize}
\end{definition}

In the above definition algorithms $\mathtt{RCSP}.\mathtt{genQuery}(),\mathtt{RCSP}.\mathtt{prove}(),\mathtt{RCSP}.\mathtt{verify}()$ and $\mathtt{RCSP}.\mathtt{resolve}()$ implicitly  take $a$ as another input and  execute only if $a=1$; however, for the sake of simplicity we  avoided explicitly stating it  in the definition. Informally, a recurring contingent service payment (RC-S-P)  scheme satisfies correctness and security. At a high level, correctness requires that by the end of the protocol's execution  (that involves  honest client and server) the client receives all $z$ valid service proofs while the server gets paid  for the proofs, without the involvement of the arbiter. More specifically, it requires that the server accepts an honest client's encoded data and query while the honest client accepts the server's valid service proof (and no one is identified as misbehaving party). Moreover, the honest client gets back all its deposited coins minus the service payment, the honest server gets back all its deposited coins  plus the service payment and the arbiter receives nothing. It is formally stated as below.



\begin{definition}[RC-S-P Correctness] A recurring contingent service payment  scheme  is  correct   if for any function $F, Q, M, E, D$, and auxiliary information $\text{aux}$, the key generation algorithm produces keys $\mathtt{RCSP}.\mathtt{keyGen}(1^{\scriptscriptstyle\lambda},F)\rightarrow \bm{k}$,   such that $\forall u \in \mathtt{Domain}(F)$ if $\mathtt{RCSP}.\mathtt{cInit}(1^{\scriptscriptstyle\lambda}, u,\bm{k}, M,z,cp) \rightarrow (u^{\scriptscriptstyle *}, e, t, coin^{\scriptscriptstyle*}_{\scriptscriptstyle\mathcal C},\bm{y}_{\scriptscriptstyle \mathcal  C},\bm{y}_{\scriptscriptstyle \mathcal  S})$, $\mathtt{RCSP}.\mathtt{sInit} (u^{\scriptscriptstyle *}, e, pk, z, t,\bm{y}_{\scriptscriptstyle \mathcal C},$ $\bm{y}_{\scriptscriptstyle \mathcal S})\rightarrow (coin^{\scriptscriptstyle *}_{\scriptscriptstyle\mathcal S}, a)$,   $\forall j:\Big($$\mathtt{RCSP}.\mathtt{genQuery}(1^{\lambda}, \text{aux}, k, Q,  en)\rightarrow c^{\scriptscriptstyle *}_{\scriptscriptstyle j}$, $\mathtt{RCSP}.\mathtt{prove}(u^{\scriptscriptstyle *},  \sigma, c^{\scriptscriptstyle *}_{\scriptscriptstyle j}, pk,  en)\rightarrow (b_{\scriptscriptstyle j},m_{\scriptscriptstyle\mathcal {S},j},\pi^{\scriptscriptstyle *}_{\scriptscriptstyle j})$, \\$\mathtt{RCSP}.\mathtt{verify}(\pi^{\scriptscriptstyle *}_{\scriptscriptstyle j}, \bm{q}_{\scriptscriptstyle j},k,en)\rightarrow (d_{\scriptscriptstyle j}, m_{\scriptscriptstyle\mathcal {C},j})$$\Big), \mathtt{RCSP}.\mathtt{resolve}(\bm{m}_{\scriptscriptstyle \mathcal{C}},\bm{m}_{\scriptscriptstyle \mathcal{S}},z, {\bm{\pi}}^{\scriptscriptstyle *}, {\bm{c}}^{\scriptscriptstyle *},pk,en)\rightarrow (\bm{y}_{\scriptscriptstyle \mathcal C},\bm{y}_{\scriptscriptstyle \mathcal S})$, \\$\mathtt{RCSP}.\mathtt{pay}(\bm{y}_{\scriptscriptstyle \mathcal C},\bm{y}_{\scriptscriptstyle \mathcal S},t_{\scriptscriptstyle cp},a,coin^{\scriptscriptstyle *}_{\scriptscriptstyle\mathcal C},coin^{\scriptscriptstyle *}_{\scriptscriptstyle\mathcal S})\rightarrow (\bm{coin}_{\scriptscriptstyle\mathcal C},\bm{coin}_{\scriptscriptstyle\mathcal S},\bm{coin}_{\scriptscriptstyle\mathcal Ar})$, then $(a=1)\wedge (\bigwedge\limits^{\scriptscriptstyle z}_{\scriptscriptstyle j=1} b_{\scriptscriptstyle j}= \bigwedge\limits^{\scriptscriptstyle z}_{\scriptscriptstyle j=1} d_{\scriptscriptstyle j}=1)\wedge (\bm{y}_{\scriptscriptstyle \mathcal C}=\bm{y}_{\scriptscriptstyle \mathcal S}=0) \wedge (\sum\limits^{\scriptscriptstyle z}_{\scriptscriptstyle j=1}coin_{\scriptscriptstyle\mathcal C,j}=coin^{\scriptscriptstyle *}_{\scriptscriptstyle\mathcal C}-o\cdot z) \wedge  (\sum\limits^{\scriptscriptstyle z}_{\scriptscriptstyle j=1}coin_{\scriptscriptstyle\mathcal S,j}=coin^{\scriptscriptstyle *}_{\scriptscriptstyle\mathcal S}+o\cdot z)  \wedge (\sum\limits^{\scriptscriptstyle z}_{\scriptscriptstyle j=1}coin_{\scriptscriptstyle\mathcal Ar,j}=0)$
\end{definition}


A RC-S-P scheme is said to be security if it satisfies  three main properties,   (a)  security against malicious server, (b) security against malicious client, and (c) privacy.  In the following, we formally define each of them. Intuitively, security against a malicious server states that (at the end of the protocol execution) either the client for each verification gets a valid proof and gets back its deposit minus the service payment or the client gets its deposit back (for $j$-th verification) and the arbiter receives $l$ coins. In other words, for each  $j$-th verification, the security requires that only with a negligible probability the adversary wins if it  provides either (a)  correct evaluation of the function on the service input but it makes the client  withdraw an incorrect amount of coins (i.e. something other than its deposit minus service payment) or (b) incorrect evaluation of the function on the service input,  but either persuades the client or the arbiter to accept it (i.e.  $b_{\scriptscriptstyle j}=1$ or $y_{\scriptscriptstyle\mathcal S}[j]=0$) or makes them  withdraw incorrect amount of coins (i.e. $coin_{\scriptscriptstyle\mathcal{C},j}\neq \frac{coin_{\scriptscriptstyle\mathcal C}^{\scriptscriptstyle*}}{z}$ or  $coin_{\scriptscriptstyle\mathcal{A}r,j}\neq l$ coins).  Below, we formalize this intuition. 




%Intuitively, security against a malicious server states that (at the end of the protocol execution) either the client for each verification gets a valid proof or the client gets its coins back (for $j$-th verification) and the arbiter receives $l$ coins. In other words, for each  $j$-th verification, the adversary wins if it provides incorrect evaluation of the function on the service input,  but (with a high probability) either persuades the client or the arbiter to accept it (i.e.  $b_{\scriptscriptstyle j}=1$ or $y_{\scriptscriptstyle\mathcal S}[j]=0$) or makes them to withdraw incorrect amount of coins (i.e. $coin_{\scriptscriptstyle\mathcal{C},j}\neq \frac{coin_{\scriptscriptstyle\mathcal C}^{\scriptscriptstyle*}}{z}$ and  $coin_{\scriptscriptstyle\mathcal{A}r,j}\neq l$ coins).  Below, we formalize this intuition with an experiment, $\textbf{Exp}^{\scriptscriptstyle(1)}_{\scriptscriptstyle\mathcal {A}}[]$: 


%
%\small{
%$$
%    \begin{array}{l}
%    \mathtt{RCSP.keyGen}(1^{\lambda},F)\rightarrow \bm{k}\\
%    \mathcal{A}(1^{\scriptscriptstyle\lambda},pk, F)\rightarrow (u,\text{state})\\
%    \mathtt{RCSP}.\mathtt{cInit}(1^{\scriptscriptstyle\lambda}, u,\bm{k}, M,z,cp)\rightarrow (u^{\scriptscriptstyle *},e,t,coin^{\scriptscriptstyle*}_{\scriptscriptstyle\mathcal C})\\
%    \mathcal{A}(u^{\scriptscriptstyle *},e, pk,z, t)\rightarrow (coin^{\scriptscriptstyle *}_{\scriptscriptstyle\mathcal S},a)\\
%	\mathtt{RCSP}.\mathtt{genQuery}(1^{\lambda}, \text{aux},k,Q,j, en)\rightarrow c^{\scriptscriptstyle *}_{\scriptscriptstyle j}\\
%	
%     \mathcal{A}(\text{state},c^{\scriptscriptstyle *}_{\scriptscriptstyle j},\sigma, u^{\scriptscriptstyle *},j,en,a)\rightarrow 
%     (b_{\scriptscriptstyle j},m_{\scriptscriptstyle \mathcal{A},j},y^{\scriptscriptstyle *}_{\scriptscriptstyle j},\delta^{\scriptscriptstyle *}_{\scriptscriptstyle j})\\ 
%   \mathtt{RCSP}.\mathtt{verify}(\pi^{\scriptscriptstyle *}_{\scriptscriptstyle j}, q,k,j,en)\rightarrow (d_{\scriptscriptstyle j},m_{\scriptscriptstyle \mathcal{C},j})\\
%   \mathtt{RCSP}.\mathtt{resolve}(\bm{m}_{\scriptscriptstyle \mathcal{C}},\bm{m}_{\scriptscriptstyle \mathcal{A}},z, {\bm{\pi}}^{\scriptscriptstyle *}, {\bm{c}}^{\scriptscriptstyle *},pk,en)\rightarrow (y_{\scriptscriptstyle \mathcal C},y_{\scriptscriptstyle \mathcal S})\\
%   \mathtt{RCSP}.\mathtt{pay}(y_{\scriptscriptstyle \mathcal C},y_{\scriptscriptstyle \mathcal S},t_{\scriptscriptstyle cp})\rightarrow ({\bm{coin}}_{\scriptscriptstyle\mathcal C},{\bm{coin}}_{\scriptscriptstyle\mathcal S},{\bm{coin}}_{\scriptscriptstyle\mathcal Ar})\\
%\end{array}$$
%}
%
%
%
%The output of the experiment is $1$, i.e. $\textit{\textbf{Exp}}^{\scriptscriptstyle(1)}_{\scriptscriptstyle\mathcal {A},\text{RC-S-P}}()=1$, if the adversary wins. 
%
%\begin{definition}[RC-S-P Security Against Malicious Server]\label{deff::}  A RC-S-P  is secure against a malicious server  for a function $F$, if for any security parameter $\lambda$, any auxiliary information $aux$, any $Q$ and any probabilistic polynomial time adversary $\mathcal{A}$, there exists a negligible function $\mu(.)$, such that: 
%%\footnotesize{
%
%$$Pr[\textbf{Exp}^{\scriptscriptstyle(1)}_{\scriptscriptstyle\mathcal {A},\text{RC-S-P}}()=1]=Pr\left[
%\begin{array}{l}
%\left((F_{\scriptscriptstyle q}(u^{\scriptscriptstyle *})= y_{\scriptscriptstyle j}\ \wedge coin_{\scriptscriptstyle\mathcal{C},j}\neq  \frac{coin_{\scriptscriptstyle\mathcal C}^{\scriptscriptstyle*}}{z}-o\right)
% \vee\\
%  \left(F_{\scriptscriptstyle q}(u^{\scriptscriptstyle *})\neq y_{\scriptscriptstyle j}\ \wedge (d_{\scriptscriptstyle j}=1	\vee y_{\scriptscriptstyle \mathcal S}[j]=0 \ \vee coin_{\scriptscriptstyle\mathcal{C},j}\neq \frac{coin_{\scriptscriptstyle\mathcal C}^{\scriptscriptstyle*}}{z} \ \wedge coin_{\scriptscriptstyle\mathcal{A}r,j}\neq l)\right)\end{array}\right]\leq \mu(\lambda)$$
%\end{definition}



%_{\scriptscriptstyle q_{_{\scaleto{j}{\size}}}}


\begin{definition}[RC-S-P Security Against Malicious Server]\label{deff::RC-S-P-SecurityAgainstMaliciousServer}  A RC-S-P  is secure against a malicious server  for a function $F$, if for any security parameter $\lambda$, any auxiliary information $\text{aux}$, every $j$ (where $1\leq j\leq z$), any $Q,M, E, D$ and any probabilistic polynomial time adversary $\mathcal{A}$, there exists a negligible function $\mu(.)$, such that: 
%\footnotesize{
{\small
$$ Pr\left[
  \begin{array}{l}
  \left(F(u^{\scriptscriptstyle *}, \bm{q}_{\scriptscriptstyle j})= h_{\scriptscriptstyle j}\ \wedge coin_{\scriptscriptstyle\mathcal{C},j}\neq  \frac{coin_{\scriptscriptstyle\mathcal C}^{\scriptscriptstyle*}}{z}-o\right)
  \vee\\
\Big(
F(u^{\scriptscriptstyle *},\bm{q}_{\scriptscriptstyle j})\neq h_{\scriptscriptstyle j} \ \wedge (d_{\scriptscriptstyle j}=1	\vee y_{\scriptscriptstyle \mathcal S}[j]=0 \ \vee \\(coin_{\scriptscriptstyle\mathcal{C},j}\neq \frac{coin_{\scriptscriptstyle\mathcal C}^{\scriptscriptstyle*}}{z} \ \vee  coin_{\scriptscriptstyle\mathcal{A}r,j}\neq l)) \Big)\\
%\text{s.t.}\\
%y_{\scriptscriptstyle j}= E^{\scriptscriptstyle -1}_{\scriptscriptstyle 2}(y^{\scriptscriptstyle *}_{\scriptscriptstyle j},t_{\scriptscriptstyle qp})\\
%E^{\scriptscriptstyle -1}_{\scriptscriptstyle 2},t_{\scriptscriptstyle qp}\in en\\
\end{array} \middle |
    \begin{array}{l}
    \mathtt{RCSP.keyGen}(1^{\lambda},F)\rightarrow \bm{k}\\
    \mathcal{A}(1^{\scriptscriptstyle\lambda},pk, F)\rightarrow u\\
  
    \mathtt{RCSP}.\mathtt{cInit}(1^{\scriptscriptstyle\lambda}, u,\bm{k}, M,z,cp)\rightarrow (u^{\scriptscriptstyle *},e,t,coin^{\scriptscriptstyle*}_{\scriptscriptstyle\mathcal C}, \bm{y}_{\scriptscriptstyle \mathcal  C},\bm{y}_{\scriptscriptstyle \mathcal  S})\\
    
    \mathcal{A}(u^{\scriptscriptstyle *},e, pk,z, t,\bm{y}_{\scriptscriptstyle \mathcal  C},\bm{y}_{\scriptscriptstyle \mathcal  S})\rightarrow (coin^{\scriptscriptstyle *}_{\scriptscriptstyle\mathcal S},a)\\
	\mathtt{RCSP}.\mathtt{genQuery}(1^{\lambda}, \text{aux},k,Q, en)\rightarrow c^{\scriptscriptstyle *}_{\scriptscriptstyle j}\\
	
     \mathcal{A}(c^{\scriptscriptstyle *}_{\scriptscriptstyle j},\sigma, u^{\scriptscriptstyle *},j,en,a)\rightarrow 
     (b_{\scriptscriptstyle j},m_{\scriptscriptstyle \mathcal{A},j},h^{\scriptscriptstyle *}_{\scriptscriptstyle j},\delta^{\scriptscriptstyle *}_{\scriptscriptstyle j})\\
     
     %(\pi^{\scriptscriptstyle *}_{\scriptscriptstyle j},b_{\scriptscriptstyle j})\\
     
\mathtt{RCSP}.\mathtt{verify}(\pi^{\scriptscriptstyle *}_{\scriptscriptstyle j}, \bm{q}_{\scriptscriptstyle j}, k, en)\rightarrow (d_{\scriptscriptstyle j},m_{\scriptscriptstyle \mathcal{C},j})\\
   \mathtt{RCSP}.\mathtt{resolve}(\bm{m}_{\scriptscriptstyle \mathcal{C}},\bm{m}_{\scriptscriptstyle \mathcal{A}},z, {\bm{\pi}}^{\scriptscriptstyle *}, {\bm{c}}^{\scriptscriptstyle *},pk,en)\rightarrow (\bm{y}_{\scriptscriptstyle \mathcal C},\bm{y}_{\scriptscriptstyle \mathcal S})\\
   \mathtt{RCSP}.\mathtt{pay}(\bm{y}_{\scriptscriptstyle \mathcal C},\bm{y}_{\scriptscriptstyle \mathcal S},t_{\scriptscriptstyle cp},a,coin^{\scriptscriptstyle *}_{\scriptscriptstyle\mathcal C},coin^{\scriptscriptstyle *}_{\scriptscriptstyle\mathcal S})\rightarrow ({\bm{coin}}_{\scriptscriptstyle\mathcal C},{\bm{coin}}_{\scriptscriptstyle\mathcal S},{\bm{coin}}_{\scriptscriptstyle\mathcal Ar})\\
\end{array}    \right]\leq \mu(\lambda)$$
}
where $h_{\scriptscriptstyle j}= D(h^{\scriptscriptstyle *}_{\scriptscriptstyle j},t_{\scriptscriptstyle qp})$, $
D,t_{\scriptscriptstyle qp}\in en$, $\sigma\in e$, and the probability is taken over uniform  choice of $k,k'\stackrel{\scriptscriptstyle\$}\leftarrow \{0,1\}^{\scriptscriptstyle\lambda}$, where $k,k'\in\bm{k}$, as well as the randomness of $\mathcal{A}$ and the randomness used in witnesses $g_{\scriptscriptstyle cp},g_{\scriptscriptstyle qp}\in t$.
\end{definition}




%\begin{definition}[RC-S-P Correctness] A recurring contingent service payment  scheme  is  correct   if for any function $F$, the key generation algorithm produces keys $\mathtt{RCSP}.\mathtt{keyGen}(1^{\scriptscriptstyle\lambda},F)\rightarrow (k,k')$,   such that $\forall u \in \mathtt{Domain}(F)$ if $\mathtt{RCSP}.\mathtt{cInit}(1^{\scriptscriptstyle\lambda}, u,k,k', M,z,o,l,E_{\scriptscriptstyle 1}) \rightarrow (u^{\scriptscriptstyle *}, e, t, coin^{\scriptscriptstyle*}_{\scriptscriptstyle\mathcal C})$, $\mathtt{RCSP}.\mathtt{sInit} (u^{\scriptscriptstyle *}, e, pk, z, t, E_{\scriptscriptstyle 1})\rightarrow (coin^{\scriptscriptstyle *}_{\scriptscriptstyle\mathcal S}, a)$, \ \\ $\mathtt{RCSP}.\mathtt{genQuery}(1^{\lambda}, \text{aux}, k, Q, j, E_{\scriptscriptstyle 2}, qp, a)\rightarrow c^{\scriptscriptstyle *}_{\scriptscriptstyle j}$, $\mathtt{RCSP}.\mathtt{prove}(u^{\scriptscriptstyle *},  \sigma, c^{\scriptscriptstyle *}_{\scriptscriptstyle j}, pk, j, E^{\scriptscriptstyle -1}_{\scriptscriptstyle 2},qp,a)\rightarrow (\pi^{\scriptscriptstyle *}_{\scriptscriptstyle j}, b_{\scriptscriptstyle j})$, \\$\mathtt{RCSP}.\mathtt{verify}(\pi^{\scriptscriptstyle *}_{\scriptscriptstyle j}, q,k,j,E^{\scriptscriptstyle -1}_{\scriptscriptstyle 2},qp,a)\rightarrow d_{\scriptscriptstyle j}$, $\mathtt{RCSP}.\mathtt{resolve}(m_{\scriptscriptstyle \mathcal{C}},m_{\scriptscriptstyle \mathcal{S}},z, {\bm{\pi}}^{\scriptscriptstyle *}, {\bm{c}}^{\scriptscriptstyle *},pk,E^{\scriptscriptstyle -1}_{\scriptscriptstyle 2},t_{\scriptscriptstyle qp},a)\rightarrow (y_{\scriptscriptstyle \mathcal C},y_{\scriptscriptstyle \mathcal S})$, \\$\mathtt{RCSP}.\mathtt{pay}(y_{\scriptscriptstyle \mathcal C},y_{\scriptscriptstyle \mathcal S},t_{\scriptscriptstyle cp},a)\rightarrow (coin_{\scriptscriptstyle\mathcal C},coin_{\scriptscriptstyle\mathcal S},coin_{\scriptscriptstyle\mathcal Ar})$, s.t. $a=1\wedge (\bigwedge\limits^{\scriptscriptstyle z}_{\scriptscriptstyle j=1} b_{\scriptscriptstyle j}= \bigwedge\limits^{\scriptscriptstyle z}_{\scriptscriptstyle j=1} d_{\scriptscriptstyle j}=1)\wedge (y_{\scriptscriptstyle \mathcal C}=y_{\scriptscriptstyle \mathcal S}=0) \wedge (coin_{\scriptscriptstyle\mathcal C}=coin^{\scriptscriptstyle *}_{\scriptscriptstyle\mathcal C}-oz) \wedge  (coin_{\scriptscriptstyle\mathcal S}=coin^{\scriptscriptstyle *}_{\scriptscriptstyle\mathcal S}+oz)   \wedge (coin_{\scriptscriptstyle\mathcal Ar}=0) $
%\end{definition}

 
 
 
 
Informally, security against a malicious client requires that, for each  $j$-th verification, a malicious client with a negligible probability  wins if it provides either (a) valid metadata and query but makes the server receive incorrect amount of coins (something other than its deposit plus the service payment), or (b) invalid metadata or query but convinces the server to accept either of them, or (c) invalid query but persuades the arbiter to accept it, or  makes them to withdraw an incorrect amount of coins (i.e. $coin_{\scriptscriptstyle\mathcal{S},j}\neq \frac{coin_{\scriptscriptstyle\mathcal S}^{\scriptscriptstyle*}}{z}+o$ or  $coin_{\scriptscriptstyle\mathcal{A}r,j}\neq l$ coins). Below, we formally state it. 
 
 
 
 
 \begin{definition}[RC-S-P Security Against Malicious Client]\label{deff::RC-S-P-Security-Against Malicious-Client}  A RC-S-P  is secure against a malicious client  for a function $F$, if for any security parameter $\lambda$, every $j$ (where $1\leq j\leq z$), any $Q,M,E,D$ and any probabilistic polynomial time adversary $\mathcal{A}$, there exists a negligible function $\mu(.)$, such that: 


{\small
$$ Pr\left[
  \begin{array}{l}
 \Big(M(u^{\scriptscriptstyle *},k)= \sigma  \wedge Q(\text{aux},k)= \bm{q}_{\scriptscriptstyle j} \wedge\\
  coin_{\scriptscriptstyle\mathcal{S},j}\neq  \frac{coin_{\scriptscriptstyle\mathcal S}^{\scriptscriptstyle*}}{z}+o\Big) \vee
  \\
  \Big(M(u^{\scriptscriptstyle *},k)\neq \sigma \wedge a=1\Big)\vee\\ \Big(Q(\text{aux},k)\neq \bm{q}_{\scriptscriptstyle j} \wedge
 b_{\scriptscriptstyle j}=1  \Big)  \vee\\

  \Big(Q(\text{aux},k)\neq \bm{q}_{\scriptscriptstyle j} \wedge
  (y_{\scriptscriptstyle\mathcal{C}}[j]=0 \ \vee \\ coin_{\scriptscriptstyle\mathcal{S},j}\neq \frac{coin_{\scriptscriptstyle\mathcal S}^{\scriptscriptstyle*}}{z}+o \ \vee  coin_{\scriptscriptstyle\mathcal{A}r,j}\neq l) 
  \Big)
 \\
 
\end{array} \middle |
    \begin{array}{l}
   \mathcal{A}(1^{\lambda},F)\rightarrow (u^{\scriptscriptstyle *},z,\bm{k},e,t,coin^{\scriptscriptstyle*}_{\scriptscriptstyle\mathcal C},en, \text{aux},\bm{y}_{\scriptscriptstyle \mathcal  C},\bm{y}_{\scriptscriptstyle \mathcal  S})\\
   
   \mathtt{RCSP}.\mathtt{sInit}(u^{\scriptscriptstyle *},e, pk,z, t,\bm{y}_{\scriptscriptstyle \mathcal  C},\bm{y}_{\scriptscriptstyle \mathcal  S})\rightarrow (coin^{\scriptscriptstyle *}_{\scriptscriptstyle\mathcal S},a)\\
   
	\mathcal{A}(coin^{\scriptscriptstyle *}_{\scriptscriptstyle\mathcal S},a, \text{aux},\bm{k},Q,j, en)\rightarrow c^{\scriptscriptstyle *}_{\scriptscriptstyle j}\\

     \mathtt{RCSP}.\mathtt{prove}(u^{\scriptscriptstyle *},  \sigma,  c^{\scriptscriptstyle *}_{\scriptscriptstyle j},pk, en)\rightarrow (b_{\scriptscriptstyle j},m_{\scriptscriptstyle \mathcal{S},j},\pi^{\scriptscriptstyle *}_{\scriptscriptstyle j})\\
     
 \mathcal{A}(\pi^{\scriptscriptstyle *}_{\scriptscriptstyle j}, \bm{q}_{\scriptscriptstyle j}, k, j, en)\rightarrow (d_{\scriptscriptstyle j},m_{\scriptscriptstyle \mathcal{A},j})\\
   
   
   \mathtt{RCSP}.\mathtt{resolve}(\bm{m}_{\scriptscriptstyle \mathcal{A}},\bm{m}_{\scriptscriptstyle \mathcal{S}},z, {\bm{\pi}}^{\scriptscriptstyle *}, {\bm{c}}^{\scriptscriptstyle *},pk,en)\rightarrow (\bm{y}_{\scriptscriptstyle \mathcal C},\bm{y}_{\scriptscriptstyle \mathcal S})\\
   \mathtt{RCSP}.\mathtt{pay}(\bm{y}_{\scriptscriptstyle \mathcal C},\bm{y}_{\scriptscriptstyle \mathcal S},t_{\scriptscriptstyle cp},a,coin^{\scriptscriptstyle *}_{\scriptscriptstyle\mathcal C},coin^{\scriptscriptstyle *}_{\scriptscriptstyle\mathcal S})\rightarrow ({\bm{coin}}_{\scriptscriptstyle\mathcal C},{\bm{coin}}_{\scriptscriptstyle\mathcal S},{\bm{coin}}_{\scriptscriptstyle\mathcal Ar})\\
\end{array}    \right]\leq \mu(\lambda)$$
}
where $\bm{q}_{\scriptscriptstyle j}\in D(c^{\scriptscriptstyle *}_{\scriptscriptstyle j},t_{\scriptscriptstyle qp})$, $
D,t_{\scriptscriptstyle qp}\in en$, $\sigma\in e$, and   the probability is taken over the randomness used in witnesses $g_{\scriptscriptstyle cp},g_{\scriptscriptstyle qp}\in t$ as well as  the randomness used in proofs $\bm{w}_{\scriptscriptstyle q_{_{\scaleto{j}{\size}}}}\in c_{\scriptscriptstyle j}$ and $w_{\scriptscriptstyle\sigma}\in e$ and  the randomness of $\mathcal{A}$.
\end{definition}
 










%\small{
%$$ Pr\left[
%  \begin{array}{l}
% \Big((M(u^{\scriptscriptstyle *},k)= \sigma  \wedge Q(\text{aux},k)= q_{\scriptscriptstyle j}) \wedge\\
%  (coin_{\scriptscriptstyle\mathcal{S},j}\neq  \frac{coin_{\scriptscriptstyle\mathcal S}^{\scriptscriptstyle*}}{z}+o)\Big) \vee
%  \\
%  \Big((M(u^{\scriptscriptstyle *},k)\neq \sigma \lor Q(\text{aux},k)\neq q_{\scriptscriptstyle j}) \wedge\\
%  (a=1 \ \vee b_{\scriptscriptstyle j}=1\ \vee y_{\scriptscriptstyle\mathcal{C}}[j]=0 \ \vee \\ coin_{\scriptscriptstyle\mathcal{S},j}\neq \frac{coin_{\scriptscriptstyle\mathcal S}^{\scriptscriptstyle*}}{z}+o \ \vee  coin_{\scriptscriptstyle\mathcal{A}r,j}\neq l) 
%  \Big)
% \\
% 
%\end{array} \middle |
%    \begin{array}{l}
%   \mathcal{A}(1^{\lambda},F)\rightarrow (u^{\scriptscriptstyle *},z,\bm{k},e,t,coin^{\scriptscriptstyle*}_{\scriptscriptstyle\mathcal C},en, \text{aux},\bm{y}_{\scriptscriptstyle \mathcal  C},\bm{y}_{\scriptscriptstyle \mathcal  S})\\
%   
%   \mathtt{RCSP}.\mathtt{sInit}(u^{\scriptscriptstyle *},e, pk,z, t,\bm{y}_{\scriptscriptstyle \mathcal  C},\bm{y}_{\scriptscriptstyle \mathcal  S})\rightarrow (coin^{\scriptscriptstyle *}_{\scriptscriptstyle\mathcal S},a)\\
%   
%	\mathcal{A}(coin^{\scriptscriptstyle *}_{\scriptscriptstyle\mathcal S},a, \text{aux},\bm{k},Q,j, en)\rightarrow c^{\scriptscriptstyle *}_{\scriptscriptstyle j}\\
%
%     \mathtt{RCSP}.\mathtt{prove}(u^{\scriptscriptstyle *},  \sigma,  c^{\scriptscriptstyle *}_{\scriptscriptstyle j},pk, en)\rightarrow (b_{\scriptscriptstyle j},m_{\scriptscriptstyle \mathcal{S},j},\pi^{\scriptscriptstyle *}_{\scriptscriptstyle j})\\
%     
% \mathcal{A}(\pi^{\scriptscriptstyle *}_{\scriptscriptstyle j}, q,k,j,en)\rightarrow (d_{\scriptscriptstyle j},m_{\scriptscriptstyle \mathcal{A},j})\\
%   
%   
%   \mathtt{RCSP}.\mathtt{resolve}(\bm{m}_{\scriptscriptstyle \mathcal{A}},\bm{m}_{\scriptscriptstyle \mathcal{S}},z, {\bm{\pi}}^{\scriptscriptstyle *}, {\bm{c}}^{\scriptscriptstyle *},pk,en)\rightarrow (\bm{y}_{\scriptscriptstyle \mathcal C},\bm{y}_{\scriptscriptstyle \mathcal S})\\
%   \mathtt{RCSP}.\mathtt{pay}(\bm{y}_{\scriptscriptstyle \mathcal C},\bm{y}_{\scriptscriptstyle \mathcal S},t_{\scriptscriptstyle cp},a,coin^{\scriptscriptstyle *}_{\scriptscriptstyle\mathcal C},coin^{\scriptscriptstyle *}_{\scriptscriptstyle\mathcal S})\rightarrow ({\bm{coin}}_{\scriptscriptstyle\mathcal C},{\bm{coin}}_{\scriptscriptstyle\mathcal S},{\bm{coin}}_{\scriptscriptstyle\mathcal Ar})\\
%\end{array}    \right]\leq \mu(\lambda)$$
%}
%where $q_{\scriptscriptstyle j}\in D(c^{\scriptscriptstyle *}_{\scriptscriptstyle j},t_{\scriptscriptstyle qp})$, $
%D,t_{\scriptscriptstyle qp}\in en$, $\sigma\in e$, and   the probability is taken over the randomness used in witnesses $w_{\scriptscriptstyle cp},w_{\scriptscriptstyle qp}\in t$ as well as  the randomness used in proofs $w_{\scriptscriptstyle q_{_{\scaleto{j}{\size}}}}\in c_{\scriptscriptstyle j}$ and $w_{\scriptscriptstyle\sigma}\in e$ and  the randomness of $\mathcal{A}$.
%\end{definition}
% 
 Note, in the above definition, an honest server either does not deposit (e.g. when $a=0$) or if it deposits (i.e. agrees to serve) ultimately receives its deposit \emph{plus the service payment} (with a high probability). Informally, RC-S-P is privacy preserving if it guarantees the privacy of service input (e.g. outsourced file) and service proof's status during the private time bubble. In the following, we formally define them.
 

  \begin{definition}[RC-S-P Privacy]\label{deff::RC-S-P-Privacy}  A RC-S-P   preserves privacy, if for any security parameter $\lambda$, any auxiliary information $\text{aux}$, any $F, Q,M, E$ and $D$, the following hold:
   
\begin{enumerate}

\item For any PPT adversary $\mathcal{A}_{\scriptscriptstyle 1}$ there exists a negligible function $\mu_{\scriptscriptstyle 1}(\cdot)$ such that
  
{\small
$$ Pr\left[
  \begin{array}{l}
  \mathcal{A}_{\scriptscriptstyle 1}(\bm{c}^{\scriptscriptstyle *},coin^{\scriptscriptstyle *}_{\scriptscriptstyle S},coin^{\scriptscriptstyle *}_{\scriptscriptstyle C},g_{\scriptscriptstyle cp},  g_{\scriptscriptstyle qp},\bm{\pi}^{\scriptscriptstyle *},a)\rightarrow\beta\\
\end{array} \middle |
    \begin{array}{l}
    \mathtt{RCSP.keyGen}(1^{\lambda},F)\rightarrow \bm{k}\\
    \mathcal{A}_{\scriptscriptstyle 1}(1^{\scriptscriptstyle\lambda},pk, F)\rightarrow (u_{\scriptscriptstyle 0},u_{\scriptscriptstyle 1})\\
    \beta\stackrel{\scriptscriptstyle\$}\leftarrow\{0,1\}\\
   \mathtt{RCSP}.\mathtt{cInit}(1^{\scriptscriptstyle\lambda}, u_{\scriptscriptstyle\beta},\bm{k}, M,z,cp)\rightarrow (u_{\scriptscriptstyle\beta}^{\scriptscriptstyle *},e,t,coin^{\scriptscriptstyle*}_{\scriptscriptstyle\mathcal C}, \bm{y}_{\scriptscriptstyle \mathcal  C},\bm{y}_{\scriptscriptstyle \mathcal  S})\\
\mathtt{RCSP}.\mathtt{sInit}(u_{\scriptscriptstyle\beta}^{\scriptscriptstyle *},e, pk,z, t,\bm{y}_{\scriptscriptstyle \mathcal  C},\bm{y}_{\scriptscriptstyle \mathcal  S})\rightarrow (coin^{\scriptscriptstyle *}_{\scriptscriptstyle\mathcal S},a)\\
\mathtt{RCSP}.\mathtt{genQuery}(1^{\scriptscriptstyle\lambda}, \text{aux},k,Q_{\scriptscriptstyle\beta},j, en)\rightarrow c^{\scriptscriptstyle *}_{\scriptscriptstyle j}\\
     \mathtt{RCSP}.\mathtt{prove}(u^{\scriptscriptstyle *}_{\scriptscriptstyle\beta},  \sigma,  c^{\scriptscriptstyle *}_{\scriptscriptstyle j},pk, en)\rightarrow (b_{\scriptscriptstyle j},m_{\scriptscriptstyle \mathcal{S},j},\pi^{\scriptscriptstyle *}_{\scriptscriptstyle j})\\  
   \mathtt{RCSP}.\mathtt{verify}(\pi^{\scriptscriptstyle *}_{\scriptscriptstyle j}, \bm{q}_{\scriptscriptstyle j},k,en)\rightarrow (d_{\scriptscriptstyle j},m_{\scriptscriptstyle \mathcal{C},j})\\
\end{array}\right]\leq \frac{1}{2}+\mu_{\scriptscriptstyle 1}(\lambda)$$
}




\item For any PPT adversary $\mathcal{A}_{\scriptscriptstyle 2}$ who plays the above game, there exists a negligible function $\mu_{\scriptscriptstyle 2}(\cdot)$ such that


{\small
$$ Pr\left[\mathcal{A}_{\scriptscriptstyle 2}(\bm{c}^{\scriptscriptstyle *},coin^{\scriptscriptstyle *}_{\scriptscriptstyle \mathcal S},coin^{\scriptscriptstyle *}_{\scriptscriptstyle\mathcal C}, g_{\scriptscriptstyle cp},  g_{\scriptscriptstyle qp},\bm{\pi}^{\scriptscriptstyle *},a)\rightarrow (d_{\scriptscriptstyle j},j)\right]\leq \frac{1}{2}+ \mu_{\scriptscriptstyle 2}(\lambda)$$
}

where $\bm{c}^{\scriptscriptstyle *}=[{c}^{\scriptscriptstyle *}_{\scriptscriptstyle 1},...,{c}^{\scriptscriptstyle *}_{\scriptscriptstyle z}]$ and $\bm{\pi}^{\scriptscriptstyle *}=[\pi^{\scriptscriptstyle *}_{\scriptscriptstyle 1},...,\pi^{\scriptscriptstyle *}_{\scriptscriptstyle z}]$.  Also, depending on the choice of $\beta$, algorithm $\mathtt{RCSP}.\mathtt{genQuery}()$ outputs an encoded valid query, when the input is $Q_{\scriptscriptstyle 1}$, or outputs an encoded invalid query, when the input is $Q_{\scriptscriptstyle 0}$. The probability is taken over uniform  choice of $k,k'\stackrel{\scriptscriptstyle\$}\leftarrow \{0,1\}^{\scriptscriptstyle\lambda}$, where $k,k'\in\bm{k}$, the randomness of $\mathcal{A}_{\scriptscriptstyle 1}$ and $\mathcal{A}_{\scriptscriptstyle 2}$,  the randomness used in witnesses $g_{\scriptscriptstyle cp}, g_{\scriptscriptstyle qp}\in t$,  the randomness used in  $E$.
\end{enumerate}
\end{definition}
 
In the above definition, the experiment is allowed to produce  invalid queries. It is required that  the privacy holds regardless of the queries status, i.e. whether they are valid or invalid, as long as the are encoded and provided.  
 
 
 
%  \begin{definition}[RC-S-P Service Proof Privacy]\label{deff::RC-S-P-ServiceProofPrivacy}  A RC-S-P   preserves the privacy of service proof (status), if for any security parameter $\lambda$, any auxiliary information $\text{aux}$, any $F,Q,M,E,D$ and any probabilistic polynomial time adversary $\mathcal{A}$ that plays the  game in Definition \ref{deff::RC-S-P-Input-Privacy}, there exists a negligible function $\mu(.)$, such that: 
%
%\small{
%$$ Pr\left[\mathcal{A}(\bm{c}^{\scriptscriptstyle *},coin^{\scriptscriptstyle *}_{\scriptscriptstyle \mathcal S},coin^{\scriptscriptstyle *}_{\scriptscriptstyle\mathcal C}, g_{\scriptscriptstyle cp},  g_{\scriptscriptstyle qp},\bm{\pi}^{\scriptscriptstyle *},a)\rightarrow (d_{\scriptscriptstyle j},j)\right]\leq \frac{1}{2}+ \mu(\lambda)$$
%}
%where $\bm{c}^{\scriptscriptstyle *}=[{c}^{\scriptscriptstyle *}_{\scriptscriptstyle 1},...,{c}^{\scriptscriptstyle *}_{\scriptscriptstyle z}]$ and $\bm{\pi}^{\scriptscriptstyle *}=[\pi^{\scriptscriptstyle *}_{\scriptscriptstyle 1},...,\pi^{\scriptscriptstyle *}_{\scriptscriptstyle z}]$.  The probability is taken over uniform  choice of $k,k'\stackrel{\scriptscriptstyle\$}\leftarrow \{0,1\}^{\scriptscriptstyle\lambda}$, where $k,k'\in\bm{k}$, the randomness of $\mathcal{A}$,  the randomness used in witnesses $g_{\scriptscriptstyle cp}, g_{\scriptscriptstyle qp}\in t$,  the randomness used in  $E$.
%\end{definition}
 
 
 
 %-----------------------------------
% \begin{definition}[RC-S-P Service Input Privacy]\label{deff::RC-S-P-Input-Privacy}  A RC-S-P   preserves the privacy of service input, if for any security parameter $\lambda$, any auxiliary information $\text{aux}$, any $F, Q,M, E,D$ and any probabilistic polynomial time adversary $\mathcal{A}$, there exists a negligible function $\mu(.)$, such that: 
%
%\small{
%$$ Pr\left[
%  \begin{array}{l}
%  \mathcal{A}(\bm{c}^{\scriptscriptstyle *},coin^{\scriptscriptstyle *}_{\scriptscriptstyle S},coin^{\scriptscriptstyle *}_{\scriptscriptstyle C},g_{\scriptscriptstyle cp},  g_{\scriptscriptstyle qp},\bm{\pi}^{\scriptscriptstyle *},a)\rightarrow\beta\\
%
%
%\end{array} \middle |
%    \begin{array}{l}
%    \mathtt{RCSP.keyGen}(1^{\lambda},F)\rightarrow \bm{k}\\
%    \mathcal{A}(1^{\scriptscriptstyle\lambda},pk, F)\rightarrow (u_{\scriptscriptstyle 0},u_{\scriptscriptstyle 1})\\
%    \beta\stackrel{\scriptscriptstyle\$}\leftarrow\{0,1\}\\
%   
%   \mathtt{RCSP}.\mathtt{cInit}(1^{\scriptscriptstyle\lambda}, u_{\scriptscriptstyle\beta},\bm{k}, M,z,cp)\rightarrow (u_{\scriptscriptstyle\beta}^{\scriptscriptstyle *},e,t,coin^{\scriptscriptstyle*}_{\scriptscriptstyle\mathcal C}, \bm{y}_{\scriptscriptstyle \mathcal  C},\bm{y}_{\scriptscriptstyle \mathcal  S})\\
%
%\mathtt{RCSP}.\mathtt{sInit}(u_{\scriptscriptstyle\beta}^{\scriptscriptstyle *},e, pk,z, t,\bm{y}_{\scriptscriptstyle \mathcal  C},\bm{y}_{\scriptscriptstyle \mathcal  S})\rightarrow (coin^{\scriptscriptstyle *}_{\scriptscriptstyle\mathcal S},a)\\
%
%\mathtt{RCSP}.\mathtt{genQuery}(1^{\scriptscriptstyle\lambda}, \text{aux},k,Q,j, en)\rightarrow c^{\scriptscriptstyle *}_{\scriptscriptstyle j}\\
%	
%     
%     \mathtt{RCSP}.\mathtt{prove}(u^{\scriptscriptstyle *}_{\scriptscriptstyle\beta},  \sigma,  c^{\scriptscriptstyle *}_{\scriptscriptstyle j},pk, en)\rightarrow (b_{\scriptscriptstyle j},m_{\scriptscriptstyle \mathcal{S},j},\pi^{\scriptscriptstyle *}_{\scriptscriptstyle j})\\
%     
%   \mathtt{RCSP}.\mathtt{verify}(\pi^{\scriptscriptstyle *}_{\scriptscriptstyle j}, \bm{q}_{\scriptscriptstyle j},k,en)\rightarrow (d_{\scriptscriptstyle j},m_{\scriptscriptstyle \mathcal{C},j})\\
%\end{array}    \right]\leq \frac{1}{2}+\mu(\lambda)$$
%}
%where $\bm{c}^{\scriptscriptstyle *}=[{c}^{\scriptscriptstyle *}_{\scriptscriptstyle 1},...,{c}^{\scriptscriptstyle *}_{\scriptscriptstyle z}]$ and $\bm{\pi}^{\scriptscriptstyle *}=[\pi^{\scriptscriptstyle *}_{\scriptscriptstyle 1},...,\pi^{\scriptscriptstyle *}_{\scriptscriptstyle z}]$.  The probability is taken over uniform  choice of $k,k'\stackrel{\scriptscriptstyle\$}\leftarrow \{0,1\}^{\scriptscriptstyle\lambda}$, where $k,k'\in\bm{k}$, the randomness of $\mathcal{A}$,  the randomness used in witnesses $g_{\scriptscriptstyle cp},g_{\scriptscriptstyle qp}\in t$,  the randomness used in  $E$.
%
%\end{definition}
% 
% 
% 
%  \begin{definition}[RC-S-P Service Proof Privacy]\label{deff::RC-S-P-ServiceProofPrivacy}  A RC-S-P   preserves the privacy of service proof (status), if for any security parameter $\lambda$, any auxiliary information $\text{aux}$, any $F,Q,M,E,D$ and any probabilistic polynomial time adversary $\mathcal{A}$ that plays the  game in Definition \ref{deff::RC-S-P-Input-Privacy}, there exists a negligible function $\mu(.)$, such that: 
%
%\small{
%$$ Pr\left[\mathcal{A}(\bm{c}^{\scriptscriptstyle *},coin^{\scriptscriptstyle *}_{\scriptscriptstyle \mathcal S},coin^{\scriptscriptstyle *}_{\scriptscriptstyle\mathcal C}, g_{\scriptscriptstyle cp},  g_{\scriptscriptstyle qp},\bm{\pi}^{\scriptscriptstyle *},a)\rightarrow (d_{\scriptscriptstyle j},j)\right]\leq \frac{1}{2}+ \mu(\lambda)$$
%}
%where $\bm{c}^{\scriptscriptstyle *}=[{c}^{\scriptscriptstyle *}_{\scriptscriptstyle 1},...,{c}^{\scriptscriptstyle *}_{\scriptscriptstyle z}]$ and $\bm{\pi}^{\scriptscriptstyle *}=[\pi^{\scriptscriptstyle *}_{\scriptscriptstyle 1},...,\pi^{\scriptscriptstyle *}_{\scriptscriptstyle z}]$.  The probability is taken over uniform  choice of $k,k'\stackrel{\scriptscriptstyle\$}\leftarrow \{0,1\}^{\scriptscriptstyle\lambda}$, where $k,k'\in\bm{k}$, the randomness of $\mathcal{A}$,  the randomness used in witnesses $g_{\scriptscriptstyle cp}, g_{\scriptscriptstyle qp}\in t$,  the randomness used in  $E$.
%\end{definition}
 %-------------------------
 
 
 
 \begin{definition}[RC-S-P Security]\label{def::RC-S-P-Security} A RC-S-P scheme is secure if it satisfies security against malicious server, security against malicious client, and preserves privacy, w.r.t. Definitions \ref{deff::RC-S-P-SecurityAgainstMaliciousServer}-\ref{deff::RC-S-P-Privacy}. 
 \end{definition}
 

 
 












\input{VSID-Protocol}

% !TEX root =R-zkCSP.tex

\section{Recurring Contingent Service Payment (RC-S-P) Protocol}

In this section, we present RC-S-P Protocol. As we explained in Section \ref{sec::Overview-of-Our-Solution}, the protocol relies on the idea that the server and client can efficiently agree on private statements at the begging of the protocol. Therefore, in the following, we first present a protocol (called SAP) that satisfies that requirements and then present RC-S-P Protocol.


% !TEX root =R-zkCSP.tex

\subsection{Statement Agreement Protocol (SAP)}\label{SAP}



In this section, we explain how a client and server, mutually distrusted, can efficiently agree on a private statement, e.g.  a string or tuple, that will be used to reclaim parties' masking coins when the private bubble bursts. Informally, a statement agreement protocol (SAP) is secure if it meets four security properties. First, neither party should be able to persuade  a third-party  verifier that it has agreed with its counter-party on an invalid statement, i.e. the statement that both parties have not agreed on. Second, after they successfully agree on the statement,  an honest party should be able to successfully prove it to the verifier, i.e. an adversary cannot prevent an honest party from successfully proving it. Third, the privacy of the statement should be preserved, (against other parties than the client and server before either of them attempt to prove the agreement on the statement). Forth, after both parties reach an agreement, neither can later deny the agreement. To that end, we use a combination of smart contract and commitment scheme. The idea is as follows. Let $x$ be the statement. The client picks a random value and uses it to commit to $x$. It sends the commitment to the contract and the commitment opening (i.e. statement and the random value) to the server. The server checks if the opening matches the commitment and if so, it commits to the statement using the same random value and sends its commitment to the contract. Later on, for a party to prove to the contract, i.e. the verifier, that it has an agreed on the statement with the other party, it only sends the opening of the commitment.  The contract checks if the opening matches both commitments and accepts if it matches. The SAP protocol is provided below. It assumes that each party $\mathcal{R} \in\{\mathcal{C,S}\}$ already has a blockchain public address $adr_{\scriptscriptstyle\mathcal{R}}$ (via creating an account). 


% either neither  party can successfully prove it has  an agreement with its counter-party, or either party can prove it to the verifier. 

 \begin{enumerate}
 \item\textbf{Setup}.  

 \begin{enumerate}
  \item both parties agree on the SAP smart contract that  explicitly states their addresses, $adr_{\scriptscriptstyle\mathcal{C}}$ and $adr_{\scriptscriptstyle\mathcal{S}}$.
 \item they sign and deploy the contract to the blockchain.
   \end{enumerate}
    \item\textbf{Agreement}.
  \begin{enumerate}
   \item $\mathcal C$ picks a random value $r$, and commits to the statement, $\mathtt{Com}(x,r)=g_{\scriptscriptstyle \mathcal{C}}$
   
   
   \item $\mathcal C$ sends $\ddot{x}=(x,r)$  to  $\mathcal S$. Also, $\mathcal C$ using its account sends $g_{\scriptscriptstyle\mathcal C}$ to the contract. 
   \item $\mathcal S$ checks if has been sent from $adr_{\scriptscriptstyle \mathcal{C}}$, and $\mathtt{Ver}(g_{\scriptscriptstyle\mathcal C}, \ddot{x})\stackrel{?}=1$. If the checks pass, then  it computes $\mathtt{Com}(x,r)=g_{\scriptscriptstyle\mathcal S}$
   
   \item $\mathcal S$ using its account sends   $g_{\scriptscriptstyle\mathcal S}$ to the contract. 
    \end{enumerate}
   \item\textbf{Prove}. For either $\mathcal C$ or $\mathcal S$ to prove, it has agreement on $x$ with its counter-party, it sends $\ddot{x}=(x, r)$  to the contract. 
 \item\textbf{Verify}. Given $\ddot{x}$, the contract does the following. 
   \begin{enumerate}
   
\item\label{SAP::check-adr} ensures $g_{\scriptscriptstyle\mathcal C}$ and $g_{\scriptscriptstyle\mathcal S}$ were sent from   $adr_{\scriptscriptstyle \mathcal{C}}$ and  $adr_{\scriptscriptstyle \mathcal{S}}$  respectively. 
  
   \item\label{SAP::check-commit} ensures $\mathtt{Ver}(g_{\scriptscriptstyle\mathcal C},\ddot{x})=\mathtt{Ver}(g_{\scriptscriptstyle\mathcal S},\ddot{x}) =1$
   
   \item outputs $1$, if the checks in the two previous steps (i.e. steps \ref{SAP::check-adr} and \ref{SAP::check-commit}) were passed. Otherwise, it outputs $0$
    \end{enumerate}
 \end{enumerate}

  
  
  
  
%  \item\textbf{Agreement}.
%  \begin{enumerate}
%   \item $\mathcal S$ picks a random value: $r$, and commits to the statement: $\mathtt{H}(x||r)=g_{\scriptscriptstyle S}$
%   \item $\mathcal S$ sends $r$  to the client and sends $g_{\scriptscriptstyle\mathcal S}$ to the contract. 
%   \item $\mathcal C$ checks: $\mathtt{H}(x||r)\stackrel{?}=g_{\scriptscriptstyle \mathcal S}$. If the equation  holds, it computes $\mathtt{H}(x||r)=g_{\scriptscriptstyle\mathcal C}$
%   \item $\mathcal C$   stores $g_{\scriptscriptstyle\mathcal C}$ in the contract. 
%    \end{enumerate}
%   \item\textbf{Prove}. For either $\mathcal C$ or $\mathcal S$ to prove, it has agreement on $x$ with its counter-party, it sends $\mu=(x, r)$  to the contract. 
% \item\textbf{Verify}. Given $\mu$, the contract does the following. 
%   \begin{enumerate}
%
%   \item checks if $\mathtt{H}(x||r)=g_{\scriptscriptstyle\mathcal C}=g_{\scriptscriptstyle\mathcal S}$
%   \item outputs $1$, if the above equation holds; otherwise, it outputs $0$
%    \end{enumerate}
% \end{enumerate}



%
% \begin{enumerate}
% \item\textbf{Setup}.  Both parties agree on a  smart contract and deploy it, such that the parties public keys, $pk_{\scriptscriptstyle C}$ and $pk_{\scriptscriptstyle S}$, are encoded in the contract.
%
%  
%  \item\textbf{Agreement}.
%  \begin{enumerate}
%   \item The server picks a random value: $r$, and commits to the statement: $H(s||r)=y_{\scriptscriptstyle S}$.
%   \item The server sends $r$  to the client and sends $y_{\scriptscriptstyle S}$ to the contract. 
%   \item The client checks: $H(s||r)\stackrel{?}=y_{\scriptscriptstyle S}$. If the equation  holds, it computes $H(s||r)=y_{\scriptscriptstyle C}$.
%   \item The client   stores $y_{\scriptscriptstyle C}$ in the contract. 
%    \end{enumerate}
%   \item\textbf{Prove}. For either $C$ or $S$ to prove, it has agreement on $s$ with its counter-party, it sends $\mu=(s, r)$, in a signed transaction, to the contract. 
% \item\textbf{Verify}. Given $\mu$, the contract does the following. 
%   \begin{enumerate}
%   \item verifies the public keys related to  signatures of $y_{\scriptscriptstyle C}$ and $y_{\scriptscriptstyle S}$ match $pk_{\scriptscriptstyle C}$ and $pk_{ \scriptscriptstyle S}$ respectively.
%   \item checks if $H(s||r)=y_{\scriptscriptstyle C}=y_{\scriptscriptstyle S}$.
%   \item outputs 1, if the above equation holds; otherwise, it outputs 0.
%    \end{enumerate}
% \end{enumerate}
 
 %Note that the above protocol is one-off, which means after first party
 
 Intuitively, the first property is guaranteed due to binding property of the (hash-based) commitment scheme, while the second property is satisfied due to the security of the blockchain and smart contract, i.e. due to blockchain's liveness property an honestly generated transaction, containing the opening,  eventually gets into  chains of honest miners \cite{GarayKL15}, and due to   security and correctness of smart contracts a valid opening is always accepted by the contract. The third property is met due to the hiding property of the commitment.  The forth property is satisfied due to the security of the signature scheme that is used to sign  transactions originated from the account holders. 
 
 Note, one may simply let each party  sign the statement and send it to the other party, so later on each party can send both signatures to the contract who verifies them. But, this would not work,  as the party who first receives the other party's signature  may refuse  to send its  signature, that prevents the other party to prove that it has an agreed on the statement with its counter-party. Alternatively, one may want to use a protocol for a fair exchange of digital signature (or fair contract signing) such as \cite{BonehN00,DBLP:conf/fc/GarayJ02}. In this case, after both parties have the other party's signature, they can sign the statement themselves and send the two signatures to the contract; who first checks the validity of both  signatures. Although this satisfies the above security requirements, it yields two main efficiency and practical issues: (a) it imposes very high computation costs, as  protocols for fair exchange of signature involve generic zero-knowledge proofs and require a high number of modular exponentiations. And (b) it is impractical, because protocols for fair exchange of signature protocol support only certain signature schemes (e.g. RSA, Rabin, or Schnorr) that are not directly supported by the most predominant  smart contract framework,  Ethereum, as it only supports  Elliptic Curve Digital Signature Algorithm (EDCSA).








% !TEX root =R-zkCSP.tex


\subsection{Recurring Contingent Service Payment (RC-S-P) Protocol}\label{sec::RC-S-P-protocol}


 
 
 In this section, we present  ``recurring contingent service payment'' (RC-S-P) protocol for a generic service. It utilises a novel combination of $\text{VSID}_{\scriptscriptstyle\text{light}}$, SAP, the private time bubble notion, and symmetric key encryption schemes along with the coin masking and padding techniques. At a high level the protocol works as follows.  The  client and server use SAP to provably agree on two private statements; first statement includes  payment details, while another one specifies a secret key, $k$, and a pad's length. They also  agree on  public parameters such as (a) the private time bubble's length, that is the total number of billing cycles: $z$, plus a waiting period, $\texttt{H}$, and (b) a smart contract which  specifies $z$ and  the total amount of masked coins each party should deposit. They deploy the contract.  Each party deposits its masked coins in the contract. If either party does not deposit enough coins on time,   later each party has a chance to withdraw its coins and terminate the contract. To start using/providing the service, they invoke $\text{VSID}_{\scriptscriptstyle\text{light}}$ protocol. In particular, they engage in   $\mathtt{VSID.keyGen}(.)$, $\mathtt{VSID.setup}(.)$, and $\mathtt{VSID.serve}(.)$ algorithms. If the server decides not to serve, e.g. it detects the client's misbehaviour,  it sends $0$  within a fixed time; in this case, the parties can withdraw their deposit and terminate the contract. Otherwise, the server sends $1$ to the contract. 
 
 
 
At the end of each billing cycle, the client generates an encrypted query, by calling $\mathtt{VSID.genQuery}(.)$ and encrypting its output using the key, $k$. It pads the encrypted query and  sends the result  to the contract. The encryption and pads ensure  nothing about the client's input (e.g. outsourced file) is revealed to the public within the private time bubble.  In the same cycle, the server retrieves the query, removes the pads and decrypts the result. Then, it locally checks its validity, by calling $\mathtt{VSID.checkQuery}(.)$. If the query is rejected, the server locally stores the index of the billing cycle and then  generates a dummy proof.  Otherwise, if the server accepts the query, it generates a proof of service by calling $\mathtt{VSID.prove}(.)$. In either case, the server encrypts the proof, pads it and sends the result to the contract. Note that sending (padded encrypted) dummy proof ensures that the public, during the private time bubble, does not learn if the client generates invalid queries. 

 After the server sends the messages to the contract, the client removes the pads, decrypts the proof and locally verifies it, by calling $\mathtt{VSID.verify}(.)$. If the verification is passed, then the client knows the server has delivered the service honestly. However, if the proof is rejected,  it waits until the private time bubble passes and dispute resolution time arrives. During the dispute resolution period, in the case the client or server  rejects any proofs, it sends a ``dispute'' message to the contract. The party also invokes the arbiter,  refers it to the invalid encrypted proofs in the contract, and  sends to it the decryption key and the pads' detail. The arbiter checks the validity of the key and pads, by using SAP. If they are accepted,  then the arbiter locally removes the pads from the encrypted proofs,   decrypts the related proofs, and runs $\mathtt{VSID.identify}(.)$ to check the validity of the party's claim. The arbiter sends to the contract a report of its findings that includes  the total number of times  the server and client provided invalid proofs. In the next phase, to distribute the coins, either client or server  sends: (a) ``pay'' message, (b) the agreed statement that specifies the payment details, and (c) the statement's proof to the contract which verifies the statement and if approved it distributes the coins according to the statement's detail, and the arbiter's report. 

Now we outline why RC-S-P addresses the issues.    In the setup, if the client provides ill-formed inputs (so later it can  accuse the server) then the server can detect and avoid serving it.  After the setup, if the client avoids sending any input,  then the server still gets paid for the service it provided. Also, in the case of a dispute between the parties, their claim is checked, and  the corrupt party is identified. The corrupt party has to pay the arbiter and if that is the client, then it has to pay the server as well.  These  features not only do guarantee the server's resource is not wasted, but also ensures fairness (i.e.  if a potentially malicious server is paid, then it must have provided the service and if a potentially malicious client does not pay, then it will learn nothing). Furthermore, as during the private time bubble  (a) no plaintext proof is given to the contract, and (b) no dispute resolution and coin transfer take place on contract, the public cannot figure out the outcome of each verification. This  preserves the server's privacy. Also, because the  deposited coins are masked and the agreed statement is kept private,   nothing about the detail of the service  is leaked to the public before the bubble bursts. This preserves the client's privacy. Also, as either party can prove to the contract the validity of the agreed statement, and ask the contract to distribute the coins, the coins will be not be locked forever. 

The RC-S-P protocol is presented below. It is assumed that (a) each party $\mathcal{R} \in\{\mathcal{C,S,A}r\}$ already has a blockchain public address, $adr_{\scriptscriptstyle\mathcal{R}}$, which is known to all parties, (b) it uses that (authorised) address to send transactions to the smart contract,  (c) the contract   before recording a transaction, ensures the transaction is  originated from an authorised address, and (d) there is a public price list $pl$ known to eveyone.

\begin{enumerate}
\item \textbf{Key Generation}. $\mathtt{RCSP}.\mathtt{keyGen}(1^{\scriptscriptstyle\lambda}, F)$
\begin{enumerate}


\item  $\mathcal C$ runs  $\mathtt{VSID.keyGen}(1^{\lambda},F)\rightarrow k:=(sk,pk)$. It picks a random secret key  $\bar{k}$ for a symmetric key encryption. Also,  it sets two parameters: $pad_{\scriptscriptstyle \pi}$ and $pad_{\scriptscriptstyle q}$, where  $pad_{\scriptscriptstyle \pi}$ and $pad_{\scriptscriptstyle q}$  refer to the number of  dummy values that will be used to pad encrypted proofs and encrypted queries respectively\footnote{\scriptsize The values of  $pad_{\scriptscriptstyle \pi}$ and $pad_{\scriptscriptstyle q}$ is determined as follows, $pad_{\scriptscriptstyle \pi}=\pi_{\scriptscriptstyle max}-\pi_{\scriptscriptstyle act}$ and $pad_{\scriptscriptstyle q}=q_{\scriptscriptstyle max}-q_{\scriptscriptstyle act}$, where $\pi_{\scriptscriptstyle max}$ and $\pi_{\scriptscriptstyle act}$ refer to the maximum and actual the service's proof size  while $q_{\scriptscriptstyle max}$ and $q_{\scriptscriptstyle act}$ refer to the maximum and actual the service's query size, respectively.},  determined by the security parameter and description of $F$.  Let $qp:=(pad_{\scriptscriptstyle \pi},pad_{\scriptscriptstyle q}, \bar{k})$. The keys' size is part of the security parameter.  Let  $\bm{k}=[k,k']$, where $k':=(sk',pk')$,   $sk'=qp$ and $pk':=(adr_{\scriptscriptstyle\mathcal{C}},adr_{\scriptscriptstyle\mathcal{S}})$. 
\end{enumerate}

\

\item \textbf{Client-side Initiation}. $\mathtt{RCSP}.\mathtt{cInit}(1^\lambda, u,\bm{k}, M,z,pl,enc)$
\begin{enumerate}

\item Calls  $\mathtt{VSID.setup}(1^{\lambda}, u,k,M)\rightarrow (u^{\scriptscriptstyle *}, {pp},e)$, to encode  service input, and generate metadata. {It appends $pp$ to $qp$}.



\item  Calls $\mathtt{SAP.init}(1^{\scriptscriptstyle\lambda}, adr_{\scriptscriptstyle\mathcal{C}}, adr_{\scriptscriptstyle\mathcal{S}},qp )\rightarrow(r_{\scriptscriptstyle qp}, g_{\scriptscriptstyle qp}, adr_{\scriptscriptstyle\text{SAP}_{\scriptscriptstyle 1}})$, to initiate an agreement (with $\mathcal{S}$) on $qp$.  Let $T_{\scriptscriptstyle qp}:=(\ddot{x}_{\scriptscriptstyle qp}, g_{\scriptscriptstyle qp})$ be proof/query encoding token,  where  $\ddot{x}_{\scriptscriptstyle qp}:=(qp,r_{\scriptscriptstyle qp})$ is the opening and $g_{\scriptscriptstyle qp}$ is the commitment stored on the contract as a result of running   SAP. 


\item   Sets coin parameters as   follows,   $o$: the amount of coins for each accepting proof, and $l$: the amount of coins  to cover the cost of each potential dispute resolution, given  price list $pl$.  %It appends $o$ and $l$ to $k'$.




\item  Sets  $cp:=(o,o_{\scriptscriptstyle max},l,l_{\scriptscriptstyle max},z)$, where $o_{\scriptscriptstyle max}$ is the maximum amount of coins  for an accepting service proof,  $l_{\scriptscriptstyle max}$  is the maximum amount of coins to resolve a potential dispute, and $z$ is the number of  service proofs/verifications. Then, $\mathcal C$ calls $\mathtt{SAP.init}(1^{\scriptscriptstyle\lambda}, adr_{\scriptscriptstyle\mathcal{C}}, adr_{\scriptscriptstyle\mathcal{S}},cp )\rightarrow(r_{\scriptscriptstyle cp},g_{\scriptscriptstyle cp},adr_{\scriptscriptstyle\text{SAP}_{\scriptscriptstyle 2}})$, to initiate an agreement (with $\mathcal{S}$) on $cp$. Let $T_{\scriptscriptstyle cp}:=(\ddot{x}_{\scriptscriptstyle cp},g_{\scriptscriptstyle cp})$ be coin encoding token,  where  $\ddot{x}_{\scriptscriptstyle cp}:=(cp,r_{\scriptscriptstyle cp})$ is the opening and $g_{\scriptscriptstyle cp}$ is the commitment stored on the contract as a result of executing   SAP.  





\item Set parameters  $coin^{\scriptscriptstyle *}_{\scriptscriptstyle\mathcal C}=z\cdot (o_{\scriptscriptstyle max}+l_{\scriptscriptstyle max})$ and $p_{\scriptscriptstyle\mathcal{S}}=z\cdot l_{\scriptscriptstyle max}$, where $coin^{\scriptscriptstyle *}_{\scriptscriptstyle\mathcal C}$ and $p_{\scriptscriptstyle\mathcal{S}}$ are   the total number of masked coins $\mathcal C$ and $\mathcal S$ should deposit respectively. It signs and deploys a smart contract, SC, that explicitly specifies  parameters $z$,  $coin^{\scriptscriptstyle *}_{\scriptscriptstyle\mathcal C}$, $p_{\scriptscriptstyle\mathcal{S}}$,  $adr_{\scriptscriptstyle\text{SAP}_{\scriptscriptstyle 1}}$, and $adr_{\scriptscriptstyle\text{SAP}_{\scriptscriptstyle 2}}$. Let $adr_{\scriptscriptstyle \text{SC}}$ be the address of the deployed SC, and $T:=\{T_{\scriptscriptstyle qp},T_{\scriptscriptstyle cp}\}$. It sets a set of time points/windows, $\texttt{Time}:\{ \texttt{T}_{\scriptscriptstyle 0},..., \texttt{T}_{\scriptscriptstyle 3},\texttt{G}_{\scriptscriptstyle 1,1},...,\texttt{G}_{\scriptscriptstyle z,2},\texttt{H}, \texttt{K}_{\scriptscriptstyle 1},...,\texttt{K}_{\scriptscriptstyle 3}, \texttt{L}\}$, that are explicitly specified in the contract who will accept a certain party's message only in a specified  time point/window. The time allocation will become clear in the next phases. 


\item Sets four counters $[y_{\scriptscriptstyle\mathcal C},y'_{\scriptscriptstyle\mathcal C},y_{\scriptscriptstyle\mathcal S}, y'_{\scriptscriptstyle\mathcal S}]$ in the contract, where their initial value is $0$. Let $\bm{y}: [y_{\scriptscriptstyle\mathcal C},y'_{\scriptscriptstyle\mathcal C},y_{\scriptscriptstyle\mathcal S}, y'_{\scriptscriptstyle\mathcal S}, \texttt{Time},adr_{\scriptscriptstyle \text{SC}}]$

\item Deposits $coin^{\scriptscriptstyle *}_{\scriptscriptstyle\mathcal C}$ coins in the contract.

\item Sends $u^{\scriptscriptstyle *}, e, \ddot{x}_{\scriptscriptstyle qp},\ddot{x}_{\scriptscriptstyle cp}$, and $p_{\scriptscriptstyle\mathcal{S}}$ (along with $adr_{\scriptscriptstyle \text{SC}}$) to $\mathcal S$. It sends  $(pk,pk')$ to SC. Let $\texttt{T}_{\scriptscriptstyle 0}$ be the time that the above process finishes. 



\end{enumerate}

\
\item \textbf{Server-side Initiation}\label{RCSP::Server-side-Initiation}. $\mathtt{RCSP}.\mathtt{sInit}(u^{\scriptscriptstyle *}, e, pk, z, T, p_{\scriptscriptstyle\mathcal{S}} ,\bm{y},enc)$

\begin{enumerate}

\item Checks the parameters  in $T$ (e.g. $qp$ and $cp$) and  in SC (e.g. $p_{\scriptscriptstyle\mathcal{S}}, \bm{y}$) and ensures sufficient amount of coins has been deposited by $\mathcal C$.  

\item Calls $\mathtt{SAP.agree}(qp,r_{\scriptscriptstyle qp},g_{\scriptscriptstyle qp},adr_{\scriptscriptstyle\mathcal{C}},adr_{\scriptscriptstyle\text{SAP}_{1}})\rightarrow (g'_{\scriptscriptstyle qp},b_{\scriptscriptstyle 1})$ and $\mathtt{SAP.agree}(cp,r_{\scriptscriptstyle cp},g_{\scriptscriptstyle cp},adr_{\scriptscriptstyle\mathcal{C}},adr_{\scriptscriptstyle\text{SAP}_{2}})\rightarrow (g'_{\scriptscriptstyle cp},b_{\scriptscriptstyle 2})$, to verify the correctness of tokens in $T$ and to agree on the tokens' parameters, where $qp,r_{\scriptscriptstyle qp}\in \ddot{x}_{\scriptscriptstyle qp}, \text{and }  cp,r_{\scriptscriptstyle cp} \in  \ddot{x}_{\scriptscriptstyle cp}$. Recall that if both $\mathcal{C}$ and $\mathcal{S}$ are honest, then $g_{\scriptscriptstyle qp}=g'_{\scriptscriptstyle qp}$ and $g_{\scriptscriptstyle cp}=g'_{\scriptscriptstyle cp}$. 




\item If any above check is rejected, then it sets $a=0$. Otherwise, it  calls $\mathtt{VSID.serve}(u^{\scriptscriptstyle *},e,pk, {pp})\rightarrow a$. 

\item Sends $a$ and $coin^{\scriptscriptstyle *}_{\scriptscriptstyle\mathcal S}=p_{\scriptscriptstyle\mathcal{S}}$ coins to SC at time $\texttt{T}_{\scriptscriptstyle 1}$, where  $coin^{\scriptscriptstyle *}_{\scriptscriptstyle\mathcal S}=\bot$ if $a=0$
\end{enumerate}
Note, $\mathcal S$  and  $\mathcal C$ can withdraw their coins at time $\texttt{T}_{\scriptscriptstyle 2}$, if  $\mathcal S$ sends $a=0$, fewer coins than $p_{\scriptscriptstyle\mathcal{S}}$, or nothing to the SC. To withdraw, $\mathcal S$  or  $\mathcal C$  simply sends a ``pay'' message to $\mathtt{RCSP}.\mathtt{pay}(.)$ algorithm (only) at time $\texttt{T}_{\scriptscriptstyle 2}$.

\

\textbf{\textit{Billing-cycles Onset}}. $\mathcal{C}$ and $\mathcal{S}$ engage in the following three phases, i.e. phase \ref{Billing-cycles-genQuery}-\ref{Client-sideProofVerification},  at the end of every $j$-th billing cycle, where $1\leq j\leq z$. Each $j$-th cycle includes two  time points, $\texttt{G}_{\scriptscriptstyle j,1}$ and $\texttt{G}_{\scriptscriptstyle j,2}$, where $\texttt{G}_{\scriptscriptstyle j,2}>\texttt{G}_{\scriptscriptstyle j,1}$, and $\texttt{G}_{\scriptscriptstyle 1,1}>\texttt{T}_{\scriptscriptstyle 2}$ 



\

\item\textbf{Client-side Query Generation}.\label{Billing-cycles-genQuery} $\mathtt{RCSP}.\mathtt{genQuery}(1^\lambda, \text{aux},k,Q, T_{\scriptscriptstyle qp}, enc)$

\begin{enumerate}
\item Calls $\mathtt{VSID.genQuery}(1^{\lambda},  \text{aux},k,Q, {pp})\rightarrow c_{\scriptscriptstyle j}:=(\bm{q}_{\scriptscriptstyle j},\bm{w}_{\scriptscriptstyle q_{_{\scaleto{j}{\size}}}})$, to generate a query-proof pair. 
\item Encrypts the pair, $\mathtt{Enc}(\bar{k},c_{\scriptscriptstyle j})=c'_{\scriptscriptstyle j}$, where $\bar{k}\in T_{\scriptscriptstyle qp}$. Then, it pads (each element of) the result with ${pad}_{\scriptscriptstyle q}\in T_{\scriptscriptstyle qp}$ random values that are picked uniformly at random from the encryption's output range, $U$. Let $c^{\scriptscriptstyle *}_{\scriptscriptstyle j}$ be the result. 
\item Sends the padded encrypted query-proof pair, $c^{\scriptscriptstyle *}_{\scriptscriptstyle j}$, to SC at time $\texttt{G}_{\scriptscriptstyle j,1}$
\end{enumerate}

\

\item\textbf{Server-side Proof Generation}. $\mathtt{RCSP}.\mathtt{prove}(u^{\scriptscriptstyle *},  \sigma,  c^{\scriptscriptstyle *}_{\scriptscriptstyle j}, pk, T_{\scriptscriptstyle qp}, enc)$


\begin{enumerate}

\item Constructs an empty  vector, $\bm{m}_{\scriptscriptstyle\mathcal S}=\bot$, if $j=1$.  
\item Removes the pads from $c^{\scriptscriptstyle *}_{\scriptscriptstyle j}$, using  parameters of $T_{\scriptscriptstyle qp}$. Let  $c'_{\scriptscriptstyle j}$ be the result. Next, it  decrypts  the result, $\mathtt{Dec(}\bar{k},c'_{\scriptscriptstyle j})=c_{\scriptscriptstyle j}$. Then,  it   runs $\mathtt{VSID.checkQuery}(c_{\scriptscriptstyle j}, pk, {pp})\rightarrow b_{\scriptscriptstyle j}$, to check the correctness of the queries.



\begin{itemize}
\item[$\bullet$] If $\mathcal S$ accepts the query, i.e. $b_{\scriptscriptstyle j}=1$, then  calls $\mathtt{VSID.prove}(u^{\scriptscriptstyle *},\sigma,c_{\scriptscriptstyle j},pk,{pp})\rightarrow \pi_{\scriptscriptstyle j}$, to generate the service proof. In this case, $\mathcal S$ encrypts it, $\mathtt{Enc}(\bar{k},\pi_{\scriptscriptstyle j})=\pi'_{\scriptscriptstyle j}$. Next, it   pads (every element of) the encrypted proof with ${pad}_{\scriptscriptstyle \pi}\in T_{\scriptscriptstyle qp}$ random values  picked uniformly at random from $U$. Let $\pi^{\scriptscriptstyle *}_{\scriptscriptstyle j}$ be the result. It sends the padded encrypted proof  to SC at time $\texttt{G}_{\scriptscriptstyle j,2}$ 

\item[$\bullet$] Otherwise (if $\mathcal S$ rejects the query), it appends  $j$  to $\bm{m}_{\scriptscriptstyle\mathcal{S}}$,  constructs a dummy proof $\pi'_{\scriptscriptstyle j}$, picked uniformly at random from $U$,  pads the result as above, and sends the padded dummy proof, $\pi^{\scriptscriptstyle *}_{\scriptscriptstyle j}$, to SC at time $\texttt{G}_{\scriptscriptstyle j,2}$ %, if the above proofs are rejected, i.e. $b_{\scriptscriptstyle j}=0$.

\end{itemize}
When $j=z$  and $\bm{m}_{\scriptscriptstyle\mathcal{S}}\neq\bot$, $\mathcal S$ sets    $m_{\scriptscriptstyle\mathcal S}:=(\bm{m}_{\scriptscriptstyle\mathcal{S}},adr_{\scriptscriptstyle \text{SC}})$.

\end{enumerate}

\

\item \textbf{Client-side Proof Verification}.\label{Client-sideProofVerification} $\mathtt{RCSP}.\mathtt{verify}(\pi^{\scriptscriptstyle *}_{\scriptscriptstyle j}, \bm{q}_{\scriptscriptstyle j},k,T_{\scriptscriptstyle qp},enc)$
\begin{enumerate}


\item Constructs an empty vector, $\bm{m}_{\scriptscriptstyle\mathcal C}=\bot$, if $j=1$. 

\item Removes the pads from $\pi^{\scriptscriptstyle *}_{\scriptscriptstyle j}$, utilising  parameters of $T_{\scriptscriptstyle qp}$. Let $\pi'_{\scriptscriptstyle j}$ be the result. It decrypts the service proof: $\mathtt{Dec}(\bar{k},\pi'_{\scriptscriptstyle j})=\pi''_{\scriptscriptstyle j}$ and then calls $\mathtt{VSID.verify}(\pi''_{\scriptscriptstyle j}, \bm{q}_{\scriptscriptstyle j},k, {pp})\rightarrow d_{\scriptscriptstyle j}$, to verify the proof. Note that if $\pi'_{\scriptscriptstyle j}=\mathtt{Enc}(\bar{k},\pi_{\scriptscriptstyle j})$, then $\pi''_{\scriptscriptstyle j}=\pi_{\scriptscriptstyle j}$. 

\begin{itemize}
\item[$\bullet$] If $\pi''_{\scriptscriptstyle j}$ passes the verification (i.e. $d_{\scriptscriptstyle j}=1$), then $\mathcal C$ concludes that the service  for this verification has been delivered successfully. 

\item[$\bullet$]  Otherwise (when $\pi''_{\scriptscriptstyle j}$ is rejected), $\mathcal C$ appends $j$  to $\bm{m}_{\scriptscriptstyle\mathcal{C}}$. 


\end{itemize}
\end{enumerate}
When $j=z$  and $\bm{m}_{\scriptscriptstyle\mathcal{C}}\neq\bot$, $\mathcal C$ sets    ${m}_{\scriptscriptstyle\mathcal C}:=(\bm{m}_{\scriptscriptstyle\mathcal{C}},adr_{\scriptscriptstyle \text{SC}},e')$, where  $e'$  contains the opening of $\mathtt{Com}_{\scriptscriptstyle sk}$ or  $\bot$, as stated in Remark \ref{remark::light-VSID}.




\

\item \textbf{Dispute Resolution}.\label{RCSP::Dispute-Resolution} $\mathtt{RCSP}.\mathtt{resolve}(m_{\scriptscriptstyle \mathcal{C}},m_{\scriptscriptstyle \mathcal{S}}, z, {\bm{\pi}}^{\scriptscriptstyle *}, {\bm{c}}^{\scriptscriptstyle *}, pk, T_{\scriptscriptstyle qp},enc)$



 The phase takes place only in  case of dispute, e.g. when $\mathcal C$ and/or $\mathcal S$ reject any proofs in the previous phases. %or  either party does not receive its counterparty's message on time. %A party who accepts all proofs does not need to get involved in this phase. 

\begin{enumerate}
%\item $\mathcal C$ and/or $\mathcal S$ at time $K_{\scriptscriptstyle 1}>G_{\scriptscriptstyle z,2}+H$ sends ``dispute'' message to the contract. 


\item The arbiter sets  counters: $y_{\scriptscriptstyle\mathcal C}, y'_{\scriptscriptstyle\mathcal C},y_{\scriptscriptstyle\mathcal S}$ and  $y'_{\scriptscriptstyle\mathcal S}$, that are initially set to $0$,  before time $\texttt{K}_{\scriptscriptstyle 1}$,  where $\texttt{K}_{\scriptscriptstyle 1}>\texttt{G}_{\scriptscriptstyle z,2}+\texttt{H}$. 
\item $\mathcal C$  sends $m_{\scriptscriptstyle\mathcal C}$ and $\ddot{x}_{\scriptscriptstyle qp}$ to the arbiter at time $\texttt{K}_{\scriptscriptstyle 1}$. Or,  $\mathcal S$ sends $m_{\scriptscriptstyle\mathcal S}$ and $\ddot{x}_{\scriptscriptstyle qp}$ to the arbiter at time $\texttt{K}_{\scriptscriptstyle 1}$ %,   where $|\vv{\bm{v}}_{\scriptscriptstyle\mathcal{C}}|\neq 0$ and $|\vv{\bm{v}}_{\scriptscriptstyle\mathcal{S}}|\neq 0$. 

 

\item\label{arbiter-checks-statement} At time $\texttt{K}_{\scriptscriptstyle 2}$, the arbiter checks the validity of  statement $\ddot{x}_{\scriptscriptstyle qp}$ sent by each party  $\forall \mathcal R\in \{\mathcal {C,S}\}$. To  do so, it sends  each $\ddot{x}_{\scriptscriptstyle qp}$ to SAP contract which  returns  either $1$ or $0$. The arbiter  constructs an empty vector: $\bm{v}$.  If  party $\mathcal R$'s statement is accepted, then 
 it appends every  element of $\bm{m}_{\scriptscriptstyle\mathcal{R}}$ to $\bm{v}$. It ensures $\bm{v}$ contains only distinct elements which are in the range $[1,z]$. Otherwise (if the party's statement is rejected) it discards the party's request, ${m}_{\scriptscriptstyle\mathcal{R}}$. It proceeds to the next step if $\bm{v}$ is not empty, otherwise it halts. 






\item\label{the-arbiter-loop} The arbiter  for every element $i\in\bm{v}$:

\begin{enumerate}

 
\item removes the pads from the related encrypted    query-proof pair  and from encrypted service proof. Let $c'_{\scriptscriptstyle i}$ and $\pi'_{\scriptscriptstyle i}$ be the result. %unpadded encrypted  query-proof pair  and encrypted service proof  respectively. 
\item decrypts the encrypted query-proof pair and encrypted service proof as follows, $ \mathtt{Dec}(\bar{k},c'_{\scriptscriptstyle i})=c_{\scriptscriptstyle i}$ and 
$ \mathtt{Dec}(\bar{k},\pi'_{\scriptscriptstyle i})=\pi''_{\scriptscriptstyle i}$ 
\item\label{arbiter-identify} calls   $\mathtt{VSID.identify}(\pi''_{\scriptscriptstyle i},c_{\scriptscriptstyle i}, k, e', {pp})\rightarrow I_{\scriptscriptstyle i}$
\begin{enumerate}
\item[$\bullet$] if $I_{\scriptscriptstyle i}=\mathcal C$, then it increments $y_{\scriptscriptstyle\mathcal C}$ by $1$ 
\item[$\bullet$]  if $I_{\scriptscriptstyle i}=\mathcal S$, then it increments $y_{\scriptscriptstyle\mathcal S}$ by $1$ 
\item[$\bullet$]  if $I_{\scriptscriptstyle i}=\bot$, then it increments  $y'_{\scriptscriptstyle\mathcal C}$ or $y'_{\scriptscriptstyle\mathcal S}$ by $1$, if $i$ is in  the complaint of $\mathcal {C}$ or $\mathcal {S}$ respectively. 
\end{enumerate}
\end{enumerate}

 
Let $\texttt{K}_{\scriptscriptstyle 3}$ be the time that the arbiter finishes the above checks.

\item The arbiter at time $\texttt{K}_{\scriptscriptstyle 3}$ sends $[y_{\scriptscriptstyle\mathcal C},y_{\scriptscriptstyle\mathcal S},  y'_{\scriptscriptstyle\mathcal C},y'_{\scriptscriptstyle\mathcal S}]$  to SC who accordingly overwrites the  elements it holds (i.e. elements of $\bm{y}$) by  the related vectors elements the arbiter sent.  
\end{enumerate}

\

\item \textbf{Coin Transfer}\label{RCSP::CoinTransfer}. $\mathtt{RCSP}.\mathtt{pay}(\bm{y},T_{\scriptscriptstyle cp},a,p_{\scriptscriptstyle\mathcal S}, coin^{\scriptscriptstyle *}_{\scriptscriptstyle\mathcal C},coin^{\scriptscriptstyle *}_{\scriptscriptstyle\mathcal S})$
\begin{enumerate}
 
 \item If SC  receives  ``pay'' message  at time $\texttt{T}_{2}$, where $a=0$ or $coins^{*}_{\scriptscriptstyle\mathcal{S}}<p_{\scriptscriptstyle\mathcal{S}}$, then it sends $coin^{\scriptscriptstyle *}_{\scriptscriptstyle\mathcal C}$ coins to $\mathcal C$ and $coin^{\scriptscriptstyle *}_{\scriptscriptstyle\mathcal S}$ coins to $\mathcal S$. In other words, the parties can withdraw their coins if they do not reach an agreement at the end of   phase \ref{RCSP::Server-side-Initiation}, i.e. server-side initiation.  Otherwise (i.e. they reach an agreement), they take the following step. 

\item Either $\mathcal C$ or $\mathcal S$ send ``pay'' message and the statement, $\ddot{x}_{\scriptscriptstyle cp}\in T_{\scriptscriptstyle cp}$,  to SC at time $\texttt{L}>\texttt{K}_{\scriptscriptstyle 3}$
\item SC checks the validity of the statement by sending $\ddot{x}_{\scriptscriptstyle cp}$ to SAP contract which  returns  either $1$ or $0$. SC only proceeds to the next step if the output is $1$. 
\item SC distributes the coins to the parties as follows:

\begin{itemize}
\item[$\bullet$]   $coin^{\scriptscriptstyle *}_{\scriptscriptstyle\mathcal C}-o(z-y_{\scriptscriptstyle\mathcal S})-l(y_{\scriptscriptstyle\mathcal C}+y'_{\scriptscriptstyle\mathcal C})$ coins  to $\mathcal C$
\item[$\bullet$] $coin^{\scriptscriptstyle *}_{\scriptscriptstyle\mathcal S}+o(z-y_{\scriptscriptstyle\mathcal S})-l(y_{\scriptscriptstyle\mathcal S}+y'_{\scriptscriptstyle\mathcal S})$ coins to $\mathcal S$
\item[$\bullet$] $l(y_{\scriptscriptstyle\mathcal S}+y_{\scriptscriptstyle\mathcal C}+y'_{\scriptscriptstyle\mathcal S}+y'_{\scriptscriptstyle\mathcal C})$ coins to the arbiter.  
\end{itemize}

\end{enumerate}
\end{enumerate}



\begin{remark} If all parties behave honestly,   then the server receives all its deposit back plus the amount of coins they initially agreed to pay the sever if it  delivers accepting proofs for all $z$ cycles, i.e. in total it receives $coin^{\scriptscriptstyle *}_{\scriptscriptstyle\mathcal S}+o\cdot z$ coins. Also, in this case an honest client receives all coins minus the amount of coins paid to the server for delivering accepting proofs for $z$ cycles, i.e. in total it receives $coin^{\scriptscriptstyle *}_{\scriptscriptstyle\mathcal C}-o\cdot z$ coins. However, the arbiter receives no coins, as it is never invoked. 
\end{remark}

\begin{remark} As stated in Section \ref{sec::RC-S-P-Definition},   algorithms $\mathtt{RCSP}.\mathtt{genQuery}(.),\mathtt{RCSP}.\mathtt{prove}(.),\mathtt{RCSP}.\mathtt{verify}(.)$ and $\mathtt{RCSP}.\mathtt{resolve}(.)$ implicitly  take $a, coin^{\scriptscriptstyle*}_{\scriptscriptstyle\mathcal{S}}, p_{\scriptscriptstyle\mathcal{S}}$ as another inputs and  execute only if $a=1$ and $coin^{\scriptscriptstyle*}_{\scriptscriptstyle\mathcal{S}}=p_{\scriptscriptstyle\mathcal{S}}$. For the sake of simplicity we  avoided explicitly stating it in the protocol.   
\end{remark}

\begin{remark}
Keeping track of  $(y'_{\scriptscriptstyle\mathcal C},y'_{\scriptscriptstyle\mathcal S})$ enables the arbiter to make malicious parties, who \emph{unnecessarily} invoke  it for  accepting proofs in step \ref{arbiter-identify}, pay  for the verifications it performs. 
\end{remark}


%\begin{remark}
%Keeping track of  $(y'_{\scriptscriptstyle\mathcal C},y'_{\scriptscriptstyle\mathcal S})$ enables the arbiter to make malicious parties, who \emph{unnecessarily} invoke  it for invalid statement in step \ref{arbiter-checks-statement} or  accepting proofs in step \ref{arbiter-identify}, pay  for the verifications it performs. 
%\end{remark}

%\begin{remark}
%If the client or server  receives no input from its counter-party, it still needs to invoke the arbiter, so that party can be paid (back) fairly. 
%\end{remark}





\begin{remark} The VSID scheme  does not (need to) preserve the privacy of the proofs. However, in RC-S-P  protocol  each proof's privacy must be preserved, for a certain time; otherwise, the proof itself can leak its status, e.g. when it can be publicly verified. This is the reason in RC-S-P protocol,  \emph{encrypted} proofs are sent to the contract.   
\end{remark}


\begin{remark}
For the sake of simplicity, in the above protocol, we assumed that each arbiter's invocation has a fixed  cost regardless of the number of steps it takes. To define a fine grained costing, one can simply allocate to each step the arbiter takes a certain rate and also separate counter  for the client and server.  
\end{remark}

\begin{remark}
In the case where $\mathtt{VSID.verify}(.)$ is privately verifiable and    the server invokes the arbiter, the client needs to provide inputs to the arbiter too. Otherwise (when it is publicly verifiable and  the server invokes the arbiter), the client's involvement is not required in the dispute resolution phase.  In contrast, if the client invokes the arbiter, the server's involvement is not required in that phase, regardless of the type of verifiability $\mathtt{VSID.verify}(.)$ supports. 
\end{remark}




 
 \input{RCSP-Proof}
 
 






% !TEX root =R-zkCSP.tex

\section{Recurring Contingent PoR Payment (RC-PoR-P) Protocol}


In this section, we present recurring contingent PoR payment (RC-PoR-P) that is a concrete instantiation of the generic recurring contingent service payment (RC-S-P), when the verifiable service is PoR. Nevertheless, RC-PoR-P offers  two primary added features. Specifically, unlike RC-S-P, it (a) does not use any zero-knowledge proofs (even though either client or server can still be  malicious) which significantly improves  costs, and (b)  has a much lower  arbiter-side computation cost; as we will show later, this also allows a smart contract efficiently plays the arbiter's role. In the following, first we  explain how the features are satisfied.

\

\noindent\textit{\textbf{Avoiding the Use of Zero-knowledge Proofs}}. In general, the majority of  PoR's are in the security model where a client is honest while the server is potentially malicious. They rely on metadata that is either a set of tags (e.g. MAC's or signatures) or a root of a Merkle tree,  constructed on  file blocks to ensure the file's availability. In the case where a client can also  be malicious,  if tags are used then using zero-knowledge proofs seem  an obvious choice, as it allows the client to guarantee to the server that the tags have been constructed correctly without leaking verification keys. However, this imposes significant computation and communication costs. We observed that using a Merkle tree would benefit our protocol from a couple of perspectives; in short, it removes the need for zero-knowledge proofs and it supports proof of misbehaviour. Our first observation is that if a Merkle tree is used to generate a metadata, then   there would be no need for the client to use zero-knowledge proofs to prove the correctness of the metadata to the server. Instead, the server can efficiently check the metadata's correctness, by reconstructing the Merkle tree on top of the file blocks. 

\


\noindent\textit{\textbf{Reducing Arbiter-side Cost}}.  As stated above, RC-PoR-P uses a Merkle tree-based PoR. In this case, each $j$-th proof contains a set of Merkle tree paths that are encoded and stored on a smart contract. For each verification, the client decoded and then verifies all paths. In a naive approach, when the client rejects  proofs of $j$-th verification,  it raises a dispute and let the arbiter verify the proofs, i.e. \emph{all paths} in the proofs \footnote{In a Merkle tree-based PoR, the number of proofs that are sent to a verifier for each verification is linear with the number of challenges, e.g. 460 challenges  to ensure 99\% of file blocks is retrievable. In contrast, in a tag-based PoR, in each verification, the verifier receives only a few proofs.}. Instead, we use the idea of proof of misbehaviour, put forth in \cite{}. In particular, in $j$-th verification, if the client detects invalid proofs, then  it  sends  details of only one invalid proof/path to the arbiter  who  decodes that single proof and checks its validity (as apposed to decoding and checking all proofs). This significantly reduces the arbiter computation cost. 


To present RC-PoR-P protocol, we will use the same approach  used to present RC-S-P. In particular, first we present the verifiable service, that is a (modified) Merkle tree-based PoR. Then, we upgrade it to the one that supports  identifiable abort, denoted by PoRID. Next,   we use PoRID to build RC-PoR-P. 


% !TEX root =R-zkCSP.tex



\subsection{Modified Merkle tree-based PoR}\label{protocol::PoR}

In this section, we present a modified version of the standard Merkle tree-based PoR, detonated by PoR. At a high level, the protocol works as follows. The client encodes its file using an error-correcting code, splits the encoded file into blocks, and constructs a Merkle tree on top of the blocks. It locally keeps the tree's root and sends the blocks to the server who rebuilds the tree on the blocks. At a verification time, the client sends a pseudorandom function's key to the server who derives a predetermined number of pseudorandom  indices of the blocks, that indicates which blocks have been challenged. The server for each challenged block generates a Merkle tree proof and sends all proofs to the client. The client, given the root and   key, verifies all  proofs. If all proofs are accepted, then the client outputs $1$ and concludes that its file is retrievable (with a high probability). However, if  it rejects a set of proofs, it outputs $0$ along with an index of the challenged block whose proof was rejected. In the following, we first present the protocol and then elaborate on the  modifications we have applied. 

%, i.e. a verifier outputs (the index of) a rejected proof.  Recall, that in the standard PoR protocol the client is honest and the server is potentially malicious. 

\begin{enumerate}

\item\textbf{Client-side Setup}. $\mathtt{PoR.setup}(1^{\lambda},u)$  
\begin{enumerate}
\item Uses an error correcting code, e.g. Reed-Solomon codes, to encode  the file: $u$. Let $u'$ be the encoded file. It splits $u'$  into  blocks as follows, $u^{\scriptscriptstyle *}=u^{\scriptscriptstyle '}_{\scriptscriptstyle 0}||0,...,u^{\scriptscriptstyle '}_{\scriptscriptstyle m}||m$
\item Generates  metadata: $\sigma$, by constructing Merkle tree on blocks of $u^{\scriptscriptstyle *}$, i.e. $\mathtt{MT.genTree}(u^{\scriptscriptstyle *})$. Let $\sigma$ be the root of the resulting tree, and $\beta$ be a security parameter. It sets public  parameters as $pp:= (\sigma,\beta,m,\zeta)$, where $\zeta:=(\psi,\eta, \iota)$ is a $\mathtt{PRF}$'s description, as it was defined in Section \ref{preliminaries}
\item Sends $pp$ and $u^{\scriptscriptstyle *}$ to $\mathcal{S}$
\end{enumerate}

\

\item\textbf{Client-side Query Generation}. $\mathtt{PoR.genQuery}(1^{\scriptscriptstyle\lambda}, pp)$ 
\begin{enumerate}
\item\label{key-chalenge} Picks a random key $\hat{k}$ of a pseudorandom function $\mathtt{PRF}$, i.e. $\hat{k}\stackrel{\scriptscriptstyle\$}\leftarrow\{0,1\}^{\scriptscriptstyle\psi}$. It ensures the function outputs distinct values, i.e. $\forall i,j\in [0,m]: (\mathtt{PRF}(\hat{k},i)\bmod m+1)\neq(\mathtt{PRF}(\hat{k},j)\bmod m+1)$, where $i\neq j$

\item It sends $\hat{k}$ to $\mathcal{S}$
 
 \end{enumerate}
 
 \
 
\item\label{PoR-server-prove}\textbf{Server-side Proof Generation}. $\mathtt{PoR.prove}(u^{\scriptscriptstyle *},\hat{k},pp)$ 
 \begin{enumerate}
\item Derives $\beta$ pseudorandom indices from $\hat{k}$ as follows. $\forall i,1\leq i\leq \beta: q_{\scriptscriptstyle i}=\mathtt{PRF}(\hat{k},i)\bmod m+1$. Let ${\bm{q}}=[q_{\scriptscriptstyle 1},..., q_{\scriptscriptstyle \beta}]$

\item For each random index $q_{\scriptscriptstyle i}$, generates a Merkle tree proof: $\pi_{\scriptscriptstyle{q_{_{\scriptscriptstyle i}}}}$, by  running Merkle tree proof generator function on $u^{\scriptscriptstyle *}$, i.e. $\mathtt{MT.prove}(u^{*},q_{\scriptscriptstyle i})$. The final result  is $ {\bm{\pi}}=[(u^{\scriptscriptstyle *}_{\scriptscriptstyle q_{_{\scriptscriptstyle i}}},\pi_{\scriptscriptstyle q_{_{\scriptscriptstyle i}}})]_{\scriptscriptstyle q_{\scriptscriptstyle i}\in {\bm{q}}}$, where $i$-th element in $ {\bm{\pi}}$ corresponds to $i$-th pseudorandom value: $q_{\scriptscriptstyle i}$ and   each $\pi_{\scriptscriptstyle{q_{_{\scriptscriptstyle i}}}}$ is path in the tree that proves its corresponding block: $u^{\scriptscriptstyle *}_{\scriptscriptstyle q_{_{\scriptscriptstyle i}}}$ is a leaf node of the  tree. 

\item Sends $ {\bm{\pi}}$ to $\mathcal{C}$
 \end{enumerate}
 
 \
 
\item\label{PoR-verify}\textbf{Client-side Proof Verification}. $\mathtt{PoR.verify}( {\bm{\pi}},\bm{q},pp)$ 
\begin{enumerate}

\item If $|\bm{\pi}|=|\bm{q}|=1$, then set $\beta=1$. This step is taken only in the case where single proof and query is provided to a third-party verifier (e.g. in the case of proof of misbehaviour). 
%\item Regenerates $\beta$ pseudorandom indices that are derived from $\hat{k}$ as follows. $\forall i,1\leq i\leq \beta: q_{\scriptscriptstyle i}=\mathtt{PRF}(\hat{k},i)\bmod m+1$. Let $ {\bm{q}}=[q_{\scriptscriptstyle 1},..., q_{\scriptscriptstyle \beta}]$
\item Checks if the server has sent proofs related to all challenged file blocks. To do that, for all $i$ (where $1\leq i\leq \beta$),  it first parses every  element of $\pi$ as follows, $\mathtt{parse}(u^{\scriptscriptstyle *}_{\scriptscriptstyle q_{_{\scriptscriptstyle i}}})=u^{\scriptscriptstyle '}_{\scriptscriptstyle q_{_{\scriptscriptstyle i}}}||q_{_{\scriptscriptstyle i}}$, and then checks if its index: $q_{_{\scriptscriptstyle i}}$ equals $i$-th element of $ {\bm{q}}$. If all checks pass, then it proceeds to the next step. Otherwise,  it outputs $ {\bm{d}}:[0,i]$, where $i$ refers to the index of the  element in $ {\bm{\pi}}$ that does not pass the check. 


\item Checks if every path in  $ {\bm{\pi}}$ is valid and corresponds to the root, by calling $\mathtt{MT.verify}(u^{\scriptscriptstyle *}_{\scriptscriptstyle q_{_{\scriptscriptstyle i}}},\pi_{\scriptscriptstyle q_{_{\scriptscriptstyle i}}},\sigma)$. If all checks pass, it outputs $ {\bm{d}}=[1,\bot]$ (where $\bot$ denotes empty); otherwise, it outputs $ {\bm{d}}:[0,i]$, where $i$ refers to the index of the  element in $ {\bm{\pi}}$ that does not pass the check. 
\end{enumerate}
\end{enumerate}



\begin{theorem}
The PoR scheme, presented in Section \ref{protocol::PoR}, is $\epsilon$-sound, w.r.t. Definitions \ref{extractable}, if Merkle tree and pseudorandom function $\mathtt{PRF}$, are secure. 

\end{theorem}


 The above protocol differs from the standard Merkle tree-based PoR from two perspectives; First and far most,  in step \ref{PoR-verify} in addition to outputting a binary value, the client outputs only one index of a rejected proof. This will enable any third-party who is given that index (and vectors of proofs and challenges) to verify the client's claim by checking only that proof, i.e. proof of misbehaviour. Second,   in step \ref{key-chalenge} instead of sending $\beta$ challenges, we allow the client to send only a key of a pseudorandom function to the server who can derive a set of challenges form it. This will ultimately  lead to a decrease in costs too, i.e. the client's communication and a smart contract's storage costs.   
 
 \begin{proof}[sketch]
 As stated above, the proposed PoR differs from the standard Merkle tree-based PoR from a couple of perspectives. However, the changes do not affect the security and soundness of the proposed PoR and its security proof is similar to the existing Merkle tree-based PoR schemes, e.g. \cite{DBLP:conf/ccs/HaleviHPS11,MillerPermacoin,DBLP:journals/iacr/JuelsK07}. Alternatively, our protocol can be proven based on the security analysis of the PoR schemes that use  MACs or BLS signatures, e.g. \cite{DBLP:conf/asiacrypt/ShachamW08}. In this case, the extractor design (in the Merkle tree-based PoR)  would be simpler as it does not need to extract blocks from a linear combination of MAC's or signatures, as the  blocks are included in PoR proofs, i.e. they are part of the Merkle tree proofs. 
 
 Intuitively, in either case, the extractor interacts with any adversarial prover that passes non-negligible $\epsilon$ fraction of audits. It initialises an empty array. Then it  challenges a subset of file blocks and asks the prover to generate a proof.  If the received proof passes the verification, then it adds the related block (in the proof) to the array. It then rewinds the prover and challenges a fresh set of blocks, and repeats the process  many times. Since, the prover has a good chance of passing the audit, it is easy to show that the extractor can eventually extract a large fraction of the entire file. Due to the security, i.e. authenticity, of the Merkle tree, the retrieved values are the valid and correct file blocks and due to security of the pseudorandom function, the challenges (or the function's outputs) are not predictable. After collecting sufficient number of blocks, the extractor can use the error correcting code to decode and recover the entire file blocks, given the retrieved ones. 
 \hfill\(\Box\)\end{proof}
   
   \begin{remark}
   Recall,  the generic definition of a verifiable service scheme (i.e. Definition \ref{service-def}) involves three algorithms: $F$, $M$, and $Q$. However, the three algorithms are implicit in the original definition of  PoR and accordingly in  PoR protocols. In the following, we explain how each algorithm is defined in PoR context.  $M$ is an algorithm that processes a file and generates metadata. For instance, when PoR uses a Merkle tree (to ensure the file's integrity and availability), then $M$ refers to the Merkle tree's algorithm that constructs a tree on top of the file blocks. Also,  $F$ is an algorithm that, during generating a PoR proof, processes a subset of the outsourced file, given the client's query (or challenged file blocks). For instance, if a PoR  utilises a Merkle tree, then $F$ refers to the algorithm that generates Merkle tree's proofs, i.e.  membership of the challenged file blocks. Furthermore, $Q$ can be  a pseudorandom function that generates a set of pseudorandom strings in a certain range, e.g. file block's indices.   
   \end{remark}



% !TEX root =R-zkCSP.tex


\subsection{PoRID Protocol}

In this section, we propose ``PoR with identifiable abort'' (PoRID) that is a concrete instantiation of $\text{VSID}_{\scriptscriptstyle\text{light}}$. It is built upon the PoR protocol, presented in the previous section and is in the same security model as $\text{VSID}_{\scriptscriptstyle\text{light}}$ is, i.e. either $\mathcal C$ or $\mathcal S$ can be malicious. In PoRID similar to $\text{VSID}_{\scriptscriptstyle\text{light}}$, $\mathcal C$ and $\mathcal S$ use a bulletin board to exchange signed messages.  In the protocol, at setup $\mathcal C$ encodes its file and generates public parameters and  metadata.  It posts the public parameters and metadata to the bulletin board and   sends the encoded file to $\mathcal S$ who runs a few lightweight checks to ensure the correctness of the public parameters and metadata. It agrees to serve, if it is convinced of their correctness. Later, when $\mathcal C$ wants to ensure the availability of its outsourced file, it  generates and posts a query  to the board.   $\mathcal S$  checks the correctness of the query, by performing a couple of highly efficient verifications. 

The server-side prove and client-side verify algorithms are similar to those in PoR with a difference that $\mathcal S$ posts the PoR proofs (i.e. output of prove algorithm) to the board. In  case of any dispute, $\mathcal C$ or $\mathcal S$ invokes the arbiter who, given the signed posted messages, checks the proofs  to identify a corrupt party. In particular,  it first checks the validity of the query (regardless of the party who invokes it). However, if  is invoked by  $\mathcal C$, it also checks  only one of the PoR proofs that the client claims it is invalid. Thus, it is much more efficient than $\text{VSID}_{\scriptscriptstyle\text{light}}$ as it does not need any zero-knowledge proofs (mainly due to the use of Merkle tree) and requires the  arbiter to check only one of the proofs (due  to the idea of proof of misbehaviour). PoRID protocol is presented below.




%In this section, we propose PoR with identifiable abort: PoRID. Here we consider the light version of  PoRID due to its efficiency. 

%However, if $M$ takes as inputs private parameters, then the client commits to the parameters and  


\begin{enumerate}
%\item\textbf{Key Generation}. $\mathtt{VSID.KeyGen}(1^{\lambda},F)$ 
%\begin{enumerate}
%\item Calls  $\mathtt{VS.KeyGen}(1^{\lambda},F)$ to generate a pair of secret and public keys, $k:(sk,pk)$.
%%\item Commits to the secret keys: $sk\in k$. This yields: $\mathtt{Com}(sk)$.
%\item Posts $pk$ to a bulletin board. 
%\end{enumerate}
%\item \textbf{keyGen}. $\mathtt{PoR.keyGen}(1^{\lambda})$. $\mathcal{C}$ sets $pk$ as public key, initially empty.    
\item\textbf{Client-side Setup}. $\mathtt{PoRID.setup}(1^{\lambda}, u)$
\begin{enumerate}
\item Calls  $\mathtt{PoR.setup}(1^{\lambda}, u)\rightarrow (u^{\scriptscriptstyle *},pk)$, that results in public parameters $pk:=(\sigma,\beta,m,\zeta)$ and encoded file: $u^{\scriptscriptstyle *}=u^{\scriptscriptstyle '}_{\scriptscriptstyle 0}||0,...,u^{\scriptscriptstyle '}_{\scriptscriptstyle m}||m$. Recall,  $\zeta:=(\psi,\eta, \iota)$ is the $\mathtt{PRF}$'s description.

\item Posts $pk$ to the bulletin board and sends $u^{\scriptscriptstyle *}$ to $\mathcal{S}$ 
\end{enumerate}

\item\textbf{Server-side Setup}. $\mathtt{PoRID.serve}(u^{\scriptscriptstyle *},pk)$
%\begin{enumerate}

Verifies the correctness of public parameters:
\begin{enumerate}
\item rebuilds  the Merkle tree on $u^{\scriptscriptstyle *}$ and checks the resulting root equals $\sigma$
\item checks $|u^{*}|=m$ and  $\beta\leq m$
\end{enumerate}
 If the proofs are accepted, then it proceeds to the next step; otherwise, it halts.
 
%\item Posts $1$ to the board,  if the above checks pass; otherwise, it posts $0$ to the board.  

%\end{enumerate}

\item\textbf{Client-side Query Generation}. $\mathtt{PoRID.genQuery}(1^{\lambda},  pk)$ 
\begin{enumerate}
\item Calls $\mathtt{PoR.genQuery}(1^{\lambda}, pk)\rightarrow \hat{k}$, to generate a key, $\hat{k}$ %whose element have been sorted in ascending order. 
%\item Sorts the queries in ascending order. Let vector $ {\bm{q}}'$ contain the ordered random indices (i.e.  $ {\bm{q}}'[i]< {\bm{q}}'[i+1]$, where $i\in [0,\beta-1]$).
\item Posts $\hat{k}$ to the board. 
\end{enumerate}




\item\textbf{Server-side Query Verification}. $\mathtt{PoRID.checkQuery}(\hat{k}, pk)$
\begin{enumerate}
\item  Checks if  $\hat{k}$ is not empty, i.e. $\hat{k}\neq \bot$, and is in the key's universe, i.e. $\hat{k}\in \{0,1\}^{\scriptscriptstyle\psi}$%If it is empty, then it outputs $I=\mathcal C$
\item  If the checks pass, then it outputs $b=1$; otherwise, it outputs $b=0$
% If the check passes, then it proceeds to the next step; otherwise, it aborts.

%\item Verifies if  the query: $ {\bm{q}}$ has been constructed correctly as follows. 
\end{enumerate}
%
%\begin{enumerate}
%\item  checks the query vector size: $| {\bm{q}}|\stackrel{\scriptscriptstyle ?}=\beta$. If the equation does not hold, then it posts  $b=0$ to the board and aborts. Otherwise, it proceeds to the next step. 
%\item verifies that the elements of $ {\bm{q}}$ are in ascending order, by checking if there are two consecutive elements of the vector: $(q_{\scriptscriptstyle i},q_{\scriptscriptstyle i+1})\in  {\bm{q}}$, such that: $q_{\scriptscriptstyle i}\not<q_{\scriptscriptstyle i+1}$.  If the pair is found, then it posts  $b:(0,i)$ to the board and aborts. Otherwise, it proceeds to the next step. 
%\end{enumerate}
%\item Posts $b=1$ to the board, if the above two checks pass.  
%\end{enumerate}

\item\textbf{Server-side Service Proof Generation}. $\mathtt{PoRID.prove}(u^{\scriptscriptstyle *},\hat{k},pk)$ 
\begin{enumerate}
\item Calls $\mathtt{PoR.prove}(u^{\scriptscriptstyle *},\hat{k},pk)\rightarrow  {\bm{\pi}}$, to generate proof vector: $ {\bm{\pi}}$ 
\item Posts $ {\bm{\pi}}$ to the board. 
\end{enumerate}

\item\textbf{Client-side Proof Verification}. $\mathtt{PoRID.verify}( {\bm{\pi}},\hat{k} ,pk)$
%\begin{enumerate}

 Calls $\mathtt{PoR.verify}( {\bm{\pi}},\hat{k} ,pk)\rightarrow  {\bm{d}}$, to verify the proof. If $ {\bm{d}}[0] = 1$,  it accepts the proof; otherwise, it rejects it.
%\item Posts $  {\bm{d}}$ to the board. 
%\end{enumerate}

\item\textbf{Arbiter-side Identification}. $\mathtt{PoRID.identify}( {\bm{\pi}}, g,\hat{k},pk)$ 

This algorithm can be invoked by $\mathcal{C}$ or $\mathcal{S}$, in the case of dispute. If it is invoked by $\mathcal{C}$, then $g$ refers to a rejected proof's  index; however, if it is invoked by $\mathcal{S}$, then $g$ is null, i.e.  $g=\bot$. The arbiter performs as follows. 


\begin{enumerate}

%\item Calls $\mathtt{VSID.checkQuery}(c, pk)\rightarrow b$. If $b=1$, then it proceeds to the next step. Otherwise, it outputs $I=\mathcal C$ and aborts. 
%\item   Ensures $ {\bm{d}}[0]=0$; otherwise, it aborts.

\item Ensures  query $\hat{k}$ is well-structured by calling $\mathtt{PoRID.checkQuery}(\hat{k}, pk)$. If it returns $b=0$, then it outputs $I=\mathcal C$ and halts; otherwise, it proceeds to the next step. 


%not empty (i.e. $\hat{k}\neq \bot$). If it is empty, then it outputs $I=\mathcal C$

\item Derives the related challenged block's index from $\hat{k}$, by computing $q_{\scriptscriptstyle g}=\mathtt{PRF}(\hat{k},g)\bmod m+1$ 



\item If $g\neq \bot$, then verifies only $g$-th proof, by calling $\mathtt{PoR.verify}( {\bm{\pi}}[g],q_{\scriptscriptstyle g},pk)\rightarrow  {\bm{d}}'$. If $ {\bm{d}}'[0]=0$, then it outputs $I=\mathcal S$. Otherwise, it outputs $I=\bot$
\end{enumerate}
\end{enumerate}

\begin{theorem}
The PoRID protocol satisfies the $\epsilon$-soundness, inputs well-formedness, and detectable abort properties, w.r.t. Definitions \ref{extractable}, \ref{deff::VSID-Inputs-Well-formedess}, and \ref{def::VSID-Identifiable-Abort}, if PoR is $\epsilon$-sound and the Merkle tree and signature schemes are secure.
\end{theorem}


\begin{proof}[sketch]
The $\epsilon$-soundness of PoRID directly stems from the security of PoR scheme, i.e. $\epsilon$-soundness. Specifically,
in PoRID the (honest) client makes block-box calls to the algorithms of PoR, to ensure the soundness. The latter scheme's soundness ensures that an extractor can recover the entire file interacting with a corrupt server who passes  $\epsilon$ fraction of challenges.




 \hfill\(\Box\)\end{proof}


\input{Recurring-Contingent-PoR-Payment-protocol.tex}
%In the case where the party's outgoing message input is of large size, the parties can post a hash of the data to the board and send the data directly to its counter party. This requires, the other party to simply check if the data matches the representation and then sends to the board a short string declaring it has received it. 








%In the above definition,  $\mathtt{Service.verify}()$ algorithm  allows a verifier to detect only a misbehaving server; nevertheless, it is not suitable to  detect a misbehaving client who may try to falsely accuse the server.  Thus, $\mathtt{Service.resolve()}$ algorithm has also been incorporated in the definition, to allow an arbiter to detect either party's misbehaviour and resolve any dispute between them.  Informally, a service scheme has two main properties: correctness and soundness. The correctness requires that for any key, the verification algorithm accepts a proof generated by an honest prover. The soundness requires that if a prover convinces the verifier (i.e. client or arbiter), with a high probability, then the service has been provided by the prover. {\color {blue} it seems undeniability and accountability/liability, i.e. a malicious party can be identified and held accountable, are needed too. Thin if it's needed here on in the main payment protocol.}
%The exact formalisation of soundness however totally depends on the kind of service provided. 



%Thus, since the two party are mutually distrusted, the above definition has also incorporated Service.resolve() algorithm that  allows an arbiter to resolve any dispute between client and server. 




% !TEX root =R-zkCSP.tex

\subsection{Recurring Contingent PoR Payment (RC-PoR-P) Protocol}\label{sec::RC-PoR-P-protocol}



In this section, we present RC-PoR-P,  a concrete instantiation of the generic RC-S-P.  Although RC-PoR-P and RC-S-P have some overlaps, they have many differences too. Therefore, we provide the protocol’s overview and its detailed description below.
At a high level the protocol works as follows. The client and server utilise SAP to provably agree on two private statements, one statement includes payment details, and another one specifies a secret key, $\bar{k}$, and a pad's details, that will be used to encode sensitive messages they send to the contract. Moreover, they  agree on public parameters such as the private time bubble's length (that is the total number of billing cycles: $z$, plus a waiting period, $\texttt{H}$) and a smart contract that specifies $z$ and the total amount of masked coins each party should deposit. They deploy the contract. Each party deposits its masked coins in the contract within a fixed time. If a party does not deposit enough coins on time, then the parties have a chance to withdraw their coins and terminate the contract after a certain time. 

To start using/providing the service, the client invokes $\mathtt{PoRID.setup}(.)$ to encode the file and generate  metadata and public parameters. It sends  encryption of the metadata and public parameters to the smart contract. Also, it  sends the encoded file to the server who decrypts them and using the encoded file checks their correctness by calling $\mathtt{PoRID.server}(.)$. If the server decides not to serve, it sends to the contract $0$ within a fixed time; in this case, the parties can withdraw their deposit and terminate the contract.  At the end of each billing cycle, the client generates a query, by calling $\mathtt{PoRID.genQuery(.)}$ and sends the query's encryption to the contract. In the same cycle, the server retrieves the query from the contract, decrypts and locally checks its correctness, by calling $\mathtt{PoRID.checkQuery(.)}$. If the query is rejected, the server locally stores the index of that billing cycle and generates  dummy PoR proofs. However, if the server accepts the query, it generates PoR proofs by calling $\mathtt{PoRID.prove(.)}$. Then, in either case, the server encrypts the proofs, pads them and sends the result to the contract. Next, the client removes the pads, decrypts the proofs and locally verifies them, by calling $\mathtt{PoRID.verify(.)}$. If the verification is passed, then the client knows the file is retrievable with a high probability. But, if the proof is rejected, then it locally stores the index of that billing cycle and details of one of the invalid proofs (in the same cycle). It  waits until the private time bubble passes and dispute resolution time arrives. 

During the dispute resolution period, in case the server rejects the query or the client rejects the PoR proofs, that party sends to the arbiter (a) details of invalid proofs and (b)  the statement that contains the decryption key and padding detail. The arbiter checks the validity of the statement first. If it accepts the statement, then it removes the pads and decrypts the values whose indices were provided by the parties. Then, the arbiter checks the party’s claim by calling $\mathtt{PoRID.checkQuery (.)}$ and $\mathtt{PoRID.identify (.)}$ if the server or client calls the contract respectively. The arbiter also keeps track of the number of times each party misbehaved.  After the arbiter processes the parties' claim, it tells to the contract how many times each party misbehaved.  In the next phase, to distribute the coins, either client or server sends to the contract: (a) “pay” message, (b) the agreed statement that specifies the payment details, and (c) the statement’s proof. The contract verifies the statement and if it is approved, then it distributes the coins according to the statement’s detail, and the number of times each party misbehaved. 



\begin{enumerate}
\item\label{RC-PoR-P::KeyGen} \textbf{Key Generation}. $\mathtt{RCPoRP}.\mathtt{keyGen}(1^{\scriptscriptstyle\lambda})$
\begin{enumerate}

\item  $\mathcal C$  picks a random secret key  $\bar{k}$ for a symmetric key encryption. It also sets  parameter $pad_{\scriptscriptstyle \pi}$ which is the number of  dummy values that will be used to pad encrypted proofs,  let $qp:=(pad_{\scriptscriptstyle \pi}, \bar{k})$. The key's size is part of the security parameter. Let  ${k}:=(sk',pk')$, where  $sk'=qp$ and $pk':=(adr_{\scriptscriptstyle\mathcal{C}},adr_{\scriptscriptstyle\mathcal{S}})$. 





\end{enumerate}

\

\item \textbf{Client-side Initiation}. $\mathtt{RCPoRP}.\mathtt{cInit}(1^\lambda, u,{k}, z,pl)$
\begin{enumerate}

\item Calls $\mathtt{PoRID.setup}(1^{\lambda}, u)\rightarrow (u^{\scriptscriptstyle *},pp)$ to encode service input. It appends $pp$ to $qp$.


\item  Calls $\mathtt{SAP.init}(1^{\scriptscriptstyle\lambda}, adr_{\scriptscriptstyle\mathcal{C}}, adr_{\scriptscriptstyle\mathcal{S}},qp )\rightarrow(r_{\scriptscriptstyle qp}, g_{\scriptscriptstyle qp}, adr_{\scriptscriptstyle\text{SAP}_{\scriptscriptstyle 1}})$, to initiate an agreement (with $\mathcal{S}$) on $qp$.  Let $T_{\scriptscriptstyle qp}:=(\ddot{x}_{\scriptscriptstyle qp}, g_{\scriptscriptstyle qp})$ be proof/query encoding token,  where  $\ddot{x}_{\scriptscriptstyle qp}:=(qp,r_{\scriptscriptstyle qp})$ is the opening and $g_{\scriptscriptstyle qp}$ is the commitment stored on the contract as a result of running   SAP. 


\item   Sets coin parameters (given  price list $pl$) as  follows,  $o$: the amount of coins for each accepting proof,  and $l$ the amount of coin $\mathcal C$  or $\mathcal S$ needs to send to a smart contract to resolve a potential dispute. 






\item\label{RC-PoR-P::set-cp}  Sets  $cp:=(o,o_{\scriptscriptstyle max}, l,  l_{\scriptscriptstyle max},z)$, where $o_{\scriptscriptstyle max}$ is the maximum amount of coins  for an accepting service proof,  $l_{\scriptscriptstyle max}$  is the maximum amount of coins to resolve a potential dispute, and $z$ is the number of  service proofs/verifications. Then, $\mathcal C$ calls $\mathtt{SAP.init}(1^{\scriptscriptstyle\lambda}, adr_{\scriptscriptstyle\mathcal{C}}, adr_{\scriptscriptstyle\mathcal{S}},cp )\rightarrow(r_{\scriptscriptstyle cp},g_{\scriptscriptstyle cp},adr_{\scriptscriptstyle\text{SAP}_{\scriptscriptstyle 2}})$, to initiate an agreement (with $\mathcal{S}$) on $cp$. Let $T_{\scriptscriptstyle cp}:=(\ddot{x}_{\scriptscriptstyle cp},g_{\scriptscriptstyle cp})$ be coin encoding token,  where  $\ddot{x}_{\scriptscriptstyle cp}:=(cp,r_{\scriptscriptstyle cp})$ is the opening and $g_{\scriptscriptstyle cp}$ is the commitment stored on the contract as a result of executing   SAP.  





\item Set parameters  $coin^{\scriptscriptstyle *}_{\scriptscriptstyle\mathcal C}=z\cdot (o_{\scriptscriptstyle max}+l_{\scriptscriptstyle max})$ and $p_{\scriptscriptstyle\mathcal{S}}=z\cdot l_{\scriptscriptstyle max}$, where $coin^{\scriptscriptstyle *}_{\scriptscriptstyle\mathcal C}$ and $p_{\scriptscriptstyle\mathcal{S}}$ are   the total number of masked coins $\mathcal C$ and $\mathcal S$ should deposit respectively. It signs and deploys a smart contract, SC, that explicitly specifies  parameters $z$,  $coin^{\scriptscriptstyle *}_{\scriptscriptstyle\mathcal C}$, $p_{\scriptscriptstyle\mathcal{S}}$,  $adr_{\scriptscriptstyle\text{SAP}_{\scriptscriptstyle 1}}$, and $adr_{\scriptscriptstyle\text{SAP}_{\scriptscriptstyle 2}}$. Let $adr_{\scriptscriptstyle \text{SC}}$ be the address of the deployed SC, and $T:=\{T_{\scriptscriptstyle qp},T_{\scriptscriptstyle cp}\}$. It sets a set of time points/windows, $\texttt{Time}:\{ \texttt{T}_{\scriptscriptstyle 0},..., \texttt{T}_{\scriptscriptstyle 3},\texttt{G}_{\scriptscriptstyle 1,1},...,\texttt{G}_{\scriptscriptstyle z,2},\texttt{H}, \texttt{K}_{\scriptscriptstyle 1},...,\texttt{K}_{\scriptscriptstyle 6}, \texttt{L}\}$, that are explicitly specified in SC who will accept a certain party's message only in a specified  time point/window. Time allocations will become clear in the next phases. 


\item\label{setcounters} Sets four counters $[y_{\scriptscriptstyle\mathcal C},y'_{\scriptscriptstyle\mathcal C},y_{\scriptscriptstyle\mathcal S},  y'_{\scriptscriptstyle\mathcal S}]$ in the contract, where their initial value is $0$. Let $\bm{y}: [y_{\scriptscriptstyle\mathcal C},y'_{\scriptscriptstyle\mathcal C},y_{\scriptscriptstyle\mathcal S},  y'_{\scriptscriptstyle\mathcal S},\texttt{Time},adr_{\scriptscriptstyle \text{SC}}]$

\item Deposits $coin^{\scriptscriptstyle *}_{\scriptscriptstyle\mathcal C}$ coins in the contract. It  sends $u^{\scriptscriptstyle *}, e, \ddot{x}_{\scriptscriptstyle qp},\ddot{x}_{\scriptscriptstyle cp}$, and $p_{\scriptscriptstyle\mathcal{S}}$ (along with $adr_{\scriptscriptstyle \text{SC}}$) to $\mathcal S$. It also sends  $pk'$ to SC. Let $\texttt{T}_{\scriptscriptstyle 0}$ be the time that the above process finishes. 




\end{enumerate}

\
\item \textbf{Server-side Initiation}\label{RCPoRP::Server-side-Initiation}. $\mathtt{RCPoRP}.\mathtt{sInit}(u^{\scriptscriptstyle *},  z, T, p_{\scriptscriptstyle\mathcal{S}} ,\bm{y})$

\begin{enumerate}

\item Checks the parameters  in $T$ (e.g. $qp$ and $cp$) and  in SC (e.g. $p_{\scriptscriptstyle\mathcal{S}}, \bm{y}$) and ensures sufficient amount of coins has been deposited by $\mathcal C$.  

\item Calls $\mathtt{SAP.agree}(qp,r_{\scriptscriptstyle qp},g_{\scriptscriptstyle qp},adr_{\scriptscriptstyle\mathcal{C}},adr_{\scriptscriptstyle\text{SAP}_{1}})\rightarrow (g'_{\scriptscriptstyle qp},b_{\scriptscriptstyle 1})$ and $\mathtt{SAP.agree}(cp,r_{\scriptscriptstyle cp},g_{\scriptscriptstyle cp},adr_{\scriptscriptstyle\mathcal{C}},adr_{\scriptscriptstyle\text{SAP}_{2}})\rightarrow (g'_{\scriptscriptstyle cp},b_{\scriptscriptstyle 2})$, to verify the correctness of tokens in $T$ and to agree on the tokens' parameters, where $qp,r_{\scriptscriptstyle qp}\in \ddot{x}_{\scriptscriptstyle qp}, \text{and }  cp,r_{\scriptscriptstyle cp} \in  \ddot{x}_{\scriptscriptstyle cp}$. Recall, if both $\mathcal{C}$ and $\mathcal{S}$ are honest, then $g_{\scriptscriptstyle qp}=g'_{\scriptscriptstyle qp}$ and $g_{\scriptscriptstyle cp}=g'_{\scriptscriptstyle cp}$. 




\item If any above check is rejected, then it sets $a=0$. Otherwise, it  calls $\mathtt{PoRID.serve}(u^{\scriptscriptstyle *}, {pp})\rightarrow a$. 

\item Sends $a$ and $coin^{\scriptscriptstyle *}_{\scriptscriptstyle\mathcal S}=p_{\scriptscriptstyle\mathcal{S}}$ coins to SC at time $\texttt{T}_{\scriptscriptstyle 1}$, where  $coin^{\scriptscriptstyle *}_{\scriptscriptstyle\mathcal S}=\bot$ if $a=0$
\end{enumerate}
Note, $\mathcal S$  and  $\mathcal C$ can withdraw their coins at time $\texttt{T}_{\scriptscriptstyle 2}$, if  $\mathcal S$ sends $a=0$, fewer coins than $p_{\scriptscriptstyle\mathcal{S}}$, or nothing to the SC. To withdraw, $\mathcal S$  or  $\mathcal C$  simply sends a ``pay'' message to $\mathtt{RCPoRP}.\mathtt{pay}(.)$ algorithm only at time $\texttt{T}_{\scriptscriptstyle 2}$.

\




\textbf{\textit{Billing-cycles Onset}}. $\mathcal{C}$ and $\mathcal{S}$ engage in the following three phases, i.e. phase \ref{Billing-cycles-genQuery}-\ref{Client-sideProofVerification},  at the end of every $j$-th billing cycle, where $1\leq j\leq z$. Each $j$-th cycle includes two  time points, $\texttt{G}_{\scriptscriptstyle j,1}$ and $\texttt{G}_{\scriptscriptstyle j,2}$, where $\texttt{G}_{\scriptscriptstyle j,2}>\texttt{G}_{\scriptscriptstyle j,1}$, and $\texttt{G}_{\scriptscriptstyle 1,1}>\texttt{T}_{\scriptscriptstyle 2}$ 


\

\item\textbf{Client-side Query Generation}.\label{Billing-cycles-genQuery} $\mathtt{RCPoRP}.\mathtt{genQuery}(1^\lambda,  T_{\scriptscriptstyle qp})$

\begin{enumerate}

\item Calls $\mathtt{PoRID.genQuery}(1^{\lambda},  pp)\rightarrow \hat{k}_{\scriptscriptstyle j}$ to generate a query, where  $ pp\in T_{\scriptscriptstyle qp}$


\item Sends encrypted query  $\hat{k}^{\scriptscriptstyle *}_{\scriptscriptstyle j}=\mathtt{Enc}(\bar{k},\hat{k}_{\scriptscriptstyle j})$ to SC at time $\texttt{G}_{\scriptscriptstyle j,1}$


\end{enumerate}

\

\item\textbf{Server-side Proof Generation}. $\mathtt{RCPoRP}.\mathtt{prove}(u^{\scriptscriptstyle *},  \hat{k}^{\scriptscriptstyle *}_{\scriptscriptstyle j}, T_{\scriptscriptstyle qp})$


\begin{enumerate}

\item Constructs an empty  vector, $\bm{m}_{\scriptscriptstyle\mathcal S}=\bot$, if $j=1$.  

\item Decrypts the query, $\hat{k}_{\scriptscriptstyle j}=\mathtt{Dec}(\bar{k},\hat{k}^{\scriptscriptstyle *}_{\scriptscriptstyle j})$

\item Calls $\mathtt{PoRID.checkQuery}(\hat{k}_{\scriptscriptstyle j}, pp)\rightarrow b_{\scriptscriptstyle j}$ to check the query's correctness.

\begin{itemize}
\item[$\bullet$] If it accepts the query, then it calls $\mathtt{PoRID.prove}(u^{\scriptscriptstyle *},\hat{k}_{\scriptscriptstyle j},pp)\rightarrow  {\bm{\pi}}_{\scriptscriptstyle j}$, to generate a PoR  proof. In this case,  $\mathcal S$ encrypts every  proof in the proof vector, i.e. $\forall g, 1\leq g\leq| {\bm{\pi}}_{\scriptscriptstyle j}|: \mathtt{Enc}(\bar{k}, {\bm{\pi}}_{\scriptscriptstyle j}[g])= {\bm{\pi}}'_{\scriptscriptstyle j}[g]$, where $\bar{k}\in T_{\scriptscriptstyle qp}$. Let vector $ {\bm{\pi}}'_{\scriptscriptstyle j}$ contain the encryption of all proofs. It pads every encrypted proof in $ {\bm{\pi}}'_{\scriptscriptstyle j}$ with ${pad}_{\scriptscriptstyle \pi}\in T_{\scriptscriptstyle qp}$ random values that are picked from the encryption's output range $U$, (by appending the random values to the encrypted proofs vector). Let $\bm{\pi}^{\scriptscriptstyle *}_{\scriptscriptstyle j}$ be the result. It sends the padded encrypted proofs  to SC at time $\texttt{G}_{\scriptscriptstyle j,2}$ 
\item[$\bullet$] Otherwise (if $\mathcal S$ rejects the query), it appends $j$  to $ {\bm{m}}_{\scriptscriptstyle\mathcal{S}}$,  constructs a dummy proof $ {\bm{\pi}}'_{\scriptscriptstyle j}$ whose elements are randomly picked from $U$,  pads the result as above, and sends the result, $\bm{\pi}^{\scriptscriptstyle *}_{\scriptscriptstyle j}$, to SC at time $\texttt{G}_{\scriptscriptstyle j,2}$

\end{itemize}

When $j=z$  and $\bm{m}_{\scriptscriptstyle\mathcal{S}}\neq\bot$, it sets    $m_{\scriptscriptstyle\mathcal S}:=\bm{m}_{\scriptscriptstyle\mathcal{S}}$.

\end{enumerate}




\


\item \textbf{Client-side Proof Verification}.\label{Client-sideProofVerification} $\mathtt{RCPoRP}.\mathtt{verify}(\bm{\pi}^{\scriptscriptstyle *}_{\scriptscriptstyle j}, \hat{k}_{\scriptscriptstyle j},T_{\scriptscriptstyle qp})$
\begin{enumerate}


\item Constructs an empty vector, $\bm{m}_{\scriptscriptstyle\mathcal C}=\bot$, if $j=1$. 

\item Removes the pads from $\bm{\pi}^{\scriptscriptstyle *}_{\scriptscriptstyle j}$, utilising  parameters of $T_{\scriptscriptstyle qp}$. Let $\bm{\pi}'_{\scriptscriptstyle j}$ be the result. It decrypts the service proofs $\mathtt{Dec}(\bar{k},\bm{\pi}'_{\scriptscriptstyle j})=\bm{\pi}_{\scriptscriptstyle j}$ and then calls $\mathtt{PoRID.verify}(\bm{\pi}_{\scriptscriptstyle j}, \hat{k}_{\scriptscriptstyle j}, {pp})\rightarrow \bm{d}_{\scriptscriptstyle j}$, to verify the proof.  

\begin{itemize}
\item[$\bullet$] If $ {\bm{\pi}}_{\scriptscriptstyle j}$ passes the verification, i.e. $ {\bm{d}}_{\scriptscriptstyle j}[0]=1$, then $\mathcal C$  concludes that the service for this verification has been delivered successfully.  


\item[$\bullet$] Otherwise (if proof $ {\bm{\pi}}_{\scriptscriptstyle j}$ is rejected, i.e. $ {\bm{d}}_{\scriptscriptstyle j}[0]=0$),  it sets $g= {\bm{d}}_{\scriptscriptstyle j}[1]$ and appends vector $[j,g]$  to $ {\bm{m}}_{\scriptscriptstyle\mathcal{C}}$. Recall, $ {\bm{d}}_{\scriptscriptstyle j}[1]$ refers to a rejected proof's index in proof vector $ {\bm{\pi}}_{\scriptscriptstyle j}$. 

\end{itemize}
\end{enumerate}
When $j=z$  and $\bm{m}_{\scriptscriptstyle\mathcal{C}}\neq\bot$, $\mathcal C$ sets    ${m}_{\scriptscriptstyle\mathcal C}:=\bm{m}_{\scriptscriptstyle\mathcal{C}}$.




\




\item \textbf{Dispute Resolution}. \label{RCPoRP::Dispute-Resolution} $\mathtt{RCPoRP}. \mathtt{resolve}(m_{\scriptscriptstyle \mathcal{C}},m_{\scriptscriptstyle \mathcal{S}}, z, {\bm{\pi}}^{\scriptscriptstyle *}, {\bm{q}^{\scriptscriptstyle *}},  T_{\scriptscriptstyle qp})$



The phase takes place only in  case of dispute, i.e. when $\mathcal C$ rejects service proofs or $\mathcal S$ rejects the queries. %or either party does not receive its counterparty's message on time. %A party who accepts all proofs does not need to get involved in this phase. 

\begin{enumerate}



\item The arbiter sets  counters: $y_{\scriptscriptstyle\mathcal C}, y'_{\scriptscriptstyle\mathcal C},y_{\scriptscriptstyle\mathcal S}$ and  $y'_{\scriptscriptstyle\mathcal S}$, that are initially set to $0$,  before time $\texttt{K}_{\scriptscriptstyle 1}$,  where $\texttt{K}_{\scriptscriptstyle 1}>\texttt{G}_{\scriptscriptstyle z,2}+\texttt{H}$. 

\item $\mathcal S$  sends ${m}_{\scriptscriptstyle\mathcal S}$ and $\ddot{x}_{\scriptscriptstyle qp}$ to the arbiter, at time $\texttt{K}_{\scriptscriptstyle 1}$.

\item The arbiter after receiving ${m}_{\scriptscriptstyle\mathcal S}$, does the following at time $\texttt{K}_{\scriptscriptstyle 2}$. 
\begin{enumerate}




\item Checks the validity of  statement $\ddot{x}_{\scriptscriptstyle qp}$, by  sending it to SAP contract which  returns   $1$ or $0$. If the output is $0$, then it discards the server's complaint, $m_{\scriptscriptstyle\mathcal S}$, and does not take steps \ref{RCPoRP::check-v} and \ref{sc-check-server-claim}. Otherwise, it proceeds to the next step.



\item\label{RCPoRP::check-v} Removes from $ {\bm{v}}_{\scriptscriptstyle\mathcal{S}}$ any element that is duplicated or is not in the range $[1,z]$. It also constructs an empty vector $ {\bm{v}}$.

\item\label{sc-check-server-claim}  For any element $i\in {\bm{v}}_{\scriptscriptstyle\mathcal{S}}$:
\begin{itemize}

\item[$\bullet$] Fetches the related encrypted query $\hat{k}^{\scriptscriptstyle *}_{\scriptscriptstyle i}\in \bm{q}^{\scriptscriptstyle *}$   from SC, and   decrypts it, $\hat{k}_{\scriptscriptstyle i}=\mathtt{Dec}(\bar{k},\hat{k}^{\scriptscriptstyle *}_{\scriptscriptstyle i})$

%\item[$\bullet$] Fetches the related query, $\hat{k}_{\scriptscriptstyle i}\in \bm{q}$,   from SC 
\item[$\bullet$]Checks if  the query is well-formed, by calling $\mathtt{PoRID.checkQuery}(\hat{k}_{\scriptscriptstyle i}, pp)\rightarrow b_{\scriptscriptstyle i}$. If the query is rejected, i.e.  $b_{\scriptscriptstyle i}=0$, then it increments $y_{\scriptscriptstyle\mathcal C}$ by $1$ and appends $i$ to $ {\bm{v}}$. Otherwise (if the query is accepted) it increments $y'_{\scriptscriptstyle\mathcal S}$  by $1$. 
\end{itemize}
Let $\texttt{K}_{\scriptscriptstyle 3}$ be the time the arbiter finishes the above checks.
\end{enumerate}





\item $\mathcal C$  sends $m_{\scriptscriptstyle\mathcal C}$ and $\ddot{x}_{\scriptscriptstyle qp}$ to the arbiter, at time $\texttt{K}_{\scriptscriptstyle 4}$




\item The arbiter after receiving $m_{\scriptscriptstyle\mathcal C}$, does the following, at time $\texttt{K}_{\scriptscriptstyle 5}$. 
\begin{enumerate}
\item Checks the validity of  statement $\ddot{x}_{\scriptscriptstyle qp}$, by   sending $\ddot{x}_{\scriptscriptstyle qp}$ to SAP contract which  returns  either $1$ or $0$. If the output is $0$, then it discards the client's complaint, ${m}_{\scriptscriptstyle\mathcal C}$, and does not take steps \ref{sc-checks-client-claim}-\ref{the-contract-loop}. Otherwise, it proceeds to the next step. 

 \item\label{sc-checks-client-claim} Ensures each vector   $ {\bm{m}}\in  {\bm{m}}_{\scriptscriptstyle\mathcal{C}}$ is well-formed. In particular, it ensures there exist no two vectors: $ {\bm{m}}, {\bm{m}}'\in  {\bm{m}}_{\scriptscriptstyle\mathcal{C}}$ such that $ {\bm{m}}[0]= {\bm{m}}'[0]$.  If such vectors  exist, it deletes the redundant ones from $ {\bm{m}}_{\scriptscriptstyle\mathcal{C}}$. This ensures no two claims refer to the same verification.  It removes any vector $ {\bm{m}}$ from $ {\bm{m}}_{\scriptscriptstyle\mathcal{C}}$ if $ {\bm{m}}[0]$ is not in the range $[1,z]$ or if $ {\bm{m}}[0]\in  {\bm{v}}$. Note the latter check  (i.e. $ {\bm{m}}[0]\in  {\bm{v}}$) ensures $\mathcal C$ cannot hold $\mathcal S$ accountable if $\mathcal C$  generated an ill-formed query for the same verification.   



\item\label{the-contract-loop} For every  vector $ {\bm{m}}\in {\bm{m}}_{\scriptscriptstyle \mathcal{C}}$:
\begin{itemize}
\item[$\bullet$]  Retrieves details of a proof that was rejected in each $i$-th verification.  In particular, it sets   $i= {\bm{m}}[0]$ and $g= {\bm{m}}[1]$. Recall that $g$ refers to the index of a rejected proof in the proof vector which was generated for $i$-th verification, i.e. $ {\bm{\pi}}_{\scriptscriptstyle i}$ 


\item[$\bullet$] Fetches the related encrypted query $\hat{k}^{\scriptscriptstyle *}_{\scriptscriptstyle i}\in \bm{q}^{\scriptscriptstyle *}$   from SC, and   decrypts it, $\hat{k}_{\scriptscriptstyle i}=\mathtt{Dec}(\bar{k},\hat{k}^{\scriptscriptstyle *}_{\scriptscriptstyle i})$



\item[$\bullet$]  Removes the pads only from $g$-th padded encrypted proof. Let $ {\bm{\pi}}'_{\scriptscriptstyle i}[g]$ be the result. Next, it decrypts the  encrypted proof, $\mathtt{Dec}(\bar{k}, {\bm{\pi}}'_{\scriptscriptstyle i}[g]))= {\bm{\pi}}_{\scriptscriptstyle i}[g]$

\item[$\bullet$] \label{dummy-vector} Constructs a fresh vector: $ {\bm{\pi}}''_{\scriptscriptstyle i}$, such that  its $g$-th element equals $ {\bm{\pi}}_{\scriptscriptstyle i}[g]$ (i.e. $ {\bm{\pi}}''_{\scriptscriptstyle i}[g]=  {\bm{\pi}}_{\scriptscriptstyle i}[g] \text{ and } | {\bm{\pi}}''_{\scriptscriptstyle i}|= | {\bm{\pi}}_{\scriptscriptstyle i}|$) and the rest of its elements are dummy values. 

\item[$\bullet$] Calls   $\mathtt{PoRID.identify}( {\bm{\pi}}''_{\scriptscriptstyle i},g,\hat{k}_{\scriptscriptstyle i},pp)\rightarrow I_{\scriptscriptstyle i}$. If $I_{\scriptscriptstyle i}=\mathcal S$, then it increments $y_{\scriptscriptstyle\mathcal S}$  by $1$. If  $I_{\scriptscriptstyle i}=\bot$, then it increments $y'_{\scriptscriptstyle\mathcal {C}}$ by $1$.
\end{itemize}

\end{enumerate}

\item The arbiter at time $\texttt{K}_{\scriptscriptstyle 6}$ sends $[y_{\scriptscriptstyle\mathcal C},y_{\scriptscriptstyle\mathcal S},  y'_{\scriptscriptstyle\mathcal C},y'_{\scriptscriptstyle\mathcal S}]$  to SC who accordingly overwrites the  elements it holds (i.e. elements of $\bm{y}$) by  the related vectors elements the arbiter sent. 

\end{enumerate} %%%%%%%%%

\

\item \textbf{Coin Transfer}. \label{RCPoRP::CoinTransfer} $\mathtt{RCPoRP}.\mathtt{pay}(\bm{y},T_{\scriptscriptstyle cp},a,p_{\scriptscriptstyle\mathcal S}, coin^{\scriptscriptstyle *}_{\scriptscriptstyle\mathcal C},coin^{\scriptscriptstyle *}_{\scriptscriptstyle\mathcal S})$
\begin{enumerate}


 \item If SC  receives  ``pay'' message  at time $\texttt{T}_{\scriptscriptstyle 2}$, where $a=0$ or $coins^{\scriptscriptstyle *}_{\scriptscriptstyle\mathcal{S}}<p_{\scriptscriptstyle\mathcal{S}}$, then it sends $coin^{\scriptscriptstyle *}_{\scriptscriptstyle\mathcal C}$ coins to $\mathcal C$ and $coin^{\scriptscriptstyle *}_{\scriptscriptstyle\mathcal S}$ coins to $\mathcal S$. Otherwise (i.e. they reach an agreement), they take the following step. 

\item Either $\mathcal C$ or $\mathcal S$ sends ``pay'' message and  statement $\ddot{x}_{\scriptscriptstyle cp}\in T_{\scriptscriptstyle cp}$ to SC at time $\texttt{L}>\texttt{K}_{\scriptscriptstyle 6}$
\item SC checks the validity of the statement by sending it  to SAP contract that  returns  either $1$ or $0$. SC only proceeds to the next step if the output is $1$
\item SC distributes the coins to the parties as follows:

\begin{itemize}
\item[$\bullet$]   $coin^{\scriptscriptstyle *}_{\scriptscriptstyle\mathcal C}-o(z-y_{\scriptscriptstyle\mathcal S})-l(y_{\scriptscriptstyle\mathcal C}+y'_{\scriptscriptstyle\mathcal C})$ coins  to $\mathcal C$
\item[$\bullet$] $coin^{\scriptscriptstyle *}_{\scriptscriptstyle\mathcal S}+o(z-y_{\scriptscriptstyle\mathcal S})-l(y_{\scriptscriptstyle\mathcal S}+y'_{\scriptscriptstyle\mathcal S})$ coins to $\mathcal S$
\item[$\bullet$] $l(y_{\scriptscriptstyle\mathcal S}+y_{\scriptscriptstyle\mathcal C}+y'_{\scriptscriptstyle\mathcal S}+y'_{\scriptscriptstyle\mathcal C})$ coins to the arbiter. 



\end{itemize}

\end{enumerate}
\end{enumerate}


\begin{remark}
The reason in step \ref{dummy-vector} vector $ {\bm{\pi}}''_{\scriptscriptstyle i}$ is constructed  is to let SC make \emph{black-box}  use of $\mathtt{PoRID.identify}(.)$. Alternatively, SC could decrypt all proofs in $\mathtt{Enc}(\bar{k}, {\bm{\pi}}_{\scriptscriptstyle i})$ and pass them to $\mathtt{PoRID.identify}(.)$. However, this approach would impose a high cost, as all proofs have to be decrypted. 
\end{remark}

\begin{remark}
In general, a transaction that is sent   to a smart contract should  cover the cost of the contract's execution. Therefore, in the above protocol, if a party unnecessarily invokes a contract for an accepting proof, it has to pay the execution cost in advance. This is the reason the above protocol (unlike RC-S-P protocol) does not need to track the number of times a party unnecessarily invokes the contract.    
\end{remark}


\begin{remark} In the protocol, for the sake of simplicity, it is assumed that the cost imposed by a malicious client to the arbiter  (to resolve a dispute) is the same as the cost imposed by a malicious server. To relax the assumption, we can simply introduce another parameter $l'$.   
 We let $l$ and $l'$ be the amounts of coin a malicious client and malicious server must pay to the arbiter respectively. In this case, (a) in step \ref{RC-PoR-P::set-cp}, the client appends $l'$ to $cp$ and (b) in the coin transfer  phase, the amounts of coin each party receives would be as follow:    $coin^{\scriptscriptstyle *}_{\scriptscriptstyle\mathcal C}-o(z-y_{\scriptscriptstyle\mathcal S})-l(y_{\scriptscriptstyle\mathcal C}+y'_{\scriptscriptstyle\mathcal C})$ coins  to $\mathcal C$,  $coin^{\scriptscriptstyle *}_{\scriptscriptstyle\mathcal S}+o(z-y_{\scriptscriptstyle\mathcal S})-l'(y_{\scriptscriptstyle\mathcal S}+ y'_{\scriptscriptstyle\mathcal S})$ coins to $\mathcal S$, and $l(y_{\scriptscriptstyle\mathcal C}+y'_{\scriptscriptstyle\mathcal C})+l'(y_{\scriptscriptstyle\mathcal S}+y'_{\scriptscriptstyle\mathcal S})$ coins to the arbiter.
\end{remark}

%\textcolor{purple} {Explain how we can slight modify the protocol to capture the fact  that the amounts of coin the client pays for a dispute resolution is different to the amount paid by the server to resolve a dispute. The idea is we introduce another value $l'$, we let $l$ and $l'$ be the amount of coin the client and server pay to resolve a dispute.  We add $l'$ to $cp$. Accordingly, we would have $coin^{\scriptscriptstyle *}_{\scriptscriptstyle\mathcal C}-o(z-y_{\scriptscriptstyle\mathcal S})+l'\cdot y_{\scriptscriptstyle\mathcal S}-l\cdot y_{\scriptscriptstyle\mathcal C}$ coins  to $\mathcal C$ and  $coin^{\scriptscriptstyle *}_{\scriptscriptstyle\mathcal S}+o(z-y_{\scriptscriptstyle\mathcal S})+l\cdot y_{\scriptscriptstyle\mathcal C}-l'\cdot y_{\scriptscriptstyle\mathcal S}$ coins to $\mathcal S$}
%


\subsubsection{Delegating the Arbiter's Role to  a Smart Contract}

In the above protocol, due to the efficiency of arbiter-side algorithm, i.e. $\mathtt{RCSPoR.resolve}(.)$,  we can  totally delegate the arbiter’s role to the smart contract, SC. In this case, the involvement of the third-party arbiter is not needed anymore.  However, to have the new variant of RC-PoR-P,   some adjustments need to be applied to the original RC-PoR-P’s protocol and definition, primarily from  two  perspectives. First,    the way a party pays to resolve a dispute would change that ultimately affects the amounts of coin each party receives at the coin distribution phase. Recall, in the RC-PoR-P and RC-S-P (presented in sections \ref{sec::RC-PoR-P-protocol} and \ref{sec::RC-S-P-protocol} respectively)  the party who raises dispute does not pay the arbiter when it sends to it a dispute query. Instead, loosely speaking, the arbiter at the coin distribution phase is paid by a misbehaving party. In contrast, when the arbiter’s role is played by a smart contract, the party who raises dispute and sends a dispute query to the contract (due to the nature of smart contracts' platform) has to pay the contract before the contract processes its query. This means, an honest party who sends a complaint to the contract needs to be compensated (by the corrupt party) for the amounts  of coin it sent to the contract to resolve the dispute. Therefore, the amounts of coin each party receives at the coin distribution phase would change, compare to the original RC-PoR-P protocol. Second, there would be no need to keep track of the number of times a party unnecessarily raises a dispute, as it pays the contract when it sends a query, before the contract processes its claim. In Appendix \ref{sec::RC-S-P-Definition2}, we provide a generic definition for RC-S-P for the case where the arbiter's role can be played by a smart contract. The generic definition also captures the new variant of RC-PoR-P. Moreover, in Appendix \ref{sec::R-PoR-P-protocol-Withohut-Arbiter}, we elaborate on how  the new variant of RC-PoR-P can be constructed and we prove its security.  

 





% !TEX root =R-zkCSP.tex

 \begin{lemma}
 If SAP and blockchain are secure and  PoRID scheme supports correctness, soundness, and detectable abort, then RC-PoR-P is secure against malicious server, w.r.t. Definition \ref{deff::RC-S-P-SecurityAgainstMaliciousServer-2}. 
 \end{lemma}
 

 \begin{proof}[sketch]
 First, we  consider event  
 
  $$\Big(F(u^{\scriptscriptstyle *}, \bm{q}_{\scriptscriptstyle j},{pp})= h_{\scriptscriptstyle j}\ \wedge (coin_{\scriptscriptstyle\mathcal{C},j}\neq  \frac{coin_{\scriptscriptstyle\mathcal C}^{\scriptscriptstyle*}}{z}-o)\Big)$$
   that captures the case where the server provides an accepting  proof, i.e. PoR, but makes an honest client withdraw  incorrect amounts of coin, i.e. $coin_{\scriptscriptstyle\mathcal{C},j}\neq  \frac{coin_{\scriptscriptstyle\mathcal C}^{\scriptscriptstyle*}}{z}-o$. Note, in RC-PoR-P protocol, the total coins the client should receive after $z$ verifications is   $coin^{\scriptscriptstyle *}_{\scriptscriptstyle\mathcal C}-o(z-y_{\scriptscriptstyle\mathcal S})+l(y_{\scriptscriptstyle\mathcal S}-y_{\scriptscriptstyle\mathcal C})$. Since we focus on  $j$-th verification, the amounts of  coin that should be credited to the client for $j$-th verification is
 
 
% 
% $F(u^{\scriptscriptstyle *},\bm{q}_{\scriptscriptstyle j})= h_{\scriptscriptstyle j}\ \wedge coin_{\scriptscriptstyle\mathcal{C},j}\neq  \frac{coin_{\scriptscriptstyle\mathcal C}^{\scriptscriptstyle*}}{z}-o$ that captures the case where the server provides an accepting service proof but makes an honest client withdraw an incorrect amounts of coin, i.e. $coin_{\scriptscriptstyle\mathcal{C},j}\neq  \frac{coin_{\scriptscriptstyle\mathcal C}^{\scriptscriptstyle*}}{z}-o$. Recall, in the protocol, the total coins the client should receive after $z$ verifications is $coin^{\scriptscriptstyle *}_{\scriptscriptstyle\mathcal C}-o(z-y_{\scriptscriptstyle\mathcal S})-l\cdot y_{\scriptscriptstyle\mathcal C}$. Since we focus on  $j$-th verification, the amount of  coins that should be credited to the client for that verification is
  
  \begin{equation}\label{equ::what-client-recives-in-j-2}
   coin_{\scriptscriptstyle\mathcal{C},j}=\frac{coin^{\scriptscriptstyle *}_{\scriptscriptstyle\mathcal C}}{z}-o(1-y_{\scriptscriptstyle\mathcal {S},j})+l(y_{\scriptscriptstyle\mathcal {S},j}-y_{\scriptscriptstyle\mathcal {C},j})
     \end{equation}

    
As the  proof is valid, an honest client accepts it and does not raise any dispute. But, the server  would be able to make the client  withdraw incorrect amounts of coin, if it manages to either convince the contract that the client has misbehaved, by making the contract output  $y_{\scriptscriptstyle\mathcal {C},j}=1$ through dispute resolution phase, or submit to the contract, at the coin transfer phase,  an accepting  statement $\ddot{x}'_{\scriptscriptstyle cp}$ other than what was agreed at the initiation phase, i.e. $\ddot{x}'_{\scriptscriptstyle cp}\neq\ddot{x}_{\scriptscriptstyle cp}$, so it can change the payments' parameters, e.g. $l$ or $o$. Nevertheless, it cannot falsely accuse the client of misbehaviour. As,  due to the security of SAP, it cannot  convince the contract to accept different query's parameters other than what was agreed with the client in the initiation phase. In particular, it cannot persuade the contract to accept  $\ddot{x}'_{\scriptscriptstyle qp}$ such that $\ddot{x}'_{\scriptscriptstyle qp}\neq \ddot{x}_{\scriptscriptstyle qp}$, except with a negligible probability, $\mu(\lambda)$.  Furthermore, if the adversary provides a valid statement then, due to the correctness of PoRID, values  $y_{\scriptscriptstyle\mathcal {C}}$ and $y_{\scriptscriptstyle\mathcal {S}}$ are not incremented by $1$  in $j$-th verification, i.e. $y_{\scriptscriptstyle\mathcal{C},j}=y_{\scriptscriptstyle\mathcal{S},j}=0$. Therefore, according to Equation \ref{equ::what-client-recives-in-j-2}, the client is credited $\frac{coin_{\scriptscriptstyle\mathcal C}^{\scriptscriptstyle*}}{z}-o$ coins for that verification, with a high probability. Note,  due to the security of SAP, the server cannot change the payment parameters by persuading the contract to accept any statement $\ddot{x}'_{\scriptscriptstyle cp}$ other than what was agreed initially between the client and server, except with  probability $\mu(\lambda)$.  We now move on to event 
     
    
    
  $$  \Big(F(u^{\scriptscriptstyle *},\bm{q}_{\scriptscriptstyle j}, {pp})\neq h_{\scriptscriptstyle j} \ \wedge (d_{\scriptscriptstyle j}=1	\vee y_{\scriptscriptstyle \mathcal{S},j}=0 \ \vee \  {coin_{\scriptscriptstyle\mathcal{C},j}\neq \frac{coin_{\scriptscriptstyle\mathcal C}^{\scriptscriptstyle*}}{z}+l} ) \Big)$$
    
    
    
     
     It captures the case where the server provides an invalid  proof but either persuades the client to accept the proof, or  persuades the contract to accept the proof  or makes the client  withdraw incorrect amounts of coin, i.e.    $coin_{\scriptscriptstyle\mathcal{C},j}\neq \frac{coin_{\scriptscriptstyle\mathcal C}^{\scriptscriptstyle*}}{z}+l$.   Nevertheless, due to the soundness of PoRID, the probability that a corrupt server can convince an honest client to accept invalid proof, i.e. outputs $d_{\scriptscriptstyle j}=1$, is  $\mu(\lambda)$. So, the client detects it with a high probability and raises a dispute. 
     
     \
     
     \
     
     Also, the server may try to make the contract keep $y_{\scriptscriptstyle \mathcal {S},j}=0$. For $y_{\scriptscriptstyle \mathcal {S},j}=0$ to happen, it has to make the contract  recognise the client as the misbehaving party, i.e. makes the contract output $y_{\scriptscriptstyle \mathcal {C},j}=0$. In this case, the client's complaint would not be processed by the contract; therefore,  $y_{\scriptscriptstyle \mathcal {S},j}$ remains $0$.   Nevertheless, as we discussed above, the probability that the adversary makes the contract  recognise the client as misbehaving is  $\mu(\lambda)$. Therefore, with a high probability  $y_{\scriptscriptstyle \mathcal {S},j}=1$ and $y_{\scriptscriptstyle \mathcal {C},j}=0$, after the contract  is invoked by the client or  server.  So, according to Equation \ref{equ::what-client-recives-in-j-2}, the client is credited $\frac{coin_{\scriptscriptstyle\mathcal C}^{\scriptscriptstyle*}}{z}+l$ coins for $j$-th verification, with a high probability. The adversary may  try to  make the client withdraw  incorrect amounts of coin, e.g. in the case where it does not succeed in convincing the client or contract. To this end,  at the coin transfer phase, it has to send a  different accepting statement  than what was initially agreed with the client. But, it would succeed only with  probability $\mu(\lambda)$, due to the security of SAP. Furthermore, in   general, due to the security of SAP, the adversary cannot block an honest client's  messages, ``pay'' and $\ddot{x}_{\scriptscriptstyle cp}$, to the contract in the coin transfer phase.
  \hfill\(\Box\)\end{proof}

%We first consider event $F(u^{\scriptscriptstyle *},\bm{q}_{\scriptscriptstyle j})= h_{\scriptscriptstyle j}\ \wedge coin_{\scriptscriptstyle\mathcal{C},j}\neq  \frac{coin_{\scriptscriptstyle\mathcal C}^{\scriptscriptstyle*}}{z}-o$ that captures the case where the server provides an accepting service proof but makes an honest client withdraw an incorrect amounts of coin, i.e. $coin_{\scriptscriptstyle\mathcal{C},j}\neq  \frac{coin_{\scriptscriptstyle\mathcal C}^{\scriptscriptstyle*}}{z}-o$. Since the proof is valid, an honest client accepts it and does not raise any dispute. 
%  However, the server  would be able to make the client  withdraw incorrect amounts of coins, if it manages to either convince the arbiter that the client has misbehaved (by making the arbiter increment $y_{\scriptscriptstyle\mathcal {C}}$  by $1$ through dispute resolution phase), or submit to the contract, at the coin transfer phase,  an accepting  statement $\ddot{x}'_{\scriptscriptstyle cp}$ other than what was agreed at the initiation phase, i.e. $\ddot{x}_{\scriptscriptstyle cp}$. Nevertheless, it cannot falsely accuse the client of misbehaviour. Because,  due to the security of SAP, it cannot  convince the arbiter to accept different decryption key or pads other than what was agreed with the client in the initiation phase; specifically, it cannot persuade the arbiter to accept  $\ddot{x}'_{\scriptscriptstyle qp}$, where $\ddot{x}'_{\scriptscriptstyle qp}\neq \ddot{x}_{\scriptscriptstyle qp}$, except with a negligible probability, $\mu(\lambda)$. This ensures that the honest client's message is accessed by  the arbiter with a high probability, as the arbiter can extract the client's message using valid pad information and decryption key.  On the other hand, if the adversary provides a valid statement, i.e. $\ddot{x}_{\scriptscriptstyle qp}$, then due to the correctness of VSID, algorithm $\mathtt{VSID.identify}()$ outputs $I_{\scriptscriptstyle j}=\bot$. Therefore, due to the security of SAP and correctness of VSID,   $y_{\scriptscriptstyle\mathcal {C}}$ and $y_{\scriptscriptstyle\mathcal {S}}$ are not incremented by $1$  in $j$-th verification, i.e. $y_{\scriptscriptstyle\mathcal{C}}[j]=y_{\scriptscriptstyle\mathcal{S}}[j]=0$.  Moreover,  due to the security of SAP, the server cannot persuade the contract to accept any statement other than what was agreed initially between the client and server, except with a negligible probability $\mu(\lambda)$ when it finds the hash function's collision. Recall, in the protocol, the total coins the client receives after $z$ verifications is $coin^{\scriptscriptstyle *}_{\scriptscriptstyle\mathcal C}-o(z-y_{\scriptscriptstyle\mathcal S})-l\cdot y_{\scriptscriptstyle\mathcal C}$. Since we focus on each $j$-th verification, the amount of  coins credited to the client for that verification is
%  
%  \begin{equation}\label{equ::what-client-recives-in-j}
%   \frac{coin^{\scriptscriptstyle *}_{\scriptscriptstyle\mathcal C}}{z}-o(1-y_{\scriptscriptstyle\mathcal S}[j])-l\cdot y_{\scriptscriptstyle\mathcal C}[j]
%     \end{equation}
%     
%     In the above event, as we discussed,  with a high probability $y_{\scriptscriptstyle\mathcal{C}}[j]=y_{\scriptscriptstyle\mathcal{S}}[j]=0$ , so the client is credited $\frac{coin_{\scriptscriptstyle\mathcal C}^{\scriptscriptstyle*}}{z}-o$ coins for that verification. 
%     
   




\begin{lemma}
 If SAP and blockchain are secure and  PoRID scheme supports correctness, inputs well-formedness, and detectable abort, then RC-PoR-P is secure against malicious client, w.r.t. Definition \ref{deff::RC-S-P-Security-Against Malicious-Client-2}. 
\end{lemma}


\begin{proof}[sketch]
  First, we consider event 
  
   $$\Big((M(u^{\scriptscriptstyle *},k,{pp})= \sigma \  \wedge \ Q(\text{aux},k, {pp})= \bm{q}_{\scriptscriptstyle j})\ \wedge \
(coin_{\scriptscriptstyle\mathcal{S},j}\neq  \frac{coin_{\scriptscriptstyle\mathcal S}^{\scriptscriptstyle*}}{z}+o)\Big)$$
  
  
  It captures the case where the client provides  accepting metadata and query  but makes the server withdraw an incorrect amounts of coin, i.e. $coin_{\scriptscriptstyle\mathcal{S},j}\neq  \frac{coin_{\scriptscriptstyle\mathcal S}^{\scriptscriptstyle*}}{z}+o$. According to  RC-PoR-P protocol, the total coins the server should receive after $z$ verifications is $coin^{\scriptscriptstyle *}_{\scriptscriptstyle\mathcal S}+o(z-y_{\scriptscriptstyle\mathcal S})+l(y_{\scriptscriptstyle\mathcal C}-y_{\scriptscriptstyle\mathcal S})$. As we focus on  $j$-th verification, the amount of  coins that should be credited to the server for $j$-th verification is
  
  \begin{equation}\label{equ::what-server-recives-in-j-2}
   coin_{\scriptscriptstyle\mathcal{S},j}=\frac{coin^{\scriptscriptstyle *}_{\scriptscriptstyle\mathcal S}}{z}+o(1-y_{\scriptscriptstyle\mathcal {S},j})+l(y_{\scriptscriptstyle\mathcal {C},j}-y_{\scriptscriptstyle\mathcal {S},j})
     \end{equation}
  
  
  
  Since the metadata and query's proofs are valid, an honest server accepts them and does not raise any dispute, so we have $y_{\scriptscriptstyle\mathcal{C},j}=0$.  The client can  make the server  withdraw incorrect amounts of coins, if it manages to either convince the contract, in the dispute resolution phase,  that the server has misbehaved, i.e. makes the contract output $y_{\scriptscriptstyle\mathcal{S},j}=1$, or submit to the contract  an accepting  statement $\ddot{x}'_{\scriptscriptstyle cp}$ other than what was agreed at the initiation phase, i.e. $\ddot{x}_{\scriptscriptstyle cp}$, in the coin transfer phase. But, it cannot falsely accuse the server of misbehaviour, because due to the security of SAP, it cannot  convince the contract to accept different decryption key and pads, by providing a different accepting statement $\ddot{x}'_{\scriptscriptstyle qp}$  (where $\ddot{x}'_{\scriptscriptstyle qp}\neq \ddot{x}_{\scriptscriptstyle qp}$), than what was initially agreed with the server, except with  probability $\mu(\lambda)$.  So, with a high probability $y_{\scriptscriptstyle\mathcal{S},j}=0$. On the other hand, if the adversary provides a valid statement, i.e. $\ddot{x}_{\scriptscriptstyle qp}$, then due to the correctness of PoRID, algorithm $\mathtt{PoRID.identify}(.)$ outputs $I_{\scriptscriptstyle j}=\bot$. Thus, due to the security of SAP and correctness of PoRID,  we would have $y_{\scriptscriptstyle\mathcal{C},j}=y_{\scriptscriptstyle\mathcal{S},j}=0$ with a high probability.    That means, according to  Equation \ref{equ::what-server-recives-in-j-2}, the server is credited $\frac{coin_{\scriptscriptstyle\mathcal S}^{\scriptscriptstyle*}}{z}+o$ coins for that verification, with a  high probability. Note,  due to the security of SAP, the client cannot convince the contract to accept any accepting statement $\ddot{x}'_{\scriptscriptstyle cp}$ other than what was initially agreed  between the client and server (i.e. $\ddot{x}'_{\scriptscriptstyle cp}\neq \ddot{x}_{\scriptscriptstyle cp}$), except with   probability $\mu(\lambda)$. We now move on  to 
  
  $$\Big(M(u^{\scriptscriptstyle *},k,{pp})\neq \sigma\ \wedge \ a=1\Big)$$
     
     
    It captures the case where the server accepts  ill-formed metadata. But, due to PoRID's inputs well-formedness the probability the event happens is  $\mu(\lambda)$. So, with a high probability $a=0$. In the case where the server does not raise any dispute, instead it avoids serving the client.    Next, we turn our attention   to 
     
     $$\Big(Q(\text{aux},k,{pp})\neq \bm{q}_{\scriptscriptstyle j})\ \wedge \ (b_{\scriptscriptstyle j}=1\ \vee \ y_{\scriptscriptstyle\mathcal{C},j}=0 \ \vee \ coin_{\scriptscriptstyle\mathcal{S},j}\neq \frac{coin_{\scriptscriptstyle\mathcal S}^{\scriptscriptstyle*}}{z}+o+l)\Big)$$
     
       It considers the case where the client provides an invalid query, but  either convinces the server or contract to accept it, or  makes the server   withdraw  incorrect amounts of coin, i.e. $coin_{\scriptscriptstyle\mathcal{S},j}\neq \frac{coin_{\scriptscriptstyle\mathcal S}^{\scriptscriptstyle*}}{z}+o+l$. Due to inputs well-formedness of PoRID, the probability that the server outputs $b_{\scriptscriptstyle j}=1$, in this case is $\mu(\lambda)$.  When the server rejects the query and raises a dispute, the client may try to convince the contract, and make it output $y_{\scriptscriptstyle \mathcal {C},j}=0$, e.g. by sending a complaint. For the adversary to succeed, first it has to send a message on the server's behalf at time $\texttt{k}_{\scriptscriptstyle 1}$, but due to the security of blockchain its winning probability is at most $\mu(\lambda)$. Also,   either it has to provide a different accepting statement $\ddot{x}'_{\scriptscriptstyle qp}$, than what was initially agreed with the server (i.e. $\ddot{x}'_{\scriptscriptstyle qp}\neq \ddot{x}_{\scriptscriptstyle qp}$) and passes the verification. But,  due to the security of SAP, its probability of success is $\mu(\lambda)$. Or it has to make the contract  accept an invalid query, i.e. makes the contract output $y_{\scriptscriptstyle\mathcal{C},j}=0$, but due to the detectable abort property of PoRID, its probability of success is  $\mu(\lambda)$ too. Therefore, with a high probability we have $y_{\scriptscriptstyle\mathcal{C},j}=1$. Also, as discussed above, the client cannot make the contract recognise the  honest server as a misbehaving party with a probability significantly greater than $\mu(\lambda)$. That means with a high probability $y_{\scriptscriptstyle\mathcal{S},j}=0$. Hence, according to Equation \ref{equ::what-server-recives-in-j-2} the server is credited  $\frac{coin_{\scriptscriptstyle\mathcal S}^{\scriptscriptstyle*}}{z}+o+l$ coins for $j$-th verification.   The adversary may still try to  make the server withdraw   incorrect amounts of coin (e.g. if the adversary does not succeed in convincing the server). To this end, at the coin transfer phase, it has to send a  different accepting statement  than what was initially agreed with the server. However, due to the security of SAP, its  success probability is  $\mu(\lambda)$. Also, due to the security of SAP, the adversary cannot block an honest server's  messages, ``pay'' and $\ddot{x}_{\scriptscriptstyle cp}$, to the contract in the coin transfer phase.
 \hfill\(\Box\)\end{proof}
  
  
  Prior to proving  RC-S-P's privacy, we provide a lemma that will be used in the privacy's proof. Informally, the lemma states that encoded coins leaks no information about the actual amounts of coin $(o,l)$, agreed between the client and server. 


\begin{lemma}\label{lemma::encoded-coins} Let $\beta\stackrel{\scriptscriptstyle\$}\leftarrow\{0,1\}$, price list be $\{(o_{\scriptscriptstyle 0},l_{\scriptscriptstyle 0}),(o_{\scriptscriptstyle 1},l_{\scriptscriptstyle 1})\}$, and encoded coin amounts be $coin^{\scriptscriptstyle *}_{\scriptscriptstyle\mathcal{C}}= z\cdot(Max(o_{\scriptscriptstyle\beta},o_{\scriptscriptstyle|\beta-1|})+Max(l_{\scriptscriptstyle\beta},l_{\scriptscriptstyle|\beta-1|}))$ and $coin^{\scriptscriptstyle *}_{\scriptscriptstyle\mathcal{S}}=z\cdot(Max(l_{\scriptscriptstyle\beta}, l_{\scriptscriptstyle |\beta-1|}))$. Then, given the price list, $z$, $coin^{\scriptscriptstyle *}_{\scriptscriptstyle\mathcal{C}}$, and $coin^{\scriptscriptstyle *}_{\scriptscriptstyle\mathcal{S}}$, an adversary $\mathcal{A}$ cannot tell the value of $\beta$ with probability significantly greater than $\frac{1}{2}$ (where the probability is taken over the choice of $\beta$ and the randomness of $\mathcal{A}$).
\end{lemma}



\begin{proof}
As it is evident, the list and $z$ contains no information about $\beta$. Also, since  $z$ is a public value, we could have   $coin'^{\scriptscriptstyle *}_{\scriptscriptstyle\mathcal C}=\frac{coin^{\scriptscriptstyle *}_{\scriptscriptstyle\mathcal C}}{z} = Max(o_{\scriptscriptstyle\beta},o_{\scriptscriptstyle|\beta-1|})+Max(l_{\scriptscriptstyle\beta}, l_{\scriptscriptstyle|\beta-1|})$. It is not hard to see $coin'^{\scriptscriptstyle *}_{\scriptscriptstyle\mathcal C}$ is a function of maximum value of $(o_{\scriptscriptstyle 0},o_{\scriptscriptstyle 1})$, and maximum value of $(l_{\scriptscriptstyle 0},l_{\scriptscriptstyle 1})$. It is also  independent of $\beta$. Therefore, given the list, $z$ and $coin'^{\scriptscriptstyle *}_{\scriptscriptstyle\mathcal C}$ the adversary learns nothing about $\beta$, unless it guesses the value,  with  success probability $\frac{1}{2}$. The same also holds for $coin^{\scriptscriptstyle *}_{\scriptscriptstyle\mathcal S}$. 
 \hfill\(\Box\)\end{proof}


\begin{lemma} If SAP is secure and the encryption scheme is semantically secure, then RC-S-P preserves privacy, w.r.t. Definition \ref{deff::RC-S-P-Privacy}. 
\end{lemma}

\begin{proof}[sketch]
Due to the privacy property of SAP, that stems from the  hiding property of the commitment scheme, given the commitments $g_{\scriptscriptstyle qp}$ and $g_{\scriptscriptstyle cp}$, (that are stored in the blockchain as result of running SAP) the adversary learns no information about the committed values (e.g. $o, l, pad_{\scriptscriptstyle\pi}, pad_{\scriptscriptstyle q},$ and $\bar{k}$), except with negligible probability, $\mu_{\scriptscriptstyle 1}(\lambda)$.   Also, given  price list $pl$, encoded coins $coin^{\scriptscriptstyle *}_{\scriptscriptstyle\mathcal{C}}=z\cdot (o_{\scriptscriptstyle max}+l_{\scriptscriptstyle max})$ and  $coin^{\scriptscriptstyle *}_{\scriptscriptstyle\mathcal{S}}=z\cdot l_{\scriptscriptstyle max}$, the adversary learns nothing about the actual price agreed between the server and client,  $(o,l)$, for each verification, due to Lemma \ref{lemma::encoded-coins}. Next we analyse the privacy of padded encrypted query vector $\bm{c}^{\scriptscriptstyle *}$. For  the sake of simplicity, we focus on   $\bm{q}^{\scriptscriptstyle *}_{\scriptscriptstyle j}\in c^{\scriptscriptstyle *}_{\scriptscriptstyle j}\in\bm{c}^{\scriptscriptstyle *}$, that is a padded encrypted query vector for $j$-th verification. Let $\bm{q}_{\scriptscriptstyle j,0}$ and $\bm{q}_{\scriptscriptstyle j,1}$  be query vectors, for $j$-th verification, related to the service inputs $u_{\scriptscriptstyle 0}$ and $u_{\scriptscriptstyle 1}$ that    are picked by the adversary according to  Definition  \ref{deff::RC-S-P-Privacy} which lets  the environment pick $\beta\stackrel{\scriptscriptstyle\$}\leftarrow \{0,1\}$. Also, let $\{\bm{q}_{\scriptscriptstyle j,0},...,\bm{q}_{\scriptscriptstyle j,m}\}$ be a  list of all queries of different sizes. In the experiment, if $\bm{q}_{\scriptscriptstyle j,\beta}$  is only   encrypted (but not padded), then given the ciphertext, due to semantical security of the encryption, an adversary cannot tell if the ciphertext corresponds to $\bm{q}_{\scriptscriptstyle j,0}$ or $\bm{q}_{\scriptscriptstyle j,1}$  (accordingly to $u_{\scriptscriptstyle 0}$ or $u_{\scriptscriptstyle 1}$) with probability greater than $\frac{1}{2}+\mu_{\scriptscriptstyle 1}(\lambda)$,  under the assumption  that $Max(|\bm{q}_{\scriptscriptstyle j,0}|,...,|\bm{q}_{\scriptscriptstyle j,m}|)=|\bm{q}_{\scriptscriptstyle j,\beta}|$. The  assumption is  relaxed with the use of a pad; as each encrypted query is padded  to the queries' maximum size, i.e. $Max(|\bm{q}_{\scriptscriptstyle j,0}|,...,|\bm{q}_{\scriptscriptstyle j,m}|)$, the adversary cannot tell with probability greater than $\frac{1}{2}+\mu_{\scriptscriptstyle 1}(\lambda)$ if the  padded encrypted proof corresponds to  $\bm{q}_{\scriptscriptstyle j,0}$ or $\bm{q}_{\scriptscriptstyle j,1}$, as the padded encrypted query \emph{always has the same size} and the pad values are picked from the same range as the encryption's ciphertext are defined. The same argument holds for $\bm{w}^{\scriptscriptstyle *}_{\scriptscriptstyle q_{_{\scaleto{j}{\size}}}}\in c^{\scriptscriptstyle *}_{\scriptscriptstyle j}\in\bm{c}^{\scriptscriptstyle *}$. 

Next we analyse the privacy of padded encrypted proof vector $\bm{\pi}^{\scriptscriptstyle *}$. The argument is similar to the one presented above; however, for  the sake of completeness we provide it.  We focus on an element of the vector,  $\pi_{\scriptscriptstyle j}^{\scriptscriptstyle *}\in\bm{\pi}^{\scriptscriptstyle *}$, that is a padded encrypted proof for $j$-th verification. Let $\pi_{\scriptscriptstyle j,0}$ and $\pi_{\scriptscriptstyle j,1}$  be proofs, for $j$-th verification, related to the service inputs $u_{\scriptscriptstyle 0}$ and $u_{\scriptscriptstyle 1}$, where the inputs   are picked by the adversary, w.r.t. Definition  \ref{deff::RC-S-P-Privacy} in which  the environment picks $\beta\stackrel{\scriptscriptstyle\$}\leftarrow \{0,1\}$.  Let $\{\pi_{\scriptscriptstyle j,0},...,\pi_{\scriptscriptstyle j,m}\}$ be proof list including all proofs of different sizes. In the experiment, if $\pi_{\scriptscriptstyle j,\beta}$  is only   encrypted, then given the ciphertext, due to semantical security of the encryption, an adversary cannot tell if the ciphertext corresponds to $\pi_{\scriptscriptstyle j,0}$ or $\pi_{\scriptscriptstyle j,1}$  (accordingly to $u_{\scriptscriptstyle 0}$ or $u_{\scriptscriptstyle 1}$) with probability greater than $\frac{1}{2}+\mu_{\scriptscriptstyle 2}(\lambda)$,  if $Max(|\pi_{\scriptscriptstyle j,0}|,...,|\pi_{\scriptscriptstyle j,m}|)=|\pi_{\scriptscriptstyle j,\beta}|$. However, the  assumption is  relaxed with the use of a pad. In particular, since each encrypted proof is padded  to the proofs' maximum size, the adversary cannot tell with probability greater than $\frac{1}{2}+\mu_{\scriptscriptstyle 2}(\lambda)$ if the  padded encrypted proof corresponds to  $\pi_{\scriptscriptstyle j,0}$ or $\pi_{\scriptscriptstyle j,1}$. 


Also, since the value of $a$ is independent of $u_{\scriptscriptstyle 0}$  or $u_{\scriptscriptstyle 1}$, and only depends on whether the metadata is well-formed, it leaks nothing about the service input $u_{\scriptscriptstyle\beta}$ and $\beta$. Moreover, since each padded encrypted query and proof leak no information and always contains a fixed number of elements, an adversary cannot tell the status of a proof for each $j$-th verification (i.e. whether it is accepted or rejected) with the probability greater than $\frac{1}{2}+\mu_{\scriptscriptstyle 2}(\lambda)$, given  $\bm{c}^{\scriptscriptstyle *},coin^{\scriptscriptstyle *}_{\scriptscriptstyle \mathcal S},coin^{\scriptscriptstyle *}_{\scriptscriptstyle\mathcal C}, g_{\scriptscriptstyle cp},  g_{\scriptscriptstyle qp},$  $\bm{\pi}^{\scriptscriptstyle *},pl$, and $a$.
  \hfill\(\Box\)\end{proof}




% !TEX root =R-zkCSP.tex

\subsection{Evaluation of RC-PoR-P}
















\bibliographystyle{splncs03}
\bibliography{ref}
\appendix

%\input{Recurring-Contingent-PoR-Payment-protocol-With-arbiter}
% !TEX root =R-zkCSP.tex

\section{Definition of RC-S-P Without Arbiter's Involvement}\label{sec::RC-S-P-Definition2}
There are cases, in RC-S-P schemes, where the third-party arbiter's role  can be  efficiently delegated to a smart contract.  In this variant of RC-S-P scheme, denoted by ${\text{RC-}\overline{\text{S}}\text{-P}}$, the arbiter's involvement is not needed anymore. The primary difference between RC-S-P and ${\text{RC-}\overline{\text{S}}\text{-P}}$ is the way a party pays to resolve a dispute. In particular, in RC-S-P, the party who raises dispute does not pay the arbiter when it sends to it a dispute query. Instead, loosely speaking, the arbiter at coin distribution is paid by a misbehaving party. Whereas, in ${\text{RC-}\overline{\text{S}}\text{-P}}$, the party who raises dispute and sends a dispute query to the contract, (due to the nature of smart contracts' platform) has to pay the contract, before the contract processes its query.  In this section, we show how RC-S-P definition (presented in Section \ref{sec::RC-S-P-Definition}) can be  adjusted to  capture ${\text{RC-}\overline{\text{S}}\text{-P}}$. In the following, we highlight the main changes that should be applied to the RC-S-P definition. 

\begin{itemize}

\item[$\bullet$] In Definition \ref{RC-S-P-def}: 
\begin{itemize}



\item [$\bullet$]  Three parties are involved; namely, client, server and smart contract (so an arbiter is not involved anymore). 

\

\item [$\bullet$]  Vectors $(\bm{y}'_{\scriptscriptstyle\mathcal{C}},\bm{y}'_{\scriptscriptstyle\mathcal{S}})$ are not needed anymore. Because a misbehaving party, who unnecessarily invokes the contract, pays the contract ahead of time. Therefore, there is no need to keep track of unnecessary contract's invocation. 




\

\item [$\bullet$] $\mathtt{RCSP}.\mathtt{resolve}(.)$ is run by a smart contract. 

\

\item [$\bullet$] $\mathtt{RCSP}.\mathtt{pay}(.)$ outputs $({\bm{coin}}_{\scriptscriptstyle\mathcal C},{\bm{coin}}_{\scriptscriptstyle\mathcal S})$, so ${\bm{coin}}_{\scriptscriptstyle\mathcal{A}r}$ is excluded from the output, as a third-party arbiter plays no role anymore.
\end{itemize}

\

\item [$\bullet$] In Definition \ref{def::RC-S-P-Correctness}: only the above changes are applied to it. 

\

\item [$\bullet$] In Definition \ref{deff::RC-S-P-SecurityAgainstMaliciousServer}:  the above changes are applied to the algorithms' syntax in the experiment. Moreover, the events are slightly modified, i.e. the amount of coins each party receives. For the sake of clarity and completeness, we state the entire modified  definition below. 

\begin{definition}[${\text{RC-}\overline{\text{S}}\text{-P}}$ Security Against Malicious Server]\label{deff::RC-S-P-SecurityAgainstMaliciousServer-2}  A ${\text{RC-}\overline{\text{S}}\text{-P}}$  is secure against a malicious server, for functions $F,Q,M,D,E$, and an auxiliary information $aux$, if for    any price list $pl$, every $j$ (where $1\leq j\leq z$), and any PPT adversary $\mathcal{A}$, there exists a negligible function $\mu(\cdot)$, such that for any security parameter $\lambda$: 
{\small
$$ \Pr\left[
  \begin{array}{l}
  \Big(F(u^{\scriptscriptstyle *}, \bm{q}_{\scriptscriptstyle j}, {pp})= h_{\scriptscriptstyle j}\ \wedge\\ (coin_{\scriptscriptstyle\mathcal{C},j}\neq  \frac{coin_{\scriptscriptstyle\mathcal C}^{\scriptscriptstyle*}}{z}-o)\Big)\ \vee\\ 
\Big(
F(u^{\scriptscriptstyle *},\bm{q}_{\scriptscriptstyle j}, {pp})\neq h_{\scriptscriptstyle j} \ \wedge\\ (d_{\scriptscriptstyle j}=1	\vee y_{\scriptscriptstyle \mathcal{S},j}=0 \ \vee \   \\ {coin_{\scriptscriptstyle\mathcal{C},j}\neq \frac{coin_{\scriptscriptstyle\mathcal C}^{\scriptscriptstyle*}}{z}+l} ) \Big)\\
\end{array} \middle |
    \begin{array}{l}
    \mathtt{RCSP.keyGen}(1^{\lambda}, F)\rightarrow \bm{k}\\
    \mathcal{A}(1^\lambda,pk, F)\rightarrow u\\
  
    \mathtt{RCSP}.\mathtt{cInit}(1^\lambda, u,\bm{k}, M,z,pl,enc)\rightarrow (u^{\scriptscriptstyle *},e,T, p_{\scriptscriptstyle\mathcal S},  \bm{y}, coin^{\scriptscriptstyle*}_{\scriptscriptstyle\mathcal C})\\ 
    \mathcal{A}(u^{\scriptscriptstyle *}, e, pk, z, T, p_{\scriptscriptstyle\mathcal S},\bm{y},enc)\rightarrow (coin^{\scriptscriptstyle *}_{\scriptscriptstyle\mathcal S},a)\\
	\mathtt{RCSP}.\mathtt{genQuery}(1^\lambda, \text{aux},k,Q, T_{\scriptscriptstyle qp}, enc)\rightarrow c^{\scriptscriptstyle *}_{\scriptscriptstyle j}\\
	
     \mathcal{A}(c^{\scriptscriptstyle *}_{\scriptscriptstyle j},\sigma, u^{\scriptscriptstyle *},enc,a)\rightarrow 
     (b_{\scriptscriptstyle j},m_{\scriptscriptstyle \mathcal{S},j},h^{\scriptscriptstyle *}_{\scriptscriptstyle j},\delta^{\scriptscriptstyle *}_{\scriptscriptstyle j})\\      
\mathtt{RCSP}.\mathtt{verify}(\pi^{\scriptscriptstyle *}_{\scriptscriptstyle j}, \bm{q}_{\scriptscriptstyle j},k,T_{\scriptscriptstyle qp},enc)\rightarrow (d_{\scriptscriptstyle j},m_{\scriptscriptstyle \mathcal{C},j})\\
   \mathtt{RCSP}.\mathtt{resolve}(m_{\scriptscriptstyle \mathcal{C}},m_{\scriptscriptstyle \mathcal{S}},z, {\bm{\pi}}^{\scriptscriptstyle *}, {\bm{c}}^{\scriptscriptstyle *},pk, T_{\scriptscriptstyle qp},enc)\rightarrow \bm{y}\\
   \mathtt{RCSP}.\mathtt{pay}(\bm{y},T_{\scriptscriptstyle cp},a,p_{\scriptscriptstyle\mathcal S}, coin^{\scriptscriptstyle *}_{\scriptscriptstyle\mathcal C},coin^{\scriptscriptstyle *}_{\scriptscriptstyle\mathcal S})\rightarrow ({\bm{coin}}_{\scriptscriptstyle\mathcal C},{\bm{coin}}_{\scriptscriptstyle\mathcal S})\\
\end{array}    \right]\leq \mu(\lambda).$$
}
where $\pi^{\scriptscriptstyle *}_{\scriptscriptstyle j}:=(h^{\scriptscriptstyle *}_{\scriptscriptstyle j}, \delta^{\scriptscriptstyle *}_{\scriptscriptstyle j}), h_{\scriptscriptstyle j}= D(h^{\scriptscriptstyle *}_{\scriptscriptstyle j},T_{\scriptscriptstyle qp})$, $
D\in enc$, $\sigma\in e$, $m_{\scriptscriptstyle \mathcal{C},j}\in m_{\scriptscriptstyle \mathcal{C}}, m_{\scriptscriptstyle \mathcal{S},j}\in m_{\scriptscriptstyle \mathcal{S}}$,   $y_{\scriptscriptstyle \mathcal{S},j}\in \bm{y}_{\scriptscriptstyle \mathcal{S}}\in\bm{y}$, and $ {pp}\in T_{\scriptscriptstyle qp}$. 
\end{definition}

\

 \item[$\bullet$] In Definition \ref{deff::RC-S-P-Security-Against Malicious-Client}: similar to the previous point, only the algorithms' syntax (in the experiment) and the amount of coins each party receives changes. Below, we state the entire modified definition. 
 
 \begin{definition}[${\text{RC-}\overline{\text{S}}\text{-P}}$ Security Against Malicious Client]\label{deff::RC-S-P-Security-Against Malicious-Client-2}  A  ${\text{RC-}\overline{\text{S}}\text{-P}}$  is secure against a malicious client, for functions $F,Q,M,D,E$, and an auxiliary information $aux$, if for every $j$ (where $1\leq j\leq z$), and any probabilistic polynomial time adversary $\mathcal{A}$, there exists a negligible function $\mu(\cdot)$, such that for any security parameter $\lambda$: 

{\small
$$ \Pr\left[
  \begin{array}{l}
 \Big((M(u^{\scriptscriptstyle *},k, {pp})= \sigma  \wedge\\ Q(\text{aux},k, {pp})= \bm{q}_{\scriptscriptstyle j})\ \wedge\\
  (coin_{\scriptscriptstyle\mathcal{S},j}\neq  \frac{coin_{\scriptscriptstyle\mathcal S}^{\scriptscriptstyle*}}{z}+o )\Big)\ \vee
  \\
  
  
  \Big(M(u^{\scriptscriptstyle *}, k,  {pp})\neq \sigma \wedge a=1\Big) \ \vee\\ 


  \Big(Q(\text{aux},k,  {pp})\neq \bm{q}_{\scriptscriptstyle j}\ \wedge\\
  (b_{\scriptscriptstyle j}=1 \ \vee \ y_{\scriptscriptstyle\mathcal{C},j}=0 \ \vee \\ {coin_{\scriptscriptstyle\mathcal{S},j}\neq \frac {coin_{\scriptscriptstyle\mathcal S}^{\scriptscriptstyle*}}{z}+o + l}) 
  \Big)
 \\
 
\end{array} \middle |
    \begin{array}{l}
   \mathcal{A}(1^{\lambda}, F)\rightarrow (u^{\scriptscriptstyle *},z,\bm{k},e,T, pl, p_{\scriptscriptstyle\mathcal{S}}, coin^{\scriptscriptstyle*}_{\scriptscriptstyle\mathcal C},enc, \text{aux},\bm{y}, enc, pk)\\
   
   \mathtt{RCSP}.\mathtt{sInit}(u^{\scriptscriptstyle *}, e, pk, z, T, p_{\scriptscriptstyle\mathcal S},\bm{y},enc)\rightarrow (coin^{\scriptscriptstyle *}_{\scriptscriptstyle\mathcal S},a)\\
   
	\mathcal{A}(coin^{\scriptscriptstyle *}_{\scriptscriptstyle\mathcal S},a, 1^\lambda, \text{aux},k,Q, T_{\scriptscriptstyle qp}, enc)\rightarrow c^{\scriptscriptstyle *}_{\scriptscriptstyle j}\\

     \mathtt{RCSP}.\mathtt{prove}(u^{\scriptscriptstyle *},  \sigma,  c^{\scriptscriptstyle *}_{\scriptscriptstyle j}, pk, T_{\scriptscriptstyle qp},enc)\rightarrow (b_{\scriptscriptstyle j},m_{\scriptscriptstyle \mathcal{S},j},\pi^{\scriptscriptstyle *}_{\scriptscriptstyle j})\\
     
 \mathcal{A}(\pi^{\scriptscriptstyle *}_{\scriptscriptstyle j}, \bm{q}_{\scriptscriptstyle j},k,T_{\scriptscriptstyle qp},enc)\rightarrow (d_{\scriptscriptstyle j},m_{\scriptscriptstyle \mathcal{S},j})\\
   
   
   \mathtt{RCSP}.\mathtt{resolve}(m_{\scriptscriptstyle \mathcal{C}},m_{\scriptscriptstyle \mathcal{S}},z, {\bm{\pi}}^{\scriptscriptstyle *}, {\bm{c}}^{\scriptscriptstyle *}, pk, T_{\scriptscriptstyle qp},enc)\rightarrow \bm{y}\\
   \mathtt{RCSP}.\mathtt{pay}(\bm{y},T_{\scriptscriptstyle cp},a,p_{\scriptscriptstyle\mathcal S}, coin^{\scriptscriptstyle *}_{\scriptscriptstyle\mathcal C},coin^{\scriptscriptstyle *}_{\scriptscriptstyle\mathcal S})\rightarrow ({\bm{coin}}_{\scriptscriptstyle\mathcal C},{\bm{coin}}_{\scriptscriptstyle\mathcal S})\\
\end{array}    \right]\leq \mu(\lambda).$$
}
where $\bm{q}_{\scriptscriptstyle j}\in D(c^{\scriptscriptstyle *}_{\scriptscriptstyle j},t_{\scriptscriptstyle qp})$, $
D\in enc$, $\sigma\in e$,    $y_{\scriptscriptstyle \mathcal{C},j}\in \bm{y}_{\scriptscriptstyle \mathcal{C}}\in\bm{y}$, and ${pp}\in T_{\scriptscriptstyle qp}$.  
\end{definition}
\end{itemize}

Note that Definition \ref{deff::RC-S-P-Privacy} remains almost the same with a  minor change, that is vectors $(\bm{y}'_{\scriptscriptstyle\mathcal{C}},\bm{y}'_{\scriptscriptstyle\mathcal{S}})$ are excluded from the related algorithms input/output. 

 \begin{definition}\label{def::RC-S-P-Security-2} A ${\text{RC-}\overline{\text{S}}\text{-P}}$  is secure if it satisfies security against malicious server, security against malicious client, and preserves privacy, w.r.t. Definitions \ref{deff::RC-S-P-SecurityAgainstMaliciousServer-2},\ref{deff::RC-S-P-Security-Against Malicious-Client-2}, and \ref{deff::RC-S-P-Privacy}. 
 \end{definition}






%\small{
%$$ Pr\left[
%  \begin{array}{l}
% \Big((M(u^{\scriptscriptstyle *},k)= \sigma  \wedge Q(\text{aux},k)= q_{\scriptscriptstyle j}) \wedge\\
%  (coin_{\scriptscriptstyle\mathcal{S},j}\neq  \frac{coin_{\scriptscriptstyle\mathcal S}^{\scriptscriptstyle*}}{z}+o)\Big) \vee
%  \\
%  \Big((M(u^{\scriptscriptstyle *},k)\neq \sigma \lor Q(\text{aux},k)\neq q_{\scriptscriptstyle j}) \wedge\\
%  (a=1 \ \vee b_{\scriptscriptstyle j}=1\ \vee y_{\scriptscriptstyle\mathcal{C}}[j]=0 \ \vee \\ coin_{\scriptscriptstyle\mathcal{S},j}\neq \frac{coin_{\scriptscriptstyle\mathcal S}^{\scriptscriptstyle*}}{z}+o \ \vee  coin_{\scriptscriptstyle\mathcal{A}r,j}\neq l) 
%  \Big)
% \\
% 
%\end{array} \middle |
%    \begin{array}{l}
%   \mathcal{A}(1^{\lambda},F)\rightarrow (u^{\scriptscriptstyle *},z,\bm{k},e,t,coin^{\scriptscriptstyle*}_{\scriptscriptstyle\mathcal C},en, \text{aux},\bm{y}_{\scriptscriptstyle \mathcal  C},\bm{y}_{\scriptscriptstyle \mathcal  S})\\
%   
%   \mathtt{RCSP}.\mathtt{sInit}(u^{\scriptscriptstyle *},e, pk,z, t,\bm{y}_{\scriptscriptstyle \mathcal  C},\bm{y}_{\scriptscriptstyle \mathcal  S})\rightarrow (coin^{\scriptscriptstyle *}_{\scriptscriptstyle\mathcal S},a)\\
%   
%	\mathcal{A}(coin^{\scriptscriptstyle *}_{\scriptscriptstyle\mathcal S},a, \text{aux},\bm{k},Q,j, en)\rightarrow c^{\scriptscriptstyle *}_{\scriptscriptstyle j}\\
%
%     \mathtt{RCSP}.\mathtt{prove}(u^{\scriptscriptstyle *},  \sigma,  c^{\scriptscriptstyle *}_{\scriptscriptstyle j},pk, en)\rightarrow (b_{\scriptscriptstyle j},m_{\scriptscriptstyle \mathcal{S},j},\pi^{\scriptscriptstyle *}_{\scriptscriptstyle j})\\
%     
% \mathcal{A}(\pi^{\scriptscriptstyle *}_{\scriptscriptstyle j}, q,k,j,en)\rightarrow (d_{\scriptscriptstyle j},m_{\scriptscriptstyle \mathcal{A},j})\\
%   
%   
%   \mathtt{RCSP}.\mathtt{resolve}(\bm{m}_{\scriptscriptstyle \mathcal{A}},\bm{m}_{\scriptscriptstyle \mathcal{S}},z, {\bm{\pi}}^{\scriptscriptstyle *}, {\bm{c}}^{\scriptscriptstyle *},pk,en)\rightarrow (\bm{y}_{\scriptscriptstyle \mathcal C},\bm{y}_{\scriptscriptstyle \mathcal S})\\
%   \mathtt{RCSP}.\mathtt{pay}(\bm{y}_{\scriptscriptstyle \mathcal C},\bm{y}_{\scriptscriptstyle \mathcal S},t_{\scriptscriptstyle cp},a,coin^{\scriptscriptstyle *}_{\scriptscriptstyle\mathcal C},coin^{\scriptscriptstyle *}_{\scriptscriptstyle\mathcal S})\rightarrow ({\bm{coin}}_{\scriptscriptstyle\mathcal C},{\bm{coin}}_{\scriptscriptstyle\mathcal S},{\bm{coin}}_{\scriptscriptstyle\mathcal Ar})\\
%\end{array}    \right]\leq \mu(\lambda)$$
%}
%where $q_{\scriptscriptstyle j}\in D(c^{\scriptscriptstyle *}_{\scriptscriptstyle j},t_{\scriptscriptstyle qp})$, $
%D,t_{\scriptscriptstyle qp}\in en$, $\sigma\in e$, and   the probability is taken over the randomness used in witnesses $w_{\scriptscriptstyle cp},w_{\scriptscriptstyle qp}\in t$ as well as  the randomness used in proofs $w_{\scriptscriptstyle q_{_{\scaleto{j}{\size}}}}\in c_{\scriptscriptstyle j}$ and $w_{\scriptscriptstyle\sigma}\in e$ and  the randomness of $\mathcal{A}$.
%\end{definition}
% 
% Note, in the above definition, an honest server either does not deposit (e.g. when $a=0$) or if it deposits (i.e. agrees to serve) ultimately receives its deposit \emph{plus the service payment} (with a high probability). Informally, RC-S-P is privacy preserving if it guarantees the privacy of (a) the service input (e.g. outsourced file) and (b) the service proof's status during the private time bubble. In the following, we formally define privacy.
% 
%
%  \begin{definition}[RC-S-P Privacy]\label{deff::RC-S-P-Privacy}  A RC-S-P   preserves privacy, any auxiliary information $\text{aux}$, any number of verifications $z$, any price list $pl$, any $F, Q,M, E$ and $D$, the following hold: 
%\begin{enumerate}
%\item For any PPT adversary $\mathcal{A}_{\scriptscriptstyle 1}$ there exists a negligible function $\mu_{\scriptscriptstyle 1}(\cdot)$, such that for any security parameter $\lambda$:
% 
%{\small
%$$\Pr\left[
%  \begin{array}{l}
%  \mathcal{A}_{\scriptscriptstyle 1}(\bm{c}^{\scriptscriptstyle *},coin^{\scriptscriptstyle *}_{\scriptscriptstyle\mathcal S},coin^{\scriptscriptstyle *}_{\scriptscriptstyle\mathcal  C}, g_{\scriptscriptstyle cp}, \\ g_{\scriptscriptstyle qp}, \bm{\pi}^{\scriptscriptstyle *}, pl, a)\rightarrow\beta\\
%\end{array} \middle |
%    \begin{array}{l}
%            \mathtt{RCSP.keyGen}(1^{\lambda}, F)\rightarrow \bm{k}\\
%        \mathcal{A}_{\scriptscriptstyle 1}(1^\lambda, pk, F)\rightarrow (u_{\scriptscriptstyle 0},u_{\scriptscriptstyle 1})\\
%    \beta\stackrel{\scriptscriptstyle\$}\leftarrow\{0,1\}\\
%   \mathtt{RCSP}.\mathtt{cInit}(1^\lambda, u_{\scriptscriptstyle\beta},\bm{k}, M,z,pl,enc)\rightarrow (u^{\scriptscriptstyle *}_{\scriptscriptstyle\beta},e,T, p_{\scriptscriptstyle\mathcal S},  \bm{y}, coin^{\scriptscriptstyle*}_{\scriptscriptstyle\mathcal C})\\
%\mathtt{RCSP}.\mathtt{sInit}(u_{\scriptscriptstyle\beta}^{\scriptscriptstyle *}, e, pk, z, T, p_{\scriptscriptstyle\mathcal S},\bm{y},enc)\rightarrow (coin^{\scriptscriptstyle *}_{\scriptscriptstyle\mathcal S},a)\\
%\forall j \in [z]:\\
%\Big(\mathtt{RCSP}.\mathtt{genQuery}(1^\lambda, \text{aux},k,Q_{\scriptscriptstyle\beta}, T_{\scriptscriptstyle qp}, enc)\rightarrow c^{\scriptscriptstyle *}_{\scriptscriptstyle j}\\
%     \mathtt{RCSP}.\mathtt{prove}(u^{\scriptscriptstyle *}_{\scriptscriptstyle\beta},   \sigma,  c^{\scriptscriptstyle *}_{\scriptscriptstyle j}, pk, T_{\scriptscriptstyle qp},enc)\rightarrow (b_{\scriptscriptstyle j},m_{\scriptscriptstyle \mathcal{S},j},\pi^{\scriptscriptstyle *}_{\scriptscriptstyle j})\\  
%   \mathtt{RCSP}.\mathtt{verify}(\pi^{\scriptscriptstyle *}_{\scriptscriptstyle j}, \bm{q}_{\scriptscriptstyle j},k,T_{\scriptscriptstyle qp},enc)\rightarrow (d_{\scriptscriptstyle j},m_{\scriptscriptstyle \mathcal{C},j})\Big)\\
%\end{array}\right]\leq \frac{1}{2}+\mu_{\scriptscriptstyle 1}(\lambda).$$
%}
%
%
%
%\item For any PPT adversary $\mathcal{A}_{\scriptscriptstyle 2}$ who plays the above game, there exists a negligible function $\mu_{\scriptscriptstyle 2}(\cdot)$ such that
%
%
%{\small
%$$ Pr\left[\mathcal{A}_{\scriptscriptstyle 2}(\bm{c}^{\scriptscriptstyle *},coin^{\scriptscriptstyle *}_{\scriptscriptstyle \mathcal S},coin^{\scriptscriptstyle *}_{\scriptscriptstyle\mathcal C}, g_{\scriptscriptstyle cp},  g_{\scriptscriptstyle qp},\bm{\pi}^{\scriptscriptstyle *}, pl, a)\rightarrow (d_{\scriptscriptstyle j},j)\right]\leq \frac{1}{2}+ \mu_{\scriptscriptstyle 2}(\lambda)$$
%}
%
%where $\bm{c}^{\scriptscriptstyle *}=[{c}^{\scriptscriptstyle *}_{\scriptscriptstyle 1},...,{c}^{\scriptscriptstyle *}_{\scriptscriptstyle z}], \bm{\pi}^{\scriptscriptstyle *}=[\pi^{\scriptscriptstyle *}_{\scriptscriptstyle 1},...,\pi^{\scriptscriptstyle *}_{\scriptscriptstyle z}]$, and $ {pp}\in T_{\scriptscriptstyle qp}$.  Also, depending on the choice of $\beta$, algorithm $\mathtt{RCSP}.\mathtt{genQuery}(.)$ outputs an encoded valid query, when the input is $Q_{\scriptscriptstyle 1}$, or outputs an encoded invalid query, when the input is $Q_{\scriptscriptstyle 0}$.
%% The probability is taken over uniform  choice of $k,k'\stackrel{\scriptscriptstyle\$}\leftarrow \{0,1\}^{\scriptscriptstyle\lambda}$, where $k,k'\in\bm{k}$, the randomness of $\mathcal{A}_{\scriptscriptstyle 1}$ and $\mathcal{A}_{\scriptscriptstyle 2}$,  the randomness used in witnesses $g_{\scriptscriptstyle cp}, g_{\scriptscriptstyle qp}\in t$,  the randomness used in  $E$.
%\end{enumerate}
%\end{definition}
 
%In the above definition, the experiment is allowed to produce  invalid queries. It is required that  the privacy holds regardless of the queries status, i.e. whether they are valid or invalid, as long as the are encoded and provided.  
 
 
 
%  \begin{definition}[RC-S-P Service Proof Privacy]\label{deff::RC-S-P-ServiceProofPrivacy}  A RC-S-P   preserves the privacy of service proof (status), if for any security parameter $\lambda$, any auxiliary information $\text{aux}$, any $F,Q,M,E,D$ and any probabilistic polynomial time adversary $\mathcal{A}$ that plays the  game in Definition \ref{deff::RC-S-P-Input-Privacy}, there exists a negligible function $\mu(.)$, such that: 
%
%\small{
%$$ Pr\left[\mathcal{A}(\bm{c}^{\scriptscriptstyle *},coin^{\scriptscriptstyle *}_{\scriptscriptstyle \mathcal S},coin^{\scriptscriptstyle *}_{\scriptscriptstyle\mathcal C}, g_{\scriptscriptstyle cp},  g_{\scriptscriptstyle qp},\bm{\pi}^{\scriptscriptstyle *},a)\rightarrow (d_{\scriptscriptstyle j},j)\right]\leq \frac{1}{2}+ \mu(\lambda)$$
%}
%where $\bm{c}^{\scriptscriptstyle *}=[{c}^{\scriptscriptstyle *}_{\scriptscriptstyle 1},...,{c}^{\scriptscriptstyle *}_{\scriptscriptstyle z}]$ and $\bm{\pi}^{\scriptscriptstyle *}=[\pi^{\scriptscriptstyle *}_{\scriptscriptstyle 1},...,\pi^{\scriptscriptstyle *}_{\scriptscriptstyle z}]$.  The probability is taken over uniform  choice of $k,k'\stackrel{\scriptscriptstyle\$}\leftarrow \{0,1\}^{\scriptscriptstyle\lambda}$, where $k,k'\in\bm{k}$, the randomness of $\mathcal{A}$,  the randomness used in witnesses $g_{\scriptscriptstyle cp}, g_{\scriptscriptstyle qp}\in t$,  the randomness used in  $E$.
%\end{definition}
 
 
 
 %-----------------------------------
% \begin{definition}[RC-S-P Service Input Privacy]\label{deff::RC-S-P-Input-Privacy}  A RC-S-P   preserves the privacy of service input, if for any security parameter $\lambda$, any auxiliary information $\text{aux}$, any $F, Q,M, E,D$ and any probabilistic polynomial time adversary $\mathcal{A}$, there exists a negligible function $\mu(.)$, such that: 
%
%\small{
%$$ Pr\left[
%  \begin{array}{l}
%  \mathcal{A}(\bm{c}^{\scriptscriptstyle *},coin^{\scriptscriptstyle *}_{\scriptscriptstyle S},coin^{\scriptscriptstyle *}_{\scriptscriptstyle C},g_{\scriptscriptstyle cp},  g_{\scriptscriptstyle qp},\bm{\pi}^{\scriptscriptstyle *},a)\rightarrow\beta\\
%
%
%\end{array} \middle |
%    \begin{array}{l}
%    \mathtt{RCSP.keyGen}(1^{\lambda},F)\rightarrow \bm{k}\\
%    \mathcal{A}(1^\lambda,pk, F)\rightarrow (u_{\scriptscriptstyle 0},u_{\scriptscriptstyle 1})\\
%    \beta\stackrel{\scriptscriptstyle\$}\leftarrow\{0,1\}\\
%   
%   \mathtt{RCSP}.\mathtt{cInit}(1^\lambda, u_{\scriptscriptstyle\beta},\bm{k}, M,z,cp)\rightarrow (u_{\scriptscriptstyle\beta}^{\scriptscriptstyle *},e,t,coin^{\scriptscriptstyle*}_{\scriptscriptstyle\mathcal C}, \bm{y}_{\scriptscriptstyle \mathcal  C},\bm{y}_{\scriptscriptstyle \mathcal  S})\\
%
%\mathtt{RCSP}.\mathtt{sInit}(u_{\scriptscriptstyle\beta}^{\scriptscriptstyle *},e, pk,z, t,\bm{y}_{\scriptscriptstyle \mathcal  C},\bm{y}_{\scriptscriptstyle \mathcal  S})\rightarrow (coin^{\scriptscriptstyle *}_{\scriptscriptstyle\mathcal S},a)\\
%
%\mathtt{RCSP}.\mathtt{genQuery}(1^\lambda, \text{aux},k,Q,j, en)\rightarrow c^{\scriptscriptstyle *}_{\scriptscriptstyle j}\\
%	
%     
%     \mathtt{RCSP}.\mathtt{prove}(u^{\scriptscriptstyle *}_{\scriptscriptstyle\beta},  \sigma,  c^{\scriptscriptstyle *}_{\scriptscriptstyle j},pk, en)\rightarrow (b_{\scriptscriptstyle j},m_{\scriptscriptstyle \mathcal{S},j},\pi^{\scriptscriptstyle *}_{\scriptscriptstyle j})\\
%     
%   \mathtt{RCSP}.\mathtt{verify}(\pi^{\scriptscriptstyle *}_{\scriptscriptstyle j}, \bm{q}_{\scriptscriptstyle j},k,en)\rightarrow (d_{\scriptscriptstyle j},m_{\scriptscriptstyle \mathcal{C},j})\\
%\end{array}    \right]\leq \frac{1}{2}+\mu(\lambda)$$
%}
%where $\bm{c}^{\scriptscriptstyle *}=[{c}^{\scriptscriptstyle *}_{\scriptscriptstyle 1},...,{c}^{\scriptscriptstyle *}_{\scriptscriptstyle z}]$ and $\bm{\pi}^{\scriptscriptstyle *}=[\pi^{\scriptscriptstyle *}_{\scriptscriptstyle 1},...,\pi^{\scriptscriptstyle *}_{\scriptscriptstyle z}]$.  The probability is taken over uniform  choice of $k,k'\stackrel{\scriptscriptstyle\$}\leftarrow \{0,1\}^{\scriptscriptstyle\lambda}$, where $k,k'\in\bm{k}$, the randomness of $\mathcal{A}$,  the randomness used in witnesses $g_{\scriptscriptstyle cp},g_{\scriptscriptstyle qp}\in t$,  the randomness used in  $E$.
%
%\end{definition}
% 
% 
% 
%  \begin{definition}[RC-S-P Service Proof Privacy]\label{deff::RC-S-P-ServiceProofPrivacy}  A RC-S-P   preserves the privacy of service proof (status), if for any security parameter $\lambda$, any auxiliary information $\text{aux}$, any $F,Q,M,E,D$ and any probabilistic polynomial time adversary $\mathcal{A}$ that plays the  game in Definition \ref{deff::RC-S-P-Input-Privacy}, there exists a negligible function $\mu(.)$, such that: 
%
%\small{
%$$ Pr\left[\mathcal{A}(\bm{c}^{\scriptscriptstyle *},coin^{\scriptscriptstyle *}_{\scriptscriptstyle \mathcal S},coin^{\scriptscriptstyle *}_{\scriptscriptstyle\mathcal C}, g_{\scriptscriptstyle cp},  g_{\scriptscriptstyle qp},\bm{\pi}^{\scriptscriptstyle *},a)\rightarrow (d_{\scriptscriptstyle j},j)\right]\leq \frac{1}{2}+ \mu(\lambda)$$
%}
%where $\bm{c}^{\scriptscriptstyle *}=[{c}^{\scriptscriptstyle *}_{\scriptscriptstyle 1},...,{c}^{\scriptscriptstyle *}_{\scriptscriptstyle z}]$ and $\bm{\pi}^{\scriptscriptstyle *}=[\pi^{\scriptscriptstyle *}_{\scriptscriptstyle 1},...,\pi^{\scriptscriptstyle *}_{\scriptscriptstyle z}]$.  The probability is taken over uniform  choice of $k,k'\stackrel{\scriptscriptstyle\$}\leftarrow \{0,1\}^{\scriptscriptstyle\lambda}$, where $k,k'\in\bm{k}$, the randomness of $\mathcal{A}$,  the randomness used in witnesses $g_{\scriptscriptstyle cp}, g_{\scriptscriptstyle qp}\in t$,  the randomness used in  $E$.
%\end{definition}
 %-------------------------
 
 
 
% \begin{definition}[RC-S-P Security]\label{def::RC-S-P-Security} A RC-S-P scheme is secure if it satisfies security against malicious server, security against malicious client, and preserves privacy, w.r.t. Definitions \ref{deff::RC-S-P-SecurityAgainstMaliciousServer}-\ref{deff::RC-S-P-Privacy}. 
% \end{definition}
 

 
 

% !TEX root =R-zkCSP.tex

\subsection{Recurring Contingent PoR Payment (RC-PoR-P) Protocol}
This section presents recurring contingent PoR payment (RC-PoR-P) protocol. It is built upon PoRID protocol and is in the same security model as RC-S-P is. RC-PoR-P inherits the features of PoRID and RC-PoR-P; however, unlike RC-PoR-P, it does not use any zero-knowledge proofs and there is no third-party arbiter involved. Even though RC-PoR-P and RC-S-P have some overlaps, they have many differences too. Therefore, we provide the protocol’s overview and its detailed description below.
At a high level the protocol works as follows. The client and server utilise SAP to provably agree on two private statements, one statement includes payment details, and another one specifies a secret key, $k$, and a pad's length. Moreover, they  agree on public parameters such as the private time bubble's length (that is the total number of billing cycles: $z$, plus a waiting period, $H$) and a smart contract that specifies $z$ and the total amount of masked coins each party should deposit. They deploy the contract. Each party deposits its masked coins in the contract within a fixed time. If any party does not deposit enough coins on time, then the parties have a chance to withdraw their coins and terminate the contract after a certain time. To start using/providing the service, the client invokes $\mathtt{PoRID.setup}()$ to encode the file and generate  metadata and public parameters. It sends  encryption of the metadata and public parameters to the smart contract and sends the encoded file to the server who decrypts them and using the encoded file checks their correctness by calling $\mathtt{PoRID.server}()$. If the server decides not to serve, it sends to the contract $0$ within a fixed time; in this case, the parties can withdraw their deposit and terminate the contract. Otherwise, the server sends $1$ to the contract. 
At the end of each billing cycle, the client generates an encrypted query, by calling $\mathtt{PoRID.genQuery()}$ and encrypting its output using the key, $k$. It sends the result to the contract. In the same cycle, the server retrieves the query, and decrypts it. Then, it locally checks its correctness, by calling $\mathtt{PoRID.checkQuery()}$. If the query is rejected, the server locally stores the index of that billing cycle and generates a dummy PoR proofs. However, if the server accepts the query, it generates PoR proofs by calling $\mathtt{PoRID.prove()}$. Then, in either case, the server encrypts the proofs, pads them and sends the result to the contract. After that, the client removes the pads, decrypts the proofs and locally verifies them, by calling $\mathtt{PoRID.verify()}$. If the verification is passed, then the client knows the file is retrievable with a high probability. But, if the proof is rejected, then it locally stores the index of that billing cycle and waits until the private time bubble passes and dispute resolution time arrives. 

During the dispute resolution period, in case the server rejects the query or the client rejects the PoR proofs, that party sends to the contract (a) the indices of the billing cycles in which its counterparty provided invalid values, and (b)  the statement that contains the decryption key and padding detail. The contract checks the validity of the statement first. If it accepts the statement, then it removes the pads and decrypts the values whose indices were provided by the parties. Then, the contract checks the party’s claim by calling $\mathtt{PoRID.checkQuery ()}$ and $\mathtt{PoRID.identify ()}$ if the server or client calls the contract respectively. The contract also keeps track of the number of times each party provided invalid queries or PoR proofs.  
In the next phase, to distribute the coins, either client or server sends to the contract: (a) “pay” message, (b) the agreed statement that specifies the payment details, and (c) the statement’s proof. The contract verifies the statement and if it is approved, then the contract distributes the coins according to the statement’s detail, and the number of times each party misbehaved. 





\begin{enumerate}
\item \textbf{Key Generation}. 
\begin{enumerate}

\item  $\mathcal C$  picks a random secret key  $\bar{k}$ for a symmetric key encryption. Also, it sets  parameter $pad_{\scriptscriptstyle \pi}$ which is the number of  dummy values that will be used to pad encrypted proofs,  let $qp:(pad_{\scriptscriptstyle \pi}, \bar{k})$. The key's size is part of the security parameter. 




\item  $\mathcal C$ sets coin parameters as   follows,   $o$: the amount of coins for each accepting proof,  as well as $l$ and $l'$ which are the amount of coins $\mathcal C$ and $\mathcal S$ respectively need to send to a smart contract to resolve a potential dispute.  Let $k':\{o,l,l',qp\}$. 


\end{enumerate}

\item \textbf{Initiation}. 
\begin{enumerate}

\item For $\mathcal C$ and $\mathcal S$ to provably agree on $qp$, $\mathcal C$ sends $qp$ to  $\mathcal S$.  Next (if the $\mathcal S$ agrees on the parameters) they take the steps in the Setup and Agreement phases in the SAP, at time $T_{\scriptscriptstyle 0}$. Let $t_{\scriptscriptstyle qp}=(\ddot{x}_{\scriptscriptstyle qp},g_{\scriptscriptstyle qp})$ be proof encoding token,  where  $\ddot{x}_{\scriptscriptstyle qp}$ is the opening and $g_{\scriptscriptstyle qp}$ is the commitment stored on the contract as a result of running   SAP. 


\item Let  $cp:(o,o_{\scriptscriptstyle max},l,l',l_{\scriptscriptstyle max},z)$, where $o_{\scriptscriptstyle max}$ is the maximum amount of coins  for an accepting service proof,  $l_{\scriptscriptstyle max}$  is the maximum amount of coins to resolve a potential dispute, and $z$ is the number of  service proofs/verifications. For $\mathcal C$ and $\mathcal S$ to provably agree on $cp$, similar to the previous step, they invoke SAP, at time $T_{\scriptscriptstyle 1}$.  Let $t_{\scriptscriptstyle cp}=(\ddot{x}_{\scriptscriptstyle cp},g_{\scriptscriptstyle cp})$ be coin encoding token,  where  $\ddot{x}_{\scriptscriptstyle cp}$ is the opening and $g_{\scriptscriptstyle cp}$ is the commitment stored on the contract as a result of executing the  SAP. 


\item $\mathcal C$ sets   $coin^{\scriptscriptstyle *}_{\scriptscriptstyle\mathcal C}=z\cdot (o_{\scriptscriptstyle max}+l_{\scriptscriptstyle max})$ and $coin^{\scriptscriptstyle *}_{\scriptscriptstyle\mathcal S}=z\cdot l_{\scriptscriptstyle max}$, where $coin^{\scriptscriptstyle *}_{\scriptscriptstyle\mathcal C}$ and $coin^{\scriptscriptstyle *}_{\scriptscriptstyle\mathcal S}$ are   the total number of masked coins that $\mathcal C$ and $\mathcal S$ should deposit respectively. $\mathcal C$ signs and deploys  a smart contract, SC, that explicitly specifies  parameters $z$,  $coin^{\scriptscriptstyle *}_{\scriptscriptstyle\mathcal C}$ and $coin^{\scriptscriptstyle *}_{\scriptscriptstyle\mathcal S}$. It deposits $coin^{\scriptscriptstyle *}_{\scriptscriptstyle\mathcal C}$ coins in the contract.

\item $\mathcal C$ constructs  vector $\bm{w}_{\scriptscriptstyle\mathcal C}$, also $\mathcal S$  constructs $\bm{v}_{\scriptscriptstyle\mathcal S}$,  where the vectors are initially empty. 


\item $\mathcal C$ runs $\mathtt{PoRID.setup}(1^{\lambda}, u)\rightarrow (u^{\scriptscriptstyle *},pk)$.  It sends  encoded file $u^{\scriptscriptstyle *}$ to $\mathcal S$, and sends the public key's encryption: $\mathtt{Enc}(\bar{k},pk)$ to SC at time $T_{\scriptscriptstyle 2}$

\item $\mathcal S$ checks the above parameters, and ensures sufficient amount of coins has been deposited.  If any check is rejected, then it sets $a=0$. Otherwise, it decrypts the public key, $\mathtt{Dec}(\bar{k},\mathtt{Enc}(\bar{k},pk))=pk$. It runs $\mathtt{PoRID.serve}(u^{\scriptscriptstyle *},pk)\rightarrow a$. Next, it sends $a$ and $coin^{\scriptscriptstyle *}_{\scriptscriptstyle\mathcal S}$ coins to SC at time $T_{\scriptscriptstyle 3}$, where  $coin^{\scriptscriptstyle *}_{\scriptscriptstyle\mathcal S}=\bot$ if $a=0$



\item $\mathcal C$  and  $\mathcal S$ can withdraw their coins at time $T_{\scriptscriptstyle 4}$, if the server sends $a=0$, fewer coins than $coin^{\scriptscriptstyle *}_{\scriptscriptstyle\mathcal S}$, or nothing to the SC


%
%
%\item $\mathcal C$ constructs  vector  $ {\bm{w}}_{\scriptscriptstyle\mathcal{C}}$ and $\mathcal S$  constructs $ {\bm{v}}_{\scriptscriptstyle\mathcal S}$,  where the vectors are initially empty.   
%
%\item $\mathcal C$ and $\mathcal S$ agree on the following parameters, $z$: the number of the service proofs/verifications, $o$: the amount of coins for each accepting proof, $l$ and $l'$ which are the amount of coins $\mathcal C$ and $\mathcal S$ respectively need to send to a smart contract to resolve a dispute. Furthermore, they agree on $f_{\scriptscriptstyle\mathcal C}$ and $f_{\scriptscriptstyle\mathcal S}$ that are the amount of masking coins $\mathcal C$ and $\mathcal S$ will use respectively. 


%where $l_{\scriptscriptstyle\mathcal C}$ and $l_{\scriptscriptstyle\mathcal S}$ cover the costs of a light verification of the client's and server's inputs respectively, while $l'_{\scriptscriptstyle\mathcal S}$ covers the cost of a more involved verification of the server's input.  

%\item $\mathcal C$ and $\mathcal S$ construct  a statement of the  form: $x=(o,l,l', f_{\scriptscriptstyle\mathcal C},f_{\scriptscriptstyle\mathcal S},z)$   
%\item For $\mathcal C$ and $\mathcal S$ to provably agree on $x$, they take the steps in the Setup and Agreement phases in the SAP, at time $T_0$. Let $\mu$ be the statement's proof. 
%


%\item $\mathcal S$ picks and sends to $\mathcal C$ a random key, $\bar{k}$

%\item  $\mathcal C$ sends to $\mathcal S$ a statement of the form $x'=({pad},\bar{k})$ where  ${pad}$  specifies the number of  dummy values that will be used to pad encrypted proofs and    $\bar{k}$ is a  secret key.
%
%
%
%\item For $\mathcal C$ and $\mathcal S$ to provably agree on $x'$, they take the steps in the Setup and Agreement phases in the SAP again, at time $T_{\scriptscriptstyle 1}$. Let $\mu'$ be the proof. 



%\item For $\mathcal C$ and $\mathcal S$ to provably agree on $\bar{k}$, they take the steps in the Setup and Agreement phases in the SAP again, at time $T_{\scriptscriptstyle 1}$. Let $\mu'$ be the proof. 

%
%\item $\mathcal C$ and $\mathcal S$ agree on a smart contract: SC. Let $p_{\scriptscriptstyle\mathcal C}=z(o+l')+f_{\scriptscriptstyle\mathcal C}$ and $p_{\scriptscriptstyle\mathcal S}=zl+f_{\scriptscriptstyle\mathcal S}$ be the total number of coins the client and server should deposit respectively. SC  explicitly specifies  parameters $z$, $p_{\scriptscriptstyle\mathcal C}$ and $p_{\scriptscriptstyle\mathcal S}$. They sign and deploy the contract. 
%
%\item $\mathcal C$ and $\mathcal S$ deposit their masked coins in SC at time $T_{\scriptscriptstyle 2}$
%\item At time $T_{\scriptscriptstyle 3}$, if the deposit amount that either party has put is less than the amount stated in SC, then SC allows each party to withdraw its
%coins and terminate the contract.
%
%
%
%\item $\mathcal S$ decrypts the public key, $\mathtt{Dec}(\bar{k},\mathtt{Enc}(\bar{k},pk))=pk$. Next, it runs $\mathtt{PoRID.serve}(u^{\scriptscriptstyle *},pk)\rightarrow a$ and then sends $a$  to SC at time $T_{\scriptscriptstyle 4}$
%
%\item $\mathcal C$ and $\mathcal S$ can withdraw their coins at time $T_{\scriptscriptstyle 5}$, if the server sends $a=0$ or nothing to the SC 

\end{enumerate}
%\item \textbf{Billing-cycles}. At each time $G_{\scriptscriptstyle j}$, (where $1\leq j\leq z$ and $G_{\scriptscriptstyle 1}>T_{\scriptscriptstyle 5}$), that a proof is generated, the parties perform as follows. 

\item \textbf{Billing-cycles}. The parties do the following, at the end of  every $j$-th billing cycle, where $1\leq j\leq z$. Each $j$-th cycle includes two  time points, $G_{\scriptscriptstyle j,1}$ and $G_{\scriptscriptstyle j,2}$, where $G_{\scriptscriptstyle j,2}>G_{\scriptscriptstyle j,1}$, and $G_{\scriptscriptstyle 1,1}>T_{\scriptscriptstyle 4}$ 

%At each time $G_{\scriptscriptstyle j}$, (where $1\leq j\leq z$ and $G_{\scriptscriptstyle 1}>T_{\scriptscriptstyle 5}$), that a proof is generated, the parties perform as follows. 
\begin{enumerate}

\item $\mathcal C$ calls $\mathtt{PoRID.genQuery}(1^{\lambda},  pk)\rightarrow \hat{k}_{\scriptscriptstyle j}$, to generate a query. It sends   $\hat{k}_{\scriptscriptstyle j}$ to SC at time $G_{\scriptscriptstyle j,1}$

%\item $\mathcal S$  appends $j$ to $ {\bm{v}}_{\scriptscriptstyle\mathcal{S}}$, if the client's query was not posted on time. In this case, it does not take the next steps, for  $j$-th verification. 


%\item If $\mathcal S$ accepts the query, i.e. $b_{\scriptscriptstyle j}=1$, then it calls $\mathtt{VSID.prove}(u^{\scriptscriptstyle *},\sigma,c_{\scriptscriptstyle j},pk)\rightarrow \pi_{\scriptscriptstyle j}$, to generate the service proof. In this case, $\mathcal S$ encrypts the proof, $\mathtt{Enc}(\bar{k},\pi_{\scriptscriptstyle j})=\pi'_{\scriptscriptstyle j}$. Next, it   pads the encrypted proof with ${pad}_{\scriptscriptstyle 2}$ random values that are picked from $U$. It sends the padded encrypted proof  to SC at time $G_{\scriptscriptstyle j,2}$. Otherwise (if $\mathcal S$ rejects the query), it appends $j$  to $ {\bm{v}}_{\scriptscriptstyle\mathcal{S}}$,  constructs a dummy proof $\pi'_{\scriptscriptstyle j}\in U$,  pads the result as above, and sends the padded dummy proof to SC at time $G_{\scriptscriptstyle j,2}$.


%\item $\mathcal S$ decrypts the query,  $\mathtt{Dec(}\bar{k},\mathtt{Enc}(\bar{k},\hat{k}_{\scriptscriptstyle j}))=\hat{k}_{\scriptscriptstyle j}$.   Then, it calls $\mathtt{PoRID.checkQuery}(\hat{k}_{\scriptscriptstyle j}, pk)\rightarrow b_{\scriptscriptstyle j}$ to check the query's correctness. 


\item $\mathcal S$ calls $\mathtt{PoRID.checkQuery}(\hat{k}_{\scriptscriptstyle j}, pk)\rightarrow b_{\scriptscriptstyle j}$ to check the query's correctness. 

\begin{itemize}
\item[$\bullet$] If it accepts the query, then it calls $\mathtt{PoRID.prove}(u^{\scriptscriptstyle *},\hat{k}_{\scriptscriptstyle j},pk)\rightarrow  {\bm{\pi}}_{\scriptscriptstyle j}$, to generate a PoR  proof. In this case,  $\mathcal S$ encrypts every  proof in the proof vector, i.e. $\forall g, 1\leq g\leq| {\bm{\pi}}_{\scriptscriptstyle j}|: \mathtt{Enc}(\bar{k}, {\bm{\pi}}_{\scriptscriptstyle j}[g])= {\bm{\pi}}'_{\scriptscriptstyle j}[g]$. Let vector $ {\bm{\pi}}'_{\scriptscriptstyle j}$ contain the encryption of all proofs. It pads every encrypted proof in $ {\bm{\pi}}'_{\scriptscriptstyle j}$ with $pad_{\scriptscriptstyle \pi}$ random values that are picked from the encryption's output range $U$, (by appending the random values to the encrypted proofs vector). It sends the padded encrypted proofs  to SC at time $G_{\scriptscriptstyle j,2}$ 
\item[$\bullet$] Otherwise (if $\mathcal S$ rejects the query), it appends $j$  to $ {\bm{v}}_{\scriptscriptstyle\mathcal{S}}$,  constructs a dummy proof $ {\bm{\pi}}'_{\scriptscriptstyle j}$ whose elements are randomly picked from $U$,  pads the result as above, and sends the padded dummy proof to SC at time $G_{\scriptscriptstyle j,2}$

\end{itemize}

When $j=z$  and $\bm{v}_{\scriptscriptstyle\mathcal{S}}\neq\bot$, $\mathcal S$ sets    $\bm{m}_{\scriptscriptstyle\mathcal S}:[\bm{v}_{\scriptscriptstyle\mathcal{S}},\ddot{x}_{\scriptscriptstyle qp}, \text{``dispute''}]$.




%\item  $\mathcal C$ appends $j$ to $ {\bm{v}}_{\scriptscriptstyle\mathcal{C}}$, if  $\mathcal S$ does not post the proof on time. In this case, it does not take the next steps, for  $j$-th verification. 


\item In this phase,  $\mathcal C$ verifies the service proof. It first constructs two empty vectors, $\bm{m}_{\scriptscriptstyle\mathcal C}=\bot$ and $\bm{w}_{\scriptscriptstyle\mathcal{C}}=\bot$. Next, it removes the pads from the padded encrypted proofs and then
 decrypts the encrypted proofs: $\mathtt{Dec}(\bar{k}, {\bm{\pi}}'_{\scriptscriptstyle j})= {\bm{\pi}}_{\scriptscriptstyle j}$. Then, it   calls $\mathtt{PoRID.verify}( {\bm{\pi}}_{\scriptscriptstyle j},\hat{k}_{\scriptscriptstyle j},pk)\rightarrow  {\bm{d}}_{\scriptscriptstyle j}$, to verify them. If $ {\bm{\pi}}_{\scriptscriptstyle j}$ passes the verification, i.e. $ {\bm{d}}_{\scriptscriptstyle j}[0]=1$, then $\mathcal C$  concludes that the service for this verification has been delivered successfully. Otherwise (if proof $ {\bm{\pi}}_{\scriptscriptstyle j}$ is rejected, i.e. $ {\bm{d}}_{\scriptscriptstyle j}[0]=0$), then it sets $g= {\bm{d}}_{\scriptscriptstyle j}[1]$ and appends vector $[j,g]$  to $ {\bm{w}}_{\scriptscriptstyle\mathcal{C}}$. Recall, $ {\bm{d}}_{\scriptscriptstyle j}[1]$ refers to a rejected proof's index in proof vector $ {\bm{\pi}}_{\scriptscriptstyle j}$. If $j=z$  and $\bm{w}_{\scriptscriptstyle\mathcal{C}}\neq\bot$, $\mathcal C$ sets    $\bm{m}_{\scriptscriptstyle\mathcal C}:[\bm{w}_{\scriptscriptstyle\mathcal{C}},\ddot{x}_{\scriptscriptstyle qp}, \text{``dispute''}]$.
  

\end{enumerate}
\item \textbf{Dispute Resolution}. The phase takes place only in  case of dispute, i.e. when $\mathcal C$ rejects service proofs or $\mathcal S$ rejects the queries. %or either party does not receive its counterparty's message on time. %A party who accepts all proofs does not need to get involved in this phase. 

\begin{enumerate}

%\item $\mathcal C$  at time $K_{\scriptscriptstyle 1}>G_{\scriptscriptstyle z,2}+H$ sends ``dispute'' message to SC 

%\item $\mathcal S$ sends $ {\bm{v}}_{\scriptscriptstyle\mathcal{S}}$ to SC, at time $K_{\scriptscriptstyle 2}$ 


%\item SC  ensures $ {\bm{v}}_{\scriptscriptstyle\mathcal{S}}$ is well-formed. In particular, it removes from  $ {\bm{v}}_{\scriptscriptstyle\mathcal{S}}$ duplicated elements or  elements that are not in the range $[1,z]$. SC for every  element $i\in  {\bm{v}}_{\scriptscriptstyle\mathcal{S}}$ checks if $\mathcal C$  sent to SC an encrypted query on time, in step \ref{Billing-cycles-genQuery};  if the client did not do so, then it appends $i$ to $ {\bm{v}}_{\scriptscriptstyle{SC}}$ and   increments $u_{\scriptscriptstyle\mathcal C}$ and $y_{\scriptscriptstyle\mathcal C}$ by $1$
 
 
 % and retakes step \ref{the-arbiter-loop} with the next element of $ {\bm{v}}$. Otherwise, it proceeds to the next step. 

\item $\mathcal S$  sends $\bm{m}_{\scriptscriptstyle\mathcal S}$ to SC, at time $K_{\scriptscriptstyle 1}$, where $K_{\scriptscriptstyle 1}>G_{\scriptscriptstyle z,2}+H$

\item SC upon receiving $\bm{m}_{\scriptscriptstyle\mathcal S}$ does the following. 
\begin{enumerate}

\item Sets two  counters: $y_{\scriptscriptstyle\mathcal C}$ and $y_{\scriptscriptstyle\mathcal S}$  that are initially set to $0$. Also, it constructs an empty vector $ {\bm{v}}$ 
\item Checks the validity of  statement $\ddot{x}_{\scriptscriptstyle qp}\in \bm{m}_{\scriptscriptstyle\mathcal S}$, by  sending it to SAP contract which  returns   $1$ or $0$. If the output is $0$, then SC discards the server's complaint, $\bm{m}_{\scriptscriptstyle\mathcal S}$, and does not take steps \ref{sc-dec-server-key}-\ref{sc-check-server-claim}. Otherwise, it proceeds to the next step.

\item\label{sc-dec-server-key} Uses secret key $\bar{k}\in \ddot{x}_{\scriptscriptstyle qp}$ to decrypt the encrypted public key, $\mathtt{Dec}(\bar{k},\mathtt{Enc}(\bar{k},pk))=pk$

\item Removes from $ {\bm{v}}_{\scriptscriptstyle\mathcal{S}}$ any element that is duplicated or is not in the range $[1,z]$ 

\item\label{sc-check-server-claim}  For any element $i\in {\bm{v}}_{\scriptscriptstyle\mathcal{S}}$:
\begin{itemize}
\item[$\bullet$] Fetches the related query, $\hat{k}_{\scriptscriptstyle i}$,   from SC 
\item[$\bullet$]Checks if  the query is well-formed, by calling $\mathtt{PoRID.checkQuery}(\hat{k}_{\scriptscriptstyle i}, pk)\rightarrow b_{\scriptscriptstyle i}$. If the query is rejected, i.e. $ b_{\scriptscriptstyle i}=0$, then it increments $y_{\scriptscriptstyle\mathcal C}$ by 1 and appends $i$ to $ {\bm{v}}$
\end{itemize}
Let $K_{\scriptscriptstyle 2}$ be the time SC finishes the above checks.
\end{enumerate}





\item $\mathcal C$  sends $\bm{m}_{\scriptscriptstyle\mathcal C}$ to SC, at time $K_{\scriptscriptstyle 3}$
 %Also,  $\mathcal S$ sends $m_{\scriptscriptstyle\mathcal S}:( {\bm{v}}_{\scriptscriptstyle\mathcal{S}},\bar{k},\mu')$ to SC %,   where $| {\bm{v}}_{\scriptscriptstyle\mathcal{C}}|\neq 0$ and $| {\bm{v}}_{\scriptscriptstyle\mathcal{S}}|\neq 0$. 

%\item\label{check-client-claim} SC  ensures $ {\bm{v}}_{\scriptscriptstyle\mathcal{C}}$ is well-formed. Specifically, it  removes from  $ {\bm{v}}_{\scriptscriptstyle\mathcal{C}}$ any element that is duplicated, or   not in the range $[1,z]$.  Also, it removes from $ {\bm{v}}_{\scriptscriptstyle\mathcal{C}}$ the elements that are in $ {\bm{v}}_{\scriptscriptstyle{SC}}$. Next, SC  for every  element $i\in  {\bm{v}}_{\scriptscriptstyle\mathcal{C}}$ checks if $\mathcal S$  sent to SC an encrypted query on time, in step \ref{Billing-cycles-genQuery} and increments $u_{\scriptscriptstyle\mathcal S}$ and $y_{\scriptscriptstyle\mathcal S}$ by $1$  if the server did not do that.  It appends those $i$'s that pass the check to $ {\bm{v}}'_{\scriptscriptstyle{SC}}$. Note, $ {\bm{v}}'_{\scriptscriptstyle{SC}}$ contains the indices of verifications whose PoR proofs  were rejected by the client.


\item SC upon receiving $\bm{m}_{\scriptscriptstyle\mathcal C}$, does the following. 
\begin{enumerate}
\item Checks the validity of  statement $\ddot{x}_{\scriptscriptstyle qp}\in \bm{m}_{\scriptscriptstyle\mathcal C}$.  To do that, it sends $\ddot{x}_{\scriptscriptstyle qp}$ to SAP contract which  returns  either $1$ or $0$. If the output is $0$, then SC discards the client's complaint, $\bm{m}_{\scriptscriptstyle\mathcal C}$, and does not take steps \ref{sc-checks-client-claim}-\ref{the-contract-loop}. Otherwise, it proceeds to the next step. 

 \item\label{sc-checks-client-claim} Ensures each vector   $ {\bm{w}}\in  {\bm{w}}_{\scriptscriptstyle\mathcal{C}}$ is well-formed. In particular, it verifies there exist no two vectors: $ {\bm{w}}, {\bm{w}}'\in  {\bm{w}}_{\scriptscriptstyle\mathcal{C}}$ such that $ {\bm{w}}[0]= {\bm{w}}'[0]$.  If such vectors  exist, it deletes the redundant ones from $ {\bm{w}}_{\scriptscriptstyle\mathcal{C}}$. This ensures no two claims refer to the same verification.  Also, it removes any vector $ {\bm{w}}$ from $ {\bm{w}}_{\scriptscriptstyle\mathcal{C}}$ if $ {\bm{w}}[0]$ is not in the range $[1,z]$ or if $ {\bm{w}}[0]\in  {\bm{v}}$. Note the latter check  (i.e. $ {\bm{w}}[0]\in  {\bm{v}}$) ensures $\mathcal C$ cannot hold $\mathcal S$ accountable if $\mathcal C$ has generated an ill-formed query for the same verification.   

\item Uses secret key $\bar{k}\in \ddot{x}_{\scriptscriptstyle qp}$ to decrypt the encrypted public key, $\mathtt{Dec}(\bar{k},\mathtt{Enc}(\bar{k},pk))=pk$

\item\label{the-contract-loop} For every  vector $ {\bm{w}}\in {\bm{w}}_{\scriptscriptstyle{C}}$:
\begin{itemize}
\item[$\bullet$]  Retrieves details of a proof that was rejected in each $i$-th verification.  In particular, it sets   $i= {\bm{w}}[0]$ and $g= {\bm{w}}[1]$. Recall that $g$ refers to the index of a rejected proof in the proof vector which was generated for $i$-th verification, i.e. $ {\bm{\pi}}_{\scriptscriptstyle i}$ 

\item[$\bullet$]  Fetches the related  query, $\hat{k}_{\scriptscriptstyle i}$, from SC.

\item[$\bullet$]  Removes the pads only from $g$-th padded encrypted proof. Let $ {\bm{\pi}}'_{\scriptscriptstyle i}[g]$ be the result. Next, it decrypts the  encrypted proof, $\mathtt{Dec}(\bar{k}, {\bm{\pi}}'_{\scriptscriptstyle i}[g]))= {\bm{\pi}}_{\scriptscriptstyle i}[g]$

\item[$\bullet$] \label{dummy-vector} Constructs a fresh vector: $ {\bm{\pi}}''_{\scriptscriptstyle i}$, such that  its $g$-th element equals $ {\bm{\pi}}_{\scriptscriptstyle i}[g]$ (i.e. $ {\bm{\pi}}''_{\scriptscriptstyle i}[g]=  {\bm{\pi}}_{\scriptscriptstyle i}[g] \text{ and } | {\bm{\pi}}''_{\scriptscriptstyle i}|= | {\bm{\pi}}_{\scriptscriptstyle i}|$) and the rest of its elements are dummy values. 

\item[$\bullet$] Calls   $\mathtt{PoRID.identify}( {\bm{\pi}}''_{\scriptscriptstyle i},g,\hat{k}_{\scriptscriptstyle i},pk)\rightarrow I_{\scriptscriptstyle i}$. If $I_{\scriptscriptstyle i}=\mathcal S$, then it increments $y_{\scriptscriptstyle\mathcal S}$  by $1$. Otherwise, it does nothing. 

\end{itemize}
\end{enumerate}
Let $K_{\scriptscriptstyle 4}$ be the time that SC finishes all the above checks.% in step \ref{the-contract-loop}

\end{enumerate} %%%%%%%%%

\item \textbf{Coin Transfer}.
\begin{enumerate}
\item Either $\mathcal C$ or $\mathcal S$ sends ``pay'' message and the statement: $\ddot{x}_{\scriptscriptstyle cp}$ to SC at time $L>K_{\scriptscriptstyle 4}$
\item SC checks the validity of the statement by sending it  to SAP contract that  returns  either $1$ or $0$. SC only proceeds to the next step if the output is $1$
\item SC distributes the coins to the parties as follows:

\begin{itemize}
\item[$\bullet$]   $coin^{\scriptscriptstyle *}_{\scriptscriptstyle\mathcal C}-o(z-y_{\scriptscriptstyle\mathcal S})+l\cdot y_{\scriptscriptstyle\mathcal S}-l'\cdot y_{\scriptscriptstyle\mathcal C}$ coins  to $\mathcal C$

\item[$\bullet$] $coin^{\scriptscriptstyle *}_{\scriptscriptstyle\mathcal S}+o(z-y_{\scriptscriptstyle\mathcal S})-l\cdot y_{\scriptscriptstyle\mathcal S}+l'\cdot y_{\scriptscriptstyle\mathcal C}$ coins to $\mathcal S$
%\item[$\bullet$] $l(u_{\scriptscriptstyle\mathcal S}+u_{\scriptscriptstyle\mathcal C}+y_{\scriptscriptstyle\mathcal S}+y_{\scriptscriptstyle\mathcal C})$ coins to the arbiter.  
\end{itemize}

\end{enumerate}
\end{enumerate}


\begin{remark}
The reason in step \ref{dummy-vector} vector $ {\bm{\pi}}''_{\scriptscriptstyle i}$ is constructed  is to let SC make \emph{black-box}  use of $\mathtt{PoRID.identify}()$. Alternatively, SC could decrypt all proofs in $\mathtt{Enc}(\bar{k}, {\bm{\pi}}_{\scriptscriptstyle i})$ and pass them to $\mathtt{PoRID.identify}()$. However, this approach would impose a high cost, as all proofs have to be decrypted. 
\end{remark}

\begin{remark}
In general, a transaction that is sent   to a smart contract should  cover the cost of the contract's execution. Therefore, in the above protocol, if a party unnecessarily invokes a contract for an accepting proof, it has to pay the execution cost in advance. This is the reason the above protocol (unlike RC-S-P protocol) does not need to track the number of times a party unnecessarily invokes the contract.    
\end{remark}







%% !TEX root =R-zkCSP.tex

\section{Recurring Contingent PoR Payment Definition}\label{sec::RC-PoR-P-Definition}


\begin{definition}[RC-PoR-P Scheme]\label{RC-PoR-P-def}
A recurring contingent service payment scheme RC-S-P  $=(\mathtt{RCSP}.\mathtt{keyGen}, $ $\mathtt{RCSP}.\mathtt{cInit}, $ $\mathtt{RCSP}.\mathtt{sInit},  $ $\mathtt{RCSP}.\mathtt{genQuery},\mathtt{RCSP}.\mathtt{prove},\mathtt{RCSP}.\mathtt{verify},\mathtt{RCSP}.\mathtt{resolve},\mathtt{RCSP}.\mathtt{pay})$ involves four parties; namely, client, server, arbiter and smart contract, and consists of eight algorithms defined as follows.


\begin{itemize}

\item[$\bullet$] $\mathtt{RCPoRP}.\mathtt{keyGen}(1^{\scriptscriptstyle\lambda})\rightarrow k$.  A probabilistic algorithm run by the client. It takes as input security parameter $1^\lambda$. It outputs $\bm{k}$ that contains a pair of secret and public parameters, $k:=(sk',pk')$. It sends  $pk'$ to the smart contract.

\

\item[$\bullet$] $\mathtt{RCPoRP}.\mathtt{cInit}(1^\lambda, u, k, z, pl,enc)\rightarrow (u^{\scriptscriptstyle *},T, p_{\scriptscriptstyle\mathcal S},  \bm{y}, coin^{\scriptscriptstyle*}_{\scriptscriptstyle\mathcal C})$. It is run by the client. It takes as input $1^\lambda$,  the service input $u$, key pair  $k:=(sk', pk')$,   the total number of verifications $z$, and  price list $pl$ containing pairs of  actual coin amount for each accepting service proof and the amount for covering each potential dispute resolution's cost.  It also takes as input encoding/decoding functions $enc:=(E,D)$ used to encode/decode the service queries/proofs.  It encodes $u$, that yields  $u^{\scriptscriptstyle *}$. {It  sets $pp$ as (possibly) input dependent parameters, e.g. file size.} It computes $pp$ that contains a set of input dependent parameters $pp'$ and a  representation of $\sigma=M(u^{\scriptscriptstyle *},{pp'})$. It sets value of $p_{\scriptscriptstyle\mathcal S}$ to the total coins the server should deposit. It picks a private price pair $(o,l)\in pl$. It sets coin secret parameters $cp$ that include $(o,l)$ and parameters of $pl$, e.g. its maximum values. It constructs coin encoding token $T_{\scriptscriptstyle cp}$ containing   $cp$ and   $cp$'s witness, $g_{\scriptscriptstyle cp}$. It   constructs  encoding token $T_{\scriptscriptstyle qp}$ that contains secret  parameters $qp$ including {$pp$} and those (in $sk'$) that will be used to encode the service queries/proofs. Also  $T_{\scriptscriptstyle qp}$ contains     $qp$'s witness,  $g_{\scriptscriptstyle qp}$.   Given a valid value and its witness  anyone can check if they match. It sets a vector of parameters $\bm{y}$ that includes  binary vectors $[\bm{y}_{\scriptscriptstyle \mathcal  C}, \bm{y}_{\scriptscriptstyle \mathcal  S},\bm{y}'_{\scriptscriptstyle \mathcal  C}, \bm{y}'_{\scriptscriptstyle \mathcal  S}]$ each of which is set to $0$ and its length is $z$.  It outputs  $u^{\scriptscriptstyle *}$,   $T:=(T_{\scriptscriptstyle cp},T_{\scriptscriptstyle qp})$,  $p_{\scriptscriptstyle\mathcal S}$,    $\bm{y}$,  and the encoded coins amount  $coin^{\scriptscriptstyle*}_{\scriptscriptstyle\mathcal C}$ (that contains $o_{\scriptscriptstyle i}$ and $l_{\scriptscriptstyle i}$ coins in an encoded form). The client sends  $u^{\scriptscriptstyle *}$, $z$, $T_{\scriptscriptstyle cp} \setminus  \{g_{\scriptscriptstyle cp}\}$ and $T_{\scriptscriptstyle qp} \setminus  \{g_{\scriptscriptstyle qp}\}$ to the server and  sends $g_{\scriptscriptstyle cp},  g_{\scriptscriptstyle qp},p_{\scriptscriptstyle\mathcal S}$, and $\bm{y}$,  and $coin^{\scriptscriptstyle*}_{\scriptscriptstyle\mathcal C}$  coins to the  contract. 





%\item[$\bullet$] $\mathtt{RCSP}.\mathtt{cInit}(1^\lambda, u,\bm{k}, M,z,pl,enc)\rightarrow (u^{\scriptscriptstyle *},e,T, p_{\scriptscriptstyle\mathcal S},  \bm{y}, coin^{\scriptscriptstyle*}_{\scriptscriptstyle\mathcal C},pk^{\scriptscriptstyle *})$. It is run by the client. It takes as input $1^\lambda$,  the service input $u$,  the keys  $\bm{k}:=(k, K')$,  the metadata generator function $M$, the total number of verifications $z$, and  price list $pl:\{(o_{\scriptscriptstyle 1},l_{\scriptscriptstyle 1}),...,(o_{\scriptscriptstyle m},l_{\scriptscriptstyle m})\}$, where  each pair contains actual coin amounts for each accepting service proof $o_{\scriptscriptstyle i}$, and for covering each potential dispute resolution's cost, $l_{\scriptscriptstyle i}$.  The algorithm also takes as input encoding/decoding functions $enc:=(E,D)$ used to encode/decode the service public-key/queries/proofs.  It encodes $u$, that results in $u^{\scriptscriptstyle *}$.  It computes metadata $\sigma=M(u^{\scriptscriptstyle *},k)$ and a  proof $w_{\scriptscriptstyle\sigma}$ asserting the metadata is well-structured. It sets value of $p_{\scriptscriptstyle\mathcal S}$ to the total coins the server should deposit. It picks a private price pair $(o_{\scriptscriptstyle i},l_{\scriptscriptstyle i})\in pl$, e.g. depending on the service type or input size. It sets coin secret parameters $cp$ that include $(o_{\scriptscriptstyle i},l_{\scriptscriptstyle i})$ and parameters of $pl$, e.g. its maximum values. It constructs coin encoding token $T_{\scriptscriptstyle cp}$ containing   $cp$ and   $cp$'s witness, $g_{\scriptscriptstyle cp}$.  It   constructs  encoding token $T_{\scriptscriptstyle qp}$ comprising secret  parameters $qp\in K'$ (used to encode the public key and service queries/proofs)  and     $qp$'s witness,  $g_{\scriptscriptstyle qp}$.   Given a valid value and its witness  anyone can check if they match. It sets a vector of parameters $\bm{y}$ that includes  binary vectors $[\bm{y}_{\scriptscriptstyle \mathcal  C}, \bm{y}_{\scriptscriptstyle \mathcal  S},\bm{y}'_{\scriptscriptstyle \mathcal  C}, \bm{y}'_{\scriptscriptstyle \mathcal  S}]$ each of which is set to $0$ and its length is $z$.  It may add other  parameters, e.g. input size, to $pk$. It encodes  $pk^{\scriptscriptstyle *}=E(pk,T_{\scriptscriptstyle qp})$. It outputs encoded input $u^{\scriptscriptstyle *}$, metadata-proof pair $e:=(\sigma,w_{\scriptscriptstyle\sigma})$, the two sets $T:=(T_{\scriptscriptstyle cp},T_{\scriptscriptstyle qp})$, the total coins amount the server should deposit $p_{\scriptscriptstyle\mathcal S}$, the  binary vectors $\bm{y}$,   the encoded coins amount  $coin^{\scriptscriptstyle*}_{\scriptscriptstyle\mathcal C}$ (that contains $o_{\scriptscriptstyle i}$ and $l_{\scriptscriptstyle i}$ coins in an encoded form) and the encoded public key $pk^{\scriptscriptstyle *}$. The client sends  $u^{\scriptscriptstyle *}$, $z$,  $e$, $T_{\scriptscriptstyle cp} \setminus  \{g_{\scriptscriptstyle cp}\}$ and $T_{\scriptscriptstyle qp} \setminus  \{g_{\scriptscriptstyle qp}\}$ to the server and  sends $g_{\scriptscriptstyle cp},  g_{\scriptscriptstyle qp},p_{\scriptscriptstyle\mathcal S}$, $\bm{y}$,   $coin^{\scriptscriptstyle*}_{\scriptscriptstyle\mathcal C}$  coins, and $pk^{\scriptscriptstyle *}$ to the smart contract. 



%-------------Commented out on 26-11-20
%\item[$\bullet$] $\mathtt{RCSP}.\mathtt{cInit}(1^\lambda, u,\bm{k}, M,z,cp)\rightarrow (u^{\scriptscriptstyle *},e,t,coin^{\scriptscriptstyle*}_{\scriptscriptstyle\mathcal C}, \bm{y}_{\scriptscriptstyle \mathcal  C},\bm{y}_{\scriptscriptstyle \mathcal  S})$. It is run by the client. It takes as input $1^\lambda$,  the service input $u$,  the keys  $\bm{k}:=(k, K')$,  the metadata generator function $M$, the total number of verifications $z$, and coin secret parameters $cp$ that include a subset of  $K'$ and the actual amount of coins for each accepting service proof $o$ and  for covering each potential dispute resolution's cost $l$.  It encodes $u$, that results in $u^{\scriptscriptstyle *}$.  It computes metadata $\sigma=M(u^{\scriptscriptstyle *},k)$ and a  proof $w_{\scriptscriptstyle\sigma}$ asserting the metadata is well-structured.  It constructs coin encoding token $T_{\scriptscriptstyle cp}$ that contains   $cp$, total coins the server should deposit $p_{\scriptscriptstyle\mathcal S}$, and   $cp$'s witness, $g_{\scriptscriptstyle cp}$.  It also constructs query/proof encoding token $T_{\scriptscriptstyle qp}$ that contains  secret  parameters $qp\in K'$ (used to encode the service queries/proofs)  and     $qp$'s witness,  $g_{\scriptscriptstyle qp}$.   Given a valid value and its witness  anyone can check if they match. It constructs two binary vectors $\bm{y}_{\scriptscriptstyle \mathcal  C}$ and $\bm{y}_{\scriptscriptstyle \mathcal  S}$ that are set to $0$ and their length is $z$.  It outputs encoded input $u^{\scriptscriptstyle *}$, metadata-proof pair $e:=(\sigma,w_{\scriptscriptstyle\sigma})$, the two sets $T:=(T_{\scriptscriptstyle cp},T_{\scriptscriptstyle qp})$, the two binary vectors $\bm{y}_{\scriptscriptstyle \mathcal  C}, \bm{y}_{\scriptscriptstyle \mathcal  S}$, and  the encoded coins amount  $coin^{\scriptscriptstyle*}_{\scriptscriptstyle\mathcal C}$ (i.e. contains $o$ and $l$ coins in an encoded form). The client sends  $u^{\scriptscriptstyle *}$, $z$, $pk$, $e$, $T_{\scriptscriptstyle cp} \setminus  \{g_{\scriptscriptstyle cp},p_{\scriptscriptstyle\mathcal S}\}$ and $T_{\scriptscriptstyle qp} \setminus  \{g_{\scriptscriptstyle qp}\}$ to the server and  sends $\{g_{\scriptscriptstyle cp},  g_{\scriptscriptstyle qp},p_{\scriptscriptstyle\mathcal S}\}$, $\bm{y}_{\scriptscriptstyle \mathcal  C}$, $\bm{y}_{\scriptscriptstyle \mathcal  S}$  and $coin^{\scriptscriptstyle*}_{\scriptscriptstyle\mathcal C}$ coins to the smart contract. 
%--------------------------

%\item[$\bullet$] $\mathtt{RCSP}.\mathtt{cInit}(1^\lambda, u,\bm{k}, M,z,o,l,E_{\scriptscriptstyle 1})\rightarrow (u^{\scriptscriptstyle *},e,t,coin^{\scriptscriptstyle*}_{\scriptscriptstyle\mathcal C})$. It is run by the client. It takes as input $1^\lambda$,  the service input: $u$,  the key pair: $(k, k')$,  metadata generator function: $M$, total number of verifications: $z$,  the amount of coins for each accepting service proof: $o$,  coins amount  to cover each potential dispute resolution's cost: $l$ and  encoding function $E_{\scriptscriptstyle 1}$ to encode the coins actual amount.  It encodes $u$, that results $u^{\scriptscriptstyle *}$.  It computes a metadata, $\sigma=M(u^{\scriptscriptstyle *},k)$, and a publicly verifiable zero-knowledge proof $w_{\scriptscriptstyle\sigma}$ asserting the metadata is well-structured.  It constructs coins' token $t_{\scriptscriptstyle cp}$ that contains  coin's secret  parameters $cp$ and   sealed    parameters $sld_{\scriptscriptstyle cp}$. Given $cp$ and $sld_{\scriptscriptstyle cp}$ anyone can check if $sld_{\scriptscriptstyle cp}$ corresponds to $cp$. It constructs  encoding token $t_{\scriptscriptstyle pq}$ that contains  secret  parameters $qp\in k'$ (used to encode the service queries/proofs) and  sealed   parameters $sld_{\scriptscriptstyle qp}$, where  given $qp$ and $sld_{\scriptscriptstyle qp}$ anyone can check if $sld_{\scriptscriptstyle qp}$ corresponds to $qp$.  It outputs encoded input $u^{\scriptscriptstyle *}$, metadata-proof pair $e:(\sigma,w_{\scriptscriptstyle\sigma})$, the two sets $t:\{t_{\scriptscriptstyle cp},t_{\scriptscriptstyle qp}\}$ and  the encoded coins amount  $coin^{\scriptscriptstyle*}_{\scriptscriptstyle\mathcal C}=E_{\scriptscriptstyle 1}(z,cp)$. The server is  given $u^{\scriptscriptstyle *}$,  $e$, and $t \setminus  \{sld_{\scriptscriptstyle cp},sld_{\scriptscriptstyle qp}\}$ and  the smart contract  is given $sld_{\scriptscriptstyle cp}$,  $sld_{\scriptscriptstyle qp}$ and $coin^{\scriptscriptstyle*}_{\scriptscriptstyle\mathcal C}$ coins by the client. 


%\item[$\bullet$] $\mathtt{VS.setup}(1^{\lambda}, u,k, M)\rightarrow \sigma$. A probabilistic algorithm run by the client. It takes as input security parameter,  the service input: $u$,  the key pair: $k$ and metadata generator deterministic function: $M$, publicly known. It runs $M(u,k)$ and outputs  metadata: $\sigma$. Right after that, the server might be given $u$,  $\sigma$ and $pk$. 

\


\item[$\bullet$] $\mathtt{RCPoRP}.\mathtt{sInit}(u^{\scriptscriptstyle *},z, T, p_{\scriptscriptstyle\mathcal S},\bm{y},enc)\rightarrow (coin^{\scriptscriptstyle *}_{\scriptscriptstyle\mathcal S},a)$. It is  run by the server. It takes as input the  service encoded input $u^{\scriptscriptstyle *}$, metadata-proof pair $e:=(\sigma,w_{\scriptscriptstyle\sigma})$,   public key $pk$ (read from the contract), the total number of verifications $z$, and $T:=(T_{\scriptscriptstyle cp},T_{\scriptscriptstyle qp})$ (where $\{g_{\scriptscriptstyle cp},  g_{\scriptscriptstyle qp}\}$ are read from the smart contract). Also, it reads $p_{\scriptscriptstyle\mathcal S}$, and $\bm{y}$ from the smart contract and takes as input the encoding/decoding functions $enc:=(E,D)$. It verifies the validity of $e$ and  $T$ elements. Also, it checks elements of $\bm{y}$ and ensures element of $\bm{y}_{\scriptscriptstyle \mathcal  C}, \bm{y}_{\scriptscriptstyle \mathcal  S},\bm{y}'_{\scriptscriptstyle \mathcal  C}, \bm{y}'_{\scriptscriptstyle \mathcal  S}\in \bm{y}$ have been set to $0$. If all checks are successful, then it encodes the amount of its coins  $coin^{\scriptscriptstyle*}_{\scriptscriptstyle\mathcal S}$ and sets $a=1$. Otherwise, it sets $coin^{\scriptscriptstyle *}_{\scriptscriptstyle\mathcal S}=\bot$ and $a=0$. It outputs $coin^{\scriptscriptstyle *}_{\scriptscriptstyle\mathcal S}$ and $a$. The smart contract is given $coin^{\scriptscriptstyle *}_{\scriptscriptstyle\mathcal S}$ coins and $a$.  



%\item[$\bullet$] $\mathtt{RCSP}.\mathtt{sInit}(u^{\scriptscriptstyle *},e, pk^{\scriptscriptstyle *},z, T, p_{\scriptscriptstyle\mathcal S},\bm{y},enc)\rightarrow (coin^{\scriptscriptstyle *}_{\scriptscriptstyle\mathcal S},a)$. It is  run by the server. It takes as input the  service encoded input $u^{\scriptscriptstyle *}$, metadata-proof pair $e:=(\sigma,w_{\scriptscriptstyle\sigma})$,  encoded public key $pk^{\scriptscriptstyle *}$ (read from the contract), the total number of verifications $z$, and $T:=(T_{\scriptscriptstyle cp},T_{\scriptscriptstyle qp})$ (where $\{g_{\scriptscriptstyle cp},  g_{\scriptscriptstyle qp}\}$ are read from the smart contract). Also, it reads $p_{\scriptscriptstyle\mathcal S}$, and the parameters in $\bm{y}$ from the smart contract and takes as input the encoding/decoding functions $enc:=(E,D)$. It decodes the public key, $pk=D(pk^{\scriptscriptstyle*}, T_{\scriptscriptstyle qp})$. Then, it verifies the validity of the elements in $e$ and  $T$, with the help of $pk$. Also, it checks elements of $\bm{y}$ and ensures element of $\bm{y}_{\scriptscriptstyle \mathcal  C}, \bm{y}_{\scriptscriptstyle \mathcal  S},\bm{y}'_{\scriptscriptstyle \mathcal  C}, \bm{y}'_{\scriptscriptstyle \mathcal  S}\in \bm{y}$ have been set to $0$. If all checks are successful, then it encodes the amount of its coins  $coin^{\scriptscriptstyle*}_{\scriptscriptstyle\mathcal S}$ and sets $a=1$. Otherwise, it sets $coin^{\scriptscriptstyle *}_{\scriptscriptstyle\mathcal S}=\bot$ and $a=0$. It outputs $coin^{\scriptscriptstyle *}_{\scriptscriptstyle\mathcal S}$ and $a$. The smart contract is given $coin^{\scriptscriptstyle *}_{\scriptscriptstyle\mathcal S}$ coins and $a$.  

\


\item[$\bullet$] $\mathtt{RCPoRP}.\mathtt{genQuery}(1^\lambda, T_{\scriptscriptstyle qp})\rightarrow q_{\scriptscriptstyle j}$. A probabilistic algorithm run by the client. It takes as input  $1^\lambda$,  auxiliary information $\text{aux}$,  the key pair $k$,   query generator  function $Q$, encoding token $T_{\scriptscriptstyle qp}$ and $enc:=(E,D)$.  It computes a pair $c_{\scriptscriptstyle j}$ containing a query vector $\bm{q}_{\scriptscriptstyle j}=Q( \text{aux},k,{pp})$,  and proof $\bm{w}_{\scriptscriptstyle q_{_{\scaleto{j}{\size}}}}$ proving the query is well-structured, where {$pp\in T_{\scriptscriptstyle qp}$}. It outputs the encoding of the pair, $c^{\scriptscriptstyle *}_{\scriptscriptstyle j}=E(c_{\scriptscriptstyle j},T_{\scriptscriptstyle qp})$, and sends the output to the contract. 



%\item[$\bullet$] $\mathtt{RCSP}.\mathtt{genQuery}(1^\lambda, \text{aux},k,Q, T_{\scriptscriptstyle qp}, enc)\rightarrow c^{\scriptscriptstyle *}_{\scriptscriptstyle j}$. A probabilistic algorithm run by the client. It takes as input  $1^\lambda$,  auxiliary information $\text{aux}$,  the key pair $k$,   query generator deterministic function $Q$, encoding token $T_{\scriptscriptstyle qp}$ and $enc:=(E,D)$.  It computes a pair $c_{\scriptscriptstyle j}$ containing a query vector $\bm{q}_{\scriptscriptstyle j}=Q( \text{aux},k)$,  and proof $\bm{w}_{\scriptscriptstyle q_{_{\scaleto{j}{\size}}}}$ proving the query is well-structured. It outputs the encoding of the pair, $c^{\scriptscriptstyle *}_{\scriptscriptstyle j}=E(c_{\scriptscriptstyle j},T_{\scriptscriptstyle qp})$. The output is sent to the smart contract. 

%\scaleto{c}{1pt}

%a set of (random) challenges $c$ or empty set if the proving/verification  algorithms do not require any challenges. %The challenges set is published to a bulletin board.
\


\item[$\bullet$] $\mathtt{RCSP}.\mathtt{prove}(u^{\scriptscriptstyle *},   c^{\scriptscriptstyle *}_{\scriptscriptstyle j},pk, T_{\scriptscriptstyle qp},enc)\rightarrow (b_{\scriptscriptstyle j},m_{\scriptscriptstyle \mathcal{S},j},\pi^{\scriptscriptstyle *}_{\scriptscriptstyle j})$. It is run by the server. It  takes as input the encoded service input $u^{\scriptscriptstyle *}$, metadata $\sigma$,   encoded query pair $c^{\scriptscriptstyle *}_{\scriptscriptstyle j}$,  public key $pk$,    the encoding token $T_{\scriptscriptstyle qp}$, and $enc:=(E,D)$.   It checks the validity of decoded query $c_{\scriptscriptstyle j}=D(c^{\scriptscriptstyle *}_{\scriptscriptstyle j},T_{\scriptscriptstyle qp})$.  If it is rejected, then it sets $b_{\scriptscriptstyle j}=0$ and constructs a complaint $m_{\scriptscriptstyle \mathcal{S},j}$. Otherwise,  it   sets $b_{\scriptscriptstyle j}=1$ and $m_{\scriptscriptstyle \mathcal{S},j}=\bot$. It outputs $b_{\scriptscriptstyle j},m_{\scriptscriptstyle \mathcal{S},j}$,   and  encoded proof $\pi^{\scriptscriptstyle *}_{\scriptscriptstyle j}=E(\pi_{\scriptscriptstyle j},T_{\scriptscriptstyle qp})$, where  $\pi_{\scriptscriptstyle j}$ contains $h_{\scriptscriptstyle j}=F(u^{\scriptscriptstyle *},\bm{q}_{\scriptscriptstyle j},{pp})$ and a proof $\delta_{\scriptscriptstyle j}$  asserting the evaluation is performed correctly ($\pi_{\scriptscriptstyle j}$ may contain dummy values if $b_{\scriptscriptstyle j}=0$). The smart contract is given $\pi^{\scriptscriptstyle *}_{\scriptscriptstyle j}$.



%\item[$\bullet$] $\mathtt{RCSP}.\mathtt{prove}(u^{\scriptscriptstyle *},  \sigma,  c^{\scriptscriptstyle *}_{\scriptscriptstyle j},pk^{\scriptscriptstyle *}, T_{\scriptscriptstyle qp},enc)\rightarrow (b_{\scriptscriptstyle j},m_{\scriptscriptstyle \mathcal{S},j},\pi^{\scriptscriptstyle *}_{\scriptscriptstyle j})$. It is run by the server. It  takes as input the encoded service input $u^{\scriptscriptstyle *}$, metadata $\sigma$,   encoded query pair $c^{\scriptscriptstyle *}_{\scriptscriptstyle j}$, encoded public key $pk^{\scriptscriptstyle *}$,    the encoding token $T_{\scriptscriptstyle qp}$, and $enc:=(E,D)$.  It decodes the public key, $pk=D(pk^{\scriptscriptstyle *}, T_{\scriptscriptstyle qp})$. It checks the validity of decoded query, $c_{\scriptscriptstyle j}=D(c^{\scriptscriptstyle *}_{\scriptscriptstyle j},T_{\scriptscriptstyle qp})$.  If it is rejected, then it sets $b_{\scriptscriptstyle j}=0$ and constructs a complaint $m_{\scriptscriptstyle \mathcal{S},j}$. Otherwise,  it   sets $b_{\scriptscriptstyle j}=1$ and $m_{\scriptscriptstyle \mathcal{S},j}=\bot$. It outputs $b_{\scriptscriptstyle j},m_{\scriptscriptstyle \mathcal{S},j}$,   and  encoded proof $\pi^{\scriptscriptstyle *}_{\scriptscriptstyle j}=E(\pi_{\scriptscriptstyle j},T_{\scriptscriptstyle qp})$, where  $\pi_{\scriptscriptstyle j}$ contains $h_{\scriptscriptstyle j}=F(u^{\scriptscriptstyle *},\bm{q}_{\scriptscriptstyle j})$ and a proof $\delta_{\scriptscriptstyle j}$  asserting the evaluation is performed correctly ($\pi_{\scriptscriptstyle j}$ may contain dummy values if $b_{\scriptscriptstyle j}=0$). The smart contract is given $\pi^{\scriptscriptstyle *}_{\scriptscriptstyle j}$.


%with the help of decoded public key $pk=D(pk^{\scriptscriptstyle *}, T_{\scriptscriptstyle qp})$

\

\item[$\bullet$] $\mathtt{RCSP}.\mathtt{verify}(\pi^{\scriptscriptstyle *}_{\scriptscriptstyle j}, \bm{q}_{\scriptscriptstyle j},k,T_{\scriptscriptstyle qp},enc)\rightarrow (d_{\scriptscriptstyle j},m_{\scriptscriptstyle \mathcal{C},j})$. A deterministic algorithm run by the client. It takes as input the encoded proof $\pi^{\scriptscriptstyle *}_{\scriptscriptstyle j}$,  query vector $\bm{q}_{\scriptscriptstyle j}$, key pair $k$,  the encoding token $T_{\scriptscriptstyle qp}$ and $enc:=(E,D)$. If the decoded proof $\pi_{\scriptscriptstyle j}=D(\pi^{\scriptscriptstyle *}_{\scriptscriptstyle j},T_{\scriptscriptstyle qp})$ is rejected,  it outputs $d_{\scriptscriptstyle j}=0$ and a complaint $m_{\scriptscriptstyle \mathcal{C},j}$. Otherwise, it outputs  $d_{\scriptscriptstyle j}=1$ and $m_{\scriptscriptstyle \mathcal{C},j}=\bot$.




%, where $k=pk$  in the case where $\mathtt{VS.verify}()$ is publicly verifiable, or $k:(sk,pk)$ when it is privately verifiable

\


\item[$\bullet$] $\mathtt{RCSP}.\mathtt{resolve}(m_{\scriptscriptstyle \mathcal{C}},m_{\scriptscriptstyle \mathcal{S}},z, {\bm{\pi}}^{\scriptscriptstyle *}, {\bm{c}}^{\scriptscriptstyle *},pk, T_{\scriptscriptstyle qp},enc)\rightarrow \bm{y}$. It is run by the arbiter. It takes as input the client's complaints $m_{\scriptscriptstyle \mathcal{C}}$, the server's complaints $m_{\scriptscriptstyle \mathcal{S}}$, the total number of verifications $z$, all encoded proofs $ {\bm{\pi}}^{\scriptscriptstyle *}$, all encoded query pairs $ {\bm{c}}^{\scriptscriptstyle *}$,  public key $pk$, encoding token $T_{\scriptscriptstyle qp}$, and $enc:=(E,D)$. It verifies the token, decoded queries, and proofs.  It reads the  binary vectors $[\bm{y}_{\scriptscriptstyle \mathcal  C}, \bm{y}_{\scriptscriptstyle \mathcal  S},\bm{y}'_{\scriptscriptstyle \mathcal  C}, \bm{y}'_{\scriptscriptstyle \mathcal  S}]$ from the smart contract. It  updates $\bm{y}_{\scriptscriptstyle \mathcal E}$ by setting an element of it to one, i.e., $y_{\scriptscriptstyle \mathcal E,j}=1$, if party $\mathcal{E}\in\{\mathcal{C},\mathcal{S}\}$ has misbehaved in the $j$-th verification (i.e., provided invalid query or service proof). It also updates $\bm{y}'_{\scriptscriptstyle \mathcal E}$ if party $\mathcal{E}$ has provided a complain that does not allow it to identify a misbehaved party, in $j$-th verification, i.e. when the arbiter is unnecessarily invoked. 




%\item[$\bullet$] $\mathtt{RCSP}.\mathtt{resolve}(m_{\scriptscriptstyle \mathcal{C}},m_{\scriptscriptstyle \mathcal{S}},z, {\bm{\pi}}^{\scriptscriptstyle *}, {\bm{c}}^{\scriptscriptstyle *},pk^{\scriptscriptstyle*}, T_{\scriptscriptstyle qp},enc)\rightarrow \bm{y}$. It is run by the arbiter. It takes as input the client's complaints $m_{\scriptscriptstyle \mathcal{C}}$, the server's complaints $m_{\scriptscriptstyle \mathcal{S}}$, the total number of verifications $z$, all encoded proofs $ {\bm{\pi}}^{\scriptscriptstyle *}$, all encoded query pairs $ {\bm{c}}^{\scriptscriptstyle *}$, encoded public key $pk^{\scriptscriptstyle *}$, encoding token $T_{\scriptscriptstyle qp}$, and $enc:=(E,D)$. It decodes the public key. It verifies the token, decoded queries, and proofs.  It reads the  binary vectors $[\bm{y}_{\scriptscriptstyle \mathcal  C}, \bm{y}_{\scriptscriptstyle \mathcal  S},\bm{y}'_{\scriptscriptstyle \mathcal  C}, \bm{y}'_{\scriptscriptstyle \mathcal  S}]$ from the smart contract. It  updates $\bm{y}_{\scriptscriptstyle \mathcal E}$ by setting an element of it to one, i.e., $y_{\scriptscriptstyle \mathcal E,j}=1$, if party $\mathcal{E}\in\{\mathcal{C},\mathcal{S}\}$ has misbehaved in the $j$-th verification (i.e., provided invalid query or service proof). It also updates $\bm{y}'_{\scriptscriptstyle \mathcal E}$ if party $\mathcal{E}$ has provided a complain that does not allow it to identify a misbehaved party, in $j$-th verification. 


%\item[$\bullet$] $\mathtt{RCSP}.\mathtt{resolve}(m_{\scriptscriptstyle \mathcal{C}},m_{\scriptscriptstyle \mathcal{S}},z, {\bm{\pi}}^{\scriptscriptstyle *}, {\bm{c}}^{\scriptscriptstyle *},pk,E^{\scriptscriptstyle -1}_{\scriptscriptstyle 2},t_{\scriptscriptstyle qp},a)\rightarrow (y_{\scriptscriptstyle \mathcal C},y_{\scriptscriptstyle \mathcal S})$. It is run by the arbiter. It takes client's complaint $m_{\scriptscriptstyle \mathcal{C}}$,   server's complaints   $m_{\scriptscriptstyle \mathcal{S}}$, total number of verifications $z$, all encoded proofs $ {\bm{\pi}}^{\scriptscriptstyle *}$, all encoded queries $ {\bm{c}}^{\scriptscriptstyle *}$, public key $pk$, decoding function $E^{\scriptscriptstyle -1}_{\scriptscriptstyle 2}$, encoding token $t_{\scriptscriptstyle qp}$ and  $a$.  It proceeds only if $a=1$. It verifies the token,   decoded queries, and proofs.  It outputs total number of times the client and server misbehaved, i.e. $y_{\scriptscriptstyle \mathcal C}$ and $y_{\scriptscriptstyle \mathcal S}$ respectively. 

\
	
	
\item[$\bullet$] $\mathtt{RCSP}.\mathtt{pay}(\bm{y},T_{\scriptscriptstyle cp},a,p_{\scriptscriptstyle\mathcal S}, coin^{\scriptscriptstyle *}_{\scriptscriptstyle\mathcal C},coin^{\scriptscriptstyle *}_{\scriptscriptstyle\mathcal S})\rightarrow ({\bm{coin}}_{\scriptscriptstyle\mathcal C},{\bm{coin}}_{\scriptscriptstyle\mathcal S},{\bm{coin}}_{\scriptscriptstyle\mathcal Ar})$. It is run by the smart contract and can be invoked by the client or server. It takes as input the binary vectors $[\bm{y}_{\scriptscriptstyle \mathcal  C}, \bm{y}_{\scriptscriptstyle \mathcal  S},\bm{y}'_{\scriptscriptstyle \mathcal  C}, \bm{y}'_{\scriptscriptstyle \mathcal  S}]\in \bm{y}$ that indicate which party misbehaved, or sent invalid complaint  in each verification,  coins' token $T_{\scriptscriptstyle cp}:=\{cp,g_{\scriptscriptstyle cp}\}$, the output of the checks that server-side initiation algorithm  perfomed $a$, the total coins the server should deposit $p_{\scriptscriptstyle\mathcal S}$, and the total coins amount the client and server  deposited, i.e. $coin^{\scriptscriptstyle *}_{\scriptscriptstyle\mathcal C}$ and $coin^{\scriptscriptstyle *}_{\scriptscriptstyle\mathcal S}$ respectively. If $a=1$ and  $coin^{\scriptscriptstyle *}_{\scriptscriptstyle\mathcal S}= p_{\scriptscriptstyle\mathcal S}$, then it verifies the validity of $T_{\scriptscriptstyle cp}$. If $T_{\scriptscriptstyle cp}$ is rejected, then it aborts. 
If it is accepted, then it constructs  vector ${\bm{coin}}_{\scriptscriptstyle\mathcal I}$, where $\mathcal{I}\in\{\mathcal{C},\mathcal{S},\mathcal{A}r\}$; It sends ${{coin}}_{\scriptscriptstyle\mathcal{I},j}\in{\bm{coin}}_{\scriptscriptstyle\mathcal I}$ coins to party $\mathcal{I}$ for each $j$-th verification. Otherwise (i.e. $a=0$ or $coin^{\scriptscriptstyle *}_{\scriptscriptstyle\mathcal S}\neq p_{\scriptscriptstyle\mathcal S}$)  it only sends $coin^{\scriptscriptstyle *}_{\scriptscriptstyle\mathcal C}$ to $\mathcal C$. 	
	
	
	
%\item[$\bullet$] $\mathtt{RCSP}.\mathtt{pay}(\bm{y},T_{\scriptscriptstyle cp},a,p_{\scriptscriptstyle\mathcal S}, coin^{\scriptscriptstyle *}_{\scriptscriptstyle\mathcal C},coin^{\scriptscriptstyle *}_{\scriptscriptstyle\mathcal S})\rightarrow ({\bm{coin}}_{\scriptscriptstyle\mathcal C},{\bm{coin}}_{\scriptscriptstyle\mathcal S},{\bm{coin}}_{\scriptscriptstyle\mathcal Ar})$. It is run by the smart contract and can be invoked by the client or server. It takes as input the binary vectors $[\bm{y}_{\scriptscriptstyle \mathcal  C}, \bm{y}_{\scriptscriptstyle \mathcal  S},\bm{y}'_{\scriptscriptstyle \mathcal  C}, \bm{y}'_{\scriptscriptstyle \mathcal  S}]\in \bm{y}$ that indicate which party misbehaved, or sent invalid complaint  in each verification,  coins' token $T_{\scriptscriptstyle cp}:=\{cp,g_{\scriptscriptstyle cp}\}$, the output of the checks that server-side initiation algorithm  perfomed $a$, the total coins the server should deposit $p_{\scriptscriptstyle\mathcal S}$, and the total coins amount the client and server  deposited, i.e. $coin^{\scriptscriptstyle *}_{\scriptscriptstyle\mathcal C}$ and $coin^{\scriptscriptstyle *}_{\scriptscriptstyle\mathcal S}$ respectively. If $a=1$ and  $coin^{\scriptscriptstyle *}_{\scriptscriptstyle\mathcal S}= p_{\scriptscriptstyle\mathcal S}$, then it verifies the validity of $T_{\scriptscriptstyle cp}$. If $T_{\scriptscriptstyle cp}$ is rejected, then it aborts. 
%If it is accepted, then it constructs  vector ${\bm{coin}}_{\scriptscriptstyle\mathcal I}$, where $\mathcal{I}\in\{\mathcal{C},\mathcal{S},\mathcal{A}r\}$; It sends ${{coin}}_{\scriptscriptstyle\mathcal{I},j}\in{\bm{coin}}_{\scriptscriptstyle\mathcal I}$ coins to party $\mathcal{I}$ for each $j$-th verification. Otherwise (i.e. $a=0$ or $coin^{\scriptscriptstyle *}_{\scriptscriptstyle\mathcal S}\neq p_{\scriptscriptstyle\mathcal S}$)  it only sends $coin^{\scriptscriptstyle *}_{\scriptscriptstyle\mathcal C}$ to $\mathcal C$. 

%sets $coin_{\scriptscriptstyle\mathcal C,0}=coin^{\scriptscriptstyle *}_{\scriptscriptstyle\mathcal C}, \bm{coin}_{\scriptscriptstyle\mathcal S}=\bot$, and $\bm{coin}_{\scriptscriptstyle\mathcal Ar}=\bot$.  %It outputs ${\bm{coin}}_{\scriptscriptstyle\mathcal C},{\bm{coin}}_{\scriptscriptstyle\mathcal S}$, and ${\bm{coin}}_{\scriptscriptstyle\mathcal Ar}$

%sends $coin_{\scriptscriptstyle\mathcal C}, coin_{\scriptscriptstyle\mathcal S}$, and $coin_{\scriptscriptstyle\mathcal Ar}$ coins to the client, server, and arbiter respectively. Otherwise, it does nothing. 

\end{itemize}
\end{definition}

In the above definition, algorithms $\mathtt{RCSP}.\mathtt{genQuery}(.),\mathtt{RCSP}.\mathtt{prove}(.),\mathtt{RCSP}.\mathtt{verify}(.)$ and $\mathtt{RCSP}.\mathtt{resolve}(.)$ implicitly  take $a, coin^{\scriptscriptstyle*}_{\scriptscriptstyle\mathcal{S}}, p_{\scriptscriptstyle\mathcal{S}}$ as another inputs and  execute only if $a=1$ and $coin^{\scriptscriptstyle*}_{\scriptscriptstyle\mathcal{S}}=p_{\scriptscriptstyle\mathcal{S}}$; however, for the sake of simplicity we  avoided explicitly stating it  in the definition. 

A recurring contingent service payment (RC-S-P)  scheme satisfies correctness and security. At a high level, correctness requires that by the end of the protocol's execution  (that involves  honest client and server) the client receives all $z$ valid service proofs while the server gets paid  for the proofs, without the involvement of the arbiter. More specifically, it requires that the server accepts an honest client's encoded data and query while the honest client accepts the server's valid service proof (and no one is identified as misbehaving party). Moreover, the honest client gets back all its deposited coins minus the service payment, the honest server gets back all its deposited coins  plus the service payment and the arbiter receives nothing. It is formally stated as below.



\begin{definition}[RC-S-P Correctness] A recurring contingent service payment  scheme  is  correct   if for any functions $F, Q, M, E, D$, any price list $pl$, and any auxiliary information $\text{aux}_{\scriptscriptstyle 1},...,\text{aux}_{\scriptscriptstyle z}$, the key generation algorithm produces keys $\mathtt{RCSP}.\mathtt{keyGen}(1^\lambda, F)\rightarrow \bm{k}$,   such that for any service input $u$, if $\mathtt{RCSP}.\mathtt{cInit}(1^\lambda, u,\bm{k}, M,z,pl,enc) \rightarrow (u^{\scriptscriptstyle *},e,T, p_{\scriptscriptstyle\mathcal S},  \bm{y}, coin^{\scriptscriptstyle*}_{\scriptscriptstyle\mathcal C})$,  $\mathtt{RCSP}.\mathtt{sInit} (u^{\scriptscriptstyle *},e, pk,z, T, p_{\scriptscriptstyle\mathcal S},\bm{y},enc)\rightarrow (coin^{\scriptscriptstyle *}_{\scriptscriptstyle\mathcal S},a)$,   $\forall j:\Big($$\mathtt{RCSP}.\mathtt{genQuery}(1^\lambda, \text{aux}_{\scriptscriptstyle j},k,\\Q, T_{\scriptscriptstyle qp}, enc)\rightarrow c^{\scriptscriptstyle *}_{\scriptscriptstyle j}$, $\mathtt{RCSP}.\mathtt{prove}(u^{\scriptscriptstyle *},  \sigma,  c^{\scriptscriptstyle *}_{\scriptscriptstyle j},pk, T_{\scriptscriptstyle qp},enc)\rightarrow (b_{\scriptscriptstyle j},m_{\scriptscriptstyle\mathcal {S},j},\pi^{\scriptscriptstyle *}_{\scriptscriptstyle j})$, $\mathtt{RCSP}.\mathtt{verify}(\pi^{\scriptscriptstyle *}_{\scriptscriptstyle j}, \bm{q}_{\scriptscriptstyle j},k,T_{\scriptscriptstyle qp},enc)\rightarrow (d_{\scriptscriptstyle j}, m_{\scriptscriptstyle\mathcal {C},j})$$\Big),\\ \mathtt{RCSP}.\mathtt{resolve}(m_{\scriptscriptstyle \mathcal{C}},m_{\scriptscriptstyle \mathcal{S}},z, {\bm{\pi}}^{\scriptscriptstyle *}, {\bm{c}}^{\scriptscriptstyle *},pk, T_{\scriptscriptstyle qp},enc)\rightarrow \bm{y}$, $\mathtt{RCSP}.\mathtt{pay}(\bm{y},T_{\scriptscriptstyle cp},a,p_{\scriptscriptstyle\mathcal S}, coin^{\scriptscriptstyle *}_{\scriptscriptstyle\mathcal C},coin^{\scriptscriptstyle *}_{\scriptscriptstyle\mathcal S})\rightarrow (\bm{coin}_{\scriptscriptstyle\mathcal C},\bm{coin}_{\scriptscriptstyle\mathcal S},\bm{coin}_{\scriptscriptstyle\mathcal Ar})$, then $(a=1)\wedge (\bigwedge\limits^{\scriptscriptstyle z}_{\scriptscriptstyle j=1} b_{\scriptscriptstyle j}= \bigwedge\limits^{\scriptscriptstyle z}_{\scriptscriptstyle j=1} d_{\scriptscriptstyle j}=1)\wedge (\bm{y}_{\scriptscriptstyle \mathcal C}=\bm{y}_{\scriptscriptstyle \mathcal S}=\bm{y}'_{\scriptscriptstyle \mathcal C}=\bm{y}'_{\scriptscriptstyle \mathcal S}=0) \wedge (\sum\limits^{\scriptscriptstyle z}_{\scriptscriptstyle j=1}coin_{\scriptscriptstyle\mathcal C,j}=coin^{\scriptscriptstyle *}_{\scriptscriptstyle\mathcal C}-o\cdot z) \wedge  (\sum\limits^{\scriptscriptstyle z}_{\scriptscriptstyle j=1}coin_{\scriptscriptstyle\mathcal S,j}=coin^{\scriptscriptstyle *}_{\scriptscriptstyle\mathcal S}+o\cdot z)  \wedge (\sum\limits^{\scriptscriptstyle z}_{\scriptscriptstyle j=1}coin_{\scriptscriptstyle\mathcal Ar,j}=0)$, where $\bm{y}_{\scriptscriptstyle \mathcal C},\bm{y}_{\scriptscriptstyle \mathcal S},\bm{y}'_{\scriptscriptstyle \mathcal C},\bm{y}'_{\scriptscriptstyle \mathcal S}\in \bm{y}$
\end{definition}

%\textcolor{blue}{XXXX Thomas: something is not right in the definition of correctness... we write $\forall j$ but the execution of $\mathtt{RCSP}.\mathtt{genQuery}$ does not depend on $j$. Not sure how to fix it yet.XXXX}

A RC-S-P scheme is said to be secure if it satisfies  three main properties: (a)  security against malicious server, (b) security against malicious client, and (c) privacy.  In the following, we formally define each of them.

 Intuitively, security against a malicious server states that (at the end of the protocol execution) either (i) for each verification the client gets a valid proof and gets back its deposit minus the service payment, or (ii) the client gets its deposit back (for $j$-th verification) and the arbiter receives $l$ coins, or (iii) if it unnecessarily invokes the arbiter, then it has to pay the arbiter. In particular, for each $j$-th verification, the security requires that only with a negligible probability the adversary wins, if it  provides either (a)  correct evaluation of the function on the service input but it either makes the client  withdraw an incorrect amount of coins (i.e. something other than its deposit minus service payment) or makes the arbiter withdraw incorrect amount of coins if it unnecessarily invokes the arbiter, or (b) incorrect evaluation of the function on the service input,  but either persuades the client or the arbiter to accept it (i.e.,  $b_{\scriptscriptstyle j}=1$ or $y_{\scriptscriptstyle\mathcal S,j}=0$) or makes them  withdraw incorrect amount of coins (i.e., $coin_{\scriptscriptstyle\mathcal{C},j}\neq \frac{coin_{\scriptscriptstyle\mathcal C}^{\scriptscriptstyle*}}{z}$ or  $coin_{\scriptscriptstyle\mathcal{A}r,j}\neq l$ coins).  Below, we formalize this intuition. 




%Intuitively, security against a malicious server states that (at the end of the protocol execution) either the client for each verification gets a valid proof or the client gets its coins back (for $j$-th verification) and the arbiter receives $l$ coins. In other words, for each  $j$-th verification, the adversary wins if it provides incorrect evaluation of the function on the service input,  but (with a high probability) either persuades the client or the arbiter to accept it (i.e.  $b_{\scriptscriptstyle j}=1$ or $y_{\scriptscriptstyle\mathcal S}[j]=0$) or makes them to withdraw incorrect amount of coins (i.e. $coin_{\scriptscriptstyle\mathcal{C},j}\neq \frac{coin_{\scriptscriptstyle\mathcal C}^{\scriptscriptstyle*}}{z}$ and  $coin_{\scriptscriptstyle\mathcal{A}r,j}\neq l$ coins).  Below, we formalize this intuition with an experiment, $\textbf{Exp}^{\scriptscriptstyle(1)}_{\scriptscriptstyle\mathcal {A}}[]$: 


%
%\small{
%$$
%    \begin{array}{l}
%    \mathtt{RCSP.keyGen}(1^{\lambda},F)\rightarrow \bm{k}\\
%    \mathcal{A}(1^\lambda,pk, F)\rightarrow (u,\text{state})\\
%    \mathtt{RCSP}.\mathtt{cInit}(1^\lambda, u,\bm{k}, M,z,cp)\rightarrow (u^{\scriptscriptstyle *},e,t,coin^{\scriptscriptstyle*}_{\scriptscriptstyle\mathcal C})\\
%    \mathcal{A}(u^{\scriptscriptstyle *},e, pk,z, t)\rightarrow (coin^{\scriptscriptstyle *}_{\scriptscriptstyle\mathcal S},a)\\
%	\mathtt{RCSP}.\mathtt{genQuery}(1^{\lambda}, \text{aux},k,Q,j, en)\rightarrow c^{\scriptscriptstyle *}_{\scriptscriptstyle j}\\
%	
%     \mathcal{A}(\text{state},c^{\scriptscriptstyle *}_{\scriptscriptstyle j},\sigma, u^{\scriptscriptstyle *},j,en,a)\rightarrow 
%     (b_{\scriptscriptstyle j},m_{\scriptscriptstyle \mathcal{A},j},y^{\scriptscriptstyle *}_{\scriptscriptstyle j},\delta^{\scriptscriptstyle *}_{\scriptscriptstyle j})\\ 
%   \mathtt{RCSP}.\mathtt{verify}(\pi^{\scriptscriptstyle *}_{\scriptscriptstyle j}, q,k,j,en)\rightarrow (d_{\scriptscriptstyle j},m_{\scriptscriptstyle \mathcal{C},j})\\
%   \mathtt{RCSP}.\mathtt{resolve}(\bm{m}_{\scriptscriptstyle \mathcal{C}},\bm{m}_{\scriptscriptstyle \mathcal{A}},z, {\bm{\pi}}^{\scriptscriptstyle *}, {\bm{c}}^{\scriptscriptstyle *},pk,en)\rightarrow (y_{\scriptscriptstyle \mathcal C},y_{\scriptscriptstyle \mathcal S})\\
%   \mathtt{RCSP}.\mathtt{pay}(y_{\scriptscriptstyle \mathcal C},y_{\scriptscriptstyle \mathcal S},t_{\scriptscriptstyle cp})\rightarrow ({\bm{coin}}_{\scriptscriptstyle\mathcal C},{\bm{coin}}_{\scriptscriptstyle\mathcal S},{\bm{coin}}_{\scriptscriptstyle\mathcal Ar})\\
%\end{array}$$
%}
%
%
%
%The output of the experiment is $1$, i.e. $\textit{\textbf{Exp}}^{\scriptscriptstyle(1)}_{\scriptscriptstyle\mathcal {A},\text{RC-S-P}}()=1$, if the adversary wins. 
%
%\begin{definition}[RC-S-P Security Against Malicious Server]\label{deff::}  A RC-S-P  is secure against a malicious server  for a function $F$, if for any security parameter $\lambda$, any auxiliary information $aux$, any $Q$ and any probabilistic polynomial time adversary $\mathcal{A}$, there exists a negligible function $\mu(.)$, such that: 
%%\footnotesize{
%
%$$Pr[\textbf{Exp}^{\scriptscriptstyle(1)}_{\scriptscriptstyle\mathcal {A},\text{RC-S-P}}()=1]=Pr\left[
%\begin{array}{l}
%\left((F_{\scriptscriptstyle q}(u^{\scriptscriptstyle *})= y_{\scriptscriptstyle j}\ \wedge coin_{\scriptscriptstyle\mathcal{C},j}\neq  \frac{coin_{\scriptscriptstyle\mathcal C}^{\scriptscriptstyle*}}{z}-o\right)
% \vee\\
%  \left(F_{\scriptscriptstyle q}(u^{\scriptscriptstyle *})\neq y_{\scriptscriptstyle j}\ \wedge (d_{\scriptscriptstyle j}=1	\vee y_{\scriptscriptstyle \mathcal S}[j]=0 \ \vee coin_{\scriptscriptstyle\mathcal{C},j}\neq \frac{coin_{\scriptscriptstyle\mathcal C}^{\scriptscriptstyle*}}{z} \ \wedge coin_{\scriptscriptstyle\mathcal{A}r,j}\neq l)\right)\end{array}\right]\leq \mu(\lambda)$$
%\end{definition}



%_{\scriptscriptstyle q_{_{\scaleto{j}{\size}}}}


\begin{definition}[RC-PoR-P Security Against Malicious Server]\label{deff::RC-S-P-SecurityAgainstMaliciousServer}  A RC-PoR-P  is secure against a malicious server  for a function $F$, if for  any price list $pl$, every $j$ (where $1\leq j\leq z$), and any PPT adversary $\mathcal{A}$, there exists a negligible function $\mu(\cdot)$, such that for any security parameter $\lambda$: 
%\footnotesize{
{\small
$$ \Pr\left[
  \begin{array}{l}
  \Big(F(u^{\scriptscriptstyle *}, \bm{q}_{\scriptscriptstyle j}, {pp})= h_{\scriptscriptstyle j}\ \wedge (coin_{\scriptscriptstyle\mathcal{C},j}\neq  \frac{coin_{\scriptscriptstyle\mathcal C}^{\scriptscriptstyle*}}{z}-o)\Big)\ \vee\\ 

\Big(
F(u^{\scriptscriptstyle *},\bm{q}_{\scriptscriptstyle j}, {pp})\neq h_{\scriptscriptstyle j} \ \wedge (d_{\scriptscriptstyle j}=1	\vee y_{\scriptscriptstyle \mathcal{S},j}=0 \ \vee \\coin_{\scriptscriptstyle\mathcal{C},j}\neq \frac{coin_{\scriptscriptstyle\mathcal C}^{\scriptscriptstyle*}}{z}+l) \Big)\\
%\text{s.t.}\\
%y_{\scriptscriptstyle j}= E^{\scriptscriptstyle -1}_{\scriptscriptstyle 2}(y^{\scriptscriptstyle *}_{\scriptscriptstyle j},t_{\scriptscriptstyle qp})\\
%E^{\scriptscriptstyle -1}_{\scriptscriptstyle 2},t_{\scriptscriptstyle qp}\in en\\
\end{array} \middle |
    \begin{array}{l}
    \mathtt{RCPoRP.keyGen}(1^{\lambda})\rightarrow \bm{k}\\
    \mathcal{A}(1^\lambda,pk)\rightarrow u\\
  
   \mathtt{RCPoRP}.\mathtt{cInit}(1^\lambda, u,\bm{k}, z,pl)\rightarrow (u^{\scriptscriptstyle *},T, p_{\scriptscriptstyle\mathcal S},  \bm{y}, coin^{\scriptscriptstyle*}_{\scriptscriptstyle\mathcal C})\\
    
    \mathcal{A}(u^{\scriptscriptstyle *},  pk, z, T, p_{\scriptscriptstyle\mathcal S},\bm{y})\rightarrow (coin^{\scriptscriptstyle *}_{\scriptscriptstyle\mathcal S},a)\\
	\mathtt{RCPoRP}.\mathtt{genQuery}(1^\lambda,  T_{\scriptscriptstyle qp})\rightarrow q_{\scriptscriptstyle j}\\
	
     \mathcal{A}( u^{\scriptscriptstyle *}, q_{\scriptscriptstyle j}, a)\rightarrow 
     (b_{\scriptscriptstyle j},m_{\scriptscriptstyle \mathcal{S},j},\bm{\pi}^{\scriptscriptstyle *}_{\scriptscriptstyle j})\\
     
     %(\pi^{\scriptscriptstyle *}_{\scriptscriptstyle j},b_{\scriptscriptstyle j})\\
     
\mathtt{RCPoRP}.\mathtt{verify}(\bm{\pi}^{\scriptscriptstyle *}_{\scriptscriptstyle j}, {q}_{\scriptscriptstyle j},T_{\scriptscriptstyle qp})\rightarrow (d_{\scriptscriptstyle j},m_{\scriptscriptstyle \mathcal{C},j})\\
  \mathtt{resolve}(m_{\scriptscriptstyle \mathcal{C}},m_{\scriptscriptstyle \mathcal{S}}, z, {\bm{\pi}}^{\scriptscriptstyle *}, {\bm{q}},  T_{\scriptscriptstyle qp})\rightarrow \bm{y}\\
   
   \mathtt{RCPoRP}.\mathtt{pay}(\bm{y},T_{\scriptscriptstyle cp},a,p_{\scriptscriptstyle\mathcal S}, coin^{\scriptscriptstyle *}_{\scriptscriptstyle\mathcal C},coin^{\scriptscriptstyle *}_{\scriptscriptstyle\mathcal S})\rightarrow ({\bm{coin}}_{\scriptscriptstyle\mathcal C},{\bm{coin}}_{\scriptscriptstyle\mathcal S})\\
\end{array}    \right]\leq \mu(\lambda).$$
}
where $h_{\scriptscriptstyle j}= D(h^{\scriptscriptstyle *}_{\scriptscriptstyle j},T_{\scriptscriptstyle qp})$, $
D\in enc$, $\sigma\in e$, $m_{\scriptscriptstyle \mathcal{C},j}\in m_{\scriptscriptstyle \mathcal{C}}, m_{\scriptscriptstyle \mathcal{S},j}\in m_{\scriptscriptstyle \mathcal{S}}$,   $y_{\scriptscriptstyle \mathcal{S},j}\in \bm{y}_{\scriptscriptstyle \mathcal{S}}\in\bm{y}$, and $ {pp}\in T_{\scriptscriptstyle qp}$. 



%\textcolor{blue}{XXXX Thomas: We never define what $m_{\scriptscriptstyle \mathcal{A}}$ is. The $\mathtt{RCSP}.\mathtt{resolve}$ algorithm takes as input $m_{\scriptscriptstyle \mathcal{C}},m_{\scriptscriptstyle \mathcal{S}}$, not $m_{\scriptscriptstyle \mathcal{C}},m_{\scriptscriptstyle \mathcal{A}}$. One extra question is: how is $m_{\scriptscriptstyle \mathcal{S}}$ created in the experiment? --> Aydin--addressed} 


%Also, the probability is taken over uniform  choice of $k,k'\stackrel{\scriptscriptstyle\$}\leftarrow \{0,1\}^{\scriptscriptstyle\lambda}$, where $k,k'\in\bm{k}$, as well as the randomness of $\mathcal{A}$ and the randomness used in witnesses $g_{\scriptscriptstyle cp},g_{\scriptscriptstyle qp}\in t$.
\end{definition}




%\begin{definition}[RC-S-P Correctness] A recurring contingent service payment  scheme  is  correct   if for any function $F$, the key generation algorithm produces keys $\mathtt{RCSP}.\mathtt{keyGen}(1^\lambda,F)\rightarrow (k,k')$,   such that $\forall u \in \mathtt{Domain}(F)$ if $\mathtt{RCSP}.\mathtt{cInit}(1^\lambda, u,k,k', M,z,o,l,E_{\scriptscriptstyle 1}) \rightarrow (u^{\scriptscriptstyle *}, e, t, coin^{\scriptscriptstyle*}_{\scriptscriptstyle\mathcal C})$, $\mathtt{RCSP}.\mathtt{sInit} (u^{\scriptscriptstyle *}, e, pk, z, t, E_{\scriptscriptstyle 1})\rightarrow (coin^{\scriptscriptstyle *}_{\scriptscriptstyle\mathcal S}, a)$, \ \\ $\mathtt{RCSP}.\mathtt{genQuery}(1^{\lambda}, \text{aux}, k, Q, j, E_{\scriptscriptstyle 2}, qp, a)\rightarrow c^{\scriptscriptstyle *}_{\scriptscriptstyle j}$, $\mathtt{RCSP}.\mathtt{prove}(u^{\scriptscriptstyle *},  \sigma, c^{\scriptscriptstyle *}_{\scriptscriptstyle j}, pk, j, E^{\scriptscriptstyle -1}_{\scriptscriptstyle 2},qp,a)\rightarrow (\pi^{\scriptscriptstyle *}_{\scriptscriptstyle j}, b_{\scriptscriptstyle j})$, \\$\mathtt{RCSP}.\mathtt{verify}(\pi^{\scriptscriptstyle *}_{\scriptscriptstyle j}, q,k,j,E^{\scriptscriptstyle -1}_{\scriptscriptstyle 2},qp,a)\rightarrow d_{\scriptscriptstyle j}$, $\mathtt{RCSP}.\mathtt{resolve}(m_{\scriptscriptstyle \mathcal{C}},m_{\scriptscriptstyle \mathcal{S}},z, {\bm{\pi}}^{\scriptscriptstyle *}, {\bm{c}}^{\scriptscriptstyle *},pk,E^{\scriptscriptstyle -1}_{\scriptscriptstyle 2},t_{\scriptscriptstyle qp},a)\rightarrow (y_{\scriptscriptstyle \mathcal C},y_{\scriptscriptstyle \mathcal S})$, \\$\mathtt{RCSP}.\mathtt{pay}(y_{\scriptscriptstyle \mathcal C},y_{\scriptscriptstyle \mathcal S},t_{\scriptscriptstyle cp},a)\rightarrow (coin_{\scriptscriptstyle\mathcal C},coin_{\scriptscriptstyle\mathcal S},coin_{\scriptscriptstyle\mathcal Ar})$, s.t. $a=1\wedge (\bigwedge\limits^{\scriptscriptstyle z}_{\scriptscriptstyle j=1} b_{\scriptscriptstyle j}= \bigwedge\limits^{\scriptscriptstyle z}_{\scriptscriptstyle j=1} d_{\scriptscriptstyle j}=1)\wedge (y_{\scriptscriptstyle \mathcal C}=y_{\scriptscriptstyle \mathcal S}=0) \wedge (coin_{\scriptscriptstyle\mathcal C}=coin^{\scriptscriptstyle *}_{\scriptscriptstyle\mathcal C}-oz) \wedge  (coin_{\scriptscriptstyle\mathcal S}=coin^{\scriptscriptstyle *}_{\scriptscriptstyle\mathcal S}+oz)   \wedge (coin_{\scriptscriptstyle\mathcal Ar}=0) $
%\end{definition}

 
 
 
 
Informally, security against a malicious client requires that, for each  $j$-th verification, a malicious client with a negligible probability  wins if it provides either (a) valid metadata and query but either makes the server receive incorrect amount of coins (something other than its deposit plus the service payment), or makes the arbiter withdraw incorrect amount of coins if it unnecessarily invokes the arbiter,  or (b) invalid metadata or query but convinces the server to accept either of them (i.e. the invalid metadata or query), or (c) invalid query but persuades the arbiter to accept it, or  makes them withdraw an incorrect amount of coins (i.e. $coin_{\scriptscriptstyle\mathcal{S},j}\neq \frac{coin_{\scriptscriptstyle\mathcal S}^{\scriptscriptstyle*}}{z}+o$ or  $coin_{\scriptscriptstyle\mathcal{A}r,j}\neq l$ coins). Below, we formally state the property. 
 
 %\textcolor{blue}{XXXX Thomas: what does ``them'' refer to? XXXX}--> Aydin: the invalid metadata or query
 
 
 \begin{definition}[RC-S-P Security Against Malicious Client]\label{deff::RC-S-P-Security-Against Malicious-Client}  A RC-S-P  is secure against a malicious client  for a function $F$, for every $j$ (where $1\leq j\leq z$), any $Q,M,E,D$ and any probabilistic polynomial time adversary $\mathcal{A}$, there exists a negligible function $\mu(\cdot)$, such that for any security parameter $\lambda$: 


{\small
$$ \Pr\left[
  \begin{array}{l}
 \Big((M(u^{\scriptscriptstyle *},k, {pp})= \sigma  \wedge Q(\text{aux},k, {pp})= \bm{q}_{\scriptscriptstyle j})\ \wedge\\
  (coin_{\scriptscriptstyle\mathcal{S},j}\neq  \frac{coin_{\scriptscriptstyle\mathcal S}^{\scriptscriptstyle*}}{z}+o) \Big)\ \vee
  \\
  
  
  \Big(M(u^{\scriptscriptstyle *}, k,  {pp})\neq \sigma \wedge a=1\Big) \ \vee\\ 


  \Big(Q(\text{aux},k,  {pp})\neq \bm{q}_{\scriptscriptstyle j}\ \wedge
  (b_{\scriptscriptstyle j}=1 \ \vee \ y_{\scriptscriptstyle\mathcal{C},j}=0 \ \vee \\ coin_{\scriptscriptstyle\mathcal{S},j}\neq \frac{coin_{\scriptscriptstyle\mathcal S}^{\scriptscriptstyle*}}{z}+o+l)\Big)
 \\
 
\end{array} \middle |
    \begin{array}{l}
   \mathcal{A}(1^{\lambda}, F)\rightarrow (u^{\scriptscriptstyle *},z,\bm{k},e,T, pl, p_{\scriptscriptstyle\mathcal{S}}, coin^{\scriptscriptstyle*}_{\scriptscriptstyle\mathcal C},enc, \text{aux},\bm{y}, enc, pk)\\
   
   \mathtt{RCSP}.\mathtt{sInit}(u^{\scriptscriptstyle *}, e, pk, z, T, p_{\scriptscriptstyle\mathcal S},\bm{y},enc)\rightarrow (coin^{\scriptscriptstyle *}_{\scriptscriptstyle\mathcal S},a)\\
   
	\mathcal{A}(coin^{\scriptscriptstyle *}_{\scriptscriptstyle\mathcal S},a, 1^\lambda, \text{aux},k,Q, T_{\scriptscriptstyle qp}, enc)\rightarrow c^{\scriptscriptstyle *}_{\scriptscriptstyle j}\\

     \mathtt{RCSP}.\mathtt{prove}(u^{\scriptscriptstyle *},  \sigma,  c^{\scriptscriptstyle *}_{\scriptscriptstyle j}, pk, T_{\scriptscriptstyle qp},enc)\rightarrow (b_{\scriptscriptstyle j},m_{\scriptscriptstyle \mathcal{S},j},\pi^{\scriptscriptstyle *}_{\scriptscriptstyle j})\\
     
 \mathcal{A}(\pi^{\scriptscriptstyle *}_{\scriptscriptstyle j}, \bm{q}_{\scriptscriptstyle j},k,T_{\scriptscriptstyle qp},enc)\rightarrow (d_{\scriptscriptstyle j},m_{\scriptscriptstyle \mathcal{S},j})\\
   
   
   \mathtt{RCSP}.\mathtt{resolve}(m_{\scriptscriptstyle \mathcal{C}},m_{\scriptscriptstyle \mathcal{S}},z, {\bm{\pi}}^{\scriptscriptstyle *}, {\bm{c}}^{\scriptscriptstyle *}, pk, T_{\scriptscriptstyle qp},enc)\rightarrow \bm{y}\\
   \mathtt{RCSP}.\mathtt{pay}(\bm{y},T_{\scriptscriptstyle cp},a,p_{\scriptscriptstyle\mathcal S}, coin^{\scriptscriptstyle *}_{\scriptscriptstyle\mathcal C},coin^{\scriptscriptstyle *}_{\scriptscriptstyle\mathcal S})\rightarrow ({\bm{coin}}_{\scriptscriptstyle\mathcal C},{\bm{coin}}_{\scriptscriptstyle\mathcal S},{\bm{coin}}_{\scriptscriptstyle\mathcal Ar})\\
\end{array}    \right]\leq \mu(\lambda).$$
}
where $\bm{q}_{\scriptscriptstyle j}\in D(c^{\scriptscriptstyle *}_{\scriptscriptstyle j},t_{\scriptscriptstyle qp})$, $
D\in enc$, $\sigma\in e$, $y'_{\scriptscriptstyle \mathcal{C},j}\in \bm{y}'_{\scriptscriptstyle \mathcal{C}}\in\bm{y}$,   $y_{\scriptscriptstyle \mathcal{C},j}\in \bm{y}_{\scriptscriptstyle \mathcal{C}}\in\bm{y}$, and ${pp}\in T_{\scriptscriptstyle qp}$.  
\textcolor{blue}{XXXX Thomas: again here, the coins amount is fixed to $o$. Also I have not understood what is the meaning and use of $\bm{y}'_{\scriptscriptstyle \mathcal{C}},\bm{y}'_{\scriptscriptstyle \mathcal{S}}$ XXXX}
%Moreover,  the probability is taken over the randomness used in witnesses $g_{\scriptscriptstyle cp},g_{\scriptscriptstyle qp}\in t$ as well as  the randomness used in proofs $\bm{w}_{\scriptscriptstyle q_{_{\scaleto{j}{\size}}}}\in c_{\scriptscriptstyle j}$ and $w_{\scriptscriptstyle\sigma}\in e$ and  the randomness of $\mathcal{A}$.
\end{definition}
 










%\small{
%$$ Pr\left[
%  \begin{array}{l}
% \Big((M(u^{\scriptscriptstyle *},k)= \sigma  \wedge Q(\text{aux},k)= q_{\scriptscriptstyle j}) \wedge\\
%  (coin_{\scriptscriptstyle\mathcal{S},j}\neq  \frac{coin_{\scriptscriptstyle\mathcal S}^{\scriptscriptstyle*}}{z}+o)\Big) \vee
%  \\
%  \Big((M(u^{\scriptscriptstyle *},k)\neq \sigma \lor Q(\text{aux},k)\neq q_{\scriptscriptstyle j}) \wedge\\
%  (a=1 \ \vee b_{\scriptscriptstyle j}=1\ \vee y_{\scriptscriptstyle\mathcal{C}}[j]=0 \ \vee \\ coin_{\scriptscriptstyle\mathcal{S},j}\neq \frac{coin_{\scriptscriptstyle\mathcal S}^{\scriptscriptstyle*}}{z}+o \ \vee  coin_{\scriptscriptstyle\mathcal{A}r,j}\neq l) 
%  \Big)
% \\
% 
%\end{array} \middle |
%    \begin{array}{l}
%   \mathcal{A}(1^{\lambda},F)\rightarrow (u^{\scriptscriptstyle *},z,\bm{k},e,t,coin^{\scriptscriptstyle*}_{\scriptscriptstyle\mathcal C},en, \text{aux},\bm{y}_{\scriptscriptstyle \mathcal  C},\bm{y}_{\scriptscriptstyle \mathcal  S})\\
%   
%   \mathtt{RCSP}.\mathtt{sInit}(u^{\scriptscriptstyle *},e, pk,z, t,\bm{y}_{\scriptscriptstyle \mathcal  C},\bm{y}_{\scriptscriptstyle \mathcal  S})\rightarrow (coin^{\scriptscriptstyle *}_{\scriptscriptstyle\mathcal S},a)\\
%   
%	\mathcal{A}(coin^{\scriptscriptstyle *}_{\scriptscriptstyle\mathcal S},a, \text{aux},\bm{k},Q,j, en)\rightarrow c^{\scriptscriptstyle *}_{\scriptscriptstyle j}\\
%
%     \mathtt{RCSP}.\mathtt{prove}(u^{\scriptscriptstyle *},  \sigma,  c^{\scriptscriptstyle *}_{\scriptscriptstyle j},pk, en)\rightarrow (b_{\scriptscriptstyle j},m_{\scriptscriptstyle \mathcal{S},j},\pi^{\scriptscriptstyle *}_{\scriptscriptstyle j})\\
%     
% \mathcal{A}(\pi^{\scriptscriptstyle *}_{\scriptscriptstyle j}, q,k,j,en)\rightarrow (d_{\scriptscriptstyle j},m_{\scriptscriptstyle \mathcal{A},j})\\
%   
%   
%   \mathtt{RCSP}.\mathtt{resolve}(\bm{m}_{\scriptscriptstyle \mathcal{A}},\bm{m}_{\scriptscriptstyle \mathcal{S}},z, {\bm{\pi}}^{\scriptscriptstyle *}, {\bm{c}}^{\scriptscriptstyle *},pk,en)\rightarrow (\bm{y}_{\scriptscriptstyle \mathcal C},\bm{y}_{\scriptscriptstyle \mathcal S})\\
%   \mathtt{RCSP}.\mathtt{pay}(\bm{y}_{\scriptscriptstyle \mathcal C},\bm{y}_{\scriptscriptstyle \mathcal S},t_{\scriptscriptstyle cp},a,coin^{\scriptscriptstyle *}_{\scriptscriptstyle\mathcal C},coin^{\scriptscriptstyle *}_{\scriptscriptstyle\mathcal S})\rightarrow ({\bm{coin}}_{\scriptscriptstyle\mathcal C},{\bm{coin}}_{\scriptscriptstyle\mathcal S},{\bm{coin}}_{\scriptscriptstyle\mathcal Ar})\\
%\end{array}    \right]\leq \mu(\lambda)$$
%}
%where $q_{\scriptscriptstyle j}\in D(c^{\scriptscriptstyle *}_{\scriptscriptstyle j},t_{\scriptscriptstyle qp})$, $
%D,t_{\scriptscriptstyle qp}\in en$, $\sigma\in e$, and   the probability is taken over the randomness used in witnesses $w_{\scriptscriptstyle cp},w_{\scriptscriptstyle qp}\in t$ as well as  the randomness used in proofs $w_{\scriptscriptstyle q_{_{\scaleto{j}{\size}}}}\in c_{\scriptscriptstyle j}$ and $w_{\scriptscriptstyle\sigma}\in e$ and  the randomness of $\mathcal{A}$.
%\end{definition}
% 
 Note, in the above definition, an honest server either does not deposit (e.g. when $a=0$) or if it deposits (i.e. agrees to serve) ultimately receives its deposit \emph{plus the service payment} (with a high probability). Informally, RC-S-P is privacy preserving if it guarantees the privacy of (a) the service input (e.g. outsourced file) and (b) the service proof's status during the private time bubble. In the following, we formally define privacy.
 

  \begin{definition}[RC-S-P Privacy]\label{deff::RC-S-P-Privacy}  A RC-S-P   preserves privacy, any auxiliary information $\text{aux}$, any number of verifications $z$, any price list $pl$, any $F, Q,M, E$ and $D$, the following hold:
   
\begin{enumerate}

\item For any PPT adversary $\mathcal{A}_{\scriptscriptstyle 1}$ there exists a negligible function $\mu_{\scriptscriptstyle 1}(\cdot)$, such that for any security parameter $\lambda$:
  
{\small
$$\Pr\left[
  \begin{array}{l}
  \mathcal{A}_{\scriptscriptstyle 1}(\bm{c}^{\scriptscriptstyle *},coin^{\scriptscriptstyle *}_{\scriptscriptstyle S},coin^{\scriptscriptstyle *}_{\scriptscriptstyle C}, g_{\scriptscriptstyle cp}, \\ g_{\scriptscriptstyle qp}, \bm{\pi}^{\scriptscriptstyle *}, pl, a)\rightarrow\beta\\
\end{array} \middle |
    \begin{array}{l}
            \mathtt{RCSP.keyGen}(1^{\lambda}, F)\rightarrow \bm{k}\\
        \mathcal{A}_{\scriptscriptstyle 1}(1^\lambda, pk, F)\rightarrow (u_{\scriptscriptstyle 0},u_{\scriptscriptstyle 1})\\


    \beta\stackrel{\scriptscriptstyle\$}\leftarrow\{0,1\}\\

   \mathtt{RCSP}.\mathtt{cInit}(1^\lambda, u_{\scriptscriptstyle\beta},\bm{k}, M,z,pl,enc)\rightarrow (u^{\scriptscriptstyle *}_{\scriptscriptstyle\beta},e,T, p_{\scriptscriptstyle\mathcal S},  \bm{y}, coin^{\scriptscriptstyle*}_{\scriptscriptstyle\mathcal C})\\
\mathtt{RCSP}.\mathtt{sInit}(u_{\scriptscriptstyle\beta}^{\scriptscriptstyle *}, e, pk, z, T, p_{\scriptscriptstyle\mathcal S},\bm{y},enc)\rightarrow (coin^{\scriptscriptstyle *}_{\scriptscriptstyle\mathcal S},a)\\

\forall j \in [z]:\\

\Big(\mathtt{RCSP}.\mathtt{genQuery}(1^\lambda, \text{aux},k,Q_{\scriptscriptstyle\beta}, T_{\scriptscriptstyle qp}, enc)\rightarrow c^{\scriptscriptstyle *}_{\scriptscriptstyle j}\\
     \mathtt{RCSP}.\mathtt{prove}(u^{\scriptscriptstyle *}_{\scriptscriptstyle\beta},   \sigma,  c^{\scriptscriptstyle *}_{\scriptscriptstyle j}, pk, T_{\scriptscriptstyle qp},enc)\rightarrow (b_{\scriptscriptstyle j},m_{\scriptscriptstyle \mathcal{S},j},\pi^{\scriptscriptstyle *}_{\scriptscriptstyle j})\\  
   \mathtt{RCSP}.\mathtt{verify}(\pi^{\scriptscriptstyle *}_{\scriptscriptstyle j}, \bm{q}_{\scriptscriptstyle j},k,T_{\scriptscriptstyle qp},enc)\rightarrow (d_{\scriptscriptstyle j},m_{\scriptscriptstyle \mathcal{C},j})\Big)\\
\end{array}\right]\leq \frac{1}{2}+\mu_{\scriptscriptstyle 1}(\lambda).$$
}




\item For any PPT adversary $\mathcal{A}_{\scriptscriptstyle 2}$ who plays the above game, there exists a negligible function $\mu_{\scriptscriptstyle 2}(\cdot)$ such that


{\small
$$ Pr\left[\mathcal{A}_{\scriptscriptstyle 2}(\bm{c}^{\scriptscriptstyle *},coin^{\scriptscriptstyle *}_{\scriptscriptstyle \mathcal S},coin^{\scriptscriptstyle *}_{\scriptscriptstyle\mathcal C}, g_{\scriptscriptstyle cp},  g_{\scriptscriptstyle qp},\bm{\pi}^{\scriptscriptstyle *}, pl, a)\rightarrow (d_{\scriptscriptstyle j},j)\right]\leq \frac{1}{2}+ \mu_{\scriptscriptstyle 2}(\lambda)$$
}

where $\bm{c}^{\scriptscriptstyle *}=[{c}^{\scriptscriptstyle *}_{\scriptscriptstyle 1},...,{c}^{\scriptscriptstyle *}_{\scriptscriptstyle z}], \bm{\pi}^{\scriptscriptstyle *}=[\pi^{\scriptscriptstyle *}_{\scriptscriptstyle 1},...,\pi^{\scriptscriptstyle *}_{\scriptscriptstyle z}]$, and $ {pp}\in T_{\scriptscriptstyle qp}$.  Also, depending on the choice of $\beta$, algorithm $\mathtt{RCSP}.\mathtt{genQuery}(.)$ outputs an encoded valid query, when the input is $Q_{\scriptscriptstyle 1}$, or outputs an encoded invalid query, when the input is $Q_{\scriptscriptstyle 0}$.
% The probability is taken over uniform  choice of $k,k'\stackrel{\scriptscriptstyle\$}\leftarrow \{0,1\}^{\scriptscriptstyle\lambda}$, where $k,k'\in\bm{k}$, the randomness of $\mathcal{A}_{\scriptscriptstyle 1}$ and $\mathcal{A}_{\scriptscriptstyle 2}$,  the randomness used in witnesses $g_{\scriptscriptstyle cp}, g_{\scriptscriptstyle qp}\in t$,  the randomness used in  $E$.
\end{enumerate}
\end{definition}
 
In the above definition, the experiment is allowed to produce  invalid queries. It is required that  the privacy holds regardless of the queries status, i.e. whether they are valid or invalid, as long as the are encoded and provided.  
 
 
 
%  \begin{definition}[RC-S-P Service Proof Privacy]\label{deff::RC-S-P-ServiceProofPrivacy}  A RC-S-P   preserves the privacy of service proof (status), if for any security parameter $\lambda$, any auxiliary information $\text{aux}$, any $F,Q,M,E,D$ and any probabilistic polynomial time adversary $\mathcal{A}$ that plays the  game in Definition \ref{deff::RC-S-P-Input-Privacy}, there exists a negligible function $\mu(.)$, such that: 
%
%\small{
%$$ Pr\left[\mathcal{A}(\bm{c}^{\scriptscriptstyle *},coin^{\scriptscriptstyle *}_{\scriptscriptstyle \mathcal S},coin^{\scriptscriptstyle *}_{\scriptscriptstyle\mathcal C}, g_{\scriptscriptstyle cp},  g_{\scriptscriptstyle qp},\bm{\pi}^{\scriptscriptstyle *},a)\rightarrow (d_{\scriptscriptstyle j},j)\right]\leq \frac{1}{2}+ \mu(\lambda)$$
%}
%where $\bm{c}^{\scriptscriptstyle *}=[{c}^{\scriptscriptstyle *}_{\scriptscriptstyle 1},...,{c}^{\scriptscriptstyle *}_{\scriptscriptstyle z}]$ and $\bm{\pi}^{\scriptscriptstyle *}=[\pi^{\scriptscriptstyle *}_{\scriptscriptstyle 1},...,\pi^{\scriptscriptstyle *}_{\scriptscriptstyle z}]$.  The probability is taken over uniform  choice of $k,k'\stackrel{\scriptscriptstyle\$}\leftarrow \{0,1\}^{\scriptscriptstyle\lambda}$, where $k,k'\in\bm{k}$, the randomness of $\mathcal{A}$,  the randomness used in witnesses $g_{\scriptscriptstyle cp}, g_{\scriptscriptstyle qp}\in t$,  the randomness used in  $E$.
%\end{definition}
 
 
 
 %-----------------------------------
% \begin{definition}[RC-S-P Service Input Privacy]\label{deff::RC-S-P-Input-Privacy}  A RC-S-P   preserves the privacy of service input, if for any security parameter $\lambda$, any auxiliary information $\text{aux}$, any $F, Q,M, E,D$ and any probabilistic polynomial time adversary $\mathcal{A}$, there exists a negligible function $\mu(.)$, such that: 
%
%\small{
%$$ Pr\left[
%  \begin{array}{l}
%  \mathcal{A}(\bm{c}^{\scriptscriptstyle *},coin^{\scriptscriptstyle *}_{\scriptscriptstyle S},coin^{\scriptscriptstyle *}_{\scriptscriptstyle C},g_{\scriptscriptstyle cp},  g_{\scriptscriptstyle qp},\bm{\pi}^{\scriptscriptstyle *},a)\rightarrow\beta\\
%
%
%\end{array} \middle |
%    \begin{array}{l}
%    \mathtt{RCSP.keyGen}(1^{\lambda},F)\rightarrow \bm{k}\\
%    \mathcal{A}(1^\lambda,pk, F)\rightarrow (u_{\scriptscriptstyle 0},u_{\scriptscriptstyle 1})\\
%    \beta\stackrel{\scriptscriptstyle\$}\leftarrow\{0,1\}\\
%   
%   \mathtt{RCSP}.\mathtt{cInit}(1^\lambda, u_{\scriptscriptstyle\beta},\bm{k}, M,z,cp)\rightarrow (u_{\scriptscriptstyle\beta}^{\scriptscriptstyle *},e,t,coin^{\scriptscriptstyle*}_{\scriptscriptstyle\mathcal C}, \bm{y}_{\scriptscriptstyle \mathcal  C},\bm{y}_{\scriptscriptstyle \mathcal  S})\\
%
%\mathtt{RCSP}.\mathtt{sInit}(u_{\scriptscriptstyle\beta}^{\scriptscriptstyle *},e, pk,z, t,\bm{y}_{\scriptscriptstyle \mathcal  C},\bm{y}_{\scriptscriptstyle \mathcal  S})\rightarrow (coin^{\scriptscriptstyle *}_{\scriptscriptstyle\mathcal S},a)\\
%
%\mathtt{RCSP}.\mathtt{genQuery}(1^\lambda, \text{aux},k,Q,j, en)\rightarrow c^{\scriptscriptstyle *}_{\scriptscriptstyle j}\\
%	
%     
%     \mathtt{RCSP}.\mathtt{prove}(u^{\scriptscriptstyle *}_{\scriptscriptstyle\beta},  \sigma,  c^{\scriptscriptstyle *}_{\scriptscriptstyle j},pk, en)\rightarrow (b_{\scriptscriptstyle j},m_{\scriptscriptstyle \mathcal{S},j},\pi^{\scriptscriptstyle *}_{\scriptscriptstyle j})\\
%     
%   \mathtt{RCSP}.\mathtt{verify}(\pi^{\scriptscriptstyle *}_{\scriptscriptstyle j}, \bm{q}_{\scriptscriptstyle j},k,en)\rightarrow (d_{\scriptscriptstyle j},m_{\scriptscriptstyle \mathcal{C},j})\\
%\end{array}    \right]\leq \frac{1}{2}+\mu(\lambda)$$
%}
%where $\bm{c}^{\scriptscriptstyle *}=[{c}^{\scriptscriptstyle *}_{\scriptscriptstyle 1},...,{c}^{\scriptscriptstyle *}_{\scriptscriptstyle z}]$ and $\bm{\pi}^{\scriptscriptstyle *}=[\pi^{\scriptscriptstyle *}_{\scriptscriptstyle 1},...,\pi^{\scriptscriptstyle *}_{\scriptscriptstyle z}]$.  The probability is taken over uniform  choice of $k,k'\stackrel{\scriptscriptstyle\$}\leftarrow \{0,1\}^{\scriptscriptstyle\lambda}$, where $k,k'\in\bm{k}$, the randomness of $\mathcal{A}$,  the randomness used in witnesses $g_{\scriptscriptstyle cp},g_{\scriptscriptstyle qp}\in t$,  the randomness used in  $E$.
%
%\end{definition}
% 
% 
% 
%  \begin{definition}[RC-S-P Service Proof Privacy]\label{deff::RC-S-P-ServiceProofPrivacy}  A RC-S-P   preserves the privacy of service proof (status), if for any security parameter $\lambda$, any auxiliary information $\text{aux}$, any $F,Q,M,E,D$ and any probabilistic polynomial time adversary $\mathcal{A}$ that plays the  game in Definition \ref{deff::RC-S-P-Input-Privacy}, there exists a negligible function $\mu(.)$, such that: 
%
%\small{
%$$ Pr\left[\mathcal{A}(\bm{c}^{\scriptscriptstyle *},coin^{\scriptscriptstyle *}_{\scriptscriptstyle \mathcal S},coin^{\scriptscriptstyle *}_{\scriptscriptstyle\mathcal C}, g_{\scriptscriptstyle cp},  g_{\scriptscriptstyle qp},\bm{\pi}^{\scriptscriptstyle *},a)\rightarrow (d_{\scriptscriptstyle j},j)\right]\leq \frac{1}{2}+ \mu(\lambda)$$
%}
%where $\bm{c}^{\scriptscriptstyle *}=[{c}^{\scriptscriptstyle *}_{\scriptscriptstyle 1},...,{c}^{\scriptscriptstyle *}_{\scriptscriptstyle z}]$ and $\bm{\pi}^{\scriptscriptstyle *}=[\pi^{\scriptscriptstyle *}_{\scriptscriptstyle 1},...,\pi^{\scriptscriptstyle *}_{\scriptscriptstyle z}]$.  The probability is taken over uniform  choice of $k,k'\stackrel{\scriptscriptstyle\$}\leftarrow \{0,1\}^{\scriptscriptstyle\lambda}$, where $k,k'\in\bm{k}$, the randomness of $\mathcal{A}$,  the randomness used in witnesses $g_{\scriptscriptstyle cp}, g_{\scriptscriptstyle qp}\in t$,  the randomness used in  $E$.
%\end{definition}
 %-------------------------
 
 
 
 \begin{definition}[RC-S-P Security]\label{def::RC-S-P-Security} A RC-S-P scheme is secure if it satisfies security against malicious server, security against malicious client, and preserves privacy, w.r.t. Definitions \ref{deff::RC-S-P-SecurityAgainstMaliciousServer}-\ref{deff::RC-S-P-Privacy}. 
 \end{definition}
 

 
 


\end{document}
