% !TEX root =R-zkCSP.tex

\section{Recurring Contingent PoR Payment (RC-PoR-P) Protocol}


In this section, we present recurring contingent PoR payment (RC-PoR-P) that is a concrete instantiation of the generic recurring contingent service payment (RC-S-P), when the verifiable service is PoR. Nevertheless, RC-PoR-P offers  two primary added features. Specifically, unlike RC-S-P, it (a) does not use any zero-knowledge proofs (even though either client or server can still be  malicious) which significantly improves  costs, and (b)  has a much lower  arbiter-side computation cost; as we will show later, this also allows a smart contract efficiently plays the arbiter's role. In the following, first we  explain how the features are satisfied.

\

\noindent\textit{\textbf{Avoiding the Use of Zero-knowledge Proofs}}. In general, the majority of  PoR's are in the security model where a client is honest while the server is potentially malicious. They rely on metadata that is either a set of tags (e.g. MAC's or signatures) or a root of a Merkle tree,  constructed on  file blocks to ensure the file's availability. In the case where a client can also  be malicious,  if tags are used then using zero-knowledge proofs seem  an obvious choice, as it allows the client to guarantee to the server that the tags have been constructed correctly without leaking verification keys. However, this imposes significant computation and communication costs. We observed that using a Merkle tree would benefit our protocol from a couple of perspectives; in short, it removes the need for zero-knowledge proofs and it supports proof of misbehaviour. Our first observation is that if a Merkle tree is used to generate a metadata, then   there would be no need for the client to use zero-knowledge proofs to prove the correctness of the metadata to the server. Instead, the server can efficiently check the metadata's correctness, by reconstructing the Merkle tree on top of the file blocks. 

\


\noindent\textit{\textbf{Reducing Arbiter-side Cost}}.  As stated above, RC-PoR-P uses a Merkle tree-based PoR. In this case, each $j$-th proof contains a set of Merkle tree paths that are encoded and stored on a smart contract. For each verification, the client decoded and then verifies all paths. In a naive approach, when the client rejects  proofs of $j$-th verification,  it raises a dispute and let the arbiter verify the proofs, i.e. \emph{all paths} in the proofs \footnote{In a Merkle tree-based PoR, the number of proofs that are sent to a verifier for each verification is linear with the number of challenges, e.g. 460 challenges  to ensure 99\% of file blocks is retrievable. In contrast, in a tag-based PoR, in each verification, the verifier receives only a few proofs.}. Instead, we use the idea of proof of misbehaviour, put forth in \cite{}. In particular, in $j$-th verification, if the client detects invalid proofs, then  it  sends  details of only one invalid proof/path to the arbiter  who  decodes that single proof and checks its validity (as apposed to decoding and checking all proofs). This significantly reduces the arbiter computation cost. 


To present RC-PoR-P protocol, we will use the same approach  used to present RC-S-P. In particular, first we present the verifiable service, that is a (modified) Merkle tree-based PoR. Then, we upgrade it to the one that supports  identifiable abort, denoted by PoRID. Next,   we use PoRID to build RC-PoR-P. 


% !TEX root =R-zkCSP.tex



\subsection{Modified Merkle tree-based PoR}\label{protocol::PoR}

In this section, we present a modified version of the standard Merkle tree-based PoR, detonated by PoR. At a high level, the protocol works as follows. The client encodes its file using an error-correcting code, splits the encoded file into blocks, and constructs a Merkle tree on top of the blocks. It locally keeps the tree's root and sends the blocks to the server who rebuilds the tree on the blocks. At a verification time, the client sends a pseudorandom function's key to the server who derives a predetermined number of pseudorandom  indices of the blocks, that indicates which blocks have been challenged. The server for each challenged block generates a Merkle tree proof and sends all proofs to the client. The client, given the root and   key, verifies all  proofs. If all proofs are accepted, then the client outputs $1$ and concludes that its file is retrievable (with a high probability). However, if  it rejects a set of proofs, it outputs $0$ along with an index of the challenged block whose proof was rejected. In the following, we first present the protocol and then elaborate on the  modifications we have applied. 

%, i.e. a verifier outputs (the index of) a rejected proof.  Recall, that in the standard PoR protocol the client is honest and the server is potentially malicious. 

\begin{enumerate}

\item\textbf{Client-side Setup}. $\mathtt{PoR.setup}(1^{\lambda},u)$  
\begin{enumerate}
\item Uses an error correcting code, e.g. Reed-Solomon codes, to encode  the file: $u$. Let $u'$ be the encoded file. It splits $u'$  into  blocks as follows, $u^{\scriptscriptstyle *}=u^{\scriptscriptstyle '}_{\scriptscriptstyle 0}||0,...,u^{\scriptscriptstyle '}_{\scriptscriptstyle m}||m$
\item Generates  metadata: $\sigma$, by constructing Merkle tree on blocks of $u^{\scriptscriptstyle *}$, i.e. $\mathtt{MT.genTree}(u^{\scriptscriptstyle *})$. Let $\sigma$ be the root of the resulting tree, and $\beta$ be a security parameter. It sets public  parameters as $pp:= (\sigma,\beta,m,\zeta)$, where $\zeta:=(\psi,\eta, \iota)$ is a $\mathtt{PRF}$'s description, as it was defined in Section \ref{preliminaries}
\item Sends $pp$ and $u^{\scriptscriptstyle *}$ to $\mathcal{S}$
\end{enumerate}

\

\item\textbf{Client-side Query Generation}. $\mathtt{PoR.genQuery}(1^{\scriptscriptstyle\lambda}, pp)$ 
\begin{enumerate}
\item\label{key-chalenge} Picks a random key $\hat{k}$ of a pseudorandom function $\mathtt{PRF}$, i.e. $\hat{k}\stackrel{\scriptscriptstyle\$}\leftarrow\{0,1\}^{\scriptscriptstyle\psi}$. It ensures the function outputs distinct values, i.e. $\forall i,j\in [0,m]: (\mathtt{PRF}(\hat{k},i)\bmod m+1)\neq(\mathtt{PRF}(\hat{k},j)\bmod m+1)$, where $i\neq j$

\item It sends $\hat{k}$ to $\mathcal{S}$
 
 \end{enumerate}
 
 \
 
\item\label{PoR-server-prove}\textbf{Server-side Proof Generation}. $\mathtt{PoR.prove}(u^{\scriptscriptstyle *},\hat{k},pp)$ 
 \begin{enumerate}
\item Derives $\beta$ pseudorandom indices from $\hat{k}$ as follows. $\forall i,1\leq i\leq \beta: q_{\scriptscriptstyle i}=\mathtt{PRF}(\hat{k},i)\bmod m+1$. Let ${\bm{q}}=[q_{\scriptscriptstyle 1},..., q_{\scriptscriptstyle \beta}]$

\item For each random index $q_{\scriptscriptstyle i}$, generates a Merkle tree proof: $\pi_{\scriptscriptstyle{q_{_{\scriptscriptstyle i}}}}$, by  running Merkle tree proof generator function on $u^{\scriptscriptstyle *}$, i.e. $\mathtt{MT.prove}(u^{*},q_{\scriptscriptstyle i})$. The final result  is $ {\bm{\pi}}=[(u^{\scriptscriptstyle *}_{\scriptscriptstyle q_{_{\scriptscriptstyle i}}},\pi_{\scriptscriptstyle q_{_{\scriptscriptstyle i}}})]_{\scriptscriptstyle q_{\scriptscriptstyle i}\in {\bm{q}}}$, where $i$-th element in $ {\bm{\pi}}$ corresponds to $i$-th pseudorandom value: $q_{\scriptscriptstyle i}$ and   each $\pi_{\scriptscriptstyle{q_{_{\scriptscriptstyle i}}}}$ is path in the tree that proves its corresponding block: $u^{\scriptscriptstyle *}_{\scriptscriptstyle q_{_{\scriptscriptstyle i}}}$ is a leaf node of the  tree. 

\item Sends $ {\bm{\pi}}$ to $\mathcal{C}$
 \end{enumerate}
 
 \
 
\item\label{PoR-verify}\textbf{Client-side Proof Verification}. $\mathtt{PoR.verify}( {\bm{\pi}},\bm{q},pp)$ 
\begin{enumerate}

\item If $|\bm{\pi}|=|\bm{q}|=1$, then set $\beta=1$. This step is taken only in the case where single proof and query is provided to a third-party verifier (e.g. in the case of proof of misbehaviour). 
%\item Regenerates $\beta$ pseudorandom indices that are derived from $\hat{k}$ as follows. $\forall i,1\leq i\leq \beta: q_{\scriptscriptstyle i}=\mathtt{PRF}(\hat{k},i)\bmod m+1$. Let $ {\bm{q}}=[q_{\scriptscriptstyle 1},..., q_{\scriptscriptstyle \beta}]$
\item Checks if the server has sent proofs related to all challenged file blocks. To do that, for all $i$ (where $1\leq i\leq \beta$),  it first parses every  element of $\pi$ as follows, $\mathtt{parse}(u^{\scriptscriptstyle *}_{\scriptscriptstyle q_{_{\scriptscriptstyle i}}})=u^{\scriptscriptstyle '}_{\scriptscriptstyle q_{_{\scriptscriptstyle i}}}||q_{_{\scriptscriptstyle i}}$, and then checks if its index: $q_{_{\scriptscriptstyle i}}$ equals $i$-th element of $ {\bm{q}}$. If all checks pass, then it proceeds to the next step. Otherwise,  it outputs $ {\bm{d}}:[0,i]$, where $i$ refers to the index of the  element in $ {\bm{\pi}}$ that does not pass the check. 


\item Checks if every path in  $ {\bm{\pi}}$ is valid and corresponds to the root, by calling $\mathtt{MT.verify}(u^{\scriptscriptstyle *}_{\scriptscriptstyle q_{_{\scriptscriptstyle i}}},\pi_{\scriptscriptstyle q_{_{\scriptscriptstyle i}}},\sigma)$. If all checks pass, it outputs $ {\bm{d}}=[1,\bot]$ (where $\bot$ denotes empty); otherwise, it outputs $ {\bm{d}}:[0,i]$, where $i$ refers to the index of the  element in $ {\bm{\pi}}$ that does not pass the check. 
\end{enumerate}
\end{enumerate}



\begin{theorem}
The PoR scheme, presented in Section \ref{protocol::PoR}, is $\epsilon$-sound, w.r.t. Definitions \ref{extractable}, if Merkle tree and pseudorandom function $\mathtt{PRF}$, are secure. 

\end{theorem}


 The above protocol differs from the standard Merkle tree-based PoR from two perspectives; First and far most,  in step \ref{PoR-verify} in addition to outputting a binary value, the client outputs only one index of a rejected proof. This will enable any third-party who is given that index (and vectors of proofs and challenges) to verify the client's claim by checking only that proof, i.e. proof of misbehaviour. Second,   in step \ref{key-chalenge} instead of sending $\beta$ challenges, we allow the client to send only a key of a pseudorandom function to the server who can derive a set of challenges form it. This will ultimately  lead to a decrease in costs too, i.e. the client's communication and a smart contract's storage costs.   
 
 \begin{proof}[sketch]
 As stated above, the proposed PoR differs from the standard Merkle tree-based PoR from a couple of perspectives. However, the changes do not affect the security and soundness of the proposed PoR and its security proof is similar to the existing Merkle tree-based PoR schemes, e.g. \cite{DBLP:conf/ccs/HaleviHPS11,MillerPermacoin,DBLP:journals/iacr/JuelsK07}. Alternatively, our protocol can be proven based on the security analysis of the PoR schemes that use  MACs or BLS signatures, e.g. \cite{DBLP:conf/asiacrypt/ShachamW08}. In this case, the extractor design (in the Merkle tree-based PoR)  would be simpler as it does not need to extract blocks from a linear combination of MAC's or signatures, as the  blocks are included in PoR proofs, i.e. they are part of the Merkle tree proofs. 
 
 Intuitively, in either case, the extractor interacts with any adversarial prover that passes non-negligible $\epsilon$ fraction of audits. It initialises an empty array. Then it  challenges a subset of file blocks and asks the prover to generate a proof.  If the received proof passes the verification, then it adds the related block (in the proof) to the array. It then rewinds the prover and challenges a fresh set of blocks, and repeats the process  many times. Since, the prover has a good chance of passing the audit, it is easy to show that the extractor can eventually extract a large fraction of the entire file. Due to the security, i.e. authenticity, of the Merkle tree, the retrieved values are the valid and correct file blocks and due to security of the pseudorandom function, the challenges (or the function's outputs) are not predictable. After collecting sufficient number of blocks, the extractor can use the error correcting code to decode and recover the entire file blocks, given the retrieved ones. 
 \hfill\(\Box\)\end{proof}
   
   \begin{remark}
   Recall,  the generic definition of a verifiable service scheme (i.e. Definition \ref{service-def}) involves three algorithms: $F$, $M$, and $Q$. However, the three algorithms are implicit in the original definition of  PoR and accordingly in  PoR protocols. In the following, we explain how each algorithm is defined in PoR context.  $M$ is an algorithm that processes a file and generates metadata. For instance, when PoR uses a Merkle tree (to ensure the file's integrity and availability), then $M$ refers to the Merkle tree's algorithm that constructs a tree on top of the file blocks. Also,  $F$ is an algorithm that, during generating a PoR proof, processes a subset of the outsourced file, given the client's query (or challenged file blocks). For instance, if a PoR  utilises a Merkle tree, then $F$ refers to the algorithm that generates Merkle tree's proofs, i.e.  membership of the challenged file blocks. Furthermore, $Q$ can be  a pseudorandom function that generates a set of pseudorandom strings in a certain range, e.g. file block's indices.   
   \end{remark}



% !TEX root =R-zkCSP.tex


\subsection{PoRID Protocol}

In this section, we propose ``PoR with identifiable abort'' (PoRID) that is a concrete instantiation of $\text{VSID}_{\scriptscriptstyle\text{light}}$. It is built upon the PoR protocol, presented in the previous section and is in the same security model as $\text{VSID}_{\scriptscriptstyle\text{light}}$ is, i.e. either $\mathcal C$ or $\mathcal S$ can be malicious. In PoRID similar to $\text{VSID}_{\scriptscriptstyle\text{light}}$, $\mathcal C$ and $\mathcal S$ use a bulletin board to exchange signed messages.  In the protocol, at setup $\mathcal C$ encodes its file and generates public parameters and  metadata.  It posts the public parameters and metadata to the bulletin board and   sends the encoded file to $\mathcal S$ who runs a few lightweight checks to ensure the correctness of the public parameters and metadata. It agrees to serve, if it is convinced of their correctness. Later, when $\mathcal C$ wants to ensure the availability of its outsourced file, it  generates and posts a query  to the board.   $\mathcal S$  checks the correctness of the query, by performing a couple of highly efficient verifications. 

The server-side prove and client-side verify algorithms are similar to those in PoR with a difference that $\mathcal S$ posts the PoR proofs (i.e. output of prove algorithm) to the board. In  case of any dispute, $\mathcal C$ or $\mathcal S$ invokes the arbiter who, given the signed posted messages, checks the proofs  to identify a corrupt party. In particular,  it first checks the validity of the query (regardless of the party who invokes it). However, if  is invoked by  $\mathcal C$, it also checks  only one of the PoR proofs that the client claims it is invalid. Thus, it is much more efficient than $\text{VSID}_{\scriptscriptstyle\text{light}}$ as it does not need any zero-knowledge proofs (mainly due to the use of Merkle tree) and requires the  arbiter to check only one of the proofs (due  to the idea of proof of misbehaviour). PoRID protocol is presented below.






\begin{enumerate}
   
\item\textbf{Client-side Setup}. $\mathtt{PoRID.setup}(1^{\lambda}, u)$
\begin{enumerate}
\item Calls  $\mathtt{PoR.setup}(1^{\lambda}, u)\rightarrow (u^{\scriptscriptstyle *},pp)$, that results in public parameters $pp:=(\sigma,\beta,m,\zeta)$ and encoded file: $u^{\scriptscriptstyle *}=u^{\scriptscriptstyle '}_{\scriptscriptstyle 0}||0,...,u^{\scriptscriptstyle '}_{\scriptscriptstyle m}||m$. Recall,  $\zeta:=(\psi,\eta, \iota)$ is the $\mathtt{PRF}$'s description.

\item Posts $pp$ to the bulletin board and sends $u^{\scriptscriptstyle *}$ to $\mathcal{S}$ 
\end{enumerate}

\

\item\textbf{Server-side Setup}. $\mathtt{PoRID.serve}(u^{\scriptscriptstyle *},pp)$
%\begin{enumerate}

Verifies the correctness of public parameters:
\begin{enumerate}
\item rebuilds  the Merkle tree on $u^{\scriptscriptstyle *}$ and checks the resulting root equals $\sigma$
\item checks $|u^{*}|=m$ and  $\beta\leq m$
\end{enumerate}
 If the proofs are accepted, then  it outputs $a = 1$ and proceeds to the next step; otherwise, it outputs $a = 0$ and halts.
 
 


\

\item\textbf{Client-side Query Generation}. $\mathtt{PoRID.genQuery}(1^{\lambda},  pp)$ 
\begin{enumerate}
\item Calls $\mathtt{PoR.genQuery}(1^{\lambda}, pp)\rightarrow \hat{k}$, to generate a key, $\hat{k}$ %whose element have been sorted 
\item Posts $\hat{k}$ to the board. 
\end{enumerate}


\

\item\textbf{Server-side Query Verification}. $\mathtt{PoRID.checkQuery}(\hat{k}, pp)$
\begin{enumerate}
\item  Checks if  $\hat{k}$ is not empty, i.e. $\hat{k}\neq \bot$, and is in the key's universe, i.e. $\hat{k}\in \{0,1\}^{\scriptscriptstyle\psi}$%If it is empty, then it outputs $I=\mathcal C$
\item  If the checks pass, then it outputs $b=1$; otherwise, it outputs $b=0$


\end{enumerate}


\

\item\textbf{Server-side Service Proof Generation}. $\mathtt{PoRID.prove}(u^{\scriptscriptstyle *},\hat{k},pp)$ 
\begin{enumerate}
\item Calls $\mathtt{PoR.prove}(u^{\scriptscriptstyle *},\hat{k},pp)\rightarrow  {\bm{\pi}}$, to generate proof vector: $ {\bm{\pi}}$ 
\item Posts $ {\bm{\pi}}$ to the board. 
\end{enumerate}

\

\item\textbf{Client-side Proof Verification}\label{PoRID::Client-side-Proof-Verification}. $\mathtt{PoRID.verify}( {\bm{\pi}},\hat{k} ,pp)$
%\begin{enumerate}

 Calls $\mathtt{PoR.verify}( {\bm{\pi}},\hat{k} ,pp)\rightarrow  {\bm{d}}$, to verify the proof. If $ {\bm{d}}[0] = 1$,  it accepts the proof; otherwise, it rejects it.


\

\item\textbf{Arbiter-side Identification}\label{PoRID::Arbiter-side-Identification}. $\mathtt{PoRID.identify}( {\bm{\pi}}, g,\hat{k},pp)$ 

This algorithm can be invoked by $\mathcal{C}$ or $\mathcal{S}$, in the case of dispute. If it is invoked by $\mathcal{C}$, then $g$ refers to a rejected proof's  index; however, if it is invoked by $\mathcal{S}$, then $g$ is null, i.e.  $g=\bot$. The arbiter performs as follows. 


\begin{enumerate}



\item Ensures  query $\hat{k}$ is well-structured by calling $\mathtt{PoRID.checkQuery}(\hat{k}, pp)$. If it returns $b=0$, then it outputs $I=\mathcal C$ and halts; otherwise, it proceeds to the next step. 




\item Derives the related challenged block's index from $\hat{k}$, by computing $q_{\scriptscriptstyle g}=\mathtt{PRF}(\hat{k},g)\bmod m+1$ 



\item If $g\neq \bot$, then verifies only $g$-th proof, by setting $\hat{\bm{\pi}}={\bm{\pi}}[g], \hat{\bm{q}}=q_{\scriptscriptstyle g}$ and then calling $\mathtt{PoR.verify}( \hat{\bm{\pi}},\hat{\bm{q}},pp)\rightarrow  {\bm{d}}'$. If $ {\bm{d}}'[0]=0$, then it outputs $I=\mathcal S$. Otherwise, it outputs $I=\bot$
\end{enumerate}
\end{enumerate}

\begin{theorem}
The PoRID protocol satisfies the $\epsilon$-soundness, inputs well-formedness, and detectable abort properties, w.r.t. Definitions \ref{extractable}, \ref{deff::VSID-Inputs-Well-formedess}, and \ref{def::VSID-Identifiable-Abort}, if PoR is $\epsilon$-sound and the  signature scheme is secure.
\end{theorem}


\begin{proof}[sketch]
The $\epsilon$-soundness of PoRID directly stems from the security of PoR scheme, i.e. $\epsilon$-soundness. Specifically,
in PoRID the (honest) client makes black-box calls to the algorithms of PoR, to ensure the soundness. The latter scheme's soundness ensures that an extractor can recover the entire file interacting with a corrupt server who passes  $\epsilon$ fraction of challenges. On the other hand, the inputs well-formedness holds for the following reasons. The metadata generation algorithm, i.e. the Merkle tree algorithm that builds a tree and computes a root, is deterministic and involves only public  parameters. Thus, given the tree's leaves (i.e. file blocks), its parameters, and the root, anyone can reconstruct it, check if it yields the same root, and verify the tree's parameters. Also, a query contains a single random key, $\hat{k}$, whose correctness can be checked deterministically, i.e. by checking $\hat{k}\neq \bot$ and  $\hat{k}\in \{0,1\}^{\scriptscriptstyle\psi}$. The detectable abort property holds as long as the soundness and inputs well-formedness hold and the signature scheme is secure. The reason is that algorithm $\mathtt{PoRID.identify}(.)$, which ensures detected abort, is a wrapper function that makes black-box calls to algorithms $\mathtt{PoRID.checkQuery}(.)$ and $\mathtt{PoR.verify}(.)$, where the former  ensures input (i.e. query) well-formedness, and the latter  ensures soundness.  Although the number of proofs that are passed to $\mathtt{PoR.verify}(.)$ in phase \ref{PoRID::Client-side-Proof-Verification}, differs from the number of proofs passed to the same algorithm in phase \ref{PoRID::Arbiter-side-Identification}, the difference does not affect the security, as due to the security of PoR (i.e.  Merkle tree) an invalid proof is detected with the same probability in both phases. The signature’s security ensures if a proof is not signed correctly, then it can also be rejected by the arbiter and the signer is held accountable for providing an ill-formed message; on the other hand, if a proof is signed correctly, then it cannot be repudiated by the signer later on that guarantees the signer is held accountable for a rejected proof it provides. 
\hfill\(\Box\)\end{proof}






% !TEX root =R-zkCSP.tex

\subsection{Recurring Contingent PoR Payment (RC-PoR-P) Protocol}\label{sec::RC-PoR-P-protocol}



In this section, we present RC-PoR-P,  a concrete instantiation of the generic RC-S-P.  Although RC-PoR-P and RC-S-P have some overlaps, they have many differences too. Therefore, we provide the protocol’s overview and its detailed description below.
At a high level the protocol works as follows. The client and server utilise SAP to provably agree on two private statements, one statement includes payment details, and another one specifies a secret key, $\bar{k}$, and a pad's details, that will be used to encode sensitive messages they send to the contract. Moreover, they  agree on public parameters such as the private time bubble's length (that is the total number of billing cycles: $z$, plus a waiting period, $\texttt{H}$) and a smart contract that specifies $z$ and the total amount of masked coins each party should deposit. They deploy the contract. Each party deposits its masked coins in the contract within a fixed time. If a party does not deposit enough coins on time, then the parties have a chance to withdraw their coins and terminate the contract after a certain time. 

To start using/providing the service, the client invokes $\mathtt{PoRID.setup}(.)$ to encode the file and generate  metadata and public parameters. It sends  encryption of the metadata and public parameters to the smart contract. Also, it  sends the encoded file to the server who decrypts them and using the encoded file checks their correctness by calling $\mathtt{PoRID.server}(.)$. If the server decides not to serve, it sends to the contract $0$ within a fixed time; in this case, the parties can withdraw their deposit and terminate the contract.  At the end of each billing cycle, the client generates a query, by calling $\mathtt{PoRID.genQuery(.)}$ and sends the query's encryption to the contract. In the same cycle, the server retrieves the query from the contract, decrypts and locally checks its correctness, by calling $\mathtt{PoRID.checkQuery(.)}$. If the query is rejected, the server locally stores the index of that billing cycle and generates  dummy PoR proofs. However, if the server accepts the query, it generates PoR proofs by calling $\mathtt{PoRID.prove(.)}$. Then, in either case, the server encrypts the proofs, pads them and sends the result to the contract. Next, the client removes the pads, decrypts the proofs and locally verifies them, by calling $\mathtt{PoRID.verify(.)}$. If the verification is passed, then the client knows the file is retrievable with a high probability. But, if the proof is rejected, then it locally stores the index of that billing cycle and details of one of the invalid proofs (in the same cycle). It  waits until the private time bubble passes and dispute resolution time arrives. 

During the dispute resolution period, in case the server rejects the query or the client rejects the PoR proofs, that party sends to the arbiter (a) details of invalid proofs and (b)  the statement that contains the decryption key and padding detail. The arbiter checks the validity of the statement first. If it accepts the statement, then it removes the pads and decrypts the values whose indices were provided by the parties. Then, the arbiter checks the party’s claim by calling $\mathtt{PoRID.checkQuery (.)}$ and $\mathtt{PoRID.identify (.)}$ if the server or client calls the contract respectively. The arbiter also keeps track of the number of times each party misbehaved.  After the arbiter processes the parties' claim, it tells to the contract how many times each party misbehaved.  In the next phase, to distribute the coins, either client or server sends to the contract: (a) “pay” message, (b) the agreed statement that specifies the payment details, and (c) the statement’s proof. The contract verifies the statement and if it is approved, then it distributes the coins according to the statement’s detail, and the number of times each party misbehaved. 



\begin{enumerate}
\item\label{RC-PoR-P::KeyGen} \textbf{Key Generation}. $\mathtt{RCPoRP}.\mathtt{keyGen}(1^{\scriptscriptstyle\lambda})$
\begin{enumerate}

\item  $\mathcal C$  picks a random secret key  $\bar{k}$ for a symmetric key encryption. It also sets  parameter $pad_{\scriptscriptstyle \pi}$ which is the number of  dummy values that will be used to pad encrypted proofs,  let $qp:=(pad_{\scriptscriptstyle \pi}, \bar{k})$. The key's size is part of the security parameter. Let  ${k}:=(sk',pk')$, where  $sk'=qp$ and $pk':=(adr_{\scriptscriptstyle\mathcal{C}},adr_{\scriptscriptstyle\mathcal{S}})$. 





\end{enumerate}

\

\item \textbf{Client-side Initiation}. $\mathtt{RCPoRP}.\mathtt{cInit}(1^\lambda, u,{k}, z,pl)$
\begin{enumerate}

\item Calls $\mathtt{PoRID.setup}(1^{\lambda}, u)\rightarrow (u^{\scriptscriptstyle *},pp)$ to encode service input. It appends $pp$ to $qp$.


\item  Calls $\mathtt{SAP.init}(1^{\scriptscriptstyle\lambda}, adr_{\scriptscriptstyle\mathcal{C}}, adr_{\scriptscriptstyle\mathcal{S}},qp )\rightarrow(r_{\scriptscriptstyle qp}, g_{\scriptscriptstyle qp}, adr_{\scriptscriptstyle\text{SAP}_{\scriptscriptstyle 1}})$, to initiate an agreement (with $\mathcal{S}$) on $qp$.  Let $T_{\scriptscriptstyle qp}:=(\ddot{x}_{\scriptscriptstyle qp}, g_{\scriptscriptstyle qp})$ be proof/query encoding token,  where  $\ddot{x}_{\scriptscriptstyle qp}:=(qp,r_{\scriptscriptstyle qp})$ is the opening and $g_{\scriptscriptstyle qp}$ is the commitment stored on the contract as a result of running   SAP. 


\item   Sets coin parameters (given  price list $pl$) as  follows,  $o$: the amount of coins for each accepting proof,  and $l$ the amount of coin $\mathcal C$  or $\mathcal S$ needs to send to a smart contract to resolve a potential dispute. 






\item\label{RC-PoR-P::set-cp}  Sets  $cp:=(o,o_{\scriptscriptstyle max}, l,  l_{\scriptscriptstyle max},z)$, where $o_{\scriptscriptstyle max}$ is the maximum amount of coins  for an accepting service proof,  $l_{\scriptscriptstyle max}$  is the maximum amount of coins to resolve a potential dispute, and $z$ is the number of  service proofs/verifications. Then, $\mathcal C$ calls $\mathtt{SAP.init}(1^{\scriptscriptstyle\lambda}, adr_{\scriptscriptstyle\mathcal{C}}, adr_{\scriptscriptstyle\mathcal{S}},cp )\rightarrow(r_{\scriptscriptstyle cp},g_{\scriptscriptstyle cp},adr_{\scriptscriptstyle\text{SAP}_{\scriptscriptstyle 2}})$, to initiate an agreement (with $\mathcal{S}$) on $cp$. Let $T_{\scriptscriptstyle cp}:=(\ddot{x}_{\scriptscriptstyle cp},g_{\scriptscriptstyle cp})$ be coin encoding token,  where  $\ddot{x}_{\scriptscriptstyle cp}:=(cp,r_{\scriptscriptstyle cp})$ is the opening and $g_{\scriptscriptstyle cp}$ is the commitment stored on the contract as a result of executing   SAP.  





\item Set parameters  $coin^{\scriptscriptstyle *}_{\scriptscriptstyle\mathcal C}=z\cdot (o_{\scriptscriptstyle max}+l_{\scriptscriptstyle max})$ and $p_{\scriptscriptstyle\mathcal{S}}=z\cdot l_{\scriptscriptstyle max}$, where $coin^{\scriptscriptstyle *}_{\scriptscriptstyle\mathcal C}$ and $p_{\scriptscriptstyle\mathcal{S}}$ are   the total number of masked coins $\mathcal C$ and $\mathcal S$ should deposit respectively. It signs and deploys a smart contract, SC, that explicitly specifies  parameters $z$,  $coin^{\scriptscriptstyle *}_{\scriptscriptstyle\mathcal C}$, $p_{\scriptscriptstyle\mathcal{S}}$,  $adr_{\scriptscriptstyle\text{SAP}_{\scriptscriptstyle 1}}$, and $adr_{\scriptscriptstyle\text{SAP}_{\scriptscriptstyle 2}}$. Let $adr_{\scriptscriptstyle \text{SC}}$ be the address of the deployed SC, and $T:=\{T_{\scriptscriptstyle qp},T_{\scriptscriptstyle cp}\}$. It sets a set of time points/windows, $\texttt{Time}:\{ \texttt{T}_{\scriptscriptstyle 0},..., \texttt{T}_{\scriptscriptstyle 3},\texttt{G}_{\scriptscriptstyle 1,1},...,\texttt{G}_{\scriptscriptstyle z,2},\texttt{H}, \texttt{K}_{\scriptscriptstyle 1},...,\texttt{K}_{\scriptscriptstyle 6}, \texttt{L}\}$, that are explicitly specified in SC who will accept a certain party's message only in a specified  time point/window. Time allocations will become clear in the next phases. 


\item\label{setcounters} Sets four counters $[y_{\scriptscriptstyle\mathcal C},y'_{\scriptscriptstyle\mathcal C},y_{\scriptscriptstyle\mathcal S},  y'_{\scriptscriptstyle\mathcal S}]$ in the contract, where their initial value is $0$. Let $\bm{y}: [y_{\scriptscriptstyle\mathcal C},y'_{\scriptscriptstyle\mathcal C},y_{\scriptscriptstyle\mathcal S},  y'_{\scriptscriptstyle\mathcal S},\texttt{Time},adr_{\scriptscriptstyle \text{SC}}]$

\item Deposits $coin^{\scriptscriptstyle *}_{\scriptscriptstyle\mathcal C}$ coins in the contract. It  sends $u^{\scriptscriptstyle *}, e, \ddot{x}_{\scriptscriptstyle qp},\ddot{x}_{\scriptscriptstyle cp}$, and $p_{\scriptscriptstyle\mathcal{S}}$ (along with $adr_{\scriptscriptstyle \text{SC}}$) to $\mathcal S$. It also sends  $pk'$ to SC. Let $\texttt{T}_{\scriptscriptstyle 0}$ be the time that the above process finishes. 




\end{enumerate}

\
\item \textbf{Server-side Initiation}\label{RCPoRP::Server-side-Initiation}. $\mathtt{RCPoRP}.\mathtt{sInit}(u^{\scriptscriptstyle *},  z, T, p_{\scriptscriptstyle\mathcal{S}} ,\bm{y})$

\begin{enumerate}

\item Checks the parameters  in $T$ (e.g. $qp$ and $cp$) and  in SC (e.g. $p_{\scriptscriptstyle\mathcal{S}}, \bm{y}$) and ensures sufficient amount of coins has been deposited by $\mathcal C$.  

\item Calls $\mathtt{SAP.agree}(qp,r_{\scriptscriptstyle qp},g_{\scriptscriptstyle qp},adr_{\scriptscriptstyle\mathcal{C}},adr_{\scriptscriptstyle\text{SAP}_{1}})\rightarrow (g'_{\scriptscriptstyle qp},b_{\scriptscriptstyle 1})$ and $\mathtt{SAP.agree}(cp,r_{\scriptscriptstyle cp},g_{\scriptscriptstyle cp},adr_{\scriptscriptstyle\mathcal{C}},adr_{\scriptscriptstyle\text{SAP}_{2}})\rightarrow (g'_{\scriptscriptstyle cp},b_{\scriptscriptstyle 2})$, to verify the correctness of tokens in $T$ and to agree on the tokens' parameters, where $qp,r_{\scriptscriptstyle qp}\in \ddot{x}_{\scriptscriptstyle qp}, \text{and }  cp,r_{\scriptscriptstyle cp} \in  \ddot{x}_{\scriptscriptstyle cp}$. Recall, if both $\mathcal{C}$ and $\mathcal{S}$ are honest, then $g_{\scriptscriptstyle qp}=g'_{\scriptscriptstyle qp}$ and $g_{\scriptscriptstyle cp}=g'_{\scriptscriptstyle cp}$. 




\item If any above check is rejected, then it sets $a=0$. Otherwise, it  calls $\mathtt{PoRID.serve}(u^{\scriptscriptstyle *}, {pp})\rightarrow a$. 

\item Sends $a$ and $coin^{\scriptscriptstyle *}_{\scriptscriptstyle\mathcal S}=p_{\scriptscriptstyle\mathcal{S}}$ coins to SC at time $\texttt{T}_{\scriptscriptstyle 1}$, where  $coin^{\scriptscriptstyle *}_{\scriptscriptstyle\mathcal S}=\bot$ if $a=0$
\end{enumerate}
Note, $\mathcal S$  and  $\mathcal C$ can withdraw their coins at time $\texttt{T}_{\scriptscriptstyle 2}$, if  $\mathcal S$ sends $a=0$, fewer coins than $p_{\scriptscriptstyle\mathcal{S}}$, or nothing to the SC. To withdraw, $\mathcal S$  or  $\mathcal C$  simply sends a ``pay'' message to $\mathtt{RCPoRP}.\mathtt{pay}(.)$ algorithm only at time $\texttt{T}_{\scriptscriptstyle 2}$.

\




\textbf{\textit{Billing-cycles Onset}}. $\mathcal{C}$ and $\mathcal{S}$ engage in the following three phases, i.e. phase \ref{Billing-cycles-genQuery}-\ref{Client-sideProofVerification},  at the end of every $j$-th billing cycle, where $1\leq j\leq z$. Each $j$-th cycle includes two  time points, $\texttt{G}_{\scriptscriptstyle j,1}$ and $\texttt{G}_{\scriptscriptstyle j,2}$, where $\texttt{G}_{\scriptscriptstyle j,2}>\texttt{G}_{\scriptscriptstyle j,1}$, and $\texttt{G}_{\scriptscriptstyle 1,1}>\texttt{T}_{\scriptscriptstyle 2}$ 


\

\item\textbf{Client-side Query Generation}.\label{Billing-cycles-genQuery} $\mathtt{RCPoRP}.\mathtt{genQuery}(1^\lambda,  T_{\scriptscriptstyle qp})$

\begin{enumerate}

\item Calls $\mathtt{PoRID.genQuery}(1^{\lambda},  pp)\rightarrow \hat{k}_{\scriptscriptstyle j}$ to generate a query, where  $ pp\in T_{\scriptscriptstyle qp}$


\item Sends encrypted query  $\hat{k}^{\scriptscriptstyle *}_{\scriptscriptstyle j}=\mathtt{Enc}(\bar{k},\hat{k}_{\scriptscriptstyle j})$ to SC at time $\texttt{G}_{\scriptscriptstyle j,1}$


\end{enumerate}

\

\item\textbf{Server-side Proof Generation}. $\mathtt{RCPoRP}.\mathtt{prove}(u^{\scriptscriptstyle *},  \hat{k}^{\scriptscriptstyle *}_{\scriptscriptstyle j}, T_{\scriptscriptstyle qp})$


\begin{enumerate}

\item Constructs an empty  vector, $\bm{m}_{\scriptscriptstyle\mathcal S}=\bot$, if $j=1$.  

\item Decrypts the query, $\hat{k}_{\scriptscriptstyle j}=\mathtt{Dec}(\bar{k},\hat{k}^{\scriptscriptstyle *}_{\scriptscriptstyle j})$

\item Calls $\mathtt{PoRID.checkQuery}(\hat{k}_{\scriptscriptstyle j}, pp)\rightarrow b_{\scriptscriptstyle j}$ to check the query's correctness.

\begin{itemize}
\item[$\bullet$] If it accepts the query, then it calls $\mathtt{PoRID.prove}(u^{\scriptscriptstyle *},\hat{k}_{\scriptscriptstyle j},pp)\rightarrow  {\bm{\pi}}_{\scriptscriptstyle j}$, to generate a PoR  proof. In this case,  $\mathcal S$ encrypts every  proof in the proof vector, i.e. $\forall g, 1\leq g\leq| {\bm{\pi}}_{\scriptscriptstyle j}|: \mathtt{Enc}(\bar{k}, {\bm{\pi}}_{\scriptscriptstyle j}[g])= {\bm{\pi}}'_{\scriptscriptstyle j}[g]$, where $\bar{k}\in T_{\scriptscriptstyle qp}$. Let vector $ {\bm{\pi}}'_{\scriptscriptstyle j}$ contain the encryption of all proofs. It pads every encrypted proof in $ {\bm{\pi}}'_{\scriptscriptstyle j}$ with ${pad}_{\scriptscriptstyle \pi}\in T_{\scriptscriptstyle qp}$ random values that are picked from the encryption's output range $U$, (by appending the random values to the encrypted proofs vector). Let $\bm{\pi}^{\scriptscriptstyle *}_{\scriptscriptstyle j}$ be the result. It sends the padded encrypted proofs  to SC at time $\texttt{G}_{\scriptscriptstyle j,2}$ 
\item[$\bullet$] Otherwise (if $\mathcal S$ rejects the query), it appends $j$  to $ {\bm{m}}_{\scriptscriptstyle\mathcal{S}}$,  constructs a dummy proof $ {\bm{\pi}}'_{\scriptscriptstyle j}$ whose elements are randomly picked from $U$,  pads the result as above, and sends the result, $\bm{\pi}^{\scriptscriptstyle *}_{\scriptscriptstyle j}$, to SC at time $\texttt{G}_{\scriptscriptstyle j,2}$

\end{itemize}

When $j=z$  and $\bm{m}_{\scriptscriptstyle\mathcal{S}}\neq\bot$, it sets    $m_{\scriptscriptstyle\mathcal S}:=\bm{m}_{\scriptscriptstyle\mathcal{S}}$.

\end{enumerate}




\


\item \textbf{Client-side Proof Verification}.\label{Client-sideProofVerification} $\mathtt{RCPoRP}.\mathtt{verify}(\bm{\pi}^{\scriptscriptstyle *}_{\scriptscriptstyle j}, \hat{k}_{\scriptscriptstyle j},T_{\scriptscriptstyle qp})$
\begin{enumerate}


\item Constructs an empty vector, $\bm{m}_{\scriptscriptstyle\mathcal C}=\bot$, if $j=1$. 

\item Removes the pads from $\bm{\pi}^{\scriptscriptstyle *}_{\scriptscriptstyle j}$, utilising  parameters of $T_{\scriptscriptstyle qp}$. Let $\bm{\pi}'_{\scriptscriptstyle j}$ be the result. It decrypts the service proofs $\mathtt{Dec}(\bar{k},\bm{\pi}'_{\scriptscriptstyle j})=\bm{\pi}_{\scriptscriptstyle j}$ and then calls $\mathtt{PoRID.verify}(\bm{\pi}_{\scriptscriptstyle j}, \hat{k}_{\scriptscriptstyle j}, {pp})\rightarrow \bm{d}_{\scriptscriptstyle j}$, to verify the proof.  

\begin{itemize}
\item[$\bullet$] If $ {\bm{\pi}}_{\scriptscriptstyle j}$ passes the verification, i.e. $ {\bm{d}}_{\scriptscriptstyle j}[0]=1$, then $\mathcal C$  concludes that the service for this verification has been delivered successfully.  


\item[$\bullet$] Otherwise (if proof $ {\bm{\pi}}_{\scriptscriptstyle j}$ is rejected, i.e. $ {\bm{d}}_{\scriptscriptstyle j}[0]=0$),  it sets $g= {\bm{d}}_{\scriptscriptstyle j}[1]$ and appends vector $[j,g]$  to $ {\bm{m}}_{\scriptscriptstyle\mathcal{C}}$. Recall, $ {\bm{d}}_{\scriptscriptstyle j}[1]$ refers to a rejected proof's index in proof vector $ {\bm{\pi}}_{\scriptscriptstyle j}$. 

\end{itemize}
\end{enumerate}
When $j=z$  and $\bm{m}_{\scriptscriptstyle\mathcal{C}}\neq\bot$, $\mathcal C$ sets    ${m}_{\scriptscriptstyle\mathcal C}:=\bm{m}_{\scriptscriptstyle\mathcal{C}}$.




\




\item \textbf{Dispute Resolution}. \label{RCPoRP::Dispute-Resolution} $\mathtt{RCPoRP}. \mathtt{resolve}(m_{\scriptscriptstyle \mathcal{C}},m_{\scriptscriptstyle \mathcal{S}}, z, {\bm{\pi}}^{\scriptscriptstyle *}, {\bm{q}^{\scriptscriptstyle *}},  T_{\scriptscriptstyle qp})$



The phase takes place only in  case of dispute, i.e. when $\mathcal C$ rejects service proofs or $\mathcal S$ rejects the queries. %or either party does not receive its counterparty's message on time. %A party who accepts all proofs does not need to get involved in this phase. 

\begin{enumerate}



\item The arbiter sets  counters: $y_{\scriptscriptstyle\mathcal C}, y'_{\scriptscriptstyle\mathcal C},y_{\scriptscriptstyle\mathcal S}$ and  $y'_{\scriptscriptstyle\mathcal S}$, that are initially set to $0$,  before time $\texttt{K}_{\scriptscriptstyle 1}$,  where $\texttt{K}_{\scriptscriptstyle 1}>\texttt{G}_{\scriptscriptstyle z,2}+\texttt{H}$. 

\item $\mathcal S$  sends ${m}_{\scriptscriptstyle\mathcal S}$ and $\ddot{x}_{\scriptscriptstyle qp}$ to the arbiter, at time $\texttt{K}_{\scriptscriptstyle 1}$.

\item The arbiter after receiving ${m}_{\scriptscriptstyle\mathcal S}$, does the following at time $\texttt{K}_{\scriptscriptstyle 2}$. 
\begin{enumerate}




\item Checks the validity of  statement $\ddot{x}_{\scriptscriptstyle qp}$, by  sending it to SAP contract which  returns   $1$ or $0$. If the output is $0$, then it discards the server's complaint, $m_{\scriptscriptstyle\mathcal S}$, and does not take steps \ref{RCPoRP::check-v} and \ref{sc-check-server-claim}. Otherwise, it proceeds to the next step.



\item\label{RCPoRP::check-v} Removes from $ {\bm{v}}_{\scriptscriptstyle\mathcal{S}}$ any element that is duplicated or is not in the range $[1,z]$. It also constructs an empty vector $ {\bm{v}}$.

\item\label{sc-check-server-claim}  For any element $i\in {\bm{v}}_{\scriptscriptstyle\mathcal{S}}$:
\begin{itemize}

\item[$\bullet$] Fetches the related encrypted query $\hat{k}^{\scriptscriptstyle *}_{\scriptscriptstyle i}\in \bm{q}^{\scriptscriptstyle *}$   from SC, and   decrypts it, $\hat{k}_{\scriptscriptstyle i}=\mathtt{Dec}(\bar{k},\hat{k}^{\scriptscriptstyle *}_{\scriptscriptstyle i})$

%\item[$\bullet$] Fetches the related query, $\hat{k}_{\scriptscriptstyle i}\in \bm{q}$,   from SC 
\item[$\bullet$]Checks if  the query is well-formed, by calling $\mathtt{PoRID.checkQuery}(\hat{k}_{\scriptscriptstyle i}, pp)\rightarrow b_{\scriptscriptstyle i}$. If the query is rejected, i.e.  $b_{\scriptscriptstyle i}=0$, then it increments $y_{\scriptscriptstyle\mathcal C}$ by $1$ and appends $i$ to $ {\bm{v}}$. Otherwise (if the query is accepted) it increments $y'_{\scriptscriptstyle\mathcal S}$  by $1$. 
\end{itemize}
Let $\texttt{K}_{\scriptscriptstyle 3}$ be the time the arbiter finishes the above checks.
\end{enumerate}





\item $\mathcal C$  sends $m_{\scriptscriptstyle\mathcal C}$ and $\ddot{x}_{\scriptscriptstyle qp}$ to the arbiter, at time $\texttt{K}_{\scriptscriptstyle 4}$




\item The arbiter after receiving $m_{\scriptscriptstyle\mathcal C}$, does the following, at time $\texttt{K}_{\scriptscriptstyle 5}$. 
\begin{enumerate}
\item Checks the validity of  statement $\ddot{x}_{\scriptscriptstyle qp}$, by   sending $\ddot{x}_{\scriptscriptstyle qp}$ to SAP contract which  returns  either $1$ or $0$. If the output is $0$, then it discards the client's complaint, ${m}_{\scriptscriptstyle\mathcal C}$, and does not take steps \ref{sc-checks-client-claim}-\ref{the-contract-loop}. Otherwise, it proceeds to the next step. 

 \item\label{sc-checks-client-claim} Ensures each vector   $ {\bm{m}}\in  {\bm{m}}_{\scriptscriptstyle\mathcal{C}}$ is well-formed. In particular, it ensures there exist no two vectors: $ {\bm{m}}, {\bm{m}}'\in  {\bm{m}}_{\scriptscriptstyle\mathcal{C}}$ such that $ {\bm{m}}[0]= {\bm{m}}'[0]$.  If such vectors  exist, it deletes the redundant ones from $ {\bm{m}}_{\scriptscriptstyle\mathcal{C}}$. This ensures no two claims refer to the same verification.  It removes any vector $ {\bm{m}}$ from $ {\bm{m}}_{\scriptscriptstyle\mathcal{C}}$ if $ {\bm{m}}[0]$ is not in the range $[1,z]$ or if $ {\bm{m}}[0]\in  {\bm{v}}$. Note the latter check  (i.e. $ {\bm{m}}[0]\in  {\bm{v}}$) ensures $\mathcal C$ cannot hold $\mathcal S$ accountable if $\mathcal C$  generated an ill-formed query for the same verification.   



\item\label{the-contract-loop} For every  vector $ {\bm{m}}\in {\bm{m}}_{\scriptscriptstyle \mathcal{C}}$:
\begin{itemize}
\item[$\bullet$]  Retrieves details of a proof that was rejected in each $i$-th verification.  In particular, it sets   $i= {\bm{m}}[0]$ and $g= {\bm{m}}[1]$. Recall that $g$ refers to the index of a rejected proof in the proof vector which was generated for $i$-th verification, i.e. $ {\bm{\pi}}_{\scriptscriptstyle i}$ 


\item[$\bullet$] Fetches the related encrypted query $\hat{k}^{\scriptscriptstyle *}_{\scriptscriptstyle i}\in \bm{q}^{\scriptscriptstyle *}$   from SC, and   decrypts it, $\hat{k}_{\scriptscriptstyle i}=\mathtt{Dec}(\bar{k},\hat{k}^{\scriptscriptstyle *}_{\scriptscriptstyle i})$



\item[$\bullet$]  Removes the pads only from $g$-th padded encrypted proof. Let $ {\bm{\pi}}'_{\scriptscriptstyle i}[g]$ be the result. Next, it decrypts the  encrypted proof, $\mathtt{Dec}(\bar{k}, {\bm{\pi}}'_{\scriptscriptstyle i}[g]))= {\bm{\pi}}_{\scriptscriptstyle i}[g]$

\item[$\bullet$] \label{dummy-vector} Constructs a fresh vector: $ {\bm{\pi}}''_{\scriptscriptstyle i}$, such that  its $g$-th element equals $ {\bm{\pi}}_{\scriptscriptstyle i}[g]$ (i.e. $ {\bm{\pi}}''_{\scriptscriptstyle i}[g]=  {\bm{\pi}}_{\scriptscriptstyle i}[g] \text{ and } | {\bm{\pi}}''_{\scriptscriptstyle i}|= | {\bm{\pi}}_{\scriptscriptstyle i}|$) and the rest of its elements are dummy values. 

\item[$\bullet$] Calls   $\mathtt{PoRID.identify}( {\bm{\pi}}''_{\scriptscriptstyle i},g,\hat{k}_{\scriptscriptstyle i},pp)\rightarrow I_{\scriptscriptstyle i}$. If $I_{\scriptscriptstyle i}=\mathcal S$, then it increments $y_{\scriptscriptstyle\mathcal S}$  by $1$. If  $I_{\scriptscriptstyle i}=\bot$, then it increments $y'_{\scriptscriptstyle\mathcal {C}}$ by $1$.
\end{itemize}

\end{enumerate}

\item The arbiter at time $\texttt{K}_{\scriptscriptstyle 6}$ sends $[y_{\scriptscriptstyle\mathcal C},y_{\scriptscriptstyle\mathcal S},  y'_{\scriptscriptstyle\mathcal C},y'_{\scriptscriptstyle\mathcal S}]$  to SC who accordingly overwrites the  elements it holds (i.e. elements of $\bm{y}$) by  the related vectors elements the arbiter sent. 

\end{enumerate} %%%%%%%%%

\

\item \textbf{Coin Transfer}. \label{RCPoRP::CoinTransfer} $\mathtt{RCPoRP}.\mathtt{pay}(\bm{y},T_{\scriptscriptstyle cp},a,p_{\scriptscriptstyle\mathcal S}, coin^{\scriptscriptstyle *}_{\scriptscriptstyle\mathcal C},coin^{\scriptscriptstyle *}_{\scriptscriptstyle\mathcal S})$
\begin{enumerate}


 \item If SC  receives  ``pay'' message  at time $\texttt{T}_{\scriptscriptstyle 2}$, where $a=0$ or $coins^{\scriptscriptstyle *}_{\scriptscriptstyle\mathcal{S}}<p_{\scriptscriptstyle\mathcal{S}}$, then it sends $coin^{\scriptscriptstyle *}_{\scriptscriptstyle\mathcal C}$ coins to $\mathcal C$ and $coin^{\scriptscriptstyle *}_{\scriptscriptstyle\mathcal S}$ coins to $\mathcal S$. Otherwise (i.e. they reach an agreement), they take the following step. 

\item Either $\mathcal C$ or $\mathcal S$ sends ``pay'' message and  statement $\ddot{x}_{\scriptscriptstyle cp}\in T_{\scriptscriptstyle cp}$ to SC at time $\texttt{L}>\texttt{K}_{\scriptscriptstyle 6}$
\item SC checks the validity of the statement by sending it  to SAP contract that  returns  either $1$ or $0$. SC only proceeds to the next step if the output is $1$
\item SC distributes the coins to the parties as follows:

\begin{itemize}
\item[$\bullet$]   $coin^{\scriptscriptstyle *}_{\scriptscriptstyle\mathcal C}-o(z-y_{\scriptscriptstyle\mathcal S})-l(y_{\scriptscriptstyle\mathcal C}+y'_{\scriptscriptstyle\mathcal C})$ coins  to $\mathcal C$
\item[$\bullet$] $coin^{\scriptscriptstyle *}_{\scriptscriptstyle\mathcal S}+o(z-y_{\scriptscriptstyle\mathcal S})-l(y_{\scriptscriptstyle\mathcal S}+y'_{\scriptscriptstyle\mathcal S})$ coins to $\mathcal S$
\item[$\bullet$] $l(y_{\scriptscriptstyle\mathcal S}+y_{\scriptscriptstyle\mathcal C}+y'_{\scriptscriptstyle\mathcal S}+y'_{\scriptscriptstyle\mathcal C})$ coins to the arbiter. 



\end{itemize}

\end{enumerate}
\end{enumerate}


\begin{remark}
The reason in step \ref{dummy-vector} vector $ {\bm{\pi}}''_{\scriptscriptstyle i}$ is constructed  is to let SC make \emph{black-box}  use of $\mathtt{PoRID.identify}(.)$. Alternatively, SC could decrypt all proofs in $\mathtt{Enc}(\bar{k}, {\bm{\pi}}_{\scriptscriptstyle i})$ and pass them to $\mathtt{PoRID.identify}(.)$. However, this approach would impose a high cost, as all proofs have to be decrypted. 
\end{remark}

\begin{remark}
In general, a transaction that is sent   to a smart contract should  cover the cost of the contract's execution. Therefore, in the above protocol, if a party unnecessarily invokes a contract for an accepting proof, it has to pay the execution cost in advance. This is the reason the above protocol (unlike RC-S-P protocol) does not need to track the number of times a party unnecessarily invokes the contract.    
\end{remark}


\begin{remark} In the protocol, for the sake of simplicity, it is assumed that the cost imposed by a malicious client to the arbiter  (to resolve a dispute) is the same as the cost imposed by a malicious server. To relax the assumption, we can simply introduce another parameter $l'$.   
 We let $l$ and $l'$ be the amounts of coin a malicious client and malicious server must pay to the arbiter respectively. In this case, (a) in step \ref{RC-PoR-P::set-cp}, the client appends $l'$ to $cp$ and (b) in the coin transfer  phase, the amounts of coin each party receives would be as follow:    $coin^{\scriptscriptstyle *}_{\scriptscriptstyle\mathcal C}-o(z-y_{\scriptscriptstyle\mathcal S})-l(y_{\scriptscriptstyle\mathcal C}+y'_{\scriptscriptstyle\mathcal C})$ coins  to $\mathcal C$,  $coin^{\scriptscriptstyle *}_{\scriptscriptstyle\mathcal S}+o(z-y_{\scriptscriptstyle\mathcal S})-l'(y_{\scriptscriptstyle\mathcal S}+ y'_{\scriptscriptstyle\mathcal S})$ coins to $\mathcal S$, and $l(y_{\scriptscriptstyle\mathcal C}+y'_{\scriptscriptstyle\mathcal C})+l'(y_{\scriptscriptstyle\mathcal S}+y'_{\scriptscriptstyle\mathcal S})$ coins to the arbiter.
\end{remark}

%\textcolor{purple} {Explain how we can slight modify the protocol to capture the fact  that the amounts of coin the client pays for a dispute resolution is different to the amount paid by the server to resolve a dispute. The idea is we introduce another value $l'$, we let $l$ and $l'$ be the amount of coin the client and server pay to resolve a dispute.  We add $l'$ to $cp$. Accordingly, we would have $coin^{\scriptscriptstyle *}_{\scriptscriptstyle\mathcal C}-o(z-y_{\scriptscriptstyle\mathcal S})+l'\cdot y_{\scriptscriptstyle\mathcal S}-l\cdot y_{\scriptscriptstyle\mathcal C}$ coins  to $\mathcal C$ and  $coin^{\scriptscriptstyle *}_{\scriptscriptstyle\mathcal S}+o(z-y_{\scriptscriptstyle\mathcal S})+l\cdot y_{\scriptscriptstyle\mathcal C}-l'\cdot y_{\scriptscriptstyle\mathcal S}$ coins to $\mathcal S$}
%


\subsubsection{Delegating the Arbiter's Role to  a Smart Contract}

In the above protocol, due to the efficiency of arbiter-side algorithm, i.e. $\mathtt{RCSPoR.resolve}(.)$,  we can  totally delegate the arbiter’s role to the smart contract, SC. In this case, the involvement of the third-party arbiter is not needed anymore.  However, to have the new variant of RC-PoR-P,   some adjustments need to be applied to the original RC-PoR-P’s protocol and definition, primarily from  two  perspectives. First,    the way a party pays to resolve a dispute would change that ultimately affects the amounts of coin each party receives at the coin distribution phase. Recall, in the RC-PoR-P and RC-S-P (presented in sections \ref{sec::RC-PoR-P-protocol} and \ref{sec::RC-S-P-protocol} respectively)  the party who raises dispute does not pay the arbiter when it sends to it a dispute query. Instead, loosely speaking, the arbiter at the coin distribution phase is paid by a misbehaving party. In contrast, when the arbiter’s role is played by a smart contract, the party who raises dispute and sends a dispute query to the contract (due to the nature of smart contracts' platform) has to pay the contract before the contract processes its query. This means, an honest party who sends a complaint to the contract needs to be compensated (by the corrupt party) for the amounts  of coin it sent to the contract to resolve the dispute. Therefore, the amounts of coin each party receives at the coin distribution phase would change, compare to the original RC-PoR-P protocol. Second, there would be no need to keep track of the number of times a party unnecessarily raises a dispute, as it pays the contract when it sends a query, before the contract processes its claim. In Appendix \ref{sec::RC-S-P-Definition2}, we provide a generic definition for RC-S-P for the case where the arbiter's role can be played by a smart contract. The generic definition also captures the new variant of RC-PoR-P. Moreover, in Appendix \ref{sec::R-PoR-P-protocol-Withohut-Arbiter}, we elaborate on how  the new variant of RC-PoR-P can be constructed and we prove its security.  

 





% !TEX root =R-zkCSP.tex

\subsection{Security Analysis of RC-PoR-P}

In this section, we analyse the security of RC-PoR-P protocol, presented in Section \ref{sec::RC-PoR-P-protocol}. We first present the protocol's primary security theorem. 



\begin{theorem}
The RC-PoR-P protocol is secure, w.r.t. Definition \ref{def::RC-S-P-Security}, if PoRID, SAP, and signature scheme are secure and the encryption scheme is semantically secure. 
\end{theorem}


 To prove the above theorem, we show that RC-PoR-P meets all security properties defined in Section \ref{sec::RC-S-P-Definition}.  Since RC-PoR-P is instantiation of RC-S-P, its proof has  similarities with the latter one. Nevertheless, for the sake of completeness we provide its proof. We start by proving that RC-PoR-P satisfies security against a malicious server.   


 \begin{lemma}\label{lemma::RC-PoR-P-SecurityAgainstMaliciousServer}
 If SAP and signature scheme are secure and  PoRID scheme supports correctness, soundness, and detectable abort, then RC-PoR-P is secure against malicious server, w.r.t. Definition \ref{deff::RC-S-P-SecurityAgainstMaliciousServer}. 
 \end{lemma}
 

 \begin{proof}[sketch] First, we  consider event    $\Big(F(u^{\scriptscriptstyle *}, \bm{q}_{\scriptscriptstyle j},{pp})= h_{\scriptscriptstyle j}\ \wedge \Big((coin_{\scriptscriptstyle\mathcal{C},j}\neq  \frac{coin_{\scriptscriptstyle\mathcal C}^{\scriptscriptstyle*}}{z}-o)  \vee (coin_{\scriptscriptstyle\mathcal{A}r,j}\neq l\ \wedge\  y'_{\scriptscriptstyle \mathcal {S},j}=1)\Big)\Big)$ that captures the case where the server provides an accepting  proof, i.e. PoR, but makes an honest client withdraw  incorrect amounts of coin, i.e. $coin_{\scriptscriptstyle\mathcal{C},j}\neq  \frac{coin_{\scriptscriptstyle\mathcal C}^{\scriptscriptstyle*}}{z}-o$, or it makes the arbiter withdraw an incorrect amount of coins, i.e. $coin_{\scriptscriptstyle\mathcal{A}r,j}\neq l$, if it unnecessarily invokes the arbiter. Because the  proof is valid, an honest client accepts it and does not raise a dispute. But, the server  could make the client  withdraw incorrect amounts of coins, if it manages to either convince the arbiter that the client has misbehaved, by making the arbiter output  $y_{\scriptscriptstyle\mathcal {C},j}=1$ through the dispute resolution phase, or submit to the contract, at the coin transfer phase,  an accepting  statement $\ddot{x}'_{\scriptscriptstyle cp}$ other than what was agreed at the initiation phase, i.e. $\ddot{x}'_{\scriptscriptstyle cp}\neq\ddot{x}_{\scriptscriptstyle cp}$, so it can change the payments' parameters, or send a message on the client's behalf to unnecessarily invoke the arbiter. Nevertheless, it cannot falsely accuse the client of misbehaviour. Because,  due to the security of SAP, it cannot  convince the arbiter to accept a different decryption key (that will be used to decrypt queries) other than what was agreed with the client in the initiation phase. In particular,  it cannot persuade the arbiter to accept  $\ddot{x}'_{\scriptscriptstyle qp}$, where $\ddot{x}'_{\scriptscriptstyle qp}\neq \ddot{x}_{\scriptscriptstyle qp}$, except with a negligible probability, $\mu(\lambda)$. This ensures that the honest client's queries are accessed by  the arbiter with a high probability.   Furthermore, if the adversary provides a valid statement, i.e. $\ddot{x}_{\scriptscriptstyle qp}$, then due to PoRID's correctness, algorithm $\mathtt{PoRID.identify}(.)$ outputs $I_{\scriptscriptstyle j}=\bot$. Therefore, due to the security of SAP and correctness of PoRID,   the following holds  $y_{\scriptscriptstyle\mathcal{C},j}=y_{\scriptscriptstyle\mathcal{S},j}=0$. Furthermore,  because of SAP's security, the server cannot change the payment parameters by convincing the contract to accept any statement $\ddot{x}'_{\scriptscriptstyle cp}$ other than what was agreed initially between the client and server, except with a negligible probability, $\mu(\lambda)$. Also, due to the signature's  security, the adversary  cannot send a message on behalf of the client, to unnecessarily invoke the arbiter and make  it output $y'_{\scriptscriptstyle\mathcal {C},j}=1$, except with   a negligible probability $\mu(\lambda)$; so with a high probability $y'_{\scriptscriptstyle\mathcal {C},j}=0$.  Recall, in RC-PoR-S or RC-S-P  protocol, according to Equation \ref{equ::what-client-recives-in-j}, the amounts of  coin that should be credited to the client for $j$-th verification is $coin_{\scriptscriptstyle\mathcal{C},j}=\frac{coin^{\scriptscriptstyle *}_{\scriptscriptstyle\mathcal C}}{z}-o(1-y_{\scriptscriptstyle\mathcal {S},j})-l(y_{\scriptscriptstyle\mathcal {C},j}+y'_{\scriptscriptstyle\mathcal {C},j})$. Since it holds that $y_{\scriptscriptstyle\mathcal{C},j}=y_{\scriptscriptstyle\mathcal{S},j}=y'_{\scriptscriptstyle\mathcal {C},j}=0$,  the client is credited $\frac{coin_{\scriptscriptstyle\mathcal C}^{\scriptscriptstyle*}}{z}-o$ coins for $j$-th verification, with a high probability.   Furthermore, as stated above, if the adversary invokes the arbiter, the arbiter with a high probability outputs $I_{\scriptscriptstyle j}=\bot$ that yields  $y'_{\scriptscriptstyle\mathcal{S},j}=1$. In  RC-PoR-P or RC-S-P protocol, according to Equation \ref{equ::arbiter}, the amounts of coin the arbiter should be credited for $j$-th verification is $coin_{\scriptscriptstyle\mathcal{A}r,j}= l(y_{\scriptscriptstyle\mathcal {S},j}+y_{\scriptscriptstyle\mathcal {C},j}+y'_{\scriptscriptstyle\mathcal {S},j}+y'_{\scriptscriptstyle\mathcal {C},j})$. As shown above $y_{\scriptscriptstyle\mathcal{C},j}=y_{\scriptscriptstyle\mathcal{S},j}=y'_{\scriptscriptstyle\mathcal{C},j}=0$ and $y'_{\scriptscriptstyle\mathcal{S},j}=1$, which   means $l$ coins  is credited to the arbiter for $j$-th verification if it is unnecessarily invoked by the adversary.  In this case, for the server to make the arbiter withdraw other than that amounts, it has to send to the contract (at the coin transfer phase) an accepting  statement $\ddot{x}'_{\scriptscriptstyle cp}$ other than what was agreed at the initiation phase, i.e. $\ddot{x}'_{\scriptscriptstyle cp}\neq\ddot{x}_{\scriptscriptstyle cp}$, so it can change the payments' parameters. But, as stated above, it cannot succeed with a probability significantly greater than $\mu(\lambda)$. 
 

 
 We now move on to event  $\Big(\Big(F(u^{\scriptscriptstyle *}, \bm{q}_{\scriptscriptstyle j}, {pp})\neq h_{\scriptscriptstyle j}\Big) \ \wedge \Big(d_{\scriptscriptstyle j}=1	\vee y_{\scriptscriptstyle \mathcal {S},j}=0 \ \vee coin_{\scriptscriptstyle\mathcal{C},j}\neq \frac{coin_{\scriptscriptstyle\mathcal C}^{\scriptscriptstyle*}}{z} \ \vee  coin_{\scriptscriptstyle\mathcal{A}r,j}\neq l\Big)\Big)$  which captures the case where the server provides an invalid  proof (i.e. PoR) but it either convinces the client to accept the proof, or  persuades the arbiter to accept it  or makes the client or arbiter withdraw  incorrect amounts of coin, i.e.    $coin_{\scriptscriptstyle\mathcal{C},j}\neq \frac{coin_{\scriptscriptstyle\mathcal C}^{\scriptscriptstyle*}}{z}$ or $coin_{\scriptscriptstyle\mathcal{A}r,j}\neq l$ respectively.   Due to the soundness of PoRID, the probability that the adversary can convince an honest client to accept invalid proof is negligible, $\mu(\lambda)$. Therefore, the client outputs $d_{\scriptscriptstyle j}=0$  with a high probability and raises a dispute. Furthermore, the server may try to make the arbiter keep $y_{\scriptscriptstyle \mathcal {S},j}=0$. For the adversary to succeed, it has to make the arbiter  identify the client as the misbehaving party, and output $y_{\scriptscriptstyle \mathcal {C},j}=1$. In this case, according to RC-PoR-P protocol, the client's complaint (for $j$-th verification) would not be processed by the arbiter. This  allows   $y_{\scriptscriptstyle \mathcal {S},j}$ to remain $0$.   But, as we argued above, the probability that the adversary makes the arbiter  recognise the client as misbehaving is at most $\mu(\lambda)$. So, with a high probability  $y_{\scriptscriptstyle \mathcal {S},j}=1$ and $y_{\scriptscriptstyle \mathcal {C},j}=0$, after the arbiter  is invoked by the client or  server. It also holds that $y'_{\scriptscriptstyle \mathcal {C},j}=y'_{\scriptscriptstyle \mathcal {S},j}=0$, because  the arbiter has already identified a misbehaving party. Moreover, due to SAP's security, the probability that the adversary succeeds in changing the payment parameters to  make the client or arbiter withdraw  incorrect amounts of coin is negligible too.    So, according to Equations \ref{equ::what-client-recives-in-j} and \ref{equ::arbiter}  the client and arbiter are credited $\frac{coin_{\scriptscriptstyle\mathcal C}^{\scriptscriptstyle*}}{z}$ and $l$ coins for $j$-th verification respectively.   Also, due to the security of SAP, the adversary cannot block an honest client's  messages, ``pay'' and $\ddot{x}_{\scriptscriptstyle cp}$, to the contract in the coin transfer phase.
  \hfill\(\Box\)\end{proof}


Next, we provide a lemma which formally states  RC-PoR-P is secure against a malicious client and then we prove the lemma. 


\begin{lemma}\label{lemma::RC-PoR-P-SecurityAgainstMaliciousClient}
 If SAP and signature scheme are secure and  PoRID scheme supports correctness, inputs well-formedness, and detectable abort, then RC-PoR-P is secure against malicious client, w.r.t. Definition \ref{deff::RC-S-P-Security-Against Malicious-Client}. 
\end{lemma}


\begin{proof}[sketch] We first consider event $\Big(\Big(M(u^{\scriptscriptstyle *},k,{pp})= \sigma   \wedge  Q(\text{aux},k, {pp})= \bm{q}_{\scriptscriptstyle j}\Big) \wedge \Big((coin_{\scriptscriptstyle\mathcal{S},j}\neq  \frac{coin_{\scriptscriptstyle\mathcal S}^{\scriptscriptstyle*}}{z}+o)\vee  (coin_{\scriptscriptstyle\mathcal{A}r,j}\neq l \wedge  y'_{\scriptscriptstyle \mathcal {C},j}=1)\Big)\Big)$. It captures the case where the client provides  accepting metadata (i.e. a Merkle tree and its root) and query  but makes the server withdraw  incorrect amounts of coin, i.e. $coin_{\scriptscriptstyle\mathcal{S},j}\neq  \frac{coin_{\scriptscriptstyle\mathcal S}^{\scriptscriptstyle*}}{z}+o$, or makes the arbiter withdraw incorrect amounts of coin, i.e. $coin_{\scriptscriptstyle\mathcal{A}r,j}\neq l$, if it unnecessarily invokes the arbiter.  Since the metadata and queries  are valid and correctly structured, an honest server accepts them and does not raise a dispute, so $y_{\scriptscriptstyle\mathcal{C},j}=0$.  However, the client could  make the server  withdraw incorrect amounts of coin, if it manages to either persuade the arbiter to recognise the server as misbehaving, i.e. makes the arbiter output $y_{\scriptscriptstyle\mathcal{S},j}=1$, or submit to the contract  an accepting  statement $\ddot{x}'_{\scriptscriptstyle cp}$ other than what was agreed at the initiation phase, i.e. $\ddot{x}_{\scriptscriptstyle cp}$ or send a message on the client's behalf to unnecessarily invoke the arbiter.  Nevertheless, it cannot falsely accuse the server of misbehaviour. Because,  due to SAP's security, it cannot  convince the arbiter to accept different decryption key and pads' detail, by providing a different accepting statement $\ddot{x}'_{\scriptscriptstyle qp}$  (where $\ddot{x}'_{\scriptscriptstyle qp}\neq \ddot{x}_{\scriptscriptstyle qp}$), than what was initially agreed with the server, except with a negligible probability, $\mu(\lambda)$. This ensures   the arbiter is given the honest server's messages, with a high probability. Therefore, with a high probability $y_{\scriptscriptstyle\mathcal{S},j}=0$. Also, if the adversary provides a valid statement, i.e. $\ddot{x}_{\scriptscriptstyle qp}$, then due to the correctness of PoRID, algorithm $\mathtt{PoRID.identify}(.)$ outputs $I_{\scriptscriptstyle j}=\bot$. So, due to the security of SAP and correctness of PoRID, the following holds $y_{\scriptscriptstyle\mathcal{C},j}=y_{\scriptscriptstyle\mathcal{S},j}=0$ with a high probability.  Moreover,  it holds that $y'_{\scriptscriptstyle\mathcal{S}} = 0$ because the honest server never invokes the arbiter when the client's queries are well-structured and due to the signature scheme's security, the client cannot send a message on the server's behalf to unnecessarily invoke the arbiter.  Note,  due to SAP's security, the client cannot change the payment parameters by convincing the contract to accept any  statement $\ddot{x}'_{\scriptscriptstyle cp}$ other than what was initially agreed  between the client and server (i.e. $\ddot{x}'_{\scriptscriptstyle cp}\neq \ddot{x}_{\scriptscriptstyle cp}$) except with a negligible  probability, $\mu(\lambda)$. Recall, according to Equation \ref{equ::what-server-recives-in-j}, in RC-PoR-P or RC-S-P protocol, the total coins the server should be credited for $j$-th verification is $coin_{\scriptscriptstyle\mathcal{S},j}=\frac{coin^{\scriptscriptstyle *}_{\scriptscriptstyle\mathcal S}}{z}+o(1-y_{\scriptscriptstyle\mathcal {S},j})-l(y_{\scriptscriptstyle\mathcal {S},j}+y'_{\scriptscriptstyle\mathcal {S},j})$. Therefore, given $y_{\scriptscriptstyle\mathcal{S},j}=y'_{\scriptscriptstyle\mathcal{S},j}=0$,  the server is credited $\frac{coin_{\scriptscriptstyle\mathcal S}^{\scriptscriptstyle*}}{z}+o$ coins for $j$-th verification, with a  high probability.  Furthermore, as stated above, if the adversary invokes the arbiter, the arbiter with a high probability outputs $I_{\scriptscriptstyle j}=\bot$ which yields  $y'_{\scriptscriptstyle\mathcal{C},j}=1$. Hence, according to Equation \ref{equ::arbiter}, the arbiter for $j$-th verification is credited $l$ coins, with a high probability.  As previously stated,  due to the security of SAP, the client cannot make the arbiter withdraw incorrect coin amounts  by changing the payment parameters  and persuading the contract to accept any statement $\ddot{x}'_{\scriptscriptstyle cp}$ other than what was agreed initially between the client and server, except with a negligible probability $\mu(\lambda)$. We now turn our attention to $\Big(M(u^{\scriptscriptstyle *},k,{pp})\neq \sigma\ \wedge \ a=1\Big)$   which captures the case where the server accepts an ill-formed metadata. But, due to PoRID's inputs well-formedness, the probability the event happens is negligible, $\mu(\lambda)$; therefore, with a high probability $a=0$. In this  case,  the server does not raise any dispute, instead it avoids serving the client. 

Next, we move on to  $\Big((Q(\text{aux},k,{pp})\neq \bm{q}_{\scriptscriptstyle j}) \wedge  (b_{\scriptscriptstyle j}=1 \vee  y_{\scriptscriptstyle\mathcal{C},j}=0  \vee  coin_{\scriptscriptstyle\mathcal{S},j}\neq \frac{coin_{\scriptscriptstyle\mathcal S}^{\scriptscriptstyle*}}{z}+o  \vee  coin_{\scriptscriptstyle\mathcal{A}r,j}\neq l )\Big)$    which considers the case where the client provides an invalid query, but  either convinces the server or arbiter to accept it, or  makes the server or arbiter  withdraw an incorrect amount of coins, i.e. $coin_{\scriptscriptstyle\mathcal{S},j}\neq \frac{coin_{\scriptscriptstyle\mathcal S}^{\scriptscriptstyle*}}{z}+o$ or  $coin_{\scriptscriptstyle\mathcal{A}r,j}\neq l$ respectively. But, due to PoRID's inputs well-formedness, the probability that the server outputs $b_{\scriptscriptstyle j}=1$ is negligible $\mu(\lambda)$. Note, when the honest server rejects the query and raises a dispute, the arbiter checks the query and sets $y_{\scriptscriptstyle\mathcal{C},j}=1$. After that, due to  RC-PoR-P design, the client cannot make the arbiter set $y_{\scriptscriptstyle\mathcal{C},j}=0$ (unless it manages to modify the blockchain's content later on, but its probability of success is negligible due to the security of blockchain).  As already stated, the client cannot make the arbiter recognise the  honest server as a misbehaving party with a probability significantly greater than $\mu(\lambda)$. That means, with a high probability $y_{\scriptscriptstyle\mathcal{S},j}=0$. Furthermore, since the arbiter  has identified a misbehaving party, the following holds   $y'_{\scriptscriptstyle\mathcal{C},j}=y'_{\scriptscriptstyle\mathcal{S},j}=0$.  The adversary may still try to  make them withdraw  incorrect amounts of coin. To this end, at the coin transfer phase, it has to send a  different accepting statement  than what was initially agreed with the server. But, due to SAP's  security, its  success probability is  $\mu(\lambda)$.  Hence, according to Equations \ref{equ::what-server-recives-in-j} and \ref{equ::arbiter} the server and arbiter are credited $\frac{coin_{\scriptscriptstyle\mathcal S}^{\scriptscriptstyle*}}{z}+o$ and $l$ coins respectively for $j$-th verification, with a high probability.  Furthermore, due to SAP's  security, the adversary cannot block an honest server's  messages, ``pay'' and $\ddot{x}_{\scriptscriptstyle cp}$, to the contract in the coin transfer phase. \hfill\(\Box\)\end{proof}
  
 

In the following, we prove  RC-PoR-P's privacy. Since the  proof has many similarities with  the  proof of RC-S-P's privacy (i.e. proof of Lemma \ref{lem::RC-S-P-Privacy}) in the following we provide only the proof's outline.  



\begin{lemma}\label{lemma::RC-PoR-P-privacy} If SAP is secure and the encryption scheme is semantically secure, then RC-PoR-P preserves privacy, w.r.t. Definition \ref{deff::RC-S-P-Privacy}. 
\end{lemma}

\begin{proof}[sketch] Briefly, due to SAP's privacy property, given  commitments $g_{\scriptscriptstyle qp}$ and $g_{\scriptscriptstyle cp}$ (stored in the blockchain as a result of running SAP) the adversary learns no information about the committed values (e.g. $o, l, pad_{\scriptscriptstyle\pi}$ and $\bar{k}$), except with negligible probability $\mu(\lambda)$. Moreover, given  price list $pl$, and the parties' encoded coins $coin^{\scriptscriptstyle *}_{\scriptscriptstyle\mathcal{C}}$ and  $coin^{\scriptscriptstyle *}_{\scriptscriptstyle\mathcal{S}}$, the adversary learns nothing about the actual price agreed between the server and client, i.e. $(o,l)$, for each verification, due to Lemma \ref{lemma::encoded-coins}.  Also, since each proof $\bm{\pi}^{\scriptscriptstyle *}_{\scriptscriptstyle j}$ is encrypted and then padded, given $\bm{\pi}^{\scriptscriptstyle *}_{\scriptscriptstyle j}$ the adversary cannot tell  whether $\bm{\pi}^{\scriptscriptstyle *}_{\scriptscriptstyle j}$ is associated with $u_{\scriptscriptstyle 0}$ or with $u_{\scriptscriptstyle 1}$ (i.e. where $u_{\scriptscriptstyle 0}$ and $u_{\scriptscriptstyle 1}$ are the adversary's choice of files), with probability significantly greater than $\frac{1}{2}+\mu(\lambda)$. As each $\hat{k}^{\scriptscriptstyle *}_{\scriptscriptstyle j}$ is an output of semantically secure symmetric key encryption and its size is fixed, it  leaks noting to the adversary.  The value of $a$ is also independent of $u_{\scriptscriptstyle 0}$  or $u_{\scriptscriptstyle 1}$, and only depends on whether the metadata is well-formed, so it leaks nothing about the choice of  input file $u_{\scriptscriptstyle\beta}$ and $\beta\in\{0,1\}$. Hence, the adversary cannot tell with a probability significantly greater than $\frac{1}{2}+\mu(\lambda)$ which file of its choice has been used as the server input. 

Furthermore, since each query and padded encrypted proof leak no information and always contain a fixed number of elements, an adversary cannot tell the status of  proof for each $j$-th verification (i.e. whether it is accepted or rejected) with the probability greater than $\frac{1}{2}+\mu(\lambda)$, given  $\hat{k}^{\scriptscriptstyle *}_{\scriptscriptstyle j},coin^{\scriptscriptstyle *}_{\scriptscriptstyle \mathcal S},coin^{\scriptscriptstyle *}_{\scriptscriptstyle\mathcal C}, g_{\scriptscriptstyle cp},  g_{\scriptscriptstyle qp},$  $\bm{\pi}^{\scriptscriptstyle *},pl$, and $a$.
  \hfill\(\Box\)\end{proof}




% !TEX root =R-zkCSP.tex

\subsection{Evaluation of RC-PoR-P}











