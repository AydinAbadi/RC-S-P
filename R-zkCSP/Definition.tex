% !TEX root =R-zkCSP.tex

\section{Definition}\label{Definition}

At a high-level, a verifiable service scheme is a two-party protocol in which a client chooses a function, $F$, and provides (an encoding of) $F$, and its input, $u$ and a query $q$ to a server.  The server is expected to evaluate $F$ on $u$ and $q$ and respond with the output. Then, the client  verifies that the output is indeed the output of the function computed on the provided input. In verifiable services, either the computation (on the input) is delegated to a server or both the computation and storage of the input is delegated to the server. A verifiable service is defined as follows. 


\begin{definition}[VS Scheme]\label{service-def}
A verifiable service scheme VS $=(\mathtt{VS.keyGen}, \mathtt{VS.setup},\mathtt{VS.genQuery},$ $\mathtt{VS.prove},\mathtt{VS.verify})$ consists of five algorithms defined as follows.






\begin{itemize}

\item[$\bullet$] $\mathtt{VS.keyGen}(1^{\lambda},F)\rightarrow k:(sk,pk)$.  A probabilistic algorithm run by the client. It takes as input security parameter and a function, $F$, that will be run on the client's input by the server. It outputs a secret/public verification key: $k$. 

\

\item[$\bullet$] $\mathtt{VS.setup}(1^{\lambda}, u,k, M)\rightarrow (u^{\scriptscriptstyle *},\sigma)$. It is run by the client. It takes as input security parameter,  the service input: $u$,  the key pair: $k$ and metadata generator deterministic function: $M$, publicly known. If an encoding is needed, then it encodes $u$, that results $u^{\scriptscriptstyle *}$; otherwise, $u^*=u$. It outputs encoded input $u^{\scriptscriptstyle *}$ and metadata $\sigma=M(u^{\scriptscriptstyle *},k)$. Right after that, the server might be given $u^{\scriptscriptstyle *}$,  $\sigma$ and $pk$. 
%\item[$\bullet$] $\mathtt{VS.setup}(1^{\lambda}, u,k, M)\rightarrow \sigma$. A probabilistic algorithm run by the client. It takes as input security parameter,  the service input: $u$,  the key pair: $k$ and metadata generator deterministic function: $M$, publicly known. It runs $M(u,k)$ and outputs  metadata: $\sigma$. Right after that, the server might be given $u$,  $\sigma$ and $pk$. 

\

\item[$\bullet$] $\mathtt{VS.genQuery}(1^{\lambda}, \text{aux},k,Q)\rightarrow q$. A probabilistic algorithm run by the client. It takes as input security parameter and auxiliary information: $\text{aux}$,  the key pair: $k$  and query generator deterministic function: $Q$,  publicly known. It outputs  a query, $q:Q( \text{aux},k)$.  Depending on service types, $q$ may be empty or contain only random strings. The output  is given to the server. 

\

%a set of (random) challenges $c$ or empty set if the proving/verification  algorithms do not require any challenges. %The challenges set is published to a bulletin board.


\item[$\bullet$] $\mathtt{VS.prove}(u^{\scriptscriptstyle *},\sigma,q,pk)\rightarrow \pi$. It is run by the server. It  takes service encoded input: $u^{\scriptscriptstyle *}$, metadata: $\sigma$,   queries $q$ and public key: $pk$. It outputs a proof pair, $\pi:(F(u^{\scriptscriptstyle *},q),\delta)$ containing the function evaluation at  input $u$, i.e. $h=F(u^{\scriptscriptstyle *},q)$ and a proof $\delta$ asserting the evaluation is performed correctly, where generating $\delta$ may involve $\sigma$. The output is given to the client. %The proof pair: $\pi$ is published to a bulletin board. 

\

\item[$\bullet$] $\mathtt{VS.verify}(\pi,q,k)\rightarrow d\in\{0,1\}$. It is run by the client. It takes the proof: $\pi$,   queries $q$, and  key  $k$, where $k=pk$  in the case where $\mathtt{VS.verify}()$ is publicly verifiable, or $k:(sk,pk)$ when it is privately verifiable. The algorithm  outputs $d=1$ if the proof is accepted; otherwise, it outputs $d=0$ 
\end{itemize}
\end{definition}


Informally, a verifiable service scheme has two main properties; namely, it is correct and sound. The correctness requires that  the verification algorithm always accepts a proof generated by an honest prover.  It is formally stated below.

\begin{definition}[VS Correctness] A verifiable service, VS, scheme is  correct,  if for any $F$,  any auxiliary information $aux$, any  $Q$, and any $M$, the key generation algorithm produces keys $\mathtt{VS.keyGen}(1^{\lambda},F)\rightarrow k:(sk,pk)$ s.t. $\forall u \in \mathtt{Domain}(F)$, if $\ \mathtt{VS.setup}(1^{\lambda}, u,k,M)$ $\rightarrow (u^{\scriptscriptstyle *},\sigma)$,  $\mathtt{VS.genQuery}(1^{\lambda}, \text{aux},k,Q)\rightarrow q$ and $\mathtt{VS.prove}(u^{\scriptscriptstyle *},\sigma,q,pk)\rightarrow \pi$, then $\mathtt{VS.verify}(\pi,q,k)\rightarrow 1$
\end{definition}

Intuitively, a verifiable service is sound  if a malicious server cannot convince the verification algorithm to accept an incorrect output (of $F$). In other words, if a prover persuades the verifier with a high probability, then the service has been provided by the prover. It is  formally stated as follows.



%%%%%%%%%%%%%%%%%%%%%%%%%%%%

\begin{definition}[VS Soundness] A verifiable service VS is sound  for a function $F$, if for  any security parameter $\lambda$, any auxiliary information $aux$, any  $Q$,  any $M$, and any probabilistic polynomial time adversaries $\mathcal{A}$, there exists a negligible function $\mu(.)$, such that: 
%\footnotesize{
\small{
$$ Pr\left[
  \begin{array}{l}
F(u^{\scriptscriptstyle *},q)\neq h \wedge d=1
\end{array} \middle | 
    \begin{array} {l}
 \mathtt{VS.keyGen}(1^{\lambda},F)\rightarrow k:(sk,pk)\\
 \mathcal{A}(1^{\scriptscriptstyle\lambda},pk, F)\rightarrow u\\
\mathtt{VS.setup}(1^{\lambda}, u,k,M)\rightarrow (u^{\scriptscriptstyle *},\sigma)\\
 \mathtt{VS.genQuery}(1^{\lambda}, \text{aux},k,Q)\rightarrow q\\
 \mathcal{A}(q,u^{\scriptscriptstyle *},\sigma)\rightarrow \pi:(h,\delta)\\
 \mathtt{VS.verify}(\pi,q,k)\rightarrow d\\
\end{array}    \right]\leq \mu(\lambda)$$
}
where the probability is taken over uniform  choice of $k\stackrel{\scriptscriptstyle\$}\leftarrow \{0,1\}^{\scriptscriptstyle\lambda}$  as well as the randomness of $\mathcal{A}$.
\end{definition}




The above generic definition captures the core requirements of a wide range of verifiable services such as verifiable outsourced storage, i.e. PoR/PDP schemes, verifiable computation, verifiable searchable encryption, verifiable information retrieval, and verifiable delegated (private) set intersection, to name a few. Any additional security properties mandated by certain services, e.g.  privacy or extractability, can be easily plugged into the above definition.  
A protocol that realises only VS's definition, would be merely secure against a malicious server and  assumes the client is honest.  Although the  assumption would suffice in certain settings and has been used before (e.g. in \cite{}), it is too strong and not suitable in the real world, especially when there are monetary incentives (e.g. service payment)  that encourage a client to misbehave. Therefore, in the following we upgrade VS's definition to allow  (a) either party to be malicious and (b)  a trusted third-party, arbiter, to identify a corrupt party. We call a verifiable service scheme with that features ``verifiable service with identifiable abort'' (VSID), inspired by the notion of secure multi-party computation with identifiable abort \cite{DBLP:conf/crypto/IshaiOZ14}. 






\



\begin{definition}[VSID Scheme]\label{service-def} A verifiable service with identifiable abort  VSID $=(\mathtt{VSID.keyGen}, $ $\mathtt{VSID.setup}, $ $\mathtt{VSID.serve},\mathtt{VSID.genQuery}$,$\mathtt{VSID.checkQuery},$ $\mathtt{VSID.prove},$ $\mathtt{VSID.verify}, \mathtt{VSID.identify})$ consists of eight algorithms defined below.


\begin{itemize}
\item[$\bullet$] $\mathtt{VSID.keyGen}(1^{\lambda},F)\rightarrow k:(sk,pk)$.  A probabilistic algorithm run by the client. It takes as input security parameter and a function, $F$, that will be run on the client's input by the server. It outputs a secret/public verification key: $k$ 

\

\item[$\bullet$] $\mathtt{VSID.setup}(1^{\lambda}, u,k,M)\rightarrow (u^{\scriptscriptstyle *},e)$. It is run by the client. It takes as input security parameter,  the service  input: $u$,  the key pair: $k$, and  metadata generator deterministic function: $M$ publicly known. If an encoding is needed, then it encodes $u$, that results $u^{\scriptscriptstyle *}$; otherwise, $u^{\scriptscriptstyle *}=u$. It outputs  $u^{\scriptscriptstyle *}$ and $e:(\sigma,w_{\sigma})$, where $\sigma=M(u^{\scriptscriptstyle *},k)$ is a metadata and $w_{\sigma}$ is a proof asserting the metadata is well-structured.  

\

\item[$\bullet$] $\mathtt{VSID.serve}(u^{\scriptscriptstyle *},e,pk)\rightarrow a\in\{0,1\}$. It is run by the server. It takes the  service encoded input $u^{\scriptscriptstyle *}$, pair $e:(\sigma,w_{\sigma})$ and public key $pk$. It outputs $a=1$, if the proof: $w_{\sigma}$ is accepted, i.e. if the metadata is well-formed. Otherwise, it outputs $a=0$ 

\

\item[$\bullet$] $\mathtt{VSID.genQuery}(1^{\lambda},  \text{aux},k,Q)\rightarrow c:(q,w_{\scriptscriptstyle q})$. A probabilistic algorithm run by the client. It takes as input security parameter, auxiliary information: $ \text{aux}$,  the key pair: $k$, and   query generator deterministic function: $Q$ publicly known. It outputs a pair: $c$ containing a query, $q:Q( \text{aux},k)$,  and proofs, $w_{q}$, proving the queries are well-structured. Depending on service types, $c$ might be  empty or  contain only random strings.

%\item[$\bullet$] $\mathtt{VSID.genQuery}(1^{\lambda}, aux,k)\rightarrow c$. A probabilistic algorithm run by the client. It takes as input security parameter and auxiliary information: $aux$, and the key pair: $k$. It outputs a set of (random) challenges $c$ or empty set if the proving/verification  algorithms do not require any challenges. %The challenges set is published to a bulletin board.

\

\item[$\bullet$] $\mathtt{VSID.checkQuery}(c, pk)\rightarrow b\in\{0,1\}$. It is run by the server. It takes as input a pair $c:(q,w_{\scriptscriptstyle q})$ including queries and their proofs, as well as public key, $pk$. It outputs $b = 1$ if the proofs: $w_{\scriptscriptstyle\sigma}$ are accepted, i.e. the queries are well-structured. Otherwise, it outputs $b = 0$

\

\item[$\bullet$] $\mathtt{VSID.prove}(u^{\scriptscriptstyle *},\sigma,c,pk)\rightarrow \pi$. It is run by the server. It  takes the  service encoded input: $u^{\scriptscriptstyle *}$, metadata: $\sigma$,  query $q\in c$ and public key: $pk$. It outputs a proof pair, $\pi:(F(u^{\scriptscriptstyle *},q),\delta)$ containing the function evaluation, i.e. $h=F(u^{\scriptscriptstyle *},q)$, and a proof $\delta$ asserting the evaluation is performed correctly, where computing $\delta$ may involve $\sigma$


 %The proof pair: $\pi$ is published to a bulletin board. 


\

\item[$\bullet$] $\mathtt{VSID.verify}(\pi,q,k)\rightarrow d\in\{0,1\}$. It is run by the client. It takes the proof: $\pi$,   queries $q$, and key pair $k$. If the proof is accepted, it outputs $d=1$; otherwise, it outputs $d=0$ 

\

\item[$\bullet$]  $\mathtt{VSID.identify}(\pi,c,k,e,u^{\scriptscriptstyle *})\rightarrow I\in \{ \mathcal{C},\mathcal{S},\bot\}$. It is run by a third-party arbiter. It takes the proof: $\pi$,  query pair: $c$,  key $k$,    $e$, and $u^{\scriptscriptstyle *}$  as input. If   proof: $w_{\scriptscriptstyle\sigma}\in e$ or $w_{\scriptscriptstyle q}\in c$  is rejected, then it outputs $I=\mathcal C$; otherwise, if proof $\pi$ is rejected it outputs $I=\mathcal S$.  However, if  $w_{\scriptscriptstyle\sigma},w_{\scriptscriptstyle q}$, and $\pi$ are accepted, it outputs   $I=\bot$ 

%If the server executes the algorithm, then $\pi$ and $c$ would be empty,  $k=pk$ and $e'=e$. In this case, if $w$ is rejected, it outputs $I=C$; otherwise, it outputs $I=\bot$. 

%$\mathcal C$ or $\mathcal S$

\end{itemize}
\end{definition}


Informally, a verifiable service with  identifiable abort scheme  has four main properties; namely, it is (a) correct, (b) sound, (c)  inputs (of clients/verifiers) are well-formed, and (d) a corrupt party can be identified by an arbiter. In the following, we formally define each of them. The correctness requires that  the verification algorithm always accepts a proof generated by an honest prover and both parties are identified as honest.  It is formally stated as follows.

\begin{definition}[VSID Correctness] A verifiable service with identifiable abort scheme  is  correct   if for any function $F,Q,M$, any auxiliary information $\text{aux}$, the key generation algorithm produces keys $\mathtt{VSID.keyGen}(1^{\lambda},F)\rightarrow k:(sk,pk)$ such that $\forall u \in \mathtt{Domain}(F)$ if $\ \mathtt{VSID.setup}(1^{\lambda}, u,k,M)$ $\rightarrow (u^{\scriptscriptstyle *},e)$, $\mathtt{VSID.serve}(u^{\scriptscriptstyle *},e,pk)\rightarrow a$, $\mathtt{VSID.genQuery}(1^{\lambda}, $ $ \text{aux},k,Q)$ $ \rightarrow$ $ c$, $\mathtt{VSID.checkQuery}$ $(c, $ $pk)\rightarrow b,$  $\mathtt{VSID.prove}(u^{\scriptscriptstyle *},\sigma,c,pk)\rightarrow \pi$, and \\$\mathtt{VSID.verify}(\pi,q,k)\rightarrow d$, then  $\mathtt{VSID.identify}(\pi,c,k,e,u^{\scriptscriptstyle *})\rightarrow I=\bot \  \  \wedge \ \ a=1 \ \ \wedge \  b=1 \ \ \wedge \  d=1$
\end{definition}



Intuitively, a VSID is sound  if a malicious server cannot convince the client to accept an incorrect output (of $F$). In other words, if a prover persuades the verifier with a high probability, then the service has been provided by the prover. It is  formally stated as follows.

\begin{definition}[VSID Soundness]\label{deff::VSID-Soundness}  A VSID  is sound for a function $F$, if for any security parameter $\lambda$, any auxiliary information $aux$, any $Q,M$ and any probabilistic polynomial time adversary $\mathcal{A}_{\scriptscriptstyle 1}$, there exists a negligible function $\mu(.)$, such that: 
%\footnotesize{
\small{
$$ Pr\left[
  \begin{array}{l}
F(u^{\scriptscriptstyle *},q)\neq h \wedge d=1
\end{array} \middle |
    \begin{array}{l}
    \mathtt{VSID.keyGen}(1^{\lambda},F)\rightarrow k:(sk,pk)\\
    \mathcal{A}_{\scriptscriptstyle 1}(1^{\scriptscriptstyle\lambda},pk, F)\rightarrow u\\
    \mathtt{VSID.setup}(1^{\lambda}, u,k,M)\rightarrow (u^{\scriptscriptstyle *},e)\\
    \mathtt{VSID.genQuery}(1^{\lambda},  \text{aux},k,Q)\rightarrow c:(q,w_{q})\\
     \mathcal{A}_{\scriptscriptstyle 1}(\text{state},c,e, u^{\scriptscriptstyle *})\rightarrow \pi:(h,\delta)\\
     \mathtt{VSID.verify}(\pi,q,k)\rightarrow d\\
\end{array}    \right]\leq \mu(\lambda)$$
}
where the probability is taken over uniform  choice of $k\stackrel{\scriptscriptstyle\$}\leftarrow \{0,1\}^{\scriptscriptstyle\lambda}$ as well as the randomness of $\mathcal{A}_{\scriptscriptstyle 1}$.
\end{definition}


A VSID has   well-formed inputs, if a malicious client cannot persuade a server to serve it on  ill-structured inputs (i.e. to accept  incorrect outputs of $M$ or $Q$). Below, we state it  formally.

\begin{definition}[VSID Inputs Well-formedness]\label{deff::VSID-Inputs-Well-formedess}  A  VSID  has  well-formed inputs,   if for any security parameter $\lambda$, any auxiliary information $\text{aux}$, any  function $F, M,Q$, and  any probabilistic polynomial time adversary $\mathcal{A}_{\scriptscriptstyle 2}$, there exists a negligible function $\mu(.)$, such that: 
%\footnotesize{
\small{
$$ Pr\left[
  \begin{array}{l}
  (M(u^{\scriptscriptstyle *},k)\neq \sigma \wedge a=1) \vee\\ (Q(\text{aux},k)\neq q) \wedge  b=1)\\
%(M(u^{\scriptscriptstyle *},k)\neq \sigma \lor Q(\text{aux},k)\neq q) \wedge\\ (a=1 \ \vee b=1)\\
    \text{s.t. }\\ k:(sk,pk),e:(\sigma,w_{\sigma})
\end{array} \middle |
    \begin{array}{l}
    
    \mathcal{A}_{\scriptscriptstyle 2}(1^{\scriptscriptstyle\lambda},F,M,Q)\rightarrow (u^{\scriptscriptstyle *},\sigma,w_{\sigma},sk,pk,\text{aux})\\
    \mathtt{VSID.serve}(u^{\scriptscriptstyle *},e,pk)\rightarrow a\\
       \mathcal{A}_{\scriptscriptstyle 2}(\text{aux},k)\rightarrow c:(q,w_{q})\\
    \mathtt{VSID.checkQuery}(c, pk)\rightarrow b\\
\end{array}    \right]\leq \mu(\lambda)$$
}
where the probability is taken over the randomness used in  proofs $w_{\scriptscriptstyle q}$ and $w_{\scriptscriptstyle\sigma}$  as well as  the randomness of $\mathcal{A}_{\scriptscriptstyle 2}$.



\end{definition}

The   above property ensures an honest server can detect  a malicious client if the client provides ill-structured inputs. It is further required that a malicious party to be identified by an honest third-party, arbiter. This ensures that in the case of dispute (or false accusation) a malicious party can be pinpointed. A VSID supports  identifiable abort if a corrupt party can escape from being identified, by the arbiter,  with only negligible probability.  More formally:

\begin{definition}[VSID Identifiable Abort]\label{def::VSID-Identifiable-Abort}  A  VSID  supports  identifiable abort   if the following holds.  Let  $(\pi,c,k,e,u^{\scriptscriptstyle *})$ be the parameters defined above. Then,  for any  security parameter $\lambda$, any auxiliary information $\text{aux}$, any  function $F, M,Q$, any probabilistic polynomial time adversary $\mathcal{A}\in \{\mathcal{A}_{\scriptscriptstyle 1},\mathcal{A}_{\scriptscriptstyle 2}\}$ who plays either of the  games in Definitions \ref{deff::VSID-Soundness} and \ref{deff::VSID-Inputs-Well-formedess}, and wants to win, there exists a negligible function $\mu(.)$, such that: 
%\footnotesize{
\small{
$$ Pr\left[
  \begin{array}{l}
\mathtt{VSID.identify}(\pi,c,k,e,u^{\scriptscriptstyle *})\rightarrow I \  \ \wedge \ \ \mathcal{A}\neq I\\

\end{array} 
    \right]\leq \mu(\lambda)$$
}
where, in case $\mathcal{A}=\mathcal{A}_{\scriptscriptstyle 1}$, the probability is taken over uniform  choice of $k\stackrel{\scriptscriptstyle\$}\leftarrow \{0,1\}^{\scriptscriptstyle\lambda}$ as well as the randomness of $\mathcal{A}_{\scriptscriptstyle 1}$, in case $\mathcal{A}=\mathcal{A}_{\scriptscriptstyle 2}$, the probability is taken over the randomness used in  proofs $w_{\scriptscriptstyle q}$ and $w_{\scriptscriptstyle\sigma}$  and the randomness of $\mathcal{A}_{\scriptscriptstyle 2}$.
\end{definition}


% !TEX root =R-zkCSP.tex

\subsection{Recurring Contingent Service Payment Definition}\label{sec::RC-S-P-Definition}


\begin{definition}[RC-S-P Scheme]\label{RC-S-P-def}
A recurring contingent service payment scheme RC-S-P  $=(\mathtt{RCSP}.\mathtt{keyGen}, $ $\mathtt{RCSP}.\mathtt{cInit}, $ $\mathtt{RCSP}.\mathtt{sInit},  $ $\mathtt{RCSP}.\mathtt{genQuery},\mathtt{RCSP}.\mathtt{prove},\mathtt{RCSP}.\mathtt{verify},\mathtt{RCSP}.\mathtt{resolve},\mathtt{RCSP}.\mathtt{pay})$ involves four parties; namely, client, server, arbiter and smart contract, and consists of eight algorithms defined as follows.


\begin{itemize}

\item[$\bullet$] $\mathtt{RCSP}.\mathtt{keyGen}(1^{\scriptscriptstyle\lambda},F)\rightarrow \bm{k}$.  A probabilistic algorithm run by the client. It takes as input security parameter $1^\lambda$  and  function $F$ that will be run on the client's input by the server. It outputs $\bm{k}$ that contains a secret/public verification key pair $k:=(sk,pk)$ and a set of secret parameters, $K'$. %The client gives $pk$ to other parties.

\

\item[$\bullet$] $\mathtt{RCSP}.\mathtt{cInit}(1^\lambda, u,\bm{k}, M,z,cp)\rightarrow (u^{\scriptscriptstyle *},e,t,coin^{\scriptscriptstyle*}_{\scriptscriptstyle\mathcal C}, \bm{y}_{\scriptscriptstyle \mathcal  C},\bm{y}_{\scriptscriptstyle \mathcal  S})$. It is run by the client. It takes as input $1^\lambda$,  the service input $u$,  the keys  $\bm{k}:=(k, K')$,  the metadata generator function $M$, the total number of verifications $z$, and coin secret parameters $cp$ that include a subset of  $K'$ and the actual amount of coins for each accepting service proof $o$ and  for covering each potential dispute resolution's cost $l$.  It encodes $u$, that results in $u^{\scriptscriptstyle *}$.  It computes metadata $\sigma=M(u^{\scriptscriptstyle *},k)$ and a  proof $w_{\scriptscriptstyle\sigma}$ asserting the metadata is well-structured.  It constructs coin encoding token $T_{\scriptscriptstyle cp}$ that contains   $cp$, total coins the server should deposit $p_{\scriptscriptstyle\mathcal S}$, and   $cp$'s witness, $g_{\scriptscriptstyle cp}$.  It also constructs query/proof encoding token $T_{\scriptscriptstyle qp}$ that contains  secret  parameters $qp\in k'$ (used to encode the service queries/proofs)  and     $qp$'s witness,  $g_{\scriptscriptstyle qp}$.   Given a valid value and its witness  anyone can check if they match. It constructs two binary vectors $\bm{y}_{\scriptscriptstyle \mathcal  C}$ and $\bm{y}_{\scriptscriptstyle \mathcal  S}$ that are set to $0$ and their length is $z$.  It outputs encoded input $u^{\scriptscriptstyle *}$, metadata-proof pair $e:=(\sigma,w_{\scriptscriptstyle\sigma})$, the two sets $T:=(T_{\scriptscriptstyle cp},T_{\scriptscriptstyle qp})$, the two binary vectors $\bm{y}_{\scriptscriptstyle \mathcal  C}, \bm{y}_{\scriptscriptstyle \mathcal  S}$, and  the encoded coins amount  $coin^{\scriptscriptstyle*}_{\scriptscriptstyle\mathcal C}$ (i.e. contains $o$ and $l$ coins in an encoded form). The client sends  $u^{\scriptscriptstyle *}$, $z$, $pk$, $e$, $T_{\scriptscriptstyle cp} \setminus  \{g_{\scriptscriptstyle cp},p_{\scriptscriptstyle\mathcal S}\}$ and $T_{\scriptscriptstyle qp} \setminus  \{g_{\scriptscriptstyle qp}\}$ to the server and  sends $\{g_{\scriptscriptstyle cp},  g_{\scriptscriptstyle qp},p_{\scriptscriptstyle\mathcal S}\}$, $\bm{y}_{\scriptscriptstyle \mathcal  C}$, $\bm{y}_{\scriptscriptstyle \mathcal  S}$  and $coin^{\scriptscriptstyle*}_{\scriptscriptstyle\mathcal C}$ coins to the smart contract. 
%\item[$\bullet$] $\mathtt{RCSP}.\mathtt{cInit}(1^\lambda, u,\bm{k}, M,z,o,l,E_{\scriptscriptstyle 1})\rightarrow (u^{\scriptscriptstyle *},e,t,coin^{\scriptscriptstyle*}_{\scriptscriptstyle\mathcal C})$. It is run by the client. It takes as input $1^\lambda$,  the service input: $u$,  the key pair: $(k, k')$,  metadata generator function: $M$, total number of verifications: $z$,  the amount of coins for each accepting service proof: $o$,  coins amount  to cover each potential dispute resolution's cost: $l$ and  encoding function $E_{\scriptscriptstyle 1}$ to encode the coins actual amount.  It encodes $u$, that results $u^{\scriptscriptstyle *}$.  It computes a metadata, $\sigma=M(u^{\scriptscriptstyle *},k)$, and a publicly verifiable zero-knowledge proof $w_{\scriptscriptstyle\sigma}$ asserting the metadata is well-structured.  It constructs coins' token $t_{\scriptscriptstyle cp}$ that contains  coin's secret  parameters $cp$ and   sealed    parameters $sld_{\scriptscriptstyle cp}$. Given $cp$ and $sld_{\scriptscriptstyle cp}$ anyone can check if $sld_{\scriptscriptstyle cp}$ corresponds to $cp$. It constructs  encoding token $t_{\scriptscriptstyle pq}$ that contains  secret  parameters $qp\in k'$ (used to encode the service queries/proofs) and  sealed   parameters $sld_{\scriptscriptstyle qp}$, where  given $qp$ and $sld_{\scriptscriptstyle qp}$ anyone can check if $sld_{\scriptscriptstyle qp}$ corresponds to $qp$.  It outputs encoded input $u^{\scriptscriptstyle *}$, metadata-proof pair $e:(\sigma,w_{\scriptscriptstyle\sigma})$, the two sets $t:\{t_{\scriptscriptstyle cp},t_{\scriptscriptstyle qp}\}$ and  the encoded coins amount  $coin^{\scriptscriptstyle*}_{\scriptscriptstyle\mathcal C}=E_{\scriptscriptstyle 1}(z,cp)$. The server is  given $u^{\scriptscriptstyle *}$,  $e$, and $t \setminus  \{sld_{\scriptscriptstyle cp},sld_{\scriptscriptstyle qp}\}$ and  the smart contract  is given $sld_{\scriptscriptstyle cp}$,  $sld_{\scriptscriptstyle qp}$ and $coin^{\scriptscriptstyle*}_{\scriptscriptstyle\mathcal C}$ coins by the client. 


%\item[$\bullet$] $\mathtt{VS.setup}(1^{\lambda}, u,k, M)\rightarrow \sigma$. A probabilistic algorithm run by the client. It takes as input security parameter,  the service input: $u$,  the key pair: $k$ and metadata generator deterministic function: $M$, publicly known. It runs $M(u,k)$ and outputs  metadata: $\sigma$. Right after that, the server might be given $u$,  $\sigma$ and $pk$. 

\

\item[$\bullet$] $\mathtt{RCSP}.\mathtt{sInit}(u^{\scriptscriptstyle *},e, pk,z, T,\bm{y}_{\scriptscriptstyle \mathcal  C},\bm{y}_{\scriptscriptstyle \mathcal  S})\rightarrow (coin^{\scriptscriptstyle *}_{\scriptscriptstyle\mathcal S},a)$. It is  run by the server. It takes as input the  service encoded input $u^{\scriptscriptstyle *}$, metadata-proof pair $e:=(\sigma,w_{\scriptscriptstyle\sigma})$,  public key $pk$, the total number of verifications $z$, sets $T:=(T_{\scriptscriptstyle cp},T_{\scriptscriptstyle qp})$ (where $\{g_{\scriptscriptstyle cp},  g_{\scriptscriptstyle qp},p_{\scriptscriptstyle\mathcal S}\}$ are read from the smart contract), and reads the two binary strings $\bm{y}_{\scriptscriptstyle \mathcal  C}, \bm{y}_{\scriptscriptstyle \mathcal  S}$ from the smart contract.  It verifies the validity of the elements in $e$ and  $T$. Also, it checks that $\bm{y}_{\scriptscriptstyle \mathcal  C}$ and $\bm{y}_{\scriptscriptstyle \mathcal  S}$ have been set to $0$. If all checks are successful, then it encodes the amount of its coins  $coin^{\scriptscriptstyle*}_{\scriptscriptstyle\mathcal S}$ and sets $a=1$. Otherwise, it sets $coin^{\scriptscriptstyle *}_{\scriptscriptstyle\mathcal S}=\bot$ and $a=0$. It outputs $coin^{\scriptscriptstyle *}_{\scriptscriptstyle\mathcal S}$ and $a$. The smart contract is given $coin^{\scriptscriptstyle *}_{\scriptscriptstyle\mathcal S}$ coins and $a$.  

\

\item[$\bullet$] $\mathtt{RCSP}.\mathtt{genQuery}(1^\lambda, \text{aux},k,Q, enc)\rightarrow c^{\scriptscriptstyle *}_{\scriptscriptstyle j}$. A probabilistic algorithm run by the client. It takes as input  $1^\lambda$,  auxiliary information $\text{aux}$,  the key pair $k$,   query generator deterministic function $Q$,  and $enc$ that contains the encoding/decoding token  $T_{\scriptscriptstyle qp}$ and  encoding/decoding functions $(E,D)$  used to encode/decode service's proofs and queries.  It computes a pair $c_{\scriptscriptstyle j}$ containing a query vector $\bm{q}_{\scriptscriptstyle j}=Q( \text{aux},k)$,  and proof $\bm{w}_{\scriptscriptstyle q_{_{\scaleto{j}{\size}}}}$ proving the query is well-structured. It outputs the encoding of the pair, $c^{\scriptscriptstyle *}_{\scriptscriptstyle j}=E(c_{\scriptscriptstyle j},t_{\scriptscriptstyle qp})$. The output is sent to the smart contract. 

%\scaleto{c}{1pt}

%a set of (random) challenges $c$ or empty set if the proving/verification  algorithms do not require any challenges. %The challenges set is published to a bulletin board.
\

\item[$\bullet$] $\mathtt{RCSP}.\mathtt{prove}(u^{\scriptscriptstyle *},  \sigma,  c^{\scriptscriptstyle *}_{\scriptscriptstyle j},pk, enc)\rightarrow (b_{\scriptscriptstyle j},m_{\scriptscriptstyle \mathcal{S},j},\pi^{\scriptscriptstyle *}_{\scriptscriptstyle j})$. It is run by the server. It  takes as input the encoded service input $u^{\scriptscriptstyle *}$, metadata $\sigma$,   encoded query pair $c^{\scriptscriptstyle *}_{\scriptscriptstyle j}$, public key $pk$,  and $enc:=(E,D,T_{\scriptscriptstyle qp})$.  It checks the validity of decoded query, $c_{\scriptscriptstyle j}=D(c^{\scriptscriptstyle *}_{\scriptscriptstyle j},T_{\scriptscriptstyle qp})$.  If it is rejected, then it sets $b_{\scriptscriptstyle j}=0$ and constructs a complaint $m_{\scriptscriptstyle \mathcal{S},j}$. Otherwise,  it   sets $b_{\scriptscriptstyle j}=1$ and $m_{\scriptscriptstyle \mathcal{S},j}=\bot$. It outputs $b_{\scriptscriptstyle j},m_{\scriptscriptstyle \mathcal{S},j}$,   and  encoded proof $\pi^{\scriptscriptstyle *}_{\scriptscriptstyle j}=E(\pi_{\scriptscriptstyle j},T_{\scriptscriptstyle qp})$, where  $\pi_{\scriptscriptstyle j}$ contains $h_{\scriptscriptstyle j}=F(u^{\scriptscriptstyle *},\bm{q}_{\scriptscriptstyle j})$ and a proof $\delta_{\scriptscriptstyle j}$  asserting the evaluation is performed correctly ($\pi_{\scriptscriptstyle j}$ may contain dummy values if $b_{\scriptscriptstyle j}=0$). The smart contract is given $\pi^{\scriptscriptstyle *}_{\scriptscriptstyle j}$.




\

\item[$\bullet$] $\mathtt{RCSP}.\mathtt{verify}(\pi^{\scriptscriptstyle *}_{\scriptscriptstyle j}, \bm{q}_{\scriptscriptstyle j},k,enc)\rightarrow (d_{\scriptscriptstyle j},m_{\scriptscriptstyle \mathcal{C},j})$. A deterministic algorithm run by the client. It takes as input the encoded proof $\pi^{\scriptscriptstyle *}_{\scriptscriptstyle j}$,  query vector $\bm{q}_{\scriptscriptstyle j}$, key pair $k$,  and $enc:=(E,D,T_{\scriptscriptstyle qp})$. If the decoded proof $\pi_{\scriptscriptstyle j}=D(\pi^{\scriptscriptstyle *}_{\scriptscriptstyle j},T_{\scriptscriptstyle qp})$ is rejected,  it outputs $d_{\scriptscriptstyle j}=0$ and a complaint $m_{\scriptscriptstyle \mathcal{C},j}$. Otherwise, it outputs  $d_{\scriptscriptstyle j}=1$ and $m_{\scriptscriptstyle \mathcal{C},j}=\bot$.




%, where $k=pk$  in the case where $\mathtt{VS.verify}()$ is publicly verifiable, or $k:(sk,pk)$ when it is privately verifiable

\


\item[$\bullet$] $\mathtt{RCSP}.\mathtt{resolve}(\bm{m}_{\scriptscriptstyle \mathcal{C}},\bm{m}_{\scriptscriptstyle \mathcal{S}},z, {\bm{\pi}}^{\scriptscriptstyle *}, {\bm{c}}^{\scriptscriptstyle *},pk,enc)\rightarrow (\bm{y}_{\scriptscriptstyle \mathcal C},\bm{y}_{\scriptscriptstyle \mathcal S})$. It is run by the arbiter. It takes as input the client's complaints $\bm{m}_{\scriptscriptstyle \mathcal{C}}$, the server's complaints $\bm{m}_{\scriptscriptstyle \mathcal{S}}$, the total number of verifications $z$, all encoded proofs $ {\bm{\pi}}^{\scriptscriptstyle *}$, all encoded query pairs $ {\bm{c}}^{\scriptscriptstyle *}$, public key $pk$, and $enc:=(E,D,T_{\scriptscriptstyle qp})$.  It verifies the token, decoded queries, and proofs.  It outputs two binary strings $\bm{y}_{\scriptscriptstyle \mathcal C}$ and $\bm{y}_{\scriptscriptstyle \mathcal S}$ of length $z$ by setting each element of $\bm{y}_{\scriptscriptstyle \mathcal E}$ to one, i.e., $y_{\scriptscriptstyle \mathcal E,j}=1$, if party $\mathcal{E}\in\{\mathcal{C},\mathcal{S}\}$ has misbehaved in the $j$-th verification (i.e., provided invalid query or service proof). 


%\item[$\bullet$] $\mathtt{RCSP}.\mathtt{resolve}(m_{\scriptscriptstyle \mathcal{C}},m_{\scriptscriptstyle \mathcal{S}},z, {\bm{\pi}}^{\scriptscriptstyle *}, {\bm{c}}^{\scriptscriptstyle *},pk,E^{\scriptscriptstyle -1}_{\scriptscriptstyle 2},t_{\scriptscriptstyle qp},a)\rightarrow (y_{\scriptscriptstyle \mathcal C},y_{\scriptscriptstyle \mathcal S})$. It is run by the arbiter. It takes client's complaint $m_{\scriptscriptstyle \mathcal{C}}$,   server's complaints   $m_{\scriptscriptstyle \mathcal{S}}$, total number of verifications $z$, all encoded proofs $ {\bm{\pi}}^{\scriptscriptstyle *}$, all encoded queries $ {\bm{c}}^{\scriptscriptstyle *}$, public key $pk$, decoding function $E^{\scriptscriptstyle -1}_{\scriptscriptstyle 2}$, encoding token $t_{\scriptscriptstyle qp}$ and  $a$.  It proceeds only if $a=1$. It verifies the token,   decoded queries, and proofs.  It outputs total number of times the client and server misbehaved, i.e. $y_{\scriptscriptstyle \mathcal C}$ and $y_{\scriptscriptstyle \mathcal S}$ respectively. 

\
	
\item[$\bullet$] $\mathtt{RCSP}.\mathtt{pay}(\bm{y}_{\scriptscriptstyle \mathcal C},\bm{y}_{\scriptscriptstyle \mathcal S},T_{\scriptscriptstyle cp},a,coin^{\scriptscriptstyle *}_{\scriptscriptstyle\mathcal C},coin^{\scriptscriptstyle *}_{\scriptscriptstyle\mathcal S})\rightarrow ({\bm{coin}}_{\scriptscriptstyle\mathcal C},{\bm{coin}}_{\scriptscriptstyle\mathcal S},{\bm{coin}}_{\scriptscriptstyle\mathcal Ar})$. It is run by the smart contract and can be invoked by the client or server. It takes as input two binary strings $\bm{y}_{\scriptscriptstyle \mathcal C}$ and $\bm{y}_{\scriptscriptstyle \mathcal S}$ that indicate which party misbehaved in each verification, and coins' token $T_{\scriptscriptstyle cp}:=\{cp,p_{\scriptscriptstyle\mathcal S},g_{\scriptscriptstyle cp}\}$. If $a=1$ and  $coin^{\scriptscriptstyle *}_{\scriptscriptstyle\mathcal S}= p_{\scriptscriptstyle\mathcal S}$, then it verifies the validity of $T_{\scriptscriptstyle cp}$. If $T_{\scriptscriptstyle cp}$ is rejected, then it aborts. 
If it is accepted, then it constructs  vector ${\bm{coin}}_{\scriptscriptstyle\mathcal I}$, where $\mathcal{I}\in\{\mathcal{C},\mathcal{S},\mathcal{A}r\}$; It sends ${{coin}}_{\scriptscriptstyle\mathcal{I},j}\in{\bm{coin}}_{\scriptscriptstyle\mathcal I}$ coins to party $\mathcal{I}$ for each $j$-th verification. Otherwise (i.e. $a=0$ or $coin^{\scriptscriptstyle *}_{\scriptscriptstyle\mathcal S}\neq p_{\scriptscriptstyle\mathcal S}\in t_{\scriptscriptstyle cp}$)  it only sends $coin^{\scriptscriptstyle *}_{\scriptscriptstyle\mathcal C}$ to $\mathcal C$. 

%sets $coin_{\scriptscriptstyle\mathcal C,0}=coin^{\scriptscriptstyle *}_{\scriptscriptstyle\mathcal C}, \bm{coin}_{\scriptscriptstyle\mathcal S}=\bot$, and $\bm{coin}_{\scriptscriptstyle\mathcal Ar}=\bot$.  %It outputs ${\bm{coin}}_{\scriptscriptstyle\mathcal C},{\bm{coin}}_{\scriptscriptstyle\mathcal S}$, and ${\bm{coin}}_{\scriptscriptstyle\mathcal Ar}$

%sends $coin_{\scriptscriptstyle\mathcal C}, coin_{\scriptscriptstyle\mathcal S}$, and $coin_{\scriptscriptstyle\mathcal Ar}$ coins to the client, server, and arbiter respectively. Otherwise, it does nothing. 

\end{itemize}
\end{definition}

In the above definition algorithms $\mathtt{RCSP}.\mathtt{genQuery},\mathtt{RCSP}.\mathtt{prove},\mathtt{RCSP}.\mathtt{verify}$ and $\mathtt{RCSP}.\mathtt{resolve}$ implicitly  take $a$ as another input and  execute only if $a=1$; however, for the sake of simplicity we  avoided explicitly stating it  in the definition. 

A recurring contingent service payment (RC-S-P)  scheme satisfies correctness and security. At a high level, correctness requires that by the end of the protocol's execution  (that involves  honest client and server) the client receives all $z$ valid service proofs while the server gets paid  for the proofs, without the involvement of the arbiter. More specifically, it requires that the server accepts an honest client's encoded data and query while the honest client accepts the server's valid service proof (and no one is identified as misbehaving party). Moreover, the honest client gets back all its deposited coins minus the service payment, the honest server gets back all its deposited coins  plus the service payment and the arbiter receives nothing. It is formally stated as below.



\begin{definition}[RC-S-P Correctness] A recurring contingent service payment  scheme  is  correct   if for any functions $F, Q, M, E, D$, and auxiliary information $\text{aux}$, the key generation algorithm produces keys $\mathtt{RCSP}.\mathtt{keyGen}(1^\lambda,F)\rightarrow \bm{k}$,   such that for any service input $u$, if $\mathtt{RCSP}.\mathtt{cInit}(1^\lambda, u,\bm{k}, M,z,cp) \rightarrow (u^{\scriptscriptstyle *}, e, T, coin^{\scriptscriptstyle*}_{\scriptscriptstyle\mathcal C},\bm{y}_{\scriptscriptstyle \mathcal  C},\bm{y}_{\scriptscriptstyle \mathcal  S})$, $\mathtt{RCSP}.\mathtt{sInit} (u^{\scriptscriptstyle *}, e, pk, z, T,\bm{y}_{\scriptscriptstyle \mathcal C},$ $\bm{y}_{\scriptscriptstyle \mathcal S})\rightarrow (coin^{\scriptscriptstyle *}_{\scriptscriptstyle\mathcal S}, a)$,   $\forall j:\Big($$\mathtt{RCSP}.\mathtt{genQuery}(1^{\lambda}, \text{aux}, k, Q,  enc)\rightarrow c^{\scriptscriptstyle *}_{\scriptscriptstyle j}$, $\mathtt{RCSP}.\mathtt{prove}(u^{\scriptscriptstyle *},  \sigma, c^{\scriptscriptstyle *}_{\scriptscriptstyle j}, pk,  enc)\rightarrow (b_{\scriptscriptstyle j},m_{\scriptscriptstyle\mathcal {S},j},\pi^{\scriptscriptstyle *}_{\scriptscriptstyle j})$, \\$\mathtt{RCSP}.\mathtt{verify}(\pi^{\scriptscriptstyle *}_{\scriptscriptstyle j}, \bm{q}_{\scriptscriptstyle j},k,enc)\rightarrow (d_{\scriptscriptstyle j}, m_{\scriptscriptstyle\mathcal {C},j})$$\Big), \mathtt{RCSP}.\mathtt{resolve}(\bm{m}_{\scriptscriptstyle \mathcal{C}},\bm{m}_{\scriptscriptstyle \mathcal{S}},z, {\bm{\pi}}^{\scriptscriptstyle *}, {\bm{c}}^{\scriptscriptstyle *},pk,enc)\rightarrow (\bm{y}_{\scriptscriptstyle \mathcal C},\bm{y}_{\scriptscriptstyle \mathcal S})$, \\$\mathtt{RCSP}.\mathtt{pay}(\bm{y}_{\scriptscriptstyle \mathcal C},\bm{y}_{\scriptscriptstyle \mathcal S},T_{\scriptscriptstyle cp},a,coin^{\scriptscriptstyle *}_{\scriptscriptstyle\mathcal C},coin^{\scriptscriptstyle *}_{\scriptscriptstyle\mathcal S})\rightarrow (\bm{coin}_{\scriptscriptstyle\mathcal C},\bm{coin}_{\scriptscriptstyle\mathcal S},\bm{coin}_{\scriptscriptstyle\mathcal Ar})$, then $(a=1)\wedge (\bigwedge\limits^{\scriptscriptstyle z}_{\scriptscriptstyle j=1} b_{\scriptscriptstyle j}= \bigwedge\limits^{\scriptscriptstyle z}_{\scriptscriptstyle j=1} d_{\scriptscriptstyle j}=1)\wedge (\bm{y}_{\scriptscriptstyle \mathcal C}=\bm{y}_{\scriptscriptstyle \mathcal S}=0) \wedge (\sum\limits^{\scriptscriptstyle z}_{\scriptscriptstyle j=1}coin_{\scriptscriptstyle\mathcal C,j}=coin^{\scriptscriptstyle *}_{\scriptscriptstyle\mathcal C}-o\cdot z) \wedge  (\sum\limits^{\scriptscriptstyle z}_{\scriptscriptstyle j=1}coin_{\scriptscriptstyle\mathcal S,j}=coin^{\scriptscriptstyle *}_{\scriptscriptstyle\mathcal S}+o\cdot z)  \wedge (\sum\limits^{\scriptscriptstyle z}_{\scriptscriptstyle j=1}coin_{\scriptscriptstyle\mathcal Ar,j}=0)$
\end{definition}
\textcolor{blue}{XXXX Thomas: something is not right in the definition of correctness... we write $\forall j$ but the execution of $\mathtt{RCSP}.\mathtt{genQuery}$ does not depend on $j$. Not sure how to fix it yet.XXXX}

A RC-S-P scheme is said to be secure if it satisfies  three main properties: (a)  security against malicious server, (b) security against malicious client, and (c) privacy.  In the following, we formally define each of them.

 Intuitively, security against a malicious server states that (at the end of the protocol execution) either (i) for each verification the client gets a valid proof and gets back its deposit minus the service payment, or (ii) the client gets its deposit back (for $j$-th verification) and the arbiter receives $l$ coins. In other words, for every $j$ and $j$-th verification, the security requires that only with a negligible probability the adversary wins, if it  provides either (a)  correct evaluation of the function on the service input but it makes the client  withdraw an incorrect amount of coins (i.e. something other than its deposit minus service payment) or (b) incorrect evaluation of the function on the service input,  but either persuades the client or the arbiter to accept it (i.e.,  $b_{\scriptscriptstyle j}=1$ or $y_{\scriptscriptstyle\mathcal S}[j]=0$) or makes them  withdraw incorrect amount of coins (i.e., $coin_{\scriptscriptstyle\mathcal{C},j}\neq \frac{coin_{\scriptscriptstyle\mathcal C}^{\scriptscriptstyle*}}{z}$ or  $coin_{\scriptscriptstyle\mathcal{A}r,j}\neq l$ coins).  Below, we formalize this intuition. 




%Intuitively, security against a malicious server states that (at the end of the protocol execution) either the client for each verification gets a valid proof or the client gets its coins back (for $j$-th verification) and the arbiter receives $l$ coins. In other words, for each  $j$-th verification, the adversary wins if it provides incorrect evaluation of the function on the service input,  but (with a high probability) either persuades the client or the arbiter to accept it (i.e.  $b_{\scriptscriptstyle j}=1$ or $y_{\scriptscriptstyle\mathcal S}[j]=0$) or makes them to withdraw incorrect amount of coins (i.e. $coin_{\scriptscriptstyle\mathcal{C},j}\neq \frac{coin_{\scriptscriptstyle\mathcal C}^{\scriptscriptstyle*}}{z}$ and  $coin_{\scriptscriptstyle\mathcal{A}r,j}\neq l$ coins).  Below, we formalize this intuition with an experiment, $\textbf{Exp}^{\scriptscriptstyle(1)}_{\scriptscriptstyle\mathcal {A}}[]$: 


%
%\small{
%$$
%    \begin{array}{l}
%    \mathtt{RCSP.keyGen}(1^{\lambda},F)\rightarrow \bm{k}\\
%    \mathcal{A}(1^\lambda,pk, F)\rightarrow (u,\text{state})\\
%    \mathtt{RCSP}.\mathtt{cInit}(1^\lambda, u,\bm{k}, M,z,cp)\rightarrow (u^{\scriptscriptstyle *},e,t,coin^{\scriptscriptstyle*}_{\scriptscriptstyle\mathcal C})\\
%    \mathcal{A}(u^{\scriptscriptstyle *},e, pk,z, t)\rightarrow (coin^{\scriptscriptstyle *}_{\scriptscriptstyle\mathcal S},a)\\
%	\mathtt{RCSP}.\mathtt{genQuery}(1^{\lambda}, \text{aux},k,Q,j, en)\rightarrow c^{\scriptscriptstyle *}_{\scriptscriptstyle j}\\
%	
%     \mathcal{A}(\text{state},c^{\scriptscriptstyle *}_{\scriptscriptstyle j},\sigma, u^{\scriptscriptstyle *},j,en,a)\rightarrow 
%     (b_{\scriptscriptstyle j},m_{\scriptscriptstyle \mathcal{A},j},y^{\scriptscriptstyle *}_{\scriptscriptstyle j},\delta^{\scriptscriptstyle *}_{\scriptscriptstyle j})\\ 
%   \mathtt{RCSP}.\mathtt{verify}(\pi^{\scriptscriptstyle *}_{\scriptscriptstyle j}, q,k,j,en)\rightarrow (d_{\scriptscriptstyle j},m_{\scriptscriptstyle \mathcal{C},j})\\
%   \mathtt{RCSP}.\mathtt{resolve}(\bm{m}_{\scriptscriptstyle \mathcal{C}},\bm{m}_{\scriptscriptstyle \mathcal{A}},z, {\bm{\pi}}^{\scriptscriptstyle *}, {\bm{c}}^{\scriptscriptstyle *},pk,en)\rightarrow (y_{\scriptscriptstyle \mathcal C},y_{\scriptscriptstyle \mathcal S})\\
%   \mathtt{RCSP}.\mathtt{pay}(y_{\scriptscriptstyle \mathcal C},y_{\scriptscriptstyle \mathcal S},t_{\scriptscriptstyle cp})\rightarrow ({\bm{coin}}_{\scriptscriptstyle\mathcal C},{\bm{coin}}_{\scriptscriptstyle\mathcal S},{\bm{coin}}_{\scriptscriptstyle\mathcal Ar})\\
%\end{array}$$
%}
%
%
%
%The output of the experiment is $1$, i.e. $\textit{\textbf{Exp}}^{\scriptscriptstyle(1)}_{\scriptscriptstyle\mathcal {A},\text{RC-S-P}}()=1$, if the adversary wins. 
%
%\begin{definition}[RC-S-P Security Against Malicious Server]\label{deff::}  A RC-S-P  is secure against a malicious server  for a function $F$, if for any security parameter $\lambda$, any auxiliary information $aux$, any $Q$ and any probabilistic polynomial time adversary $\mathcal{A}$, there exists a negligible function $\mu(.)$, such that: 
%%\footnotesize{
%
%$$Pr[\textbf{Exp}^{\scriptscriptstyle(1)}_{\scriptscriptstyle\mathcal {A},\text{RC-S-P}}()=1]=Pr\left[
%\begin{array}{l}
%\left((F_{\scriptscriptstyle q}(u^{\scriptscriptstyle *})= y_{\scriptscriptstyle j}\ \wedge coin_{\scriptscriptstyle\mathcal{C},j}\neq  \frac{coin_{\scriptscriptstyle\mathcal C}^{\scriptscriptstyle*}}{z}-o\right)
% \vee\\
%  \left(F_{\scriptscriptstyle q}(u^{\scriptscriptstyle *})\neq y_{\scriptscriptstyle j}\ \wedge (d_{\scriptscriptstyle j}=1	\vee y_{\scriptscriptstyle \mathcal S}[j]=0 \ \vee coin_{\scriptscriptstyle\mathcal{C},j}\neq \frac{coin_{\scriptscriptstyle\mathcal C}^{\scriptscriptstyle*}}{z} \ \wedge coin_{\scriptscriptstyle\mathcal{A}r,j}\neq l)\right)\end{array}\right]\leq \mu(\lambda)$$
%\end{definition}



%_{\scriptscriptstyle q_{_{\scaleto{j}{\size}}}}


\begin{definition}[RC-S-P Security Against Malicious Server]\label{deff::RC-S-P-SecurityAgainstMaliciousServer}  A RC-S-P  is secure against a malicious server  for a function $F$, if for any auxiliary information $\text{aux}$, every $j$ (where $1\leq j\leq z$), any $Q,M, E, D$ and any probabilistic polynomial time adversary $\mathcal{A}$, there exists a negligible function $\mu(\cdot)$, such that for any security parameter $\lambda$: 
%\footnotesize{
{\small
$$ Pr\left[
  \begin{array}{l}
  \left(F(u^{\scriptscriptstyle *}, \bm{q}_{\scriptscriptstyle j})= h_{\scriptscriptstyle j}\ \wedge coin_{\scriptscriptstyle\mathcal{C},j}\neq  \frac{coin_{\scriptscriptstyle\mathcal C}^{\scriptscriptstyle*}}{z}-o\right)
  \vee\\
\Big(
F(u^{\scriptscriptstyle *},\bm{q}_{\scriptscriptstyle j})\neq h_{\scriptscriptstyle j} \ \wedge (d_{\scriptscriptstyle j}=1	\vee y_{\scriptscriptstyle \mathcal S}[j]=0 \ \vee \\(coin_{\scriptscriptstyle\mathcal{C},j}\neq \frac{coin_{\scriptscriptstyle\mathcal C}^{\scriptscriptstyle*}}{z} \ \vee  coin_{\scriptscriptstyle\mathcal{A}r,j}\neq l)) \Big)\\
%\text{s.t.}\\
%y_{\scriptscriptstyle j}= E^{\scriptscriptstyle -1}_{\scriptscriptstyle 2}(y^{\scriptscriptstyle *}_{\scriptscriptstyle j},t_{\scriptscriptstyle qp})\\
%E^{\scriptscriptstyle -1}_{\scriptscriptstyle 2},t_{\scriptscriptstyle qp}\in en\\
\end{array} \middle |
    \begin{array}{l}
    \mathtt{RCSP.keyGen}(1^{\lambda},F)\rightarrow \bm{k}\\
    \mathcal{A}(1^\lambda,pk, F)\rightarrow u\\
  
    \mathtt{RCSP}.\mathtt{cInit}(1^\lambda, u,\bm{k}, M,z,cp)\rightarrow (u^{\scriptscriptstyle *},e,T,coin^{\scriptscriptstyle*}_{\scriptscriptstyle\mathcal C}, \bm{y}_{\scriptscriptstyle \mathcal  C},\bm{y}_{\scriptscriptstyle \mathcal  S})\\
    
    \mathcal{A}(u^{\scriptscriptstyle *},e, pk,z, T,\bm{y}_{\scriptscriptstyle \mathcal  C},\bm{y}_{\scriptscriptstyle \mathcal  S})\rightarrow (coin^{\scriptscriptstyle *}_{\scriptscriptstyle\mathcal S},a)\\
	\mathtt{RCSP}.\mathtt{genQuery}(1^{\lambda}, \text{aux},k,Q, enc)\rightarrow c^{\scriptscriptstyle *}_{\scriptscriptstyle j}\\
	
     \mathcal{A}(c^{\scriptscriptstyle *}_{\scriptscriptstyle j},\sigma, u^{\scriptscriptstyle *},j,enc,a)\rightarrow 
     (b_{\scriptscriptstyle j},m_{\scriptscriptstyle \mathcal{A},j},h^{\scriptscriptstyle *}_{\scriptscriptstyle j},\delta^{\scriptscriptstyle *}_{\scriptscriptstyle j})\\
     
     %(\pi^{\scriptscriptstyle *}_{\scriptscriptstyle j},b_{\scriptscriptstyle j})\\
     
\mathtt{RCSP}.\mathtt{verify}(\pi^{\scriptscriptstyle *}_{\scriptscriptstyle j}, \bm{q}_{\scriptscriptstyle j}, k, enc)\rightarrow (d_{\scriptscriptstyle j},m_{\scriptscriptstyle \mathcal{C},j})\\
   \mathtt{RCSP}.\mathtt{resolve}(\bm{m}_{\scriptscriptstyle \mathcal{C}},\bm{m}_{\scriptscriptstyle \mathcal{A}},z, {\bm{\pi}}^{\scriptscriptstyle *}, {\bm{c}}^{\scriptscriptstyle *},pk,enc)\rightarrow (\bm{y}_{\scriptscriptstyle \mathcal C},\bm{y}_{\scriptscriptstyle \mathcal S})\\
   \mathtt{RCSP}.\mathtt{pay}(\bm{y}_{\scriptscriptstyle \mathcal C},\bm{y}_{\scriptscriptstyle \mathcal S},T_{\scriptscriptstyle cp},a,coin^{\scriptscriptstyle *}_{\scriptscriptstyle\mathcal C},coin^{\scriptscriptstyle *}_{\scriptscriptstyle\mathcal S})\rightarrow ({\bm{coin}}_{\scriptscriptstyle\mathcal C},{\bm{coin}}_{\scriptscriptstyle\mathcal S},{\bm{coin}}_{\scriptscriptstyle\mathcal Ar})\\
\end{array}    \right]\leq \mu(\lambda)$$
}
where $h_{\scriptscriptstyle j}= D(h^{\scriptscriptstyle *}_{\scriptscriptstyle j},t_{\scriptscriptstyle qp})$, $
D,t_{\scriptscriptstyle qp}\in en$, $\sigma\in e$, and the probability is taken over uniform  choice of $k,k'\stackrel{\scriptscriptstyle\$}\leftarrow \{0,1\}^{\scriptscriptstyle\lambda}$, where $k,k'\in\bm{k}$, as well as the randomness of $\mathcal{A}$ and the randomness used in witnesses $g_{\scriptscriptstyle cp},g_{\scriptscriptstyle qp}\in t$.
\end{definition}




%\begin{definition}[RC-S-P Correctness] A recurring contingent service payment  scheme  is  correct   if for any function $F$, the key generation algorithm produces keys $\mathtt{RCSP}.\mathtt{keyGen}(1^\lambda,F)\rightarrow (k,k')$,   such that $\forall u \in \mathtt{Domain}(F)$ if $\mathtt{RCSP}.\mathtt{cInit}(1^\lambda, u,k,k', M,z,o,l,E_{\scriptscriptstyle 1}) \rightarrow (u^{\scriptscriptstyle *}, e, t, coin^{\scriptscriptstyle*}_{\scriptscriptstyle\mathcal C})$, $\mathtt{RCSP}.\mathtt{sInit} (u^{\scriptscriptstyle *}, e, pk, z, t, E_{\scriptscriptstyle 1})\rightarrow (coin^{\scriptscriptstyle *}_{\scriptscriptstyle\mathcal S}, a)$, \ \\ $\mathtt{RCSP}.\mathtt{genQuery}(1^{\lambda}, \text{aux}, k, Q, j, E_{\scriptscriptstyle 2}, qp, a)\rightarrow c^{\scriptscriptstyle *}_{\scriptscriptstyle j}$, $\mathtt{RCSP}.\mathtt{prove}(u^{\scriptscriptstyle *},  \sigma, c^{\scriptscriptstyle *}_{\scriptscriptstyle j}, pk, j, E^{\scriptscriptstyle -1}_{\scriptscriptstyle 2},qp,a)\rightarrow (\pi^{\scriptscriptstyle *}_{\scriptscriptstyle j}, b_{\scriptscriptstyle j})$, \\$\mathtt{RCSP}.\mathtt{verify}(\pi^{\scriptscriptstyle *}_{\scriptscriptstyle j}, q,k,j,E^{\scriptscriptstyle -1}_{\scriptscriptstyle 2},qp,a)\rightarrow d_{\scriptscriptstyle j}$, $\mathtt{RCSP}.\mathtt{resolve}(m_{\scriptscriptstyle \mathcal{C}},m_{\scriptscriptstyle \mathcal{S}},z, {\bm{\pi}}^{\scriptscriptstyle *}, {\bm{c}}^{\scriptscriptstyle *},pk,E^{\scriptscriptstyle -1}_{\scriptscriptstyle 2},t_{\scriptscriptstyle qp},a)\rightarrow (y_{\scriptscriptstyle \mathcal C},y_{\scriptscriptstyle \mathcal S})$, \\$\mathtt{RCSP}.\mathtt{pay}(y_{\scriptscriptstyle \mathcal C},y_{\scriptscriptstyle \mathcal S},t_{\scriptscriptstyle cp},a)\rightarrow (coin_{\scriptscriptstyle\mathcal C},coin_{\scriptscriptstyle\mathcal S},coin_{\scriptscriptstyle\mathcal Ar})$, s.t. $a=1\wedge (\bigwedge\limits^{\scriptscriptstyle z}_{\scriptscriptstyle j=1} b_{\scriptscriptstyle j}= \bigwedge\limits^{\scriptscriptstyle z}_{\scriptscriptstyle j=1} d_{\scriptscriptstyle j}=1)\wedge (y_{\scriptscriptstyle \mathcal C}=y_{\scriptscriptstyle \mathcal S}=0) \wedge (coin_{\scriptscriptstyle\mathcal C}=coin^{\scriptscriptstyle *}_{\scriptscriptstyle\mathcal C}-oz) \wedge  (coin_{\scriptscriptstyle\mathcal S}=coin^{\scriptscriptstyle *}_{\scriptscriptstyle\mathcal S}+oz)   \wedge (coin_{\scriptscriptstyle\mathcal Ar}=0) $
%\end{definition}

 
 
 
 
Informally, security against a malicious client requires that, for each  $j$-th verification, a malicious client with a negligible probability  wins if it provides either (a) valid metadata and query but makes the server receive incorrect amount of coins (something other than its deposit plus the service payment), or (b) invalid metadata or query but convinces the server to accept either of them, or (c) invalid query but persuades the arbiter to accept it, or  makes them to withdraw an incorrect amount of coins (i.e. $coin_{\scriptscriptstyle\mathcal{S},j}\neq \frac{coin_{\scriptscriptstyle\mathcal S}^{\scriptscriptstyle*}}{z}+o$ or  $coin_{\scriptscriptstyle\mathcal{A}r,j}\neq l$ coins). Below, we formally state it. 
 
 
 
 
 \begin{definition}[RC-S-P Security Against Malicious Client]\label{deff::RC-S-P-Security-Against Malicious-Client}  A RC-S-P  is secure against a malicious client  for a function $F$, if for any security parameter $\lambda$, every $j$ (where $1\leq j\leq z$), any $Q,M,E,D$ and any probabilistic polynomial time adversary $\mathcal{A}$, there exists a negligible function $\mu(.)$, such that: 


{\small
$$ Pr\left[
  \begin{array}{l}
 \Big(M(u^{\scriptscriptstyle *},k)= \sigma  \wedge Q(\text{aux},k)= \bm{q}_{\scriptscriptstyle j} \wedge\\
  coin_{\scriptscriptstyle\mathcal{S},j}\neq  \frac{coin_{\scriptscriptstyle\mathcal S}^{\scriptscriptstyle*}}{z}+o\Big) \vee
  \\
  \Big(M(u^{\scriptscriptstyle *},k)\neq \sigma \wedge a=1\Big)\vee\\ \Big(Q(\text{aux},k)\neq \bm{q}_{\scriptscriptstyle j} \wedge
 b_{\scriptscriptstyle j}=1  \Big)  \vee\\

  \Big(Q(\text{aux},k)\neq \bm{q}_{\scriptscriptstyle j} \wedge
  (y_{\scriptscriptstyle\mathcal{C}}[j]=0 \ \vee \\ coin_{\scriptscriptstyle\mathcal{S},j}\neq \frac{coin_{\scriptscriptstyle\mathcal S}^{\scriptscriptstyle*}}{z}+o \ \vee  coin_{\scriptscriptstyle\mathcal{A}r,j}\neq l) 
  \Big)
 \\
 
\end{array} \middle |
    \begin{array}{l}
   \mathcal{A}(1^{\lambda},F)\rightarrow (u^{\scriptscriptstyle *},z,\bm{k},e,t,coin^{\scriptscriptstyle*}_{\scriptscriptstyle\mathcal C},en, \text{aux},\bm{y}_{\scriptscriptstyle \mathcal  C},\bm{y}_{\scriptscriptstyle \mathcal  S})\\
   
   \mathtt{RCSP}.\mathtt{sInit}(u^{\scriptscriptstyle *},e, pk,z, t,\bm{y}_{\scriptscriptstyle \mathcal  C},\bm{y}_{\scriptscriptstyle \mathcal  S})\rightarrow (coin^{\scriptscriptstyle *}_{\scriptscriptstyle\mathcal S},a)\\
   
	\mathcal{A}(coin^{\scriptscriptstyle *}_{\scriptscriptstyle\mathcal S},a, \text{aux},\bm{k},Q,j, en)\rightarrow c^{\scriptscriptstyle *}_{\scriptscriptstyle j}\\

     \mathtt{RCSP}.\mathtt{prove}(u^{\scriptscriptstyle *},  \sigma,  c^{\scriptscriptstyle *}_{\scriptscriptstyle j},pk, en)\rightarrow (b_{\scriptscriptstyle j},m_{\scriptscriptstyle \mathcal{S},j},\pi^{\scriptscriptstyle *}_{\scriptscriptstyle j})\\
     
 \mathcal{A}(\pi^{\scriptscriptstyle *}_{\scriptscriptstyle j}, \bm{q}_{\scriptscriptstyle j}, k, j, en)\rightarrow (d_{\scriptscriptstyle j},m_{\scriptscriptstyle \mathcal{A},j})\\
   
   
   \mathtt{RCSP}.\mathtt{resolve}(\bm{m}_{\scriptscriptstyle \mathcal{A}},\bm{m}_{\scriptscriptstyle \mathcal{S}},z, {\bm{\pi}}^{\scriptscriptstyle *}, {\bm{c}}^{\scriptscriptstyle *},pk,en)\rightarrow (\bm{y}_{\scriptscriptstyle \mathcal C},\bm{y}_{\scriptscriptstyle \mathcal S})\\
   \mathtt{RCSP}.\mathtt{pay}(\bm{y}_{\scriptscriptstyle \mathcal C},\bm{y}_{\scriptscriptstyle \mathcal S},t_{\scriptscriptstyle cp},a,coin^{\scriptscriptstyle *}_{\scriptscriptstyle\mathcal C},coin^{\scriptscriptstyle *}_{\scriptscriptstyle\mathcal S})\rightarrow ({\bm{coin}}_{\scriptscriptstyle\mathcal C},{\bm{coin}}_{\scriptscriptstyle\mathcal S},{\bm{coin}}_{\scriptscriptstyle\mathcal Ar})\\
\end{array}    \right]\leq \mu(\lambda)$$
}
where $\bm{q}_{\scriptscriptstyle j}\in D(c^{\scriptscriptstyle *}_{\scriptscriptstyle j},t_{\scriptscriptstyle qp})$, $
D,t_{\scriptscriptstyle qp}\in en$, $\sigma\in e$, and   the probability is taken over the randomness used in witnesses $g_{\scriptscriptstyle cp},g_{\scriptscriptstyle qp}\in t$ as well as  the randomness used in proofs $\bm{w}_{\scriptscriptstyle q_{_{\scaleto{j}{\size}}}}\in c_{\scriptscriptstyle j}$ and $w_{\scriptscriptstyle\sigma}\in e$ and  the randomness of $\mathcal{A}$.
\end{definition}
 










%\small{
%$$ Pr\left[
%  \begin{array}{l}
% \Big((M(u^{\scriptscriptstyle *},k)= \sigma  \wedge Q(\text{aux},k)= q_{\scriptscriptstyle j}) \wedge\\
%  (coin_{\scriptscriptstyle\mathcal{S},j}\neq  \frac{coin_{\scriptscriptstyle\mathcal S}^{\scriptscriptstyle*}}{z}+o)\Big) \vee
%  \\
%  \Big((M(u^{\scriptscriptstyle *},k)\neq \sigma \lor Q(\text{aux},k)\neq q_{\scriptscriptstyle j}) \wedge\\
%  (a=1 \ \vee b_{\scriptscriptstyle j}=1\ \vee y_{\scriptscriptstyle\mathcal{C}}[j]=0 \ \vee \\ coin_{\scriptscriptstyle\mathcal{S},j}\neq \frac{coin_{\scriptscriptstyle\mathcal S}^{\scriptscriptstyle*}}{z}+o \ \vee  coin_{\scriptscriptstyle\mathcal{A}r,j}\neq l) 
%  \Big)
% \\
% 
%\end{array} \middle |
%    \begin{array}{l}
%   \mathcal{A}(1^{\lambda},F)\rightarrow (u^{\scriptscriptstyle *},z,\bm{k},e,t,coin^{\scriptscriptstyle*}_{\scriptscriptstyle\mathcal C},en, \text{aux},\bm{y}_{\scriptscriptstyle \mathcal  C},\bm{y}_{\scriptscriptstyle \mathcal  S})\\
%   
%   \mathtt{RCSP}.\mathtt{sInit}(u^{\scriptscriptstyle *},e, pk,z, t,\bm{y}_{\scriptscriptstyle \mathcal  C},\bm{y}_{\scriptscriptstyle \mathcal  S})\rightarrow (coin^{\scriptscriptstyle *}_{\scriptscriptstyle\mathcal S},a)\\
%   
%	\mathcal{A}(coin^{\scriptscriptstyle *}_{\scriptscriptstyle\mathcal S},a, \text{aux},\bm{k},Q,j, en)\rightarrow c^{\scriptscriptstyle *}_{\scriptscriptstyle j}\\
%
%     \mathtt{RCSP}.\mathtt{prove}(u^{\scriptscriptstyle *},  \sigma,  c^{\scriptscriptstyle *}_{\scriptscriptstyle j},pk, en)\rightarrow (b_{\scriptscriptstyle j},m_{\scriptscriptstyle \mathcal{S},j},\pi^{\scriptscriptstyle *}_{\scriptscriptstyle j})\\
%     
% \mathcal{A}(\pi^{\scriptscriptstyle *}_{\scriptscriptstyle j}, q,k,j,en)\rightarrow (d_{\scriptscriptstyle j},m_{\scriptscriptstyle \mathcal{A},j})\\
%   
%   
%   \mathtt{RCSP}.\mathtt{resolve}(\bm{m}_{\scriptscriptstyle \mathcal{A}},\bm{m}_{\scriptscriptstyle \mathcal{S}},z, {\bm{\pi}}^{\scriptscriptstyle *}, {\bm{c}}^{\scriptscriptstyle *},pk,en)\rightarrow (\bm{y}_{\scriptscriptstyle \mathcal C},\bm{y}_{\scriptscriptstyle \mathcal S})\\
%   \mathtt{RCSP}.\mathtt{pay}(\bm{y}_{\scriptscriptstyle \mathcal C},\bm{y}_{\scriptscriptstyle \mathcal S},t_{\scriptscriptstyle cp},a,coin^{\scriptscriptstyle *}_{\scriptscriptstyle\mathcal C},coin^{\scriptscriptstyle *}_{\scriptscriptstyle\mathcal S})\rightarrow ({\bm{coin}}_{\scriptscriptstyle\mathcal C},{\bm{coin}}_{\scriptscriptstyle\mathcal S},{\bm{coin}}_{\scriptscriptstyle\mathcal Ar})\\
%\end{array}    \right]\leq \mu(\lambda)$$
%}
%where $q_{\scriptscriptstyle j}\in D(c^{\scriptscriptstyle *}_{\scriptscriptstyle j},t_{\scriptscriptstyle qp})$, $
%D,t_{\scriptscriptstyle qp}\in en$, $\sigma\in e$, and   the probability is taken over the randomness used in witnesses $w_{\scriptscriptstyle cp},w_{\scriptscriptstyle qp}\in t$ as well as  the randomness used in proofs $w_{\scriptscriptstyle q_{_{\scaleto{j}{\size}}}}\in c_{\scriptscriptstyle j}$ and $w_{\scriptscriptstyle\sigma}\in e$ and  the randomness of $\mathcal{A}$.
%\end{definition}
% 
 Note, in the above definition, an honest server either does not deposit (e.g. when $a=0$) or if it deposits (i.e. agrees to serve) ultimately receives its deposit \emph{plus the service payment} (with a high probability). Informally, RC-S-P is privacy preserving if it guarantees the privacy of service input (e.g. outsourced file) and service proof's status during the private time bubble. In the following, we formally define them.
 

  \begin{definition}[RC-S-P Privacy]\label{deff::RC-S-P-Privacy}  A RC-S-P   preserves privacy, if for any security parameter $\lambda$, any auxiliary information $\text{aux}$, any $F, Q,M, E$ and $D$, the following hold:
   
\begin{enumerate}

\item For any PPT adversary $\mathcal{A}_{\scriptscriptstyle 1}$ there exists a negligible function $\mu_{\scriptscriptstyle 1}(\cdot)$ such that
  
{\small
$$ Pr\left[
  \begin{array}{l}
  \mathcal{A}_{\scriptscriptstyle 1}(\bm{c}^{\scriptscriptstyle *},coin^{\scriptscriptstyle *}_{\scriptscriptstyle S},coin^{\scriptscriptstyle *}_{\scriptscriptstyle C},g_{\scriptscriptstyle cp},  g_{\scriptscriptstyle qp},\bm{\pi}^{\scriptscriptstyle *},a)\rightarrow\beta\\
\end{array} \middle |
    \begin{array}{l}
    \mathtt{RCSP.keyGen}(1^{\lambda},F)\rightarrow \bm{k}\\
    \mathcal{A}_{\scriptscriptstyle 1}(1^\lambda,pk, F)\rightarrow (u_{\scriptscriptstyle 0},u_{\scriptscriptstyle 1})\\
    \beta\stackrel{\scriptscriptstyle\$}\leftarrow\{0,1\}\\
   \mathtt{RCSP}.\mathtt{cInit}(1^\lambda, u_{\scriptscriptstyle\beta},\bm{k}, M,z,cp)\rightarrow (u_{\scriptscriptstyle\beta}^{\scriptscriptstyle *},e,t,coin^{\scriptscriptstyle*}_{\scriptscriptstyle\mathcal C}, \bm{y}_{\scriptscriptstyle \mathcal  C},\bm{y}_{\scriptscriptstyle \mathcal  S})\\
\mathtt{RCSP}.\mathtt{sInit}(u_{\scriptscriptstyle\beta}^{\scriptscriptstyle *},e, pk,z, t,\bm{y}_{\scriptscriptstyle \mathcal  C},\bm{y}_{\scriptscriptstyle \mathcal  S})\rightarrow (coin^{\scriptscriptstyle *}_{\scriptscriptstyle\mathcal S},a)\\
\mathtt{RCSP}.\mathtt{genQuery}(1^\lambda, \text{aux},k,Q_{\scriptscriptstyle\beta},j, en)\rightarrow c^{\scriptscriptstyle *}_{\scriptscriptstyle j}\\
     \mathtt{RCSP}.\mathtt{prove}(u^{\scriptscriptstyle *}_{\scriptscriptstyle\beta},  \sigma,  c^{\scriptscriptstyle *}_{\scriptscriptstyle j},pk, en)\rightarrow (b_{\scriptscriptstyle j},m_{\scriptscriptstyle \mathcal{S},j},\pi^{\scriptscriptstyle *}_{\scriptscriptstyle j})\\  
   \mathtt{RCSP}.\mathtt{verify}(\pi^{\scriptscriptstyle *}_{\scriptscriptstyle j}, \bm{q}_{\scriptscriptstyle j},k,en)\rightarrow (d_{\scriptscriptstyle j},m_{\scriptscriptstyle \mathcal{C},j})\\
\end{array}\right]\leq \frac{1}{2}+\mu_{\scriptscriptstyle 1}(\lambda)$$
}




\item For any PPT adversary $\mathcal{A}_{\scriptscriptstyle 2}$ who plays the above game, there exists a negligible function $\mu_{\scriptscriptstyle 2}(\cdot)$ such that


{\small
$$ Pr\left[\mathcal{A}_{\scriptscriptstyle 2}(\bm{c}^{\scriptscriptstyle *},coin^{\scriptscriptstyle *}_{\scriptscriptstyle \mathcal S},coin^{\scriptscriptstyle *}_{\scriptscriptstyle\mathcal C}, g_{\scriptscriptstyle cp},  g_{\scriptscriptstyle qp},\bm{\pi}^{\scriptscriptstyle *},a)\rightarrow (d_{\scriptscriptstyle j},j)\right]\leq \frac{1}{2}+ \mu_{\scriptscriptstyle 2}(\lambda)$$
}

where $\bm{c}^{\scriptscriptstyle *}=[{c}^{\scriptscriptstyle *}_{\scriptscriptstyle 1},...,{c}^{\scriptscriptstyle *}_{\scriptscriptstyle z}]$ and $\bm{\pi}^{\scriptscriptstyle *}=[\pi^{\scriptscriptstyle *}_{\scriptscriptstyle 1},...,\pi^{\scriptscriptstyle *}_{\scriptscriptstyle z}]$.  Also, depending on the choice of $\beta$, algorithm $\mathtt{RCSP}.\mathtt{genQuery}()$ outputs an encoded valid query, when the input is $Q_{\scriptscriptstyle 1}$, or outputs an encoded invalid query, when the input is $Q_{\scriptscriptstyle 0}$. The probability is taken over uniform  choice of $k,k'\stackrel{\scriptscriptstyle\$}\leftarrow \{0,1\}^{\scriptscriptstyle\lambda}$, where $k,k'\in\bm{k}$, the randomness of $\mathcal{A}_{\scriptscriptstyle 1}$ and $\mathcal{A}_{\scriptscriptstyle 2}$,  the randomness used in witnesses $g_{\scriptscriptstyle cp}, g_{\scriptscriptstyle qp}\in t$,  the randomness used in  $E$.
\end{enumerate}
\end{definition}
 
In the above definition, the experiment is allowed to produce  invalid queries. It is required that  the privacy holds regardless of the queries status, i.e. whether they are valid or invalid, as long as the are encoded and provided.  
 
 
 
%  \begin{definition}[RC-S-P Service Proof Privacy]\label{deff::RC-S-P-ServiceProofPrivacy}  A RC-S-P   preserves the privacy of service proof (status), if for any security parameter $\lambda$, any auxiliary information $\text{aux}$, any $F,Q,M,E,D$ and any probabilistic polynomial time adversary $\mathcal{A}$ that plays the  game in Definition \ref{deff::RC-S-P-Input-Privacy}, there exists a negligible function $\mu(.)$, such that: 
%
%\small{
%$$ Pr\left[\mathcal{A}(\bm{c}^{\scriptscriptstyle *},coin^{\scriptscriptstyle *}_{\scriptscriptstyle \mathcal S},coin^{\scriptscriptstyle *}_{\scriptscriptstyle\mathcal C}, g_{\scriptscriptstyle cp},  g_{\scriptscriptstyle qp},\bm{\pi}^{\scriptscriptstyle *},a)\rightarrow (d_{\scriptscriptstyle j},j)\right]\leq \frac{1}{2}+ \mu(\lambda)$$
%}
%where $\bm{c}^{\scriptscriptstyle *}=[{c}^{\scriptscriptstyle *}_{\scriptscriptstyle 1},...,{c}^{\scriptscriptstyle *}_{\scriptscriptstyle z}]$ and $\bm{\pi}^{\scriptscriptstyle *}=[\pi^{\scriptscriptstyle *}_{\scriptscriptstyle 1},...,\pi^{\scriptscriptstyle *}_{\scriptscriptstyle z}]$.  The probability is taken over uniform  choice of $k,k'\stackrel{\scriptscriptstyle\$}\leftarrow \{0,1\}^{\scriptscriptstyle\lambda}$, where $k,k'\in\bm{k}$, the randomness of $\mathcal{A}$,  the randomness used in witnesses $g_{\scriptscriptstyle cp}, g_{\scriptscriptstyle qp}\in t$,  the randomness used in  $E$.
%\end{definition}
 
 
 
 %-----------------------------------
% \begin{definition}[RC-S-P Service Input Privacy]\label{deff::RC-S-P-Input-Privacy}  A RC-S-P   preserves the privacy of service input, if for any security parameter $\lambda$, any auxiliary information $\text{aux}$, any $F, Q,M, E,D$ and any probabilistic polynomial time adversary $\mathcal{A}$, there exists a negligible function $\mu(.)$, such that: 
%
%\small{
%$$ Pr\left[
%  \begin{array}{l}
%  \mathcal{A}(\bm{c}^{\scriptscriptstyle *},coin^{\scriptscriptstyle *}_{\scriptscriptstyle S},coin^{\scriptscriptstyle *}_{\scriptscriptstyle C},g_{\scriptscriptstyle cp},  g_{\scriptscriptstyle qp},\bm{\pi}^{\scriptscriptstyle *},a)\rightarrow\beta\\
%
%
%\end{array} \middle |
%    \begin{array}{l}
%    \mathtt{RCSP.keyGen}(1^{\lambda},F)\rightarrow \bm{k}\\
%    \mathcal{A}(1^\lambda,pk, F)\rightarrow (u_{\scriptscriptstyle 0},u_{\scriptscriptstyle 1})\\
%    \beta\stackrel{\scriptscriptstyle\$}\leftarrow\{0,1\}\\
%   
%   \mathtt{RCSP}.\mathtt{cInit}(1^\lambda, u_{\scriptscriptstyle\beta},\bm{k}, M,z,cp)\rightarrow (u_{\scriptscriptstyle\beta}^{\scriptscriptstyle *},e,t,coin^{\scriptscriptstyle*}_{\scriptscriptstyle\mathcal C}, \bm{y}_{\scriptscriptstyle \mathcal  C},\bm{y}_{\scriptscriptstyle \mathcal  S})\\
%
%\mathtt{RCSP}.\mathtt{sInit}(u_{\scriptscriptstyle\beta}^{\scriptscriptstyle *},e, pk,z, t,\bm{y}_{\scriptscriptstyle \mathcal  C},\bm{y}_{\scriptscriptstyle \mathcal  S})\rightarrow (coin^{\scriptscriptstyle *}_{\scriptscriptstyle\mathcal S},a)\\
%
%\mathtt{RCSP}.\mathtt{genQuery}(1^\lambda, \text{aux},k,Q,j, en)\rightarrow c^{\scriptscriptstyle *}_{\scriptscriptstyle j}\\
%	
%     
%     \mathtt{RCSP}.\mathtt{prove}(u^{\scriptscriptstyle *}_{\scriptscriptstyle\beta},  \sigma,  c^{\scriptscriptstyle *}_{\scriptscriptstyle j},pk, en)\rightarrow (b_{\scriptscriptstyle j},m_{\scriptscriptstyle \mathcal{S},j},\pi^{\scriptscriptstyle *}_{\scriptscriptstyle j})\\
%     
%   \mathtt{RCSP}.\mathtt{verify}(\pi^{\scriptscriptstyle *}_{\scriptscriptstyle j}, \bm{q}_{\scriptscriptstyle j},k,en)\rightarrow (d_{\scriptscriptstyle j},m_{\scriptscriptstyle \mathcal{C},j})\\
%\end{array}    \right]\leq \frac{1}{2}+\mu(\lambda)$$
%}
%where $\bm{c}^{\scriptscriptstyle *}=[{c}^{\scriptscriptstyle *}_{\scriptscriptstyle 1},...,{c}^{\scriptscriptstyle *}_{\scriptscriptstyle z}]$ and $\bm{\pi}^{\scriptscriptstyle *}=[\pi^{\scriptscriptstyle *}_{\scriptscriptstyle 1},...,\pi^{\scriptscriptstyle *}_{\scriptscriptstyle z}]$.  The probability is taken over uniform  choice of $k,k'\stackrel{\scriptscriptstyle\$}\leftarrow \{0,1\}^{\scriptscriptstyle\lambda}$, where $k,k'\in\bm{k}$, the randomness of $\mathcal{A}$,  the randomness used in witnesses $g_{\scriptscriptstyle cp},g_{\scriptscriptstyle qp}\in t$,  the randomness used in  $E$.
%
%\end{definition}
% 
% 
% 
%  \begin{definition}[RC-S-P Service Proof Privacy]\label{deff::RC-S-P-ServiceProofPrivacy}  A RC-S-P   preserves the privacy of service proof (status), if for any security parameter $\lambda$, any auxiliary information $\text{aux}$, any $F,Q,M,E,D$ and any probabilistic polynomial time adversary $\mathcal{A}$ that plays the  game in Definition \ref{deff::RC-S-P-Input-Privacy}, there exists a negligible function $\mu(.)$, such that: 
%
%\small{
%$$ Pr\left[\mathcal{A}(\bm{c}^{\scriptscriptstyle *},coin^{\scriptscriptstyle *}_{\scriptscriptstyle \mathcal S},coin^{\scriptscriptstyle *}_{\scriptscriptstyle\mathcal C}, g_{\scriptscriptstyle cp},  g_{\scriptscriptstyle qp},\bm{\pi}^{\scriptscriptstyle *},a)\rightarrow (d_{\scriptscriptstyle j},j)\right]\leq \frac{1}{2}+ \mu(\lambda)$$
%}
%where $\bm{c}^{\scriptscriptstyle *}=[{c}^{\scriptscriptstyle *}_{\scriptscriptstyle 1},...,{c}^{\scriptscriptstyle *}_{\scriptscriptstyle z}]$ and $\bm{\pi}^{\scriptscriptstyle *}=[\pi^{\scriptscriptstyle *}_{\scriptscriptstyle 1},...,\pi^{\scriptscriptstyle *}_{\scriptscriptstyle z}]$.  The probability is taken over uniform  choice of $k,k'\stackrel{\scriptscriptstyle\$}\leftarrow \{0,1\}^{\scriptscriptstyle\lambda}$, where $k,k'\in\bm{k}$, the randomness of $\mathcal{A}$,  the randomness used in witnesses $g_{\scriptscriptstyle cp}, g_{\scriptscriptstyle qp}\in t$,  the randomness used in  $E$.
%\end{definition}
 %-------------------------
 
 
 
 \begin{definition}[RC-S-P Security]\label{def::RC-S-P-Security} A RC-S-P scheme is secure if it satisfies security against malicious server, security against malicious client, and preserves privacy, w.r.t. Definitions \ref{deff::RC-S-P-SecurityAgainstMaliciousServer}-\ref{deff::RC-S-P-Privacy}. 
 \end{definition}
 

 
 


%Algorithm $\mathtt{VSID.identify}()$   allows an arbiter  to identify a misbehaving party even in the setup phase. Nevertheless, often it is sufficient  to let the arbiter pinpoint a corrupt party after the client and server agree to deal with each other, i.e. after the setup when the server runs  $\mathtt{VSID.serve}$ and outputs $1$. In this case, $\mathtt{VSID.identify}()$  only needs to take $(\pi,c,k)$ as input, and checks the validity of proofs $w_q\in c$ and $\pi$. A VSID protocol that meets the latter (lighter) requirement would impose lower costs. Such a scheme is denoted by $\text{VSID}_{\scriptscriptstyle \text{light}}$ in this paper. 


\input{VSID-Protocol}






%In the case where the party's outgoing message input is of large size, the parties can post a hash of the data to the board and send the data directly to its counter party. This requires, the other party to simply check if the data matches the representation and then sends to the board a short string declaring it has received it. 








%In the above definition,  $\mathtt{Service.verify}()$ algorithm  allows a verifier to detect only a misbehaving server; nevertheless, it is not suitable to  detect a misbehaving client who may try to falsely accuse the server.  Thus, $\mathtt{Service.resolve()}$ algorithm has also been incorporated in the definition, to allow an arbiter to detect either party's misbehaviour and resolve any dispute between them.  Informally, a service scheme has two main properties: correctness and soundness. The correctness requires that for any key, the verification algorithm accepts a proof generated by an honest prover. The soundness requires that if a prover convinces the verifier (i.e. client or arbiter), with a high probability, then the service has been provided by the prover. {\color {blue} it seems undeniability and accountability/liability, i.e. a malicious party can be identified and held accountable, are needed too. Thin if it's needed here on in the main payment protocol.}
%The exact formalisation of soundness however totally depends on the kind of service provided. 



%Thus, since the two party are mutually distrusted, the above definition has also incorporated Service.resolve() algorithm that  allows an arbiter to resolve any dispute between client and server. 


